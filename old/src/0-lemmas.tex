% !TeX root = ../new.tex

% \begin{lemma}[Splitting Lemma (Hatcher p. 147)]\label{lem:splitting}
%   For a short exact sequence \[0\to A\xrightarrow{i} B\xrightarrow{j} C\to 0\] of abelian groups the following statements are equivalent
%   \begin{enumerate}
%     \item There is a homomorphism $p: B\to A$ such that $p\circ i = \mathbf{Id}_A$.
%     \item There is a homomorphism $s: C\to B$ such that $j\circ s = \mathbf{Id}_C$.
%     \item There is an isomorphism $B\cong A\oplus C$ making the commutative diagram below, where the maps in the lower row are the obvious ones $a\mapsto (a, 0)$ and $(a,c)\mapsto c$.
%
%     \[\begin{tikzcd}[column sep=small, row sep=small]
%               &                       & B\ar[dd, "\cong"]\ar[dr,"j"]  &         & \\
%       0\ar[r] & A\ar[ur, "i"]\ar[dr]  &                               & C\ar[r] & 0\\
%               &                       & A\oplus C\ar[ur]              &         &
%     \end{tikzcd}\]
%   \end{enumerate}
% \end{lemma}

\begin{lemma}[The Five-Lemma (Hatcher p. 129)]\label{lem:five}
  In a commutative diagram of abelian groups as below, if the two rows are exact and $\alpha,\beta,\delta$, and $\e$ are isomorphisms then $\gamma$ is an isomorphism.
  \[\begin{tikzcd}
      A\ar[r, "i"]\ar[d, "\alpha"]
    & B\ar[r, "j"]\ar[d, "\beta"]
    & C\ar[r, "k"]\ar[d, "\gamma"]
    & D\ar[r, "\ell"]\ar[d, "\delta"]
    & E\ar[d, "\e"]\\
    %
      A'\ar[r, "i'"]
    & B'\ar[r, "j'"]
    & C'\ar[r, "k'"]
    & D'\ar[r, "\ell'"]
    & E'\\
  \end{tikzcd}\]

  \begin{itemize}
    \item If $\beta$ and $\delta$ are surjective and $\e$ is injective then $\gamma$ is surjective.
    \item If $\beta$ and $\delta$ are injective and $\alpha$ is surjective then $\gamma$ is injective.
  \end{itemize}
\end{lemma}

\begin{lemma}[Persistent Nerve Lemma (Chazal~\cite{chazal08towards}, Lemma 3.4)]\label{lem:rel_pers_nerve}
  Let $X\subseteq X'$ be two paracompact spaces and $Y\subseteq Y'$ be two subspaces $Y\subseteq X$, $Y'\subseteq X'$.
  Let $\mathcal{U} = \{U_a\}_{a\in A}$ and $\mathcal{U}' = \{U_a'\}_{a\in A}$ be good open covers of $X$ and $X'$, respectively, such that $U_a\subseteq U_a'$ for all $a\in A$.
  Let $\mathcal{V} = \{V_a\}_{a\in A} \subseteq \mathcal{U}$ and $\mathcal{V}' = \{V_a'\}_{a\in A}\subseteq \mathcal{U}'$ be good open subcovers of $Y$ and $Y'$, respectively, such that $V\subseteq V'$ for all $a\in A$.
  Then there exist homotopy equivalences of pairs $(\N\mathcal{U}, \N\mathcal{V})\to (X, Y)$ and $(\N\mathcal{U}', \N\mathcal{V}')$ that commute with the canonical inclusions of pairs $(X, Y)\hookrightarrow (X', Y')$ and $(\N\mathcal{U}, \N\mathcal{V})\hookrightarrow (\N\mathcal{U}', \N\mathcal{V}')$ at the homology and homotopy levels.
\end{lemma}

% \begin{theorem}[Alexander Duality]\label{thm:alexander}
%   If $D$ is a compact, locally contractible, nonempty, proper subspace of $S^d$ then for all $k$ there is an isomorphism
%   \[ \Gamma_D^k : \tilde{\hom}_k(D)\to \tilde{\hom}^{d-k-1}(S^d\setminus D). \]
%
%   If $(D, B)$ is a pair of such subspaces of $S^d$ then for all $k$ there is an isomorphism
%   \[ \Gamma_{(D,B)}^k : \tilde{\hom}_k(D, B)\to \tilde{\hom}^{d-k}(S^d\setminus B, S^d\setminus D). \]
% \end{theorem}

% \begin{lemma}[Lemma 3.2 from~\cite{chazal08towards}]\label{lem:sandwich}
%     Given a sequence $A\to B\to C\to D\to E\to F$ of homomorphisms between finite-dimensional vector spaces, if $\rk(A\to F) = \rk(C\to D)$ then this quantity also equals the rank of $B\to E$.
%     Similarly, if $A\to B\to C\to E\to F$ is a sequence of homomorphisms such that $\rk(A\to F) = \dim~C$ then $\rk(B\to E) = \dim~C$.
% \end{lemma}

% \textbf{TODO
% \begin{itemize}
%   \item Excision
% \end{itemize}}

% \begin{lemma}
%   Suppose $B\subseteq B' \subseteq A$.
%
%   If $\hom_k(B\to B')$ is surjective for all $k$ then $\hom_k((A, B)\to (A,B'))$ is surjective.
% \end{lemma}
% \begin{proof}
%   This result follows from Lemma~\ref{lem:five} applied to the long exact sequences of the pairs $(A, B)$ and $(A, B')$.
%     \begin{equation}\begin{tikzcd}\label{dgm:separate_iso}
%       \hom_k(B)\arrow{r}{i}\arrow{d}{a} &
%       \hom_k(A)\arrow{r}{j}\arrow{d}{b} &
%       \hom_k(A, B)\arrow{r}{k}\arrow{d}{c} &
%       \hom_{k-1}(B)\arrow{r}{\ell}\arrow{d}{d} &
%       \hom_{k-1}(A)\arrow{d}{e} &\\
%       %
%       \hom_k(B')\arrow{r}{i'}%\arrow{d}{a'} &
%       \hom_k(A)\arrow{r}{j'}%\arrow{d}{b'} &
%       \hom_k(A, B')\arrow{r}{k'}%\arrow{d}{c'} &
%       \hom_{k-1}(B')\arrow{r}{\ell'}%\arrow{d}{d'} &
%       \hom_{k-1}(A)%\arrow{d}{e'} \\
%       % %
%       % \hom_k(B'')\arrow{r}{i''}&
%       % \hom_k(A)\arrow{r}{j''}&
%       % \hom_k(A, B'')\arrow{r}{k''}&
%       % \hom_{k-1}(B'')\arrow{r}{\ell''} &
%       % \hom_{k-1}(A)
%     \end{tikzcd}\end{equation}
%     Because $
