% !TeX root = ../../main_socg.tex

\subsection{Proof of the Interleaving}

For $w,\alpha\in\R$ let $\DD{w}^k$ denote the $k$th persistent (relative) homology module of the filtration $\{(D\subi{w}{\alpha},B_w)\}_{\alpha\in\R}$ with respect to $B_w$, and let $\PP{w}{\e,k}$ denote the $k$th persistent (relative) homology module of $\{(P\subi{w}{\alpha}^\e,Q_w^\e)\}_{\alpha\in\R}$.
Similarly, let $\CPP{w}{\e,k}$ and $\RPP{w}{\e,k}$ denote the corresponding \Cech and Vietoris-Rips filtrations, respectively.
We will omit the dimension $k$ and write $\DD{w}$ (resp. $\PP{w}{\e}$) if a statement holds for all dimensions.

If $Q_w^\e$ surrounds $P^\e$ in $D$ let $\ext{\PP{w}{\e}}$ denote the $k$th persistent homology module of the filtration of extensions $\{(\ext{P\subi{w}{\alpha}^\e},\ext{Q_w^\e})\}$, where $\ext{P\subi{w}{a}^\e} = P\subi{w}{a}^\e \cup (D\setminus P^\e)$.
Lemma~\ref{lem:excision} can be extended to show that we have isomorphisms $\E_w^\e \in \Hom(\PP{w}{\e},\ext{\PP{w}{\e}})$
of persistence modules induced by inclusions.
If $\e < \varrho_D$ then we for any $\alpha\in\R$ the inclusion $\cech^\e(P\subi{w}{\alpha}, Q_w)\hookrightarrow (P\subi{w}{\alpha}^\e, Q_w^\e)$ is a homotopy equivalence by the Nerve Theorem.
As the module homomorphisms of $\CPP{w}{\e}$ and $\PP{w}{\e}$ are induced by inclusion we have an isomorphism $\N_w^\e\in\Hom(\CPP{w}{\e}, \PP{w}{\e})$ of persistence modules that commutes with maps induced by inclusions by the Persistent Nerve Lemma
As the isomorphisms of $\E_w^\e$ are given by excision they are induced by inclusions, so the composition $\E\N_w^\e := \E_w^\e\circ \N_w^\e$ is an isomorphism that commutes with maps induced by inclusion as well.
The following lemma uses these isomorphisms along with inclusions $\I_w^\e\in\Hom(\CPP{w}{\e}, \RPP{w}{2\e})$ and $\J_w^\e\in\Hom(\RPP{w}{\e},\CPP{w}{\e})$ to establish image module homomorphisms by maps $\Sigma_w^\e\in\Hom(\PP{w}{\e},\RPP{w}{2\e})$ and $\Upsilon_w^\e\in \Hom(\RPP{w}{\e},\PP{w}{\e})$.
%TODO Its proof, along with the existence of the maps $\E\N_w^\e$ can be found in the \fullversion.

\begin{lemma}\label{lem:rips_homomorphism_left}
  For $w\in\R$ and $\e \leq\varrho_D / 4$ let $\Lambda^\e\in\Hom(\ext{\PP{w}{\e}}, \ext{\PP{z}{2\e}})$ and $\rips\Lambda\in\Hom(\RPP{w}{2\e},\RPP{z}{4\e})$.
  Then $\tilde{\Phi}(\Sigma_w^\e,\Sigma_z^{2\e})\in\Hom(\im~\Lambda^\e,\im~\rips\Lambda)$ and $\tilde{\Psi}(\Upsilon_w^{2\e},\Upsilon_z^{4\e})\in\Hom(\im~\rips\Lambda,\im~\Lambda^{2\e})$ are image module homomorphisms.
\end{lemma}

% Suppose $Q_w^\e$ surrounds $P^\e$ in $D$ for any $w\in\R$ and $\e > 0$.
% Lemma~\ref{lem:surround_and_cover} can be extended to give homomorphisms $\DD{w-c\e}\xrightarrow{F}\E\PP{w}{\e}\xrightarrow{M}\DD{w+c\e}$ of degree $c\e$ induced by inclusions.
Suppose $Q_{\omega-2c\delta}^\delta$ surrounds $P^\delta$ in $D$ and $D\setminus B_\omega\subseteq P^\delta$.% for $\zeta\geq 2\delta$.
Then, because $f$ is $c$-Lipschitz, $B_{\omega-3c\delta}\cap P^\delta\subseteq Q_{\omega-2c\delta}^\delta$ and $B_\omega\cap P^\delta\subseteq Q_{\omega+c\delta}^{2\delta}$.
Similarly, $Q_{\omega-2c\delta}^{2\delta}\subseteq B_\omega$ and $Q_{\omega+c\delta}^{4\delta}\subseteq B_{\omega+5c\delta}$.
Therefore, by Lemma~\ref{lem:surround_and_cover}
\[ B_{\omega-3c\delta}\subseteq \E Q_{\omega-2c\delta}^\delta\subseteq\E Q_{\omega-2c\delta}^{2\delta}\subseteq B_\omega
  \subseteq \E Q_{\omega+c\delta}^{2\delta}\subseteq \E Q_{\omega+c\delta}^{4\delta}\subseteq B_{\omega+5c\delta}.\]
We have the following commutative diagrams of persistence modules where all maps are induced by inclusions.
Proof that inclusions given by Lemma~\ref{lem:surround_and_cover} extend to maps $(F, G)$ and $(M, N)$ of persistence modules can be found in the \fullversion.\\

\begin{subequations}
  \begin{minipage}{0.4\linewidth}
    \begin{equation}\label{eq:partial_left}
      \begin{tikzcd}
        \DD{\omega-3c\delta} \arrow{r}{\Gamma}\arrow{d}{F} &
        \DD{\omega} \arrow{d}{G}\\
        %
        \E\PP{\omega-2c\delta}{\delta}\arrow{r}{\Lambda} &
        \E\PP{\omega+c\delta}{2\delta}
      \end{tikzcd}
    \end{equation}
  \end{minipage}
  \begin{minipage}{0.4\linewidth}
    \begin{equation}\label{eq:partial_right}
      \begin{tikzcd}
        \E\PP{\omega-2c\delta}{2\delta} \arrow{r}{\Lambda'}\arrow{d}{M} &
        \E\PP{\omega+c\delta}{4\delta}\arrow{d}{N}\\
        %
        \DD{\omega} \arrow{r}{\Pi} &
        \DD{\omega+5c\delta}
      \end{tikzcd}
    \end{equation}
  \end{minipage}
\end{subequations}\vspace{2ex}

In the following let $\rips\Lambda\in\Hom(\RPP{\omega-2c\delta}{2\delta},\RPP{\omega+c\delta}{4\delta})$ be induced by inclusion.
Clearly, $\Phi(F, G)$ is an image module homomorphism of degree $2c\delta$ and $\Psi(M, N)$ is an image module homomorphism of degree $4c\delta$.
By Lemma~\ref{lem:rips_homomorphism_left} we have image module homomorphisms $\tilde{\Phi}(\Sigma_{\omega-2c\delta}^\delta, \Sigma_{\omega+c\delta}^{2\delta})$ and $\tilde{\Psi}(\Upsilon_{\omega-2c\delta}^{2\delta}, \Upsilon_{\omega+c\delta}^{4\delta})$.
Therefore, as the composition of image module homomorphisms are image module homomorphisms we have
\[ \rips\Phi := \tilde{\Phi}\circ\Phi\in\Hom^{2c\delta}(\im~\Gamma,\im~\rips\Lambda)\ \text{ and }\ \rips\Psi :=\Psi\circ\tilde{\Psi}\in\Hom^{4c\delta}(\im~\rips\Lambda, \im~\Pi).\]
% given by the compositions
% \[ \rips\Phi(\rips F, \rips G) := (\Sigma_{\omega-2c\delta}^\delta\circ F, \Sigma_{\omega+c\delta}^{2\delta}\circ G)\ \text{ and }\ \rips\Psi(\rips M, \rips N) := (M\circ \Upsilon_{\omega-2c\delta}^{2\delta}, N\circ\Upsilon_{\omega+c\delta}^{4\delta}).\]

Because all maps are induced by inclusions, or commute with maps induced by inclusions it can be shown that $\rips \Phi_{\rips M}$ is a partial $2c\delta$-interleaving of image modules and $\rips \Psi_{\rips G}$ is a partial $4c\delta$-interleaving of image modules by a straightforward diagram chasing argument.
Proof of these facts can be found in the \fullversion.
These maps, along with assumptions that imply $\im(\DD{\omega-3c\delta}\to \DD{\omega+5c\delta})\cong \DD{\omega}$ provide the proof of Theorem~\ref{thm:interleaving_main_2} by Lemma~\ref{thm:interleaving_main}.

\begin{theorem}\label{thm:interleaving_main_2}
  Let $\X$ be a $d$-manifold, $D\subset\X$ and $f : D\to\R$ be a $c$-Lipschitz function.
  Let $\omega\in\R$, $\delta < \varrho_D/4$ be constants such that $B_{\omega-3c\delta}$ surrounds $D$ in $\X$.
  Let $P\subset D$ be a finite subset and suppose $\hom_k(B_{\omega-3c\delta}\hookrightarrow B_\omega)$ is surjective and $\hom_k(B_\omega\hookrightarrow B_{\omega+5c\delta})$ is an isomorphism for all $k$.

  If $D\setminus B_\omega\subseteq P^\delta$ and $Q_{\omega-2c\delta}^\delta$ surrounds $P^\delta$ in $D$ then the $k$th persistent homology module of $\{\rips^{2\delta}(P\subi{\omega-2c\delta}{\alpha}, Q_{\omega-2c\delta})\hookrightarrow \rips^{4\delta}(P\subi{\omega+c\delta}{\alpha}, Q_{\omega+c\delta})\}_{\alpha\in\R}$ is $4c\delta$-interleaved with that of $\{(D\subi{\omega}{\alpha}, B_\omega)\}_{\alpha\in\R}$.
\end{theorem}
\begin{proof}
  Let $\rips\Lambda \in\Hom(\RPP{\omega-2c\delta}{2c\delta}, \RPP{\omega+c\delta}{4c\delta})$ be induced by inclusions.
  Because $D\setminus B_\omega\subseteq P^\delta$ and $Q_{\omega-2c\delta}^\delta$ surrounds $P^\delta$ in $D$ Diagrams~\ref{eq:partial_left} and~\ref{eq:partial_right} commute as all maps are induced by inclusions.
  Moreover, because $\delta < \varrho_D/4$ the isomorphisms provided by the Nerve Theorem commute with inclusions by Lemma~\ref{lem:pers_nerve}.
  So $\rips \Phi_{\rips M}(\rips F, \rips G)\in\Hom^{2c\delta}(\im~\Gamma,\im~\rips\Lambda)$ is a partial $2c\delta$-interleaving of image modules and $\rips \Psi_{\rips G} (\rips M,\rips N)\in\Hom^{4c\delta}(\im~\rips\Lambda, \im~\Pi)$ is a partial $4c\delta$-interleaving of image modules.

  As we have assumed that $\hom_k(B_{\omega-3c\delta}\hookrightarrow B_\omega)$ is surjective and $\hom_k(B_\omega)\cong\hom_k(B_{\omega+5c\delta})$ the five-lemma implies $\gamma_\alpha$ is surjective and $\pi_\alpha$ is an isomorphism (and therefore injective) for all $\alpha$.
  So $\Gamma$ is an epimorphism and $\Pi$ is a monomorphism, thus $\im~\rips\Lambda$ is $4c\delta$-interleaved with $\DD{\omega}$ by Lemma~\ref{thm:interleaving_main} as desired.
\end{proof}
