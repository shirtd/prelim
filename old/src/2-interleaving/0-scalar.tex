% !TeX root = ../../main.tex

For $z,\alpha\in\R$ let $\DD{z}^k$ denote the $k$th persistent (relative) homology module of the filtration $\{(D\subi{z}{\alpha},B_z)\}_{\alpha\in\R}$ with respect to $B_z$, and let $\PP{z}{\e,k}$ denote the $k$th persistent (relative) homology module of $\{(P\subi{z}{\alpha}^\e,Q_z^\e)\}_{\alpha\in\R}$.
Similarly, let $\CPP{z}{\e,k}$ and $\RPP{z}{\e,k}$ denote the corresponding \v Cech and Rips filtrations, respectively.
We will omit the dimension $k$ and write $\DD{z}$ (resp. $\PP{z}{\e}$) if a statement holds for all dimensions.
If $Q_z^\delta$ surrounds $P^\delta$ in $D$ let $\ext{\PP{z}{\e}}$ denote the $k$th persistent homology module of the filtration of extensions $\{(\ext{P\subi{z}{\alpha}^\e},\ext{Q_z^\e})\}$ for any $\e\geq\delta$, where $\ext{P\subi{z}{\alpha}^\e} = P\subi{z}{\alpha}^\e \cup (D\setminus P^\delta)$.

Lemma~\ref{lem:inclusions} follows directly from the definition of truncated sublevel sets.
This is used to extend Lemma~\ref{lem:surround_and_cover} to persistence modules in Lemma~\ref{lem:inclusion_hom} in order to provide a sequence of shifted homomorphisms $\DD{\omega-3c\delta}\xrightarrow{F}\E\PP{\omega-2c\delta}{\e}\xrightarrow{M}\DD{\omega}\xrightarrow{G}\E\PP{\omega+c\delta}{2\e}\xrightarrow{N}\DD{\omega+5c\delta}$ of varying degree.
These homomorphisms are then combined with those given by the Nerve Theorem and the Rips-\v Cech interleaving in Lemma~\ref{lem:partial_interleaving} to obtain partial interleavings required for our proof of Theorem~\ref{thm:interleaving_main_2}.

\begin{lemma}\label{lem:inclusions}
  If $\delta\leq\e$ and $t,\alpha\in\R$ then $P^\delta\cap D\subi{t-c\e}{\alpha-c\e}\subseteq P\subi{t}{\alpha}^\e\subseteq D\subi{t+c\e}{\alpha+c\e}$.
\end{lemma}
\begin{proof}
  Suppose $x\in P^\delta\cap D\subi{t-c\e}{\alpha-c\e}$.
  Because $x\in P^\delta$ there exists some $p\in P$ such that $\dist(x,p) < \delta$.
  Because $f$ is $c$-Lipschitz $f(p)\leq f(x) + c\dist(x,p) < f(x) + c\delta$.
  If $\alpha\leq t$ then $x\in B_{t-c\e}$ implies $f(p) < t-c\e + c\delta \leq t$ so $x\in Q_t^\e$ as $\delta\leq\e$.
  If $\alpha\geq t$ then $x\in B_{\alpha-c\e}$ which implies $f(p) \leq \alpha$ so $x\in Q_\alpha^\e$.
  So $P^\delta\cap D\subi{t-c\e}{\alpha-c\e}\subseteq P\subi{t}{\alpha}^\e$ as $P\subi{t}{\alpha} = Q_t^\e\cup Q_\alpha^\e$.

  Now, suppose $x\in P\subi{t}{\alpha}^\e$.
  If $\alpha\leq t$ then $x\in Q_t^\e\subseteq B_{t+c\e}$ because $f$ is $c$-Lipschitz.
  Similarly, $\alpha > t$ implies $x\in Q_\alpha^\e\subseteq B_{\alpha+c\e}$, so $P\subi{t}{\alpha}^\e\subseteq D\subi{t+c\e}{\alpha+c\e}$ as $D\subi{t+c\e}{\alpha+c\e} = B_{t+c\e}\cup B_{\alpha+c\e}$.
\end{proof}

\begin{lemma}\label{lem:inclusion_hom}
  Let $s + 3c\delta\leq t + 2c\delta\leq u\leq v-c\delta\leq w-5c\delta$ and $\e\in [\delta,2\delta]$.
  If $Q_{t}^\delta$ surrounds $P^\delta$ in $D$ and $D\setminus B_u\subseteq P^\delta$ then the following homomorphisms are induced by inclusions:
  \[(F, G)\in \Hom^{c\delta}(\DD{s}, \E\PP{t}{\e})\times \Hom^{2c\delta}(\DD{u}, \E\PP{v}{2\e}),\ (M, N)\in \Hom^{c\e}(\E\PP{t}{\e},\DD{u})\times\Hom^{2c\e}(\E\PP{v}{2\e}, \DD{w}).\]
\end{lemma}
\begin{proof}
  Because $Q_t^\delta$ surrounds $P^\delta$ in $D$ and $\delta\leq\e$, $t < v$ we know $Q_t^\e$ and $Q_v^\e$ surround $P^\delta$ in $D$.
  As $P^\delta\cap B_s\subseteq Q_t^\e$ and $P^\delta\cap B_u\subseteq Q_v^{2\e}$ for all $\e\in[\delta,2\delta]$ Lemma~\ref{lem:surround_and_cover} implies that we have a sequence of inclusions $B_s\subseteq \E Q_t^\e\subseteq B_u\subseteq \E Q_v^{2\e}\subseteq B_w$.

  For any $\alpha\in\R$ we know that $D\setminus P^\delta \subseteq \ext{P\subi{t}{\alpha}^\e}$ by the definition of $\ext{P\subi{t}{\alpha}^\e}$.
  Moreover, $D\setminus P^\delta\subseteq D\subi{u}{\alpha}$ because $D\setminus B_u\subseteq P^\delta$.
  Lemma~\ref{lem:inclusions} therefore implies $D\subi{s}{\alpha-c\delta}\subseteq \E P\subi{t}{\alpha}^\e\subseteq D\subi{u}{\alpha+c\e}$ as $s + c\delta\leq t \leq u - c\e$.
  So the inclusions $(D\subi{s}{\alpha-c\delta}, B_s)\subseteq (\E P\subi{t}{\alpha}^\e, \E Q_t^\e)$ induce $F\in\Hom^{c\delta}(\DD{s},\E\PP{t}{\e})$ and $(\E P\subi{t}{\alpha}^\e, \E Q_t^\e)\subseteq (D\subi{u}{\alpha+c\e}, B_u)$ induce $M\in \Hom^{c\e}(\E\PP{t}{\e}, \DD{u})$.

  By an identical argument Lemma~\ref{lem:inclusions} implies $D\subi{u}{\alpha-2c\delta}\subseteq \E P\subi{v}{\alpha}^\e\subseteq D\subi{w}{\alpha+2c\e}$ as $u+c\delta\leq v\leq w-4c\delta$.
  So $(D\subi{u}{\alpha-2c\delta},B_u)\subseteq (\E P\subi{v}{\alpha}^\e, \E Q_v^{2\e})$ induce $G\in\Hom^{2c\delta}(\DD{u},\E\PP{v}{2\e})$ and $(\E P\subi{v}{\alpha}^\e, \E Q_v^{2\e})\subseteq (D\subi{w}{\alpha+2c\e},B_w)$ induce $N\in\Hom^{2c\e}(\E\PP{v}{2\e},\DD{u})$.
\end{proof}

\begin{lemma}\label{lem:partial_interleaving}
  For $\delta < \varrho_D$ and $s + 3c\delta\leq t + 2c\delta\leq u\leq v-c\delta\leq w-5c\delta$ let
  $\Gamma\in\Hom(\DD{s},\DD{u})$,
  $\Pi\in\Hom(\DD{u},\DD{w})$, and
  $\Lambda\in\Hom(\RPP{t}{2\delta}, \RPP{v}{4\delta})$ be induced by inclusions.

  If $Q_{t}^\delta$ surrounds $P^\delta$ in $D$ and $D\setminus B_u\subseteq P^\delta$ then there is a partial $2c\delta$ interleaving $\Phi^*\in\Hom^{2c\delta}(\im~\Gamma, \im~\Lambda)$ and a partial $4c\delta$ interleaving $\Psi^*\in\Hom^{4c\delta}(\im~\Lambda, \im~\Pi)$.
\end{lemma}
\begin{proof}
  Because the shifted homomorphisms provided by Lemma~\ref{lem:inclusion_hom} are all induced by inclusions the following diagram commutes for all $\alpha\leq\beta$.
  So we have image module homomorphisms $\Phi(F, G)\in\Hom^{2c\delta}(\im~\Gamma, \im~C\circ A)$ and $\Psi(M, N)\in\Hom^{4c\delta}(\im~E\circ C, \im~\Pi)$.
  \[\begin{tikzcd}
      \hom_k(D\subi{s}{\alpha-2c\delta}, B_s) \arrow{r}{f_{\alpha-2c\delta}}\arrow{d}{\gamma_{\alpha-2c\delta}[\beta-\alpha]} &
      \hom_k(\E P\subi{t}{\alpha}^\delta, \E Q_t^\delta)\arrow{d}{c_\alpha[\beta-\alpha]\circ a_\alpha}\\
      %
      \hom_k(D\subi{u}{\beta-2c\delta}, B_u)\arrow{r}{g_{\beta-2c\delta}} &
      \hom_k(\E P\subi{v}{\beta}^{2\delta}, \E Q_v^{2\delta})
    \end{tikzcd}
    \begin{tikzcd}
      \hom_k(\E P\subi{t}{\alpha}^{2\delta}, \E Q_t^{2\delta})\arrow{d}{e_\beta\circ c_\alpha[\beta-\alpha]}\arrow{r}{m_{\alpha}} &
      \hom_k(D\subi{u}{\alpha+4c\delta}, B_u)\arrow{d}{\gamma_{\alpha+4c\delta}[\beta-\alpha]}\\
      %
      \hom_k(\E P\subi{v}{\beta}^{4\delta}, \E Q_v^{4\delta})\arrow{r}{n_\beta} &
      \hom_k(D\subi{w}{\beta+4c\delta}, B_w)
    \end{tikzcd}\]

  Because the isomorphisms provided by Lemma~\ref{lem:excision} are given by excision they are induced by inclusion, and therefore give isomorphisms $\E_z^\e \in \Hom(\PP{z}{\e},\ext{\PP{z}{\e}})$ for any $z\in\R$ such that $Q_z^\e$ surrounds $P^\delta$ in $D$.
  For any $\e < \varrho_D$ we have isomorphisms $\N_z^\e\in\Hom(\CPP{z}{\e}, \PP{z}{\e})$ that commute with maps induced by inclusions by the Persistent Nerve Lemma.
  So the compositions $\E_z^\e\circ \N_z^\e$ are isomorphisms that commute with maps induced by inclusion as well.
  These compositions, along with the Rips-\v Cech interleaving, provide maps $\E\PP{t}{\delta}\xrightarrow{F'}\RPP{t}{2\delta}\xrightarrow{M'} \E\PP{t}{2\delta}$ and $\E\PP{v}{2\delta}\xrightarrow{G'}\RPP{v}{4\delta}\xrightarrow{N'} \E\PP{v}{4\delta}$ that commute with maps induced by inclusions.
  So we have the following commutative diagram:
  %
  \begin{equation}
    \begin{tikzcd}
      \E\PP{t}{\delta}\arrow{rr}{A}\arrow{dr}{F'} & &
      \E\PP{t}{2\delta}\arrow{r}{C} &
      \E\PP{v}{2\delta}\arrow{rr}{E}\arrow{dr}{G'} & &
      \E\PP{v}{4\delta}\\
      %
      & \RPP{t}{2\delta}\arrow{ur}{M'}\arrow[to=RV, "\Lambda"] & &
      & |[alias=RV]|\RPP{v}{4\delta}\arrow{ur}{N'} &
    \end{tikzcd}
  \end{equation}
  %
  That is, we have image module homomorphisms $\Phi'(F', G')$ and $\Psi'(M', N')$ such that $A = M'\circ F'$, $E = N'\circ G'$, and $\Lambda = G'\circ C\circ M'$.
  Because all maps are induced by inclusions or commute with maps induced by inclusions $\Phi^* = \Phi'\circ \Phi\in\Hom^{2c\delta}(\im~\Gamma, \im~\Lambda)$ and $\Psi^* = \Psi\circ\Psi'\in\Hom^{4c\delta}(\im~\Lambda, \im~\Pi)$ by Lemm~\ref{lem:image_composition}.

  Because $G,M,C$ are induced by inclusions $C[3c\delta] = G\circ M$, so $\Lambda[3c\delta] = G'\circ C[3c\delta]\circ M' = G'\circ (G\circ M)\circ M'$ as $G', M'$ commute with maps induced by inclusions.
  In the same way, $\Gamma[3c\delta] = M\circ (A\circ F) = M\circ (M'\circ F')\circ F$ and $\Pi[5c\delta] = N\circ (E\circ G) = N\circ (N'\circ G')\circ G$.

  Let $F^*:= F'\circ F$, $G^*:= G'\circ G$, $M^*:=M'\circ M$, and $N^*:=N'\circ N$.
  So $\Phi^*_{M^*}$ is a partial $2c\delta$ interleaving as $\Gamma[3c\delta] = M^*\circ F^*$ and $\Lambda[3c\delta] = G^*\circ M^*$, and $\Psi^*_{G^*}$ is a partial $4c\delta$ interleaving as $\Lambda[3c\delta] = G^*\circ M^*$ and $\Pi[5c\delta] = N^*\circ G^*$.
\end{proof}

The partial interleavings given by Lemma~\ref{lem:partial_interleaving}, along with assumptions that imply $\im(\DD{\omega-3c\delta}\to \DD{\omega+5c\delta})\cong \DD{\omega}$, provide the proof of Theorem~\ref{thm:interleaving_main_2} by Lemma~\ref{thm:interleaving_main}.

\begin{theorem}\label{thm:interleaving_main_2}
  Let $\X$ be a $d$-manifold, $D\subset\X$ and $f : D\to\R$ be a $c$-Lipschitz function.
  Let $\omega\in\R$, $\delta < \varrho_D/4$ be constants such that $B_{\omega-3c\delta}$ surrounds $D$ in $\X$.
  Let $P\subset D$ be a finite subset and suppose $\hom_k(B_{\omega-3c\delta}\hookrightarrow B_\omega)$ is surjective and $\hom_k(B_\omega\hookrightarrow B_{\omega+5c\delta})$ is an isomorphism for all $k$.

  If $D\setminus B_\omega\subseteq P^\delta$ and $Q_{\omega-2c\delta}^\delta$ surrounds $P^\delta$ in $D$ then the $k$th persistent homology module of $\{\rips^{2\delta}(P\subi{\omega-2c\delta}{\alpha}, Q_{\omega-2c\delta})\hookrightarrow \rips^{4\delta}(P\subi{\omega+c\delta}{\alpha}, Q_{\omega+c\delta})\}_{\alpha\in\R}$ is $4c\delta$-interleaved with that of $\{(D\subi{\omega}{\alpha}, B_\omega)\}_{\alpha\in\R}$.
\end{theorem}
\begin{proof}
  Let $\Lambda \in\Hom(\RPP{\omega-2c\delta}{2c\delta}, \RPP{\omega+c\delta}{4c\delta})$, $\Gamma\in\Hom(\DD{\omega-3c\delta}, \DD{\omega})$, and $\Pi\in\Hom(\DD{\omega},\DD{\omega+5c\delta})$ be induced by inclusions.
  Because $\delta < \varrho_D/4$, $D\setminus B_\omega\subseteq P^\delta$ and $Q_{\omega-2c\delta}^\delta$ surrounds $P^\delta$ in $D$ we have a partial $2c\delta$ interleaving $\Phi^*\in\Hom^{2c\delta}(\im~\Gamma, \im~\Lambda)$ and a partial $4c\delta$ interleaving $\Psi^*\in\Hom^{4c\delta}(\im~\Lambda, \im~\Pi)$ by Lemma~\ref{lem:partial_interleaving}.
  As we have assumed that $\hom_k(B_{\omega-3c\delta}\hookrightarrow B_\omega)$ is surjective and $\hom_k(B_\omega)\cong\hom_k(B_{\omega+5c\delta})$ the five-lemma implies $\gamma_\alpha$ is surjective and $\pi_\alpha$ is an isomorphism (and therefore injective) for all $\alpha$.
  So $\Gamma$ is an epimorphism and $\Pi$ is a monomorphism, thus $\im~\Lambda$ is $4c\delta$-interleaved with $\DD{\omega}$ by Lemma~\ref{thm:interleaving_main} as desired.
\end{proof}


% Let $\X$ be an oriented $d$-manifold and let $D$ be a compact subset of $\X$.
% Let $f: D\to\R$ be a $c$-lipschitz function and $B_w := f^{-1}((-\infty,a])$ denote a sub-levelset of $f$ at scale $w\in\R$.
% Let $P$ be a finite collection of points in $D$ and $Q_w := P\cap B_w$.
%
% Note that $\{B_\alpha\}_{\alpha\in\R}$ is precisely the sub-levelset filtration of $f$.
% Because we will not assume coverage below some $\omega\in\R$ the persistent homology of $\{Q_\alpha^\delta\}_{\alpha\in\R}$ or even $\{Q_\alpha^\delta\setminus B_\omega\}_{\alpha\in\R}$ cannot be trusted as a reliable approximation.
% In Section~\ref{sec:truncations} we will compare the approximation we provide with approximations of the these filtrations.
% % In fact, because of the nature of homology as a global property we cannot assume either of these filtrations capture anything meaningful even for $\alpha >> \omega$.
%
% We introduce the following notation for \emph{truncated sub-levelsets} to distinguish $B_w$ and $Q_w$ as static sub-levelsets that we will take persistent homology relative to.
%
% \paragraph{Truncated Filtrations}
%
% For $w,\alpha\in\R$ let $D\subi{w}{\alpha} := B_w\cup B_\alpha$ denote the \textbf{$\omega$-truncated $\alpha$-sub-levelsets} of $f$ and let $P\subi{w}{\alpha} := P\cap D\subi{w}{\alpha}$.
% % \[ D\subi{w}{\alpha} := B_w\cup B_\alpha\ \text{ and }\ \ P\subi{w}{\alpha} := P\cap D\subi{w}{\alpha}.\]
% % Now, the pairs $(D\subi{w}{\alpha}, B_w)$ and $(P\subi{w}{\alpha}, Q_w)$ are well defined for all $\alpha\in\R$.
% %
% % Let
% % \[ \DD{w}^k := \left(\left\{\D{w}{\alpha}^k := \hom_k(D\subi{w}{\alpha},B_w)\right\}_{\alpha\in\R},\left\{d\subi{w}{\alpha,\beta}^k : \D{w}{\alpha}^k\to\D{w}{\beta}^k\right\}_{\alpha\leq\beta}\right)\]
% Let $\DD{w}^k$ denote the $k$th persistent (relative) homology module of the filtration $\{(D\subi{w}{\alpha},B_w)\}_{\alpha\in\R}$ with respect to $B_w$, and let $\PP{w}{\e,k}$
% % \[\PP{w}{\e,k} := \left(\left\{\P{w}{\e,k}{\alpha} := \hom_k(P\subi{w}{\alpha}^\e,Q_w^\e)\right\}_{\alpha\in\R}, \left\{p\subi{w}{\alpha,\beta}^{\e,k} : \P{w}{\e}{\alpha}\to\P{w}{\e}{\beta}\right\}_{\alpha\leq\beta}\right)\]
% denote the $k$th persistent (relative) homology module of $\{(P\subi{w}{\alpha}^\e,Q_w^\e)\}_{\alpha\in\R}$.
% Similarly, let $\CPP{w}{\e,k}$ and $\RPP{w}{\e,k}$ denote the corresponding \Cech and Vietoris-Rips filtrations, respectively.
% We will omit the dimension $k$ and write $\DD{w}$ (resp. $\PP{w}{\e}$) if a statement holds for all dimensions.
