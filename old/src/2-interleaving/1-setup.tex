% !TeX root = ../../main.tex

\paragraph{Extensions}

Suppose $Q_w^\e$ surrounds $P^\e$ in $D$.
Let $\ext{\PP{w}{\e}}$ denote the $k$th persistent homology module of the filtration of extensions $\{(\ext{P\subi{w}{\alpha}^\e},\ext{Q_w^\e})\}$, where $\ext{P\subi{w}{a}^\e} = P\subi{w}{a}^\e \cup (D\setminus P^\e)$.
The following lemmas apply Lemmas~\ref{lem:surround_and_cover} and~\ref{lem:excision} to construct homomorphisms between persistence modules that are induced by inclusions.

\begin{lemma}\label{lem:extension_apply}
  If $Q_w^\e$ surrounds $P^\e$ in $D$ then then there is an isomorphism $\E_w^\e \in \Hom(\PP{w}{\e},\ext{\PP{w}{\e}})$.
\end{lemma}\begin{proof}
  (See Appendix~\ref{apx:omit})
\end{proof}
\proofatend
  Because $P\subi{w}{a} := P\cap D\subi{w}{a}$ and $B_w\subseteq D\subi{w}{a}$ we know $Q_w = P\cap B_w \subseteq P\subi{w}{a}$ for all $a\in\R$.
  So
  \[\ext{Q^\e_a} = Q^\e_a\cup (D\setminus P^\e) \subseteq P\subi{w}{a}^\e \cup (D\setminus P^\e) = \ext{P\subi{w}{a}^\e}.\]
  As $(P^\e, Q_w^\e)$ is a surrounding pair in $D$, $P^\e$ is open in $D$ and $\ext{P\subi{w}{a}^\e}\subseteq D$ is such that $\ext{Q^\e_a}\subseteq \ext{P\subi{w}{a}^\e}$ it follows that
  \[\hom_k(P\subi{w}{a}^\e, Q^\e_a) = \hom_k(P^\e\cap \ext{P\subi{w}{a}^\e}, Q^\e_a) \cong\hom_k(\ext{P\subi{w}{a}^\e}, \ext{Q^\e_a})\]
  by Lemma~\ref{lem:excision}.

  Because these isomorphisms commute with inclusions we have an isomorphism $\E\subi{w}{\cdot}^\e \in \Hom(\PP{w}{\e},\ext{\PP{w}{\e}})$ defined to be the family $\{\E\subi{w}{\alpha}^\e : \P{w}{\e}{a}\to \E\P{w}{\e}{a}\}$.
\endproofatend

\begin{lemma}\label{lem:p_interleave}
 If $Q_w^\e$ surrounds $P^\e$ in $D$ and $D\setminus B_{w + \e}\subseteq P^\e$ then we have the following sequence of homomorphisms of degree $c\e$ induced by inclusions
 \[\DD{w-c\e}\xrightarrow{F}\E\PP{w}{\e}\xrightarrow{M}\DD{w+c\e}.\]
\end{lemma}
\begin{proof}
  (See Appendix~\ref{apx:omit})
\end{proof}
\proofatend
  Suppose $x\in (P^\e\cap B\subi{w-c\e}{\alpha-c\e})\setminus B_{w+\e}$.
  Because $B_{w-\e}\subset B_{w+\e}$ we know $x\notin B_{w-\e}$ so $w+c\e < f(x)\leq \alpha-c\e$ and there exists some $p\in P$ such that $\dist(x, p) < \e$.
  Because $f$ is $c$-Lipschitz it follows
  \[ f(p)\leq f(x) + c\dist(x, p) < \alpha - c\e + c\e = \alpha\]
  and
  \[ f(p)\geq f(x) - c\dist(x, p) > w+c\e-c\e = w.\]
  So $x\in P\subi{w}{\alpha}^\e$.

  Now, suppose $x\in P\subi{w}{\alpha}^\e\setminus B_{w+c\e}$.
  So $w+c\e < f(x)$ and there exists some $p\in P\subi{w}{\alpha}$ such that $\dist(x,p) < \e$.
  Because $f$ is $c$-Lipschitz it follows
  \[ f(x) \leq f(p) + c\dist(x,p) < a + c\e.\]
  So $x\in B\subi{w+c\e}{\alpha+c\e}\setminus B_{w+c\e}$.

  Because $D\setminus B_{w+c\e}\subseteq P^\e$ we know that $D\setminus P^\e \subseteq B_{w+c\e}$, so
  \[D\subi{w-c\e}{\alpha-c\e}\setminus B_{w+c\e} \subseteq P\subi{w}{\alpha}^\e\setminus B_{w+c\e}\subseteq D\subi{w+c\e}{\alpha+c\e}\setminus B_{w+c\e}\]
  implies
  \[ D\subi{w-c\e}{\alpha-c\e}\subseteq P\subi{w}{\alpha}^\e\cup (D\setminus P^\e) = \ext{P\subi{w}{\alpha}^\e} \subseteq D\subi{w+c\e}{\alpha+c\e} \]
  as desired.

  Because $f$ is $c$-Lipschitz, $B_{w-c\e}\cap P^\delta\subseteq Q_{w}^\e$ so $B_{w-c\e} \subseteq \E Q_w^\e\subseteq B_{w+c\e}$ by Lemma~\ref{lem:surround_and_cover}.
  It follows that we have homomorphisms $F\in \Hom^{c\e}(\DD{w-c\e}, \E\PP{w}{\e})$ and $M\in\Hom^{c\e}(\E\PP{w}{\e}, \DD{w+c\e})$ induced by inclusions.
\endproofatend

\paragraph{Excisive Nerves}

Let $\e < \varrho_D$ and $w\in\R$.
For any $\alpha\in\R$ the inclusion $\cech^\e(P\subi{w}{\alpha}, Q_w)\hookrightarrow (P\subi{w}{\alpha}^\e, Q_w^\e)$ is a homotopy equivalence by the Nerve Theorem.
So we have isomorphisms $\hom_k(\cech^\e(P\subi{w}{\alpha}, Q_w))\to \hom_k(P\subi{w}{\alpha}^\e, Q_w^\e)$ for all $\alpha$ that commute with inclusions by the Persistent Nerve Lemma.
% Therefore, for any $\alpha\leq\beta$ these isomorphisms commute with the module homomorphisms $\hom_k(\cech^\e(P\subi{w}{\alpha}, Q_w))\to \hom_k(\cech^\e(P\subi{w}{\beta}, Q_w))$ and $\hom_k(P\subi{w}{\alpha}^\e, Q_w^\e)\to \hom_k(P\subi{w}{\beta}^\e, Q_w^\e)$ induced by inclusions.
As the module homomorphisms of $\CPP{w}{\e}$ and $\PP{w}{\e}$ are induced by inclusion we have an isomorphism $\N_w^\e\in\Hom(\CPP{w}{\e}, \PP{w}{\e})$ of persistence modules that commutes with maps induced by inclusions.
As the isomorphisms of $\E_w^\e$ are given by excision they are induced by inclusions, so the composition $\E\N_w^\e := \E_w^\e\circ \N_w^\e$ is an isomorphism that commutes with maps induced by inclusion as well.

% \begin{lemma}\label{cor:excisive_nerve}
%   For any $w\leq z$, $\e\leq \eta < \varrho_D$ let $\Lambda\in\Hom(\E\PP{w}{\e},\E\PP{z}{\eta})$ and $\cech\Lambda\in\Hom(\CPP{w}{\e},\CPP{z}{\eta})$ be induced by inclusions.
%   Then $\E\N_w^\e$ and $\E\N_z^\eta$ are isomorphisms such that $\Lambda = \E\N_{z}^{\eta}\circ \cech\Lambda\circ (\E\N_w^\e)^{-1}$ and $\cech\Lambda = (\E\N_{z}^{\eta})^{-1}\circ \Lambda\circ \E\N_w^\e.$
% \end{lemma}
% \begin{proof}\textbf{TODO}
%   (See Appendix~\ref{apx:omit})
% \end{proof}
% \proofatend
%   \textbf{TODO}
% \endproofatend

% For any $w\in\R$ and $\e\geq 0$ let the $k$th persistent homology module of the Rips filtration $\{\rips^\e(P\subi{w}{\alpha}, Q_w)\}$ be denoted
% \[\RPP{w}{\e} := \left(\left\{\RP{w}{\e}{\alpha} := \hom_k(\rips^\e(P\subi{w}{\alpha}, Q_w))\right\}_{\alpha\in\R}, \left\{\rips p\subi{w}{\alpha,\beta}^\e : \RP{w}{\e}{\alpha}\to\RP{w}{\e}{\beta}\right\}_{\alpha\leq\beta}\right).\]
\paragraph{Rips-Cover Interleaving}

For all $w\in\R$ and $\e < \varrho_D$ let $\I_w^\e\in\Hom(\CPP{w}{\e}, \RPP{w}{2\e})$ and $\J_w^\e\in\Hom(\RPP{w}{\e},\CPP{w}{\e})$ be induced by the inclusions
\[ \cech^\e(P\subi{w}{\alpha}, Q_w)\subseteq \rips^{2\e}(P\subi{w}{\alpha},Q_w)\subseteq \cech^{2\e}(P\subi{w}{\alpha}, Q_w)\]
and define the composite maps
\[\Sigma_w^\e := \I_w^\e\circ (\E\N_w^\e)^{-1}\in \Hom(\PP{w}{\e},\RPP{w}{2\e})\ \text{ and }\ \Upsilon_w^\e := \E\N_w^{\e}\circ \J_w^{\e}\in \Hom(\RPP{w}{\e},\PP{w}{\e}).\]
% For $w \leq z$ and $2\e\leq \eta < \varrho_D$ let
% \[ \Lambda\in\Hom(\ext{\PP{w}{\e}}, \ext{\PP{w}{2\e}}),\ \rips\Lambda\in\Hom(\RPP{w}{2\e},\RPP{z}{2\eta}),\text{ and } \Lambda'\in\Hom(\ext{\PP{w}{\eta}},\ext{\PP{z}{2\eta}})\]
% be induced by inclusion.

% \begin{lemma}\label{lem:weak_rips_left}
%   For $w\in\R$ and $\e < \varrho_D$ let $\mathcal{S}\in\Hom(\ext{\PP{w}{\e}}, \ext{\PP{w}{2\e}})$ be induced by inclusion.
%   Then $(\Sigma_w^\e, \Upsilon_w^{2\e})$ factors $\mathcal{S}$ through $\RPP{w}{2\e}$.
% \end{lemma}
% \begin{proof}
%   Because $\I_w^\e$ and $\J_w^{2\e}$ are induced by inclusions $\mathcal{S} = \E\N_w^{2\e}\circ (\J_w^{2\e})\circ \I_w^\e)\circ \E\N_w^\e)^{-1}$ by Lemma~\ref{lem:pers_nerve}.
%   Therefore, by the definitions of $\Sigma_w^\e$ and $\Upsilon_w^{2\e}$, the pair $(\Sigma_w^\e, \Upsilon_w^{2\e})$ factors $\Lambda$ through $\RPP{w}{2\e}$.
% \end{proof}

For $w\leq z$ and $2\e \leq \eta < \varrho_D / 2$ let
\[ \Lambda\in\Hom(\ext{\PP{w}{\e}}, \ext{\PP{z}{\eta}}),\ \rips\Lambda\in\Hom(\RPP{w}{2\e},\RPP{z}{2\eta}),\ \text{ and }\ \Lambda'\in\Hom(\ext{\PP{w}{2\e}},\ext{\PP{z}{2\eta}})\]
be induced by inclusions.
The following lemma establishes image module homomorphisms that allow us to pass from our Rips modules to our cover modules by factoring through our \Cech module.

\begin{lemma}\label{lem:rips_homomorphism_left}
  $\tilde{\Phi}(\Sigma_w^\e,\Sigma_z^\eta)\in\Hom(\im~\Lambda,\im~\rips\Lambda)$ and $\tilde{\Psi}(\Upsilon_w^{2\e},\Upsilon_z^{2\eta})\in\Hom(\im~\rips\Lambda,\im~\Lambda')$ are image module homomorphisms.
\end{lemma}
\begin{proof}
  (See Appendix~\ref{apx:omit})
\end{proof}
\proofatend
  By Lemma~\ref{lem:pers_nerve} we have $\cech\Lambda\circ (\E\N_w^\e)^{-1} = (\E\N_z^\eta)^{-1}\circ \Lambda$ for $\cech\Lambda\in\Hom(\CPP{w}{\e},\CPP{z}{\eta})$ induced by inclusions.
  As $\rips\Lambda\circ\I_w^\e = \I_z^\eta\circ\cech\Lambda$
  \[ \rips\Lambda\circ \I_w^\e\circ(\E\N_w^\e)^{-1} = \I_z^\eta\circ\cech\Lambda\circ (\E\N_w^\e)^{-1} = \I_z^\eta\circ (\E\N_z^\eta)^{-1}\circ\Lambda.\]
  It follows that $\rips\Lambda\circ\Sigma_w^\e = \Sigma_z^\eta\circ\Lambda$ by the definition of $\Sigma$.
  So Diagram~\ref{dgm:image_homomorphism} commutes and we may therefore conclude that $\tilde{\Phi}(\Sigma_w^\e,\Sigma_z^\eta)$ is an image module homomorphism.

  By Lemma~\ref{lem:pers_nerve} we have $\E\N_z^{2\eta} \circ\cech\Lambda'  = \cech \Lambda\circ \E\N_w^{2\e}$ for $\cech\Lambda'\in\Hom(\CPP{w}{2\e},\CPP{z}{2\eta})$ induced by inclusions.
  As $\J_z^\eta\circ \rips\Lambda = \cech\Lambda'\circ\J_w^\e$
  \[ \E\N_z^{2\eta}\circ \J_z^\eta\circ \rips\Lambda = \E\N_z^{2\eta}\circ\cech\Lambda'\circ\J_w^\e = \cech \Lambda\circ \E\N_w^{2\e}\circ\J_w^\e.\]
  Once again, Diagram~\ref{dgm:image_homomorphism} commutes by the definition of $\Upsilon$, so $\tilde{\Psi}(\Upsilon_w^{2\e},\Upsilon_z^{2\eta})$ is an image module homomorphism.
\endproofatend

% Let $\mathcal{S} \in
% Corollary~\ref{cor:left_right} follows from the fact that the pair $(\Sigma_w^\e, \Upsilon_w^{2\e})$ factors the map $\ext{\PP{w}{\e}}\to\ext{\PP{w}{2\e}}$ induced by inclusions through $\RPP{w}{2\e}$ for any $w\in\R$ and $\e < \varrho_D$.

\begin{corollary}\label{cor:left_right}
  If $w\leq z$ and $2\e \leq \eta < \varrho_D / 2$ then $\tilde{\Phi}(\Sigma_w^\e,\Sigma_z^\eta)$ is a left interleaving of image modules and $\tilde{\Psi}(\Upsilon_w^{2\e},\Upsilon_z^{2\eta})$ is a right interleaving of image modules.
\end{corollary}\begin{proof}
  (See Appendix~\ref{apx:omit})
\end{proof}
\proofatend
  Because $\eta\geq 2\e$ and $w\leq z$ the pair $(\Sigma_w^\e, \Upsilon_z^\eta)$ factors $\Lambda$ through the map $\RPP{w}{2\e}\to \RPP{z}{\eta}$ induced by inclusions.
  It follows that $\tilde{\Phi}$ is a left interleaving of image modules via the composition of this map with $\Upsilon_z^\eta$.
  Similarly, $(\Upsilon_w^{2\e}, \Sigma_z^{\eta})$ factors $\rips\Lambda$ through the map $\E\PP{w}{2\e}\to \E\PP{z}{\eta}$ induced by inclusions.
  It follows that $\tilde{\Psi}$ is a right interleaving of image modules via the composition of this map with $\Sigma_z^{\eta}$.
  As all maps are induced by inclusions
\endproofatend

% Note that Lemma~\ref{lem:weak_rips_left} implies that $\tilde{\Phi}$ is a left interleaving via $\Upsilon_w^{2\e}$, and that $\tilde{\Psi}$ is a right interleaving via $\Sigma_w^\e$.
A commutative diagram of these maps can be found in Appendix~\ref{apx:diagrams} \textbf{TODO}.
