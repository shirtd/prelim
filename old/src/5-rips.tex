% !TeX root = ../new.tex

\subsection{Rips Interleaving}

For $w\in\R$ and $k\in\Z$ let
\[ \DD{w} := \left(\left\{\D{w}{\alpha} := \hom_k(D\subi{w,\alpha},B_w)\right\}_{\alpha\in\R},\left\{d_w[\alpha,\beta] : \D{w}{\alpha}\to\D{w}{\beta}\right\}_{\alpha\leq\beta}\right)\]
denote the $k$th persistent homology module of $\{(D\subi{w,\alpha},B_w)\}$.

For $w\in\R$, $\e > 0$ and $k\in\Z$ let
\[\PP{w}{\e} := \left(\left\{\P{w}{\e}{\alpha} := \hom_k(P\subi{w,\alpha}^\e,Q_w^\e)\right\}_{\alpha\in\R}, \left\{p_w^\e[\alpha,\beta] : \P{w}{\e}{\alpha}\to\P{w}{\e}{\beta}\right\}_{\alpha\leq\beta}\right)\]
denote the $k$th persistent homology module of $\{(P\subi{w,\alpha}^\e,Q_w^\e)\}$.

For any $w\in\R$ and $\e > 0$ suppose $B_{w-3c\e}$ surrounds $D$ in $\X$, $D\setminus B_w \subseteq P^\e$, and $Q_w^\e$ surrounds $P^\e$ in $D$.
We can define a filtration of extensions $\{(\ext{P\subi{w,\alpha}^\e},\ext{Q_w^\e})\}$ and let $\ext{\PP{w}{\e}}$ denote its $k$th persistent homology module.
Let
% \[ \E(w,\e) := \{\E_\alpha(w,\e) : \P_\alpha(w,\e)\to \ext{\P_\alpha(w,\e)}\} \in \Hom(\PP(w,\e),\ext{\PP(w,\e)})\]
$\E_w^\e \in \Hom(\PP{w}{\e},\ext{\PP{w}{\e}})$
denote the isomorphism of persistence modules defined to be the family of isomorphisms
\[ \{\E_w^\e[\alpha] : \P{w}{\e}{\alpha}\to \ext{\P{w}{\e}{\alpha}}\}\]
provided by from Lemma~\ref{lem:extension_apply}.
By Lemma~\ref{lem:p_interleave} we have the following sequence of maps induced by inclusion for all $\alpha$
\[ \D{w-c\e}{\alpha-c\e}\to \P{w}{\e}{\alpha}\to \D{w+c\e}{\alpha+c\e}.\]

% \begin{lemma}\label{lem:weak_interleave}
%   For any $w\in\R$, $\e\geq 0$ let $\Gamma\in\Hom(\DD{w-c\e}, \DD(w+c\e))$ be induced by inclusion.
%
%   The pair $(F(w,\e), M(w,\e))\in \Hom^{c\e}(\DD{w-c\e}, \ext{\PP(w,\e)})\times\Hom^{c\e}(\ext{\PP(w,\e)},\DD(w+c\e))$ of maps induced by inclusion is a weak $c\e$-interleaving of $\Gamma$ with $\ext{\PP(w,\e)}$.
% \end{lemma}

% \begin{lemma}\label{lem:weak_interleave_left}
%   For any $w\in\R$, $\eta\geq\e \geq 0$ let $\Gamma\in\Hom(\DD{w-c\e}, \DD{w+c\eta})$ and $\Lambda\in\Hom(\PP{w}{\e},\PP{w}{\eta})$ be induced by inclusion.
%
%   The pair
%   \[ (F_{w-c\e}, M_w^\eta)\in \Hom^{c\e}(\DD{w-c\e}, \ext{\PP{w}{\e}})\times\Hom^{c\eta}(\ext{\PP{w}{\eta}},\DD{w+c\eta})\]
%   of maps induced by inclusion is a weak $c\eta$-interleaving of $\Gamma$ with $\Lambda$.
% \end{lemma}
%
% \begin{lemma}\label{lem:weak_interleave_right}
%   For any $w\in\R$, $\eta\geq \e \geq 0$ let $\Lambda\in\Hom(\PP{w-c\e}{\e},\PP{w+c\e}{\eta})$ be induced by inclusion.
%
%   The pair
%   \[ (M_{w-c\e}^{\e}, F_w)\in \Hom^{c\e}(\ext{\PP{w-c\e}{\e}}, \DD{w})\times\Hom^{c\eta}(\DD{w}, \ext{\PP{w+c\e}{\eta}})\]
%   of maps induced by inclusion is a weak $c\eta$-interleaving of $\Lambda$ with $\DD{w}$.
% \end{lemma}


Let $\N_w^\e\in\Hom(\CPP{w}{\e}, \PP{w}{\e})$
% \[ \N(w,\e) := \{\N_\alpha(w,\e) : \P_\alpha(w,\e)\to \cech^\e\P_\alpha(w)\} \in \Hom(\PP(w,\e),\cech^\e\PP(w) \]
denote the isomorphism provided by the Nerve Theorem where
\[\CPP{w}{\e} := \left(\left\{\CP{w}{\e}{\alpha} := \hom_k(\cech^\e(P\subi{w,\alpha}),\cech^\e(Q_w))\right\}_{\alpha\in\R}, \left\{\cech p_w^\e[\alpha,\beta] : \CP{w}{\e}{\alpha}\to\CP{w}{\e}{\beta}\right\}_{\alpha\leq\beta}\right),\]
denotes the $k$th persistent homology module of the \Cech filtration.
So $\E\N_w^\e := \E_w^\e\circ \N_w^\e)\in\Hom(\CPP{w}{\e}, \ext{\PP{w}{\e}})$ is an isomorphism for all $w\in\R$ and $\e > 0$.

\begin{lemma}\label{lem:excisive_nerve}
  For all $w\leq z$, $\e\leq\eta$, and $\alpha\leq\beta$ the following diagram commutes
  \begin{equation}\begin{tikzcd}
    \CP{w}{\e}{\alpha}\arrow{r}\arrow{d}{\E\N_w^\e[\alpha]} &
    \CP{z}{\eta}{\beta}\arrow{d}{\E\N_{z}^{e'}[\beta]}\\
    %
    \P{w}{\e}{\alpha}\arrow{r} &
    \P{z}{\eta}{\beta}
  \end{tikzcd}\end{equation}
  where the horizontal arrows are induced by inclusion and vertical arrows are isomorphisms.
\end{lemma}

\begin{corollary}\label{cor:excisive_nerve}
 For any $w\leq z$, $\e\leq\eta$ let $\Lambda\in\Hom(\E\PP{w}{\e},\E\PP{z}{\eta})$ and $\cech\Lambda(\CPP{w}{\e},\CPP{z}{\eta})$ be induced by inclusions.
 Then
 \[\Lambda = \E\N_{z}^{\eta}\circ \cech\Lambda\circ (\E\N_w^\e)^{-1}\text{ and }\ \cech\Lambda = (\E\N_{z}^{\eta})^{-1}\circ \Lambda\circ \E\N_w^\e.\]
\end{corollary}

For any $w\in\R$ and $\e\geq 0$ let the $k$th persistent homology module of the Rips filtration be denoted
\[\RPP{w}{\e} := \left(\left\{\RP{w}{\e}{\alpha} := \hom_k(\rips^\e(P\subi{w,\alpha}),\rips^\e(Q_w))\right\}_{\alpha\in\R}, \left\{\rips p_w^\e[\alpha,\beta] : \RP{w}{\e}{\alpha}\to\RP{w}{\e}{\beta}\right\}_{\alpha\leq\beta}\right).\]
Let $\I_w^\e\in\Hom(\CPP{w}{\e}, \RPP{w}{2\e})$ and $\J_w^\e\in\Hom(\RPP{w}{\e},\CPP{w}{\e})$ be defined by the following maps induced by inclusion
\[ \CP{w}{\e}{\alpha}\xrightarrow{\I_w^\e[\alpha]} \RP{w}{2\e}{\alpha}\xrightarrow{\J_w^{2\e}[\alpha]}\CP{w}{2\e}{\alpha}\]
for $\alpha\in\R$.

% In the following let

\begin{lemma}\label{lem:weak_rips_left}
  Let $\Lambda\in\Hom(\ext{\PP{w}{\e}}, \ext{\PP{w}{2\e}})$ be induced by inclusions.
  Then there exists a weak interleaving
  \[ (\Sigma_w^\e, \Upsilon_w^{2\e})\in \Hom(\ext{\PP{w}{\e}}, \RPP{w}{2\e})\times \Hom(\RPP{w}{2\e},\ext{\PP{w}{2\e}})\]
  of $\Lambda$ with $\RPP{w}{2\e}$.
\end{lemma}
\begin{proof}
  Let $\cech\Lambda\in\Hom(\CPP{w}{\e},\CPP{w}{2\e})$ be induced by inclusion.
  % By Lemma~\ref{lem:excisive_nerve} we know $\E\N(w,2\e)\circ \Lambda = \cech\Lambda\circ \E\N(w,\e)$ and $\cech\Lambda\circ \E\N(w,\e)^{-1} =
  Because $\I_w^\e$ and $\J_w^{2\e}$ are induced by inclusions $\cech\Lambda = \J_w^{2\e}\circ \I_w^\e$.
  Let
  \[ \Sigma_w^\e := \I_w^\e\circ (\E\N_w^\e)^{-1}\text{and}\ \Upsilon_w^{2\e} := \E\N_w^{2\e}\circ \J_w^{2\e}.\]
  By Corollary~\ref{cor:excisive_nerve} we have
  \begin{align*}
    \Lambda &= \E\N_w^{2\e}\circ \cech\Lambda\circ (\E\N_w^\e)^{-1}\\
      &= (\E\N_w^{2\e}\circ \J_w^{2\e})\circ (\I_w^\e\circ (\E\N_w^\e)^{-1})\\
      &= \Upsilon_w^{2\e}\circ \Sigma_w^\e
  \end{align*}
  so $(\Sigma_w^\e, \Upsilon_w^{2\e})$ is a weak interleaving of $\Lambda$ with $\RPP{w}{2\e}$.
\end{proof}

In the following let $(\Sigma_w^\e, \Upsilon_w^{2\e}) := (\E\N_w^\e)^{-1}, \E\N_w^{2\e}\circ \J_w^{2\e})$ denote the pair provided by Lemma~\ref{lem:weak_rips_left} for any $w\in\R$, $\e\geq 0$.

\begin{lemma}\label{lem:rips_homomorphisms}
  For any $w\leq z$ and $\e\leq\eta$ let $\Lambda\in\Hom(\ext{\PP{w}{\e}}, \ext{\PP{z}{\eta}})$, $\Lambda'\in\Hom(\ext{\PP{w}{2\e}},\ext{\PP{z}{2\eta}})$, and $\rips\Lambda\in\Hom(\RPP{w}{\e},\RPP{w}{\eta})$ be induced by inclusions.
  Then
  \[ \tilde{\Phi}(\Sigma_w^\e,\Sigma_z^\eta)\in\Hom(\im~\Lambda,\im~\rips\Lambda)\]
  and
  \[ \tilde{\Psi}(\Upsilon_w^{2\e},\Upsilon_z^{2\eta})\in\Hom(\im~\rips\Lambda,\im~\Lambda')\]
  are image module homomorphisms.
\end{lemma}
\begin{proof}
  % Let $\cech\Lambda\in\Hom(\CPP{w}{\e},\CPP{z}{\eta})$ and $\cech\Lambda'\in\Hom(\CPP{w}{2\e},\CPP{z}{2\eta})$ be induced by inclusion.
  Because $\I_w^\e$, $\I_z^\eta$, and $\rips\Lambda$ are induced by inclusions, and letting $\cech\Lambda\in\Hom(\CPP{w}{\e},\CPP{z}{\eta})$ be induced by inclusion,
  \[ \rips\Lambda\circ\I_w^\e = \I_z^\eta\circ\cech\Lambda.\]
  Moreover, because $\cech\Lambda$ is induced by inclusions
  \[\cech\Lambda\circ(\E\N_w^\e)^{-1} = (\E\N_z^\eta)^{-1}\circ\Lambda\]
  by Lemma~\ref{lem:excisive_nerve}.
  % Therefore, by the definition of $\Sigma_w^\e$
  % \begin{align*}
  %   \rips\Lambda\circ\Sigma_w^\e &= (\rips\Lambda\circ\I_w^\e)\circ (\E\N_w^\e)^{-1}\\
  %     &= \I_z^\eta\circ(\cech\Lambda\circ (\E\N_w^\e)^{-1}\\
  %     &= \I_z^\eta\circ (\E\N_z^\eta)^{-1}\circ\Lambda\\
  %     &= \Sigma_z^\eta\circ\Lambda.
  % \end{align*}
  % So Diagram~\ref{dgm:image_homomorphism} commutes for all $\alpha\leq\beta$, and we may therefore conclude that $\tilde{\Phi}(\Sigma_w^\e,\Sigma_z^\eta)$ is an image module homomorphism.
  We therefore have the following for all $\alpha\leq\beta$ by the definition of $\Sigma_w^\e$.
  \begin{align*}
    \rips\lambda[\alpha;\beta-\alpha]\circ\sigma_w^\e[\alpha] &= (\rips\lambda[\alpha;\beta-\alpha]\circ\I_w^\e[\alpha])\circ (\E\N_w^\e)^{-1}[\alpha]\\
      &= \I_z^\eta[\beta]\circ(\cech\lambda[\alpha;\beta-\alpha]\circ (\E\N_w^\e)^{-1}[\alpha])\\
      &= \I_z^\eta[\beta]\circ (\E\N_z^\eta)^{-1}[\beta]\circ\lambda[\alpha;\beta-\alpha]\\
      &= \sigma_z^\eta[\beta]\circ\lambda[\alpha;\beta-\alpha]
  \end{align*}
  so Diagram~\ref{dgm:image_homomorphism} commutes, and we may therefore conclude that $\tilde{\Phi}(\Sigma_w^\e,\Sigma_z^\eta)$ is an image module homomorphism.

  Because $\Lambda'$ is induced by inclusions and letting $\cech\Lambda'\in\Hom(\CPP{w}{2\e},\CPP{w}{2\eta})$ be induced by inclusions
  \[\Lambda'\circ\E\N_w^{2\e} = \E\N_z^{2\eta}\circ \cech\Lambda'\]
  by Lemma~\ref{lem:excisive_nerve}.
  Because $\rips\Lambda$, $\J_w^\e$ and $\J_z^\eta$ are induced by inclusions
  \[ \cech\Lambda'\circ \J_w^{2\e} = \J_z^{2\eta}\circ\rips\Lambda.\]
  % Therefore, by the definition of $\Upsilon_w^\e$
  % \begin{align*}
  %   \Lambda'\circ \Upsilon_w^{2\e} &= (\Lambda'\circ \E\N_w^{2\e})\circ \J_w^{2\e}\\
  %     &=\E\N_z^{2\eta}\circ(\cech\lambda'\circ\J_w^{2\e})\\
  %     &=(\E\N_z^{2\eta}\circ\J_z^{2\e})\circ\rips\Lambda\\
  %     &=\Upsilon_z^{2\eta}\circ\rips\Lambda
  % \end{align*}
  % So Diagram~\ref{dgm:image_homomorphism} commutes for all $\alpha\leq\beta$, and we may therefore conclude that $\tilde{\Psi}(\Upsilon_w^{2\e},\Upsilon_z^{2\eta})$ is an image module homomorphism.
  We therefore have the following for all $\alpha\leq\beta$ by the definition of $\Upsilon_w^\e$.
  \begin{align*}
    \lambda'[\alpha;\beta-\alpha]\circ \upsilon_w^{2\e}[\alpha] &= (\lambda'[\alpha;\beta-\alpha]\circ \E\N_w^{2\e}[\alpha])\circ \J_w^{2\e}[\alpha]\\
      &=\E\N_z^{2\eta}[\beta]\circ(\cech\lambda'[\alpha;\beta-\alpha]\circ\J_w^{2\e}[\alpha])\\
      &=(\E\N_z^{2\eta}[\beta]\circ\J_z^{2\e}[\beta])\circ\rips\lambda[\alpha;\beta-\alpha]\\
      &=\upsilon_z^{2\eta}[\beta]\circ\rips\lambda[\alpha;\beta-\alpha]
  \end{align*}
  so Diagram~\ref{dgm:image_homomorphism} commutes, and we may therefore conclude that $\tilde{\Psi}(\Upsilon_w^{2\e},\Upsilon_z^{2\eta})$ is an image module homomorphism.
\end{proof}

\begin{theorem}
  Let $D\subset\X$ and $f : D\to\R$ be a $c$-Lipschitz function.
  Let $\omega\in\R$, $\zeta\geq 2\delta\geq 0$ be constants such that
  \begin{enumerate}[label=\Roman*.]
    \item $B_{\omega-c(\delta+\zeta)}$ surrounds $D$ in $\X$,
    \item $\hom_k(B_{\omega-c(\delta+\zeta)}\hookrightarrow B_\omega)$ is surjective, and
    \item $\hom_k(B_\omega)\cong\hom_k(B_{\omega+c(\delta+2\zeta)})$
  \end{enumerate}
  for all $k$.
  Let $P\subset D$ be a finite subset and $Q_w := P\cap B_w$.
  % Suppose $\hom_k(B_{\omega-c(\delta+\zeta)}\hookrightarrow B_\omega)$ is surjective and $\hom_k(B_\omega)\cong\hom_k(B_{\omega+c(\delta+2\zeta)})$ for all $k$.

  If $D\setminus B_\omega\subseteq P^\delta$ and $Q_{\omega-c\zeta}^\delta$ surrounds $P^\delta$ in $D$ then the $k$th persistent homology modules of $\{(D\subi{\omega,\alpha}, B_\omega)\}$ and
  \[
    \{(\rips^{2\delta}(P\subi{\omega-c\zeta,\alpha}), \rips^{2\delta}(Q_{\omega-c\zeta})) \hookrightarrow
      (\rips^{2\zeta}(P\subi{\omega+c\delta,\alpha}), \rips^{2\zeta}(Q_{\omega+c\delta}))\}
  \]
  are $2c\zeta$-interleaved.
\end{theorem}
\begin{proof}
  % Suppose $\b$ surrounds $D$ in $\X$, $D\setminus\B\subseteq P^\of$, and $\Q^\of$ surrounds $P^\of$ in $D$ such that
  % \[ \b\cap P^\of \subseteq \Q^\of\subseteq \B\cap P\DD{w-c\e}\subseteq \QQ^\of\subseteq \BB.\]
  By Lemma~\ref{lem:surround_and_cover} and Lemma~\ref{lem:p_interleave}
  \[ (D\subi{\omega-c(\delta+\zeta),\alpha-c\delta},\b)\subseteq (\ext{P^\e\subi{\omega-c\zeta,\alpha}},\ext{\Q^\e})\subseteq (D\subi{\omega,\alpha+c\e},\B)\]
  and
  \[ (D\subi{\omega,\alpha-c\delta},\B)\subseteq (\ext{P^\e\subi{\omega+c\delta,\alpha}},\ext{\QQ^\e})\subseteq (D\subi{\omega+c(\delta+\zeta),\alpha+c\e},\BB)\]
  for all $\alpha\in\R$ and $\delta\leq\e\leq\zeta$.

  % Let $\Gamma\in\Hom(\DD{w-c{\d},\D{w})$ and $\Lambda\in\Hom(\PP{w})
  Let $\Gamma\in\Hom(\DD{\omega-c(\delta+\zeta)},\DD{\omega})$ and $\Lambda\in\Hom(\ext{\PP{\omega-c\zeta}{\delta}},\ext{\PP{\omega+c\delta}{\zeta}})$.
  As all maps are induced by inclusion we have $\Phi(F, G)\in\Hom^{c\zeta}(\im~\Gamma, \im~\Lambda).$
  Similarly, for $\Pi\in\Hom(\DD{w},\DD{\omega+c(\delta+2\zeta)})$ and $\Lambda\in\Hom(\ext{\PP{\omega-c\zeta}{2\delta}},\ext{\PP{\omega+c\delta}{2\zeta}})$ we have $\Psi(M, N)\in\Hom^{2c\zeta}(\im~\Lambda',\im~\Pi).$

  By Lemma~\ref{lem:rips_homomorphisms} we have
  \[ \tilde{\Phi}(\Sigma_{\omega-c\zeta}^\delta,\Sigma_{\omega+c\delta}^\zeta)\in\Hom(\im~\Lambda,\im~\rips\Lambda)\]
  and
  \[ \tilde{\Psi}(\Upsilon_{\omega-c\zeta}^{2\delta},\Upsilon_{\omega+c\delta}^{2\zeta})\in\Hom(\im~\rips\Lambda,\im~\Lambda')\]
  therefore, letting
  \[ (\rips F, \rips G) := (\Sigma_{\omega-c\zeta}^\delta\circ F, \Sigma_{\omega+c\delta}^\zeta\circ G)\]
  and
  \[ (\rips F, \rips G) := (M\circ \Upsilon_{\omega-c\zeta}^{2\delta}, N\circ\Upsilon_{\omega+c\delta}^{2\zeta})\]
  it follows from Lemma~\ref{lem:image_composition} that $\rips \Phi (\rips F, \rips G)\in\Hom^{c\zeta}(\im~\Gamma,\im~\rips\Lambda)$ is an image module homomorphism of degree $c\zeta$ and $\rips\Psi(\rips M, \rips N)\in\Hom^{2c\zeta}(\im~\rips\Lambda,\im~\Pi)$ is an image module homomorphism of degree $2c\zeta$.


  Let $\mathcal{S}\in\Hom(\ext{\PP{\omega-c\zeta}{\delta}},\ext{\PP{\omega-c\zeta}{2\delta}})$, $\Theta\in\Hom(\ext{\PP{\omega-c\zeta}{2\delta}},\ext{\PP{\omega+c\delta}{\zeta}})$, and $\mathcal{T}\in\Hom(\ext{\PP{\omega+c\delta}{\zeta}},\ext{\PP{\omega+c\delta}{2\zeta}})$ be induced by inclusions so that $\Lambda = \Theta\circ \mathcal{S}$ and $\Lambda' = \Theta\circ \mathcal{T}$.

  Because $\Gamma\in\Hom(\DD{\omega-c(\delta+\zeta)},\DD{\omega})$, $F\in\Hom^{c\delta}(\DD{\omega-c(\delta+\zeta)}, \ext{\PP{\omega-c\zeta}{\delta}})$, and $M\in\Hom^{2c\delta}(\ext{\PP{\omega-c\zeta}{2\delta}}, \DD{\omega})$ are induced by inclusions $\Gamma[3c\delta] = M\circ\mathcal{S}\circ F$.
  So $(F, M)$ is a weak $2c\delta$-interleaving of $\Gamma[3c\delta]$ with $\mathcal{S}$.
  By Lemma~\ref{lem:weak_rips_left} $(\Sigma_{\omega-c\zeta}^{\delta},\Upsilon_{\omega-c\zeta}^{2\delta})$ is a weak interleaving of $\mathcal{S}$ with $\RPP{\omega-c\zeta}{2\delta}$.
  So $(\Sigma_{\omega-c\zeta}^{\delta}\circ F, M\circ \Upsilon_{\omega-c\zeta}^{2\delta}) = (\rips F, \rips M)$ is a weak $2c\delta$-interleaving of $\Gamma[3c\delta]$ with $\RPP{\omega-c\zeta}{2\delta}$ by Lemma~\ref{lem:left}.

  Similarly, because $\Pi\in\Hom(\DD{\omega},\DD{\omega+c(\delta+2\zeta)})$, $G\in\Hom^{c\zeta}(\DD{\omega}, \ext{\PP{\omega+c\delta}{\zeta}})$, and $N\in\Hom^{2c\zeta}(\ext{\PP{\omega+c\delta}{2\zeta}})$ are induced by inclusion $\Pi[3c\zeta] = N\circ \mathcal{T}\circ G$, so $(G, N)$ is a weak $2c\zeta$-interleaving of $\Pi$ with $\mathcal{T}$.
  Once again, by Lemma~\ref{lem:weak_rips_left}, $(\Sigma_{\omega+c\delta}^{\zeta},\Upsilon_{\omega+c\delta}^{2\zeta})$ is a weak $2c\zeta$-interleaving of $\mathcal{T}$ with $\RPP{\omega+c\delta}{2\zeta}$.
  So $(\Sigma_{\omega+c\delta}^{\zeta}\circ G, N\circ \Upsilon_{\omega+c\delta}^{2\zeta}) = (\rips G, \rips N)$ is a weak $2c\zeta$-interleaving of $\Pi[3c\zeta]$ with $\RPP{\omega+c\delta}{2\zeta}$ by Lemma~\ref{lem:left}.

  By Corollary~\ref{cor:excisive_nerve} we know that $\cech\Theta := (\E\N_{\omega+c\delta}^\zeta)^{-1}\circ \Theta\circ \E\N_{\omega-c\zeta}^{2\delta}$
  where, because all maps are induced by inclusions,
  \begin{align*}
    \rips\Lambda &= \I_{\omega+c\delta}^\zeta\circ\cech\Theta\circ\J_{\omega-c\zeta}^{2\delta}\\
      &= (\I_{\omega+c\delta}^\zeta\circ (\E\N_{\omega+c\delta}^\zeta)^{-1})\circ \Theta\circ (\E\N_{\omega-c\zeta}^{2\delta}\circ \J_{\omega-c\zeta}^{2\delta})\\
      &= \Sigma_{\omega+c\delta}^\zeta\circ \Theta\circ \Upsilon_{\omega-c\zeta}^{2\delta}\\
  \end{align*}
  Because $\Theta[2c\delta+c\zeta] = G\circ M$ the pair $(M, G)$ is a weak $c\zeta$-interleaving of $\Theta[2c\delta+c\zeta]$ with $\DD{\omega}$.
  Therefore,
  \begin{align*}
    \rips\Lambda[2c\delta+c\zeta] &= \Sigma_{\omega+c\delta}^\zeta\circ \Theta[2c\delta+c\zeta]\circ \Upsilon_{\omega-c\zeta}^{2\delta}\\
      &= (\Sigma_{\omega+c\delta}^\zeta\circ G)\circ (M\circ \Upsilon_{\omega-c\zeta}^{2\delta})\\
      &= \rips G\circ \rips M.
  \end{align*}
  So $(\rips M, \rips G)$ is a weak $c\zeta$-interleaving of $\rips\Lambda$ with $\DD{\omega}$.

  Because $(\rips F,\rips M)$ is a weak $2c\delta$-interleaving $\rips \Phi$ is a left $2c\delta$-interleaving of image modules.
  Because $(\rips G, \rips N)$ is a weak $2c\zeta$-interleaving $\rips \Psi$ is a right $2c\zeta$-interleaving of image modules.
  Finally, because $(\rips M, \rips G)$ is a weak $c\zeta$-interleaving $\rips \Phi$ is a right $c\zeta$-interleaving of image modules and $\rips \Psi$ is a left $c\zeta$-interleaving of image modules.
  So $\rips \Phi_{\rips M}(\rips F, \rips G)$ is a partial $c\zeta$-interleaving of image modules and $\rips \Psi_{\rips G} (\rips M,\rips N)$ is a partial $2c\zeta$-interleaving of image modules.

  As we have assumed that $\hom_k(B_{\omega-c(\delta+\zeta)}\hookrightarrow B_\omega)$ is surjective and $\hom_k(B_\omega)\cong\hom_k(B_{\omega+c(\delta+2\zeta)})$ Lemma~\ref{lem:pt_interleaving} implies $\gamma_\alpha$ is surjective and $\pi_\alpha$ is an isomorphism (and therefore injective) for all $\alpha$.
  So $\Gamma$ is an epimorphism and $\Pi$ is a monomorphism, thus $\im~\rips\Lambda$ is $2c\zeta$-interleaved with $\DD{\omega}$ by Lemma~\ref{thm:interleaving_main} as desired.
\end{proof}

% \begin{lemma}\label{lem:rips_image_interleaving}
%   For any $w\leq z$ and $\e\leq\eta$ let $\Lambda\in\Hom(\PP{w}{\e}, \PP{z}{\eta})$, $\Lambda'\in\Hom(\PP{w}{2\e},\PP{z}{2\eta})$, and $\rips\Lambda\in\Hom(\RPP{w}{\e},\RPP{w}{\eta})$ be induced by inclusions.
%   Then
%   \[ \tilde{\Phi}_{\Upsilon_w^{2\e}}(\Sigma_w^\e,\Sigma_z^\eta)\in\Hom(\im~\Lambda,\im~\rips\Lambda)\]
%   and
%   \[ \tilde{\Psi}_{\Sigma_z^\eta}(\Upsilon_w^{2\e},\Upsilon_z^{2\eta})\in\Hom(\im~\rips\Lambda,\im~\Lambda)\]
%   are partial interleavings of image modules.
% \end{lemma}

% \begin{lemma}\label{cor:weak_rips_left}
%   For any $w\in\R$, $\e \geq 0$ let $\Gamma\in\Hom(\DD{w-c\e}, \DD{w+2c\e})$ be induced by inclusions.
%
%   Then there exists a weak $2c\e$-interleaving
%   \[ (\rips F_{w-c\e}^{2\e}, \rips M{w}^{2\e})\in\Hom^{c\e}(\DD{w-c\e},\RPP{w}{2\e})\times \Hom^{2c\e}(\RPP{w}{2\e},\DD{w+2c\e}) \]
%   of $\Gamma$ with $\RPP{w}{2\e}$.
%   % $(\Sigma(w,\e)\circ F(w-c\e,\e), M(w,\eta)\circ\Upsilon(w,2\e))$ \[ \Sigma(w,\e)\circ F(w-c\e,\e)
%   % \[(F(w-c\e,\e), M(w,\eta))\in \Hom^{c\e}(\DD{w-c\e}, \ext{\PP(w,\e)})\times\Hom^{c\e}(\ext{\PP(w,\eta)},\DD(w+c\eta))\]
%   % be the weak $c\eta$-interleaving of $\Gamma$ with $\Lambda$ provided by Lemma~\ref{lem:weak_interleave_left}.
%   % Let
%   % \[ (\Sigma(w,\e), \Upsilon(w,\eta))\in \Hom(\PP(w,\e), \rips^{2\e}\PP(w))\times \Hom(\rips^{2\e}\PP(w),\PP(w,\eta))\]
%   % be the weak interleaving of $\Lambda$ with $\rips^\eta\PP(w)$ provided by Lemma~\ref{lem:weak_rips_left}.
%   % Then the pair
%   % \[ (\Sigma(w,\e)\circ F(w-c\e,\e), M(w,\eta)\circ\Upsilon(w,2\e))\in \Hom^{c\e}(\DD{w-c\e}, \rips^{2\e}\PP(w)})\times\Hom^{c\eta}(\rips^{2\e}\PP(w)},\DD(w+c\eta))\]
%   % is a weak $c\eta$-interleaving of $\Gamma$ with $\rips^{2\e}\PP(w)$.
%
% \end{lemma}
% \begin{proof}
%   Let $\Lambda\in\Hom(\PP{w}{\e},\PP{w}{2\e})$ be induced by inclusions.
%   By Lemma~\ref{lem:weak_rips_left} we have a weak interleaving
%   \[ (\Sigma_w^\e, \Upsilon_w^{2\e})\in \Hom(\PP{w}{\e}, \RPP{w}{2\e})\times \Hom(\RPP{w}{2\e},\PP{w}{2\e})\]
%   of $\Lambda$ with $\RPP{w}{2\e}$.
%   By Lemma~\ref{lem:weak_interleave_left} the pair
%   \[ (F_{w-c\e}, M_w^{2\e})\in\Hom^{c\e}(\DD{w-c\e},\E\PP{w}{\e})\times\Hom^{2c\e}(\E\PP{w}{2c\e},\DD{w+2c\e})\]
%   of maps induced by inclusion is a weak $2c\e$-interleaving of $\Gamma$ with $\Lambda$.
%   It follows that the pair $(\rips F_{w-c\e}, \rips M_w^{2\e}) := (\Sigma_w^\e\circ F_{w-c\e}, M_w^{2\e}\circ \Upsilon_w^{2\e})$ is a weak $2c\e$-interleaving of $\Gamma$ with $\RPP{w}{2\e}$ by Lemma~\ref{lem:left}.
% \end{proof}
%
% % \begin{lemma}\label{lem:weak_rips_right}
% %   For $\e\leq\eta$ and $w\leq z$ let
% %   \[ \Lambda\in\Hom(\ext{\PP(w,\e)}, \ext{\PP(z,\eta)})\text{ and } \rips\Lambda\in\Hom(\rips^\e\PP(w),\rips^{2\eta}\PP(z))\]
% %   be induced by inclusions.
% %   Then
% %   \[ (\Upsilon(w,\e), \Sigma(z,\eta))\in \Hom(\rips^{\e}\PP(w), \PP(w,\e))\times \Hom(\PP(z,\eta), \rips^{2\eta}\PP(z))\]
% %   is a weak interleaving of $\rips\Lambda$ with $\Lambda$.
% % \end{lemma}
%
%
% \begin{theorem}
%   Let $D\subset\X$ and $f : D\to\R$ be a $c$-Lipschitz function.
%   Let $\omega\in\R$, $\zeta\geq 2\delta\geq 0$ be constants such that
%   \begin{enumerate}[label=\Roman*.]
%     \item $B_{\omega-c(\delta+\zeta)}$ surrounds $D$ in $\X$,
%     \item $\hom_k(B_{\omega-c(\delta+\zeta)}\hookrightarrow B_\omega)$ is surjective, and
%     \item $\hom_k(B_\omega)\cong\hom_k(B_{\omega+c(\delta+2\zeta)})$
%   \end{enumerate}
%   for all $k$.
%   Let $P\subset D$ be a finite subset and $Q_w := P\cap B_w$.
%   % Suppose $\hom_k(B_{\omega-c(\delta+\zeta)}\hookrightarrow B_\omega)$ is surjective and $\hom_k(B_\omega)\cong\hom_k(B_{\omega+c(\delta+2\zeta)})$ for all $k$.
%
%   If $D\setminus B_\omega\subseteq P^\delta$ and $Q_{\omega-c\zeta}^\delta$ surrounds $P^\delta$ in $D$ then the $k$th persistent homology modules of $\{(D\subi{\omega,\alpha}, B_\omega)\}$ and
%   \[
%     \{(\rips^{2\delta}(P\subi{\omega-c\zeta,\alpha}), \rips^{2\delta}(Q_{\omega-c\zeta})) \hookrightarrow
%       (\rips^{2\zeta}(P\subi{\omega+c\delta,\alpha}), \rips^{2\zeta}(Q_{\omega+c\delta}))\}
%   \]
%   are $2c\zeta$-interleaved.
% \end{theorem}
% \begin{proof}
%   Let $\Gamma\in\Hom(\DD{\omega-c(\delta+\zeta)}, \DD{\omega})$ and $\Pi\in\Hom(\DD{\omega}, \DD{\omega+c(\delta+2\zeta)})$ be induced by inclusions.
%   Let $\mathcal{S}\in\Hom(\ext{\PP{\omega-c\zeta}{\delta}}, \ext{\PP{\omega-c\zeta}{2\delta}})$ and $\mathcal{T}\in\Hom(\ext{\PP{\omega+c\delta}{\zeta}}, \ext{\PP{\omega+c\delta}{2\zeta}})$ be induced by inclusions.
%
%   By Lemma~\ref{cor:weak_rips_left} the pair $(\rips F_{\omega-c(\delta+\zeta)},\rips M_{\omega-c\zeta}^{2\delta})$ is a weak $2c\delta$-interleaving of $\Gamma$ with $\RPP{\omega-c\zeta}{2\delta}$.
%   Similarly,  and the pair $(\rips F_{\omega},\rips M_{\omega+c\delta}^{2\zeta})$ is a weak $2c\zeta$-interleaving of $\Pi$ with $\RPP{\omega+c\delta}{2\zeta}$.
%
%   Because
%   \[ (\ext{P\subi{\omega-c\zeta, \alpha-2c{\delta}}^{2\delta}},\ext{Q_{\omega-c\zeta}^{2\delta}})\subseteq (D\subi{\omega, \alpha}, B_\omega)\subseteq (\ext{P\subi{\omega+c\delta, \alpha+c\zeta}^\zeta},\ext{Q_{\omega+c\delta}^\zeta}) \]
%   for all $\alpha\in\R$ the pair
%   \[ (M_{\omega-c\zeta}^{2\delta}, F_{\omega+c\delta})\in\Hom^{2c\delta}(\ext{\PP{\omega-c\zeta}{2\delta}}, \DD{\omega})\times \Hom^{c\zeta}(\DD{\omega}, \ext{\PP{\omega+c\delta}{\zeta}})\]
%   is a weak $c\zeta$-interleaving of $\Lambda\in\Hom(\ext{\PP{\omega-c\zeta}{2\delta}}, \ext{\PP{\omega+c\delta}{\zeta}})$ with $\DD{\omega}$.
%
%   % Let $\rips\Lambda \in\Hom(\rips^{2\delta}\PP(\omega-c\zeta), \rips^{2\zeta}\PP(\omega+c\delta))$ be induced by inclusion and
%   % \[\cech\Lambda =
%   % \[\cech\Lambda = \I(\omega+c\delta,\zeta)\circ \rips\Lambda \J(\omega-c\zeta,2c\delta)
%   By Corollary~\ref{cor:excisive_nerve}
%   \[\cech\Lambda = (\E\N_{\omega+c\delta}^\zeta)^{-1}\circ\Lambda\circ \E\N_{\omega-c\zeta}^{2c\delta}\]
%   so we define
%   \[\rips\Lambda := \I_{\omega+c\delta}^{\zeta}\circ \cech\Lambda\circ \J_{\omega-c\zeta}^{2\delta}\in\Hom(\RPP{\omega-c\zeta}{2\delta}, \RPP{\omega+c\delta}{2\zeta}).\]
%   Now,
%   \begin{align*}
%     \rips\Lambda &= \I_{\omega+c\delta}^\zeta\circ \cech\Lambda\circ \J_{\omega-c\zeta}^{2\delta}\\
%       &= (\I_{\omega+c\delta}^\zeta\circ (\E\N_{\omega+c\delta}^\zeta)^{-1})\circ\Lambda \circ (\E\N_{\omega-c\zeta}^{2c\delta}\circ\J_{\omega-c\zeta}^{2\delta})\\
%       &= \rips F_{\omega+c\delta}\circ \Lambda\circ \rips M_{\omega-c\zeta}^{2\delta}.
%   \end{align*}
%   So we may conclude that $(\rips M_{\omega-c\zeta}^{2\delta}, \rips F_{\omega+c\delta})$ is a weak $c\zeta$-interleaving of $\rips\Lambda$ with $\DD{\omega}$ by Lemma~\ref{lem:right}.
%
%   \textbf{TODO} For all $\alpha\leq\beta$
%   \begin{align*}
%     f_\omega\circ \gamma_\alpha[\beta-\alpha] &= m_{\omega-2c\delta}^{2\delta}[\alpha]\circ s_{\alpha}\circ f_{\omega-c(\delta+\zeta)}
%   \end{align*}
%   % \[\rips\Phi(\rips F_{\omega-c(\delta+\zeta)}^{2\delta}, \rips F_{\omega})\in\Hom^{c\zeta}(\im~\Gamma,\im~\rips\Lambda)\]
%
%
%   Because all maps commute with inclusion
%   \[ \rips\Phi_{\rips M_{\omega-c\zeta}^{2\delta}}(\rips F_{\omega-c(\delta+\zeta)}^{2\delta}, \rips F_{\omega}^{2\zeta})\in\Hom^{c\zeta}(\im~\Gamma, \im~\rips\Lambda) \]
%   is a partial $c\zeta$-interleaving of image modules, and
%   \[ \rips\Psi_{\rips F_{\omega}^{2\zeta}}(\rips M_{\omega-c\zeta}^{2\delta}, \rips M_{\omega+c\delta}^{2\zeta})\in\Hom^{2c\zeta}(\im~\rips\Lambda, \im~\Pi) \]
%   is a partial $2c\zeta$-interleaving of image modules.
%
%   By Lemma~\ref{lem:pt_interleaving} $\Gamma$ is an epimorphism and $\Pi$ is an isomorphism (and therefore a monomorphism).
%   The result follows from Theorem~\ref{thm:interleaving_main}.
%
%
%   % For
%   % % \[F(\omega-c(\delta+\zeta),\delta)\in \Hom^{c\delta}(\DD(\omega-c(\delta+\zeta)), \ext{\PP(\omega-c\zeta,\delta)})\]
%   % \[F(\delta)\in \Hom^{c\delta}(\DD(\omega-c(\delta+\zeta)), \ext{\PP(\omega-c\zeta,\delta)})\]
%   % and
%   % % \[M(\omega-c\zeta, 2\delta)\in \Hom^{2c\delta}(\ext{\PP(\omega-c\zeta, 2\delta)}, \DD(\omega))\]
%   % \[M(2\delta)\in \Hom^{2c\delta}(\ext{\PP(\omega-c\zeta, 2\delta)}, \DD(\omega))\]
%   % % induced by inclusion the pair $(F(\omega-c(\delta+\zeta),\delta)), M(\omega-c\zeta, 2\delta))$ is a weak $2c\delta$-interleaving of $\Gamma$ with $\mathcal{S}$ by Lemma~\ref{lem:weak_interleave_left}.
%   % induced by inclusions the pair $(F(\delta), M(2\delta)$ is a weak $2c\delta$-interleaving of $\Gamma$ with $\mathcal{S}$ by Lemma~\ref{lem:weak_interleave_left}.
%   % % By Lemma~\ref{lem:weak_rips_left} the pair
%   % By Lemma~\ref{lem:weak_rips_right} the $(\Sigma(\omega-c\zeta),\Upsilon(\omega-c\zeta))$ is a weak interleaving of $\mathcal{S}$ with $\rips^{2\delta}\PP(\omega-c\zeta)$, where
%   % \[ \Sigma(\omega-c\zeta) \in\Hom(\ext{\PP(\omega-c\zeta,\delta)},\rips^{2\delta}\PP(\omega-c\zeta)),\]
%   % and
%   % \[ \Upsilon(\omega-c\zeta)\in\Hom(\rips^{2\delta}\PP(\omega-c\zeta), \PP(\omega-c\zeta,2\delta)).\]
%   % It follows from Lemma~\ref{lem:left} that the pair $(\Sigma(\omega-c\zeta)\circ F(\delta), M(2\delta)\circ \Upsilon(\omega-c\zeta))$ is a weak $2c\delta$-interleaving of $\Gamma$ with $\rips^{2\delta}\PP(\omega-c\zeta)$.
%   %
%   % Using an identical process we obtain a weak $2c\zeta$-interleaving $(\Sigma(\omega+c\delta)\circ F(\zeta), M(2\zeta)\circ \Upsilon(\omega+c\delta))$ of $\Pi$ with $\rips^{2\zeta}\PP(\omega+c\delta)$.
%
% \end{proof}
%
%
% % We also define the $k$th persistent homology modules of \Cech and Rips filtrations as follows:
% % \[\cech^\e\PP(w) := \left(\left\{\cech^\e\P_\alpha(w) := \hom_k(\cech^\e(P\subi{w,\alpha}),\cech^\e(Q_w))\right\}_{\alpha\in\R}, \left\{\check{c}_\alpha^\beta(w,\e) : \cech^\e\P_\alpha(w)\to\cech^\e\P_\beta(w)\right\}_{\alpha\leq\beta}\right),\]
% % \[\rips^\e\PP(w) := \left(\left\{\rips^\e\P_\alpha(w) := \hom_k(\rips^\e(P\subi{w,\alpha}),\rips^\e(Q_w))\right\}_{\alpha\in\R}, \left\{r_\alpha^\beta(w,\e) : \rips^\e\P_\alpha(w)\to\rips^\e\P_\beta(w)\right\}_{\alpha\leq\beta}\right).\]
