% !TeX root = ../../main.tex

\subsection{Some Helpful Lemmas}

% \begin{lemma}[Splitting Lemma (Hatcher p. 147)]\label{lem:splitting}
%   For a short exact sequence \[0\to A\xrightarrow{i} B\xrightarrow{j} C\to 0\] of abelian groups the following statements are equivalent
%   \begin{enumerate}
%     \item There is a homomorphism $p: B\to A$ such that $p\circ i = \mathbf{Id}_A$.
%     \item There is a homomorphism $s: C\to B$ such that $j\circ s = \mathbf{Id}_C$.
%     \item There is an isomorphism $B\cong A\oplus C$ making the commutative diagram below, where the maps in the lower row are the obvious ones $a\mapsto (a, 0)$ and $(a,c)\mapsto c$.
%
%     \[\begin{tikzcd}[column sep=small, row sep=small]
%               &                       & B\ar[dd, "\cong"]\ar[dr,"j"]  &         & \\
%       0\ar[r] & A\ar[ur, "i"]\ar[dr]  &                               & C\ar[r] & 0\\
%               &                       & A\oplus C\ar[ur]              &         &
%     \end{tikzcd}\]
%   \end{enumerate}
% \end{lemma}

\begin{lemma}[\textbf{The Five-Lemma} (Hatcher p. 129)]\label{lem:five}
  In a commutative diagram of abelian groups as below, if the two rows are exact and $\alpha,\beta,\delta$, and $\e$ are isomorphisms then $\gamma$ is an isomorphism.
  \[\begin{tikzcd}
      A\ar[r, "i"]\ar[d, "\alpha"]
    & B\ar[r, "j"]\ar[d, "\beta"]
    & C\ar[r, "k"]\ar[d, "\gamma"]
    & D\ar[r, "\ell"]\ar[d, "\delta"]
    & E\ar[d, "\e"]\\
    %
      A'\ar[r, "i'"]
    & B'\ar[r, "j'"]
    & C'\ar[r, "k'"]
    & D'\ar[r, "\ell'"]
    & E'\\
  \end{tikzcd}\]

  \begin{itemize}
    \item If $\beta$ and $\delta$ are surjective and $\e$ is injective then $\gamma$ is surjective.
    \item If $\beta$ and $\delta$ are injective and $\alpha$ is surjective then $\gamma$ is injective.
  \end{itemize}
\end{lemma}

\begin{theorem}[\textbf{Universal Coefficient Theorem} (Munkres p. 323, Corollary 53.2)]\label{thm:univ_coef}
  Let $(A,B)$ be a topological pair.
  Then for all $k$ and any abelian group $G$ there is a natural exact sequence
  \[ 0\to\mathrm{Ext}(\hom_{k-1}(A, B), G)\to \hom^k(A, B; G)\xrightarrow{\nu} \Hom(\hom_k(A, B), G)\to 0.\]
  This sequence splits, but not naturally.
\end{theorem}

\subsection{A Helpful Diagram}

% \begin{landscape}
% \begin{scriptsize}
% \begin{centering}
% \[\begin{tikzcd}[row sep=large, column sep=scriptsize]
%   |[alias=U]| \D{\omega-c(\delta+\zeta)}{\alpha-3c\delta}
%                                   \arrow[to=V, "\gamma_{\alpha-3c\delta}{[3c\delta]}"]
%                                   \arrow[to=Sa, "f_{\omega-c(\delta+\zeta}{[\alpha-3c\delta]}"]
%   &&&& |[alias=V]| \D{\omega}{\alpha}
%                                   \arrow[to=W, "\pi_\alpha{[3c\zeta]}"]
%                                   \arrow[to=Ta, "f_\omega{[\alpha]}"]
%   &&&& |[alias=W]|
%   \D{\omega+c(\delta+2\zeta)}{\alpha+3c\zeta}\\
%   %
%   & |[alias=Sa]|
%   \P{\omega-c\zeta}{\delta}{\alpha-2c\delta}
%                                   \arrow[to=Sb, "s_{\alpha-2c\delta}"]
%                                   \arrow[to=CSa, "(\E\N_{w-c\zeta}^\delta{[\alpha-2c\delta]})^{-1}"]
%   && |[alias=Sb]| \P{\omega-c\zeta}{2\delta}{\alpha-2c\delta}
%                                   \arrow[to=Ta, "\vartheta_{\alpha-2c\delta}{[2c\delta+\zeta]}"]
%                                   \arrow[to=V, "m_{\omega-c\zeta}^{2\delta}{[\alpha-2c\delta]}"]
%   && |[alias=Ta]| \P{\omega+c\delta}{\zeta}{\alpha+c\zeta}
%                                   \arrow[to=Tb, "t_{\alpha+c\zeta}"]
%                                   \arrow[to=CTa, "(\E\N_{w+c\delta}^\zeta{[\alpha+c\zeta]})^{-1}"]
%   && |[alias=Tb]| \P{\omega+c\delta}{2\zeta}{\alpha+c\zeta}
%                                   \arrow[to=W, "m_{\omega+c\delta}^{2\zeta}{[\alpha+c\zeta]}"]
%   &\\
%   %
%   & |[alias=CSa]|
%   \CP{\omega-c\zeta}{\delta}{\alpha-2c\delta}
%                                   \arrow[to=CSb, "\check{s}_{\alpha-2c\delta}"]
%                                   \arrow[to=RS, "\I_{\omega-c\zeta}^\delta{[\alpha-2c\delta]}"']
%   && |[alias=CSb]| \CP{\omega-c\zeta}{2\delta}{\alpha-2c\delta}
%                                   \arrow[to=Sb, "\E\N_{w-c\zeta}^{2\delta}{[\alpha-2c\delta]}"]
%                                   \arrow[to=CTa, "\check{\vartheta}_{\alpha-2c\delta}{[2c\delta+\zeta]}"]
%   && |[alias=CTa]| \CP{\omega+c\delta}{\zeta}{\alpha+c\zeta}
%                                   \arrow[to=CTb, "\check{t}_{\alpha+c\zeta}"]
%                                   \arrow[to=RT, "\I_{\omega+c\delta}^\zeta{[\alpha+c\zeta]}"']
%   && |[alias=CTb]| \CP{\omega+c\delta}{2\zeta}{\alpha+c\zeta}
%                                   \arrow[to=Tb, "\E\N_{w+c\delta}^{2\zeta}{[\alpha+2c\zeta]}"]
%   &\\
%   %
%   && |[alias=RS]| \CP{\omega-c\zeta}{2\delta}{\alpha-2c\delta}
%                                   \arrow[to=CSb, "\J_{\omega-c\zeta}^{2\delta}{[\alpha-2c\delta]}"]
%                                   \arrow[to=RT, "\rips\lambda_{\alpha-c\zeta}{[c(2\delta+\zeta)]}"]
%   &&&& |[alias=RT]| \RP{\omega+c\delta}{2\zeta}{\alpha+c\zeta}
%                                   \arrow[to=CTb, "\J_{\omega+c\delta}^{2\zeta}{[\alpha+c\zeta]}"]
%   &&\\
% \end{tikzcd}\]
% \end{centering}
% \end{scriptsize}
% \end{landscape}

\begin{scriptsize}
\begin{centering}
\[\begin{tikzcd}[row sep=large, column sep=scriptsize]
  |[alias=U]| \DD{\omega-c(\delta+\zeta)}
                                  \arrow[to=V, two heads, blue, "\Gamma{[3c\delta]}"]
                                  \arrow[to=Sa,blue, "F"]
  &&&& |[alias=V]| \DD{\omega}
                                  \arrow[to=W, hook,red, "\Pi{[3c\zeta]}"]
                                  \arrow[to=Ta,red, "G"]
  &&&& |[alias=W]|
  \DD{\omega+c(\delta+2\zeta)}\\
  %
  & |[alias=Sa]|
  \E\PP{\omega-c\zeta}{\delta}
                                  \arrow[to=Sb, blue, "\mathcal{S}"]
                                  \arrow[to=CSa, "(\E\N_{w-c\zeta}^\delta)^{-1}"]
  && |[alias=Sb]| \E\PP{\omega-c\zeta}{2\delta}
                                  \arrow[to=Ta, "\Theta{[2c\delta+\zeta]}"]
                                  \arrow[to=V, blue, "M"]
  && |[alias=Ta]| \E\PP{\omega+c\delta}{\zeta}
                                  \arrow[to=Tb, red, "\mathcal{T}"]
                                  \arrow[to=CTa, "(\E\N_{w+c\delta}^\zeta)^{-1}"]
  && |[alias=Tb]| \E\PP{\omega+c\delta}{2\zeta}
                                  \arrow[to=W, red, "N"]
  &\\
  %
  & |[alias=CSa]|
  \CPP{\omega-c\zeta}{\delta}
                                  \arrow[to=CSb, green, "\cech\mathcal{S}"]
                                  \arrow[to=RS, green, "\I_{\omega-c\zeta}^\delta"']
  && |[alias=CSb]| \CPP{\omega-c\zeta}{2\delta}
                                  \arrow[to=Sb, "\E\N_{w-c\zeta}^{2\delta}"]
                                  \arrow[to=CTa, "\cech\Theta{[2c\delta+c\zeta]}"]
  && |[alias=CTa]| \CPP{\omega+c\delta}{\zeta}
                                  \arrow[to=CTb, orange, "\cech\mathcal{T}"]
                                  \arrow[to=RT, orange, "\I_{\omega+c\delta}^\zeta"']
  && |[alias=CTb]| \CPP{\omega+c\delta}{2\zeta}
                                  \arrow[to=Tb, "\E\N_{w+c\delta}^{2\zeta}"]
  &\\
  %
  && |[alias=RS]| \RPP{\omega-c\zeta}{2\delta}
                                  \arrow[to=CSb, green, "\J_{\omega-c\zeta}^{2\delta}"]
                                  \arrow[to=RT, "\rips\Lambda"]
  &&&& |[alias=RT]| \RPP{\omega+c\delta}{2\zeta}
                                  \arrow[to=CTb, orange, "\J_{\omega+c\delta}^{2\zeta}"]
  &&\\
\end{tikzcd}\]
\end{centering}
\end{scriptsize}

\begin{enumerate}
  \item Blue commutes (all inclusions) + green commutes (all inclusions) + middle left square commutes with inclusion (pers. nerve lemma) $\implies$ weak interleaving of $\Gamma$ with $\RPP{\omega-c\zeta}{2\delta}$.
  \item Red commutes (all inclusions) + orange commutes (all inclusions) + middle right square commutes with inclusion (pers. nerve lemma) $\implies$ weak interleaving of $\Pi$ with $\RPP{\omega+c\delta}{2\zeta}$.
  \item Middle commutes using the same arguments $\implies$ weak interleaving of $\rips\Lambda$ with $\DD{\omega}$
  \item weak interleaving of $\Gamma$ with $\RPP{\omega-c\zeta}{2\delta}$ + weak interleaving of $\rips\Lambda$ with $\DD{\omega}$ + path \textbf{down} from $\Gamma$ to $\rips\Lambda$ is an image module homomorphism $\implies$ partial interleaving of image modules from $\Gamma$ to $\rips\Lambda$.
  \item weak interleaving of $\rips\Lambda$ with $\DD{\omega}$ + weak interleaving of $\Pi$ with $\RPP{\omega+c\delta}{2\zeta}$ + path \textbf{up} from $\rips\Lambda$ to $\Pi$ is an image module homomorphism $\implies$ partial interleaving of image modules from $\rips\Lambda$ to $\Pi$.
  \item these two partial interleavings + $\Gamma$ epi + $\Pi$ mono $\implies$ $\im~\rips\Lambda$ is interleaved with $\DD{\omega}$ by Lemma~\ref{thm:interleaving_main}.
\end{enumerate}
