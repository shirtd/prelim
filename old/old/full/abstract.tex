% !TeX root = main.tex

\begin{abstract}
    We are studying methods to measure, analyze, visualize, summarize, and compare global behaviors of coordinate-free sensor networks over time.
    Major challenges include sensor errors, gaps in coverage, and a changing network.
    We provide a stable signature for time-varying functions on coordinate-free sensor networks that is robust to changes in the network and can be easily computed from neighborhood information.
    This is done using persistent homology, a tool from algebraic topology that is particularly well suited to providing a summary of global behavior from local information alone.
\end{abstract}

% \begin{abstract}
%     We are study methods to measure, analyze, visualize, summarize, and compare global behaviors of sensor networks over time.
%     Major challenges include sensor errors, gaps in coverage, and a changing network.
%     These issues make it difficult to distinguish network anomalies from meaningful changes in the data.
%     However, even reliable sensor data must be properly integrated in order to reflect global behaviors over time.
%     Often these global behaviors represent an event observed by the network through sensor measurements.
%     Constructing a summary of such an event that is both descriptive and discriminative is increasingly difficult when the network is allowed to change over time.
%     Persistent homology is particularly well suited to this setting as it provides a reliable summary of global behavior from local information alone.
%     This summary is robust to missing data and stable under reasonable changes to the network.
%     Moreover, persistent homology has been shown to be useful for coverage verification in \emph{coordinate-free sensor networks} in which sensor locations are not known.
%     The \emph{Topological Coverage Criterion (TCC)} of de Silva and Ghrist~\cite{desilva07coverage} uses persistent homology to verify coverage of an unknown domain by a coordinate-free sensor network, and was extended to weighted $k$-coverage in a more general setting in~\cite{cavanna2017when}.
%     In this work we extend the analysis of scalar fields to functions over time in the setting of coordinate-free networks with boundary.
%     We show that the resulting signatures are stable and can be approximated by a pair readily available simplicial complexes.
%     % these complexes are the same as those used to verify converage: end-to-end
%     We also show some preliminary results on an alternative method for function approximation that may be used for level-set analysis.
% \end{abstract}
