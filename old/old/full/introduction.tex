% !TeX root = main.tex

\section{Introduction}
\label{sec:introduction}

% We are studying methods to measure, analyze, visualize, summarize, and compare global behaviors of sensor networks over time.
% Major challenges include sensor errors, gaps in coverage, and a changing network.
% These issues make it difficult to distinguish network anomalies from meaningful changes in the data.
% However, even reliable sensor data must be properly integrated in order to reflect global behaviors over time.
% Often these global behaviors represent an event observed by the network through sensor measurements.
% Constructing a summary of such an event that is both descriptive and discriminative is increasingly difficult when the network is allowed to change over time.
% Persistent homology is particularly well suited to this setting as it provides a reliable summary of global behavior from local information alone.
% This summary is robust to missing data and stable under reasonable changes to the network.
% Moreover, persistent homology has been shown to be useful for coverage verification in \emph{coordinate-free sensor networks} in which sensor locations are not known.
% The \emph{Topological Coverage Criterion (TCC)} of de Silva and Ghrist~\cite{desilva07coverage} uses persistent homology to verify coverage of an unknown domain by a coordinate-free sensor network, and was extended to weighted $k$-coverage in a more general setting in~\cite{cavanna2017when}.
% In this work we extend the analysis of scalar fields to functions over time in the setting of coordinate-free networks with boundary.
% We show that the resulting signatures are stable and can be approximated by a pair readily available simplicial complexes.
% % these complexes are the same as those used to verify converage: end-to-end
% We also show some preliminary results on an alternative method for function approximation that may be used for level-set analysis.

Our goal is to analyze, summarize, and compare data from sensor networks in which the sensors have two main abilities: they can detect nearby sensors and they can measure some quantity about their environment.
Major challenges include sensor errors, gaps in coverage, and a changing network.
These issues make it difficult to distinguish network anomalies from meaningful changes in the data.
However, even reliable sensor data must be properly integrated in order to reflect global behaviors over time.
Often these global behaviors represent an event observed by the network through sensor measurements which give us a sample of some unknown function on an unknown domain.
The measurements sample the function values and the neighborhood information gives hints about the domain.
With this kind of data, we'd like to build a model of ``normal'' behavior of the network that is robust to both changes in the function as well as changes in the network.
% \begin{enumerate}
%     \item changes in the function,
%     \item changes in the network.
% \end{enumerate}

Persistent homology is well-suited to this setting as it provides a reliable summary of global behavior from local information alone.
This summary is robust to missing data and stable under reasonable changes to the network.
Moreover, persistent homology has been shown to be useful for coverage verification in \emph{coordinate-free sensor networks} in which sensor locations are not known.
The \emph{Topological Coverage Criterion (TCC)} of de Silva and Ghrist~\cite{desilva07coverage} uses persistent homology to verify coverage of an unknown domain by a coordinate-free sensor network, and was extended to weighted $k$-coverage in a more general setting in~\cite{cavanna2017when}.

Chazal et. al. introduced a method for approximating the persistent homology of a function, or scalar field, from a finite point sample~\cite{chazal09analysis} .
This method is well-suited to the setting of coordinate-free sensor networks as the coordinates of the sensors are not required.
The authors make some necessary assumptions about the geometry of the domain, as in the TCC, with their primary assumption being that the network covers the domain.
We therefore consider a setting in the intersection of coverage and the analysis of scalar fields by coordinate-free sensor networks in which a collection of sensors capable of verifying coverage of a domain is augmented with the ability to measure a scalar value on that domain.
Moreover, by focusing on the geometry of the level-sets of a function we provide a novel way to re-cast the geometric assumptions made in both the TCC and the analysis of scalar fields as topological properties of the function itself.
Our goal is to reconcile the theoretical foundations of these two problems as well as leverage the shared machinery required for their computation.

% Moreover, we want to consider behaviors that evolve in time.
% That is, rather than a single function $f:X\to \R$, we will have a family $(f_t)_{t\in [0,1]}$ of such functions.
% The signatures we will define will be global in that they will be aggregated (i.e. integrated) over the domain.
% We base them on persistent homology, a tool that has been previously used to give theoretical guarantees of coverage in homological sensor networks.
%
% \begin{itemize}
%     \item The network may change over time,
%     \item The use of relative homology allows us to signatures of the same function over different domains,
%     \item Robust to noise in the function and the network.
% \end{itemize}
%
% By embedding the persistence diagrams in the plane, we can trace out the change in the topology (i.e., the persistence diagram) over time.
% We call the resulting signatures Persistence Trajectories.
% An example of one such trajectory is given in Figure~\ref{fig:curves}.
%
% \figblock{%
% \begin{figure}[htbp]
%   \centering
%   \includegraphics[width=\textwidth,trim={30 170 30 170},clip]{figures/curve_callout}
%   \caption{A collection of overlaid trajectories from different networks.
%           For three time steps of one of these trajectories the function values on the corresponding networks and resulting persistence diagrams are shown.}
%   \label{fig:curves}
% \end{figure}
% }
%
% \paragraph*{Contributions}
% In this paper, we lay the foundations for a systematic study of persistence trajectories as data summaries.
% % We give the definitions for trajectories and prove stability results for the most common scenarios.
% In this work we extend the analysis of scalar fields to functions over time in the setting of coordinate-free networks with boundary.
% We show that the resulting signatures are stable and can be approximated by a pair readily available simplicial complexes.
% % these complexes are the same as those used to verify converage: end-to-end
% We also show some preliminary results on an alternative method for function approximation that may be used for level-set analysis.
% % We also show some results from experiments.
