% !TeX root = ../new.tex

\section{Rips Interleaving}

% Now, suppose $B_{\omega -3c\of}$ surrounds $D$ in $\X$ and $Q_{\omega-3c\of}^\of$ surrounds $\P$ in $D$ such that
%
% \begin{itemize}
%   \item $D\setminus \B\subseteq \P$,
%   \item $B_{\omega -3c\of}\cap \P \subseteq Q_{\omega-2c\of}\subseteq \B$, and
%   \item $\B\cap \P\subseteq \QQ^\of\subseteq B_{\omega+3c\of}$.
% \end{itemize}

% Let $\zeta=2\delta$ and suppose $\hom_k(B_{\omega-3c\delta}\to B_\omega)$ is surjective and $\hom_k(B_\omega)\cong\hom_k(B_{\omega+3c\delta})$ for all $k$.
% Let $\U, \V, \W$ be the $k$th persistent homology modules of $\{(D\subi{\omega-3c\of, \alpha}, B_{\omega-3c\of})\}$, $\{(D\subi{\omega,\alpha}, \B)\}$, $\{(D\subi{\omega+3c\of,\alpha}, B_{\omega+3c\of})\}$.
% For $\delta\leq\e\leq 2\delta$ let $\S^\e, \T^\e$ denote the $k$th persistent homology modules of $\{(P^\e\subi{\omega-2c\of,\alpha},Q_{\omega-2c\of}^\e)\}$ and $\{(P^\e\subi{\omega+c\of,\alpha},\QQ^\e)\}$ respectively.
% For all $\delta\leq\e\leq 2\delta$ and $\Lambda^\e\in\Hom(\S^\e, \T^\e)$ induced by inclusion $\im~\Lambda^\e$ and $\V$ are $c\e$ interleaved by Lemma~\ref{lem:pt_interleaving}.
%
%
% Let $\cech^\e\S$ be defined as the family $\{\cech^\e S_\alpha := \hom_k(\cech^\e(P\subi{\omega-2c\delta, \alpha}), \cech^\e(Q_{\omega-2c\delta}))\}$ and linear maps $\{(\check{s}^\e)_\alpha^\beta : \cech^\e S_\alpha\to \cech^\e S_\beta\}$ induced by inclusion.
% Define $\rips^\e\S$, $\cech^\e\T$, $\rips^\e\T$ in the same way.

\begin{lemma}\label{lem:left_interleavings}
  For some $w\in\R$, $\e > 0$ and $k\in\Z$ let $\U,\V$ denote the $k$th persistent homology modules of $\{(D\subi{w-c\e,\alpha}, B_{w-c\e})\}$ and $\{(D\subi{w+2c\e,\alpha},B_{w+c\e})\}$.
  Let $\rips^{2\e}\S$ denote the $k$th persistent homology module of $\{(\rips^{2\delta}(P\subi{w,\alpha}), \rips^{2\delta}(Q_w))\}$.

  If $D\setminus B_{w-c\e}\subseteq P^\e$ and $Q^\e_{w}$ surrounds $P^\e$ in $D$ then $\Gamma\in\Hom(\U,\V)$ induced by inclusion is partially $2\e$-interleaved with $\rips^{2\e}\S$.
\end{lemma}
\begin{proof}
  For any $\epsilon > 0$ let $\S^\epsilon$ and $\cech^\epsilon\S$ denote the $k$th persistent homology modules of $\{(\ext{P\subi{w, \alpha}^\epsilon}, \ext{Q^\epsilon_{w}})\}$ and $\{(\cech^\epsilon(P\subi{w, \alpha}),\cech^\epsilon(Q_{w}))\}$, respectively.
  First note that, because
  \[(D\subi{w-c\e,\alpha-c\e}, B_{w-c\e})\subseteq (\ext{P\subi{w, \alpha}^\e}, \ext{Q^\e_{w}})\subseteq (\ext{P\subi{w, \alpha}^{2\e}}, \ext{Q^{2\e}_{w}})\subseteq (D\subi{w+2c\e, \alpha+2c\e}, B_{w+2c\e})\]
  for all $\alpha$ we know that $\Gamma$ is $\e$-interleaved with $\S^\e$ by a pair $(F, M)$ of maps induced by inclusion and $\S^{2\e}$ by a pair $(F',M')$ of maps induced by inclusion.
  The following diagram commutes by Lemma~\ref{lem:rel_pers_nerve} where vertical maps are isomorphisms provided by the Nerve Theorem and Lemma~\ref{lem:excision} and horizontal maps are induced by inclusion for all $\alpha$.
  \begin{equation}\begin{tikzcd}
    S^\e_\alpha\arrow{r}{i_\alpha} &
    S^{2\e}_\alpha\\
    %
    \cech^\e S_\alpha\arrow{r}{\check{i}_\alpha}\arrow{u}{\eta_\alpha^\e}  &
    \cech^{2\e} S_\alpha\arrow{u}{\eta_\alpha^{2\e}}
  \end{tikzcd}\end{equation}
  Note that, because all maps are induced by inclusion, $m_\alpha = m_\alpha'\circ i_\alpha$ and $f_{\alpha-c\e}' = i_\alpha\circ f_{\alpha-c\e}$ for all $\alpha$.
  Moreover, we have the following sequence of maps induced by inclusions for all $\alpha$
  \[ \cech^\e S_\alpha\xrightarrow{k_\alpha} \rips^{2\e} S_\alpha\xrightarrow{\ell_\alpha} \cech^{2\e} S_\alpha, \]
  so $\check{i}_\alpha = \ell_\alpha\circ k_\alpha$.

  Let $\Sigma\in\Hom(\S^\e, \rips^{2\e}\S)$ be defined as the family of linear maps
  \[\{\sigma_\alpha := k_\alpha\circ (\eta_\alpha^\e)^{-1} : S_\alpha^\e\to \rips^{2\e} S_\alpha\}\]
  and $\Upsilon\in\Hom(\rips^{2\e}\S, \S^{2\e})$ be defined
  \[\{\upsilon_\alpha := \eta_\alpha^{2\delta}\circ \ell_\alpha : \rips^{2\e} S_\alpha \to S^{2\e}_\alpha\}.\]
  It follows
  \begin{align*}
    i_\alpha &= \eta_\alpha^{2\e}\circ \check{i}_\alpha\circ (\eta_\alpha^\e)^{-1}\\
      &= (\eta_\alpha^{2\e}\circ \ell_\alpha)\circ (k_\alpha\circ (\eta_\alpha^\e)^{-1})\\
      &= \upsilon_\alpha\circ\sigma_\alpha.
  \end{align*}
  So $M = M'\circ\Upsilon\circ \Sigma$ and $F' = \Upsilon\circ\Sigma\circ F$, therefore $\Gamma$ is $2c\e$-interleaved with $\rips^{2\e}\S$ by Lemma~\ref{lem:left}.
\end{proof}

% \begin{lemma}\label{lem:right_interleaving}
%   For some $w\in\R$ and $k\in\Z$ let $\V$ denote the $k$th persistent homology module of $\{(D\subi{w,\alpha}, B_{w})\}$ and let $\S^\epsilon$ and $\T^\epsilon$ denote the $k$th persistent homology modules of $\{(\rips^{\epsilon}(P\subi{w-c\epsilon,\alpha}), \rips^{\epsilon}(Q_{w-c\epsilon}))\}$ and $\{(\rips^{\epsilon}(P\subi{w+c\epsilon,\alpha}), \rips^{\epsilon}(Q_{w+c\epsilon}))\}$ for any $\epsilon > 0$, respectively.
%
%   If $D\setminus B_{w}\subseteq P^\e\setminus Q_{w-c\e}^{\e}$ and $Q^\e_{w}$ surrounds $P^\delta$ in $D$

\begin{theorem}
  Let $f: D\to \R$ be a $c$-Lipschitz function and $\omega\in\R$, $0 < 2\delta\leq\zeta$ be constants such that $\hom_k(B_{\omega-c(\delta+\zeta)}\hookrightarrow B_\omega)$ is surjective and $\hom_k(B_\omega)\cong \hom_k(B_{\omega+c(\delta+2\zeta)})$ for all $k$.

  If $D\setminus B_\omega\subseteq P^\delta$ and $Q_{\omega-c\zeta}^\delta$ surrounds $P^\delta$ in $D$ then the $k$th persistent homology modules of $\{(D\subi{\omega,\alpha},B_\omega)\}$ and
  \[ \{(\rips^{2\delta}(P\subi{\omega-c\zeta,\alpha}),\rips^{2\delta}(Q_{\omega-c\zeta}))\hookrightarrow (\rips^{2\zeta}(P\subi{\omega+c\delta,\alpha}),\rips^{2\zeta}(Q_{\omega+c\delta}))\}\]
  are $2c\zeta$-interleaved for all $k$.
\end{theorem}
\begin{proof}
  Let $\U,\V,$ and $\W$ denote the $k$th persistent homology modules of $\{(D\subi{\omega-c{\delta+\zeta},\alpha}, B_{\omega-c{\delta+\zeta}})\}$, $\{(D\subi{\omega,\alpha}, B_\omega)\}$, and $\{(D\subi{\omega+c{\delta+2\zeta},\alpha}, B_{\omega+c{\delta+2\zeta}})\}$ respectively.
  Let $\Gamma\in\Hom(\U,\V)$ and $\Pi\in\Hom(\V,\W)$.

  As in Lemma~\ref{lem:left_interleavings} let $\S^\epsilon$, $\cech^\epsilon\S$, and $\rips^\epsilon\S$ denote the $k$th persistent homology modules of $\{(\ext{P\subi{\omega-c\zeta, \alpha}^\epsilon}, \ext{Q^\epsilon_{\omega-c\zeta}})\}$, $\{(\cech^\epsilon(P\subi{\omega-c\zeta, \alpha}),\cech^\epsilon(Q_{\omega-c\zeta}))\}$, and $\{(\rips^\epsilon(P\subi{\omega-c\zeta, \alpha}),\rips^\epsilon(Q_{\omega-c\zeta}))\}$ for any $\epsilon > 0$, respectively.
  $\T^\epsilon$, $\cech^\epsilon\S$, and $\rips^\epsilon\T$ are defined similarly for pairs $(P\subi{\omega+c\delta, \alpha}^\epsilon, Q^\epsilon_{\omega+c\delta})$.

  Let $\Lambda\in\Hom(\S^\delta, \T^\zeta)$ be induced by inclusion so that $\Phi(F^\delta, G^\zeta)\in\Hom^{\zeta}(\im~\Gamma, \im~\Lambda)$ and $\Psi(M, N)\in\Hom^{\zeta}(\im~\Lambda, \im~\Pi)$ are image module homomorphisms of degree $\zeta$.
  Similarly, let $\Lambda'\in\Hom(\S^{2\delta}, \T^{2\zeta})$ be induced by inclusion so that $\Phi'(F^{2\delta}, G^{2\zeta})\in\Hom^{2\zeta}(\im~\Gamma, \im~\Lambda')$ and $\Psi'(M^{2\delta}, N^{2\zeta})\in\Hom^{2\zeta}(\im~\Lambda', \im~\Pi)$ are image module homomorphisms of degree $2\zeta$.

  % By Lemmas~\ref{lem:surround_and_cover} and~\ref{lem:p_interleave} we have the following diagram of maps of pairs induced by inclusion.
  % \begin{equation}\begin{tikzcd}[column sep=scriptsize]
  %   |[alias=U]| U_{\alpha-3c\delta}
  %                                   \arrow[to=V, "{\gamma[3c\delta]}_{\alpha-3c\delta}"]
  %                                   \arrow[to=Sa, "f^\delta_{\alpha-3c\delta}"]
  %   &&& |[alias=V]| V_{\alpha}
  %                                   \arrow[to=W, "{\pi[3c\zeta]}_\alpha"]
  %                                   \arrow[to=Ta, "g^\zeta_\alpha"]
  %   &&& |[alias=W]|
  %   W_{\alpha+3c\zeta}\\
  %   %
  %   & |[alias=Sa]|
  %   S^{\delta}_{\alpha-2c\delta}
  %                                   \arrow[to=Sb, "i_{\alpha-2c\delta}"]
  %   & |[alias=Sb]| S^{2\delta}_{\alpha-2c\delta}
  %                                   \arrow[to=Ta, "{\vartheta[2c\delta+\zeta]}_{\alpha-2c\delta}"]
  %                                   \arrow[to=V, "m_{\alpha-2c\delta}^{2\delta}"]
  %   && |[alias=Ta]| T^\zeta_{\alpha+c\zeta}
  %                                   \arrow[to=Tb, "j_{\alpha+c\zeta}"]
  %   & |[alias=Tb]| T^{2\zeta}_{\alpha+c\zeta}
  %                                   \arrow[to=W, "n^{2\zeta}_{\alpha+c\zeta}"]
  %   &
  % \end{tikzcd}\end{equation}

  By Lemma~\ref{lem:left_interleavings} $\Gamma$ is partially $2\delta$-interleaved with $\rips^{2\delta}\S$ by the pair $(\Sigma\circ F^\delta, M^{2\delta}\circ\Upsilon)$ and $\Pi$ is partially $2\zeta$-interleaved with $\rips^{2\zeta}\T$ by a pair $(\Sigma'\circ G^\zeta, N^{2\zeta}\circ\Upsilon')$.
  So we have $\Upsilon\in\Hom(\rips^{2\delta}\S,\V)$ and $\Sigma'\in\Hom(\T^\zeta, \rips^{2\zeta}\T)$ that commute with inclusion.

  Because  $\tilde{\Phi}(\Sigma\circ F^{\delta}, \Sigma'\circ G^\zeta)\in\Hom^{\zeta}(\im~\Gamma, \im~\rips\Lambda)$ as 

  For $\Theta\in\Hom(\S^{2\delta},\T^\zeta)$ induced by inclusion it follows from Lemma~\ref{lem:right} that $\rips\Lambda :=\Sigma'\circ\Theta\circ\Upsilon\in\Hom(\rips^{2\delta}\S, \rips^{2\zeta}\T)$ is partially $c\zeta$-interleaved with $\V$ by the pair $(M^{2\delta}\circ \Upsilon, \Sigma'\circ G^{\zeta})$.

  % Let $\tilde{F} := \Sigma\circ F^\delta$, $\tilde{G} := \Sigma'\circ G^{\zeta}$.
  % Because
  %
  % Let $\Phi_
  % Let $\Lambda\in\Hom(\S^\delta, \T^\zeta)$ and $\Lambda'\in\Hom(\S^{2\delta}, \T^{2\zeta})$ be induced by inclusions.
  % By Lemma~\ref{lem:pt_interleaving} we have that $\im~\Lambda$ is $c\zeta$-interleaved with $\V$
\end{proof}


% The proof of the next lemma is identical to that of~\ref{lem:gamma_interleaving}, we include it for completeness.
%
% \begin{lemma}\label{lem:gamma_interleaving}
%   $\Pi\in\Hom(\V,\W)$ induced by inclusion is partially $2\zeta$-interleaved with $\rips^{2\zeta}\T$.
% \end{lemma}
% \begin{proof}
%   First note that, because
%   \[(D\subi{\omega,\alpha-3c\zeta}, B_{\omega})\subseteq (P\subi{\omega+c\delta, \alpha-2c\zeta}^\zeta, Q^\zeta_{\omega+c\delta})\subseteq (P\subi{\omega+c\delta, \alpha-2c\zeta}^{2\zeta}, Q^{2\zeta}_{\omega+c\delta})\subseteq (D\subi{\omega+c(\delta+\zeta), \alpha}, B_{\omega+c(\delta+\zeta)})\]
%   for all $\alpha$ we know that $\Gamma$ is $c\zeta$-interleaved with $\S^\zeta$ by a pair $(G, N)$ of maps induced by inclusion and $2c\zeta$-interleaved with $\S^{2\zeta}$ by a pair $(G',N')$ of maps induced by inclusion.
%   The following diagram commutes by Lemma~\ref{lem:rel_pers_nerve} where vertical maps are isomorphisms and horizontal maps are induced by inclusion for all $\alpha$.
%   \begin{equation}\begin{tikzcd}
%     T^\zeta_\alpha\arrow{r}{i_\alpha} &
%     T^{2\zeta}_\alpha\\
%     %
%     \cech^\zeta S_\alpha\arrow{r}{j_\alpha}\arrow{u}{\eta_\alpha^\zeta}  &
%     \cech^{2\zeta} S_\alpha\arrow{u}{\eta_\alpha^{2\zeta}}
%   \end{tikzcd}\end{equation}
%   Note that, because all maps are induced by inclusion, $m_\alpha = m_\alpha'\circ i_\alpha$ and $f_{\alpha-c\delta}' = i_\alpha\circ f_{\alpha-c\delta}$ for all $\alpha$.
%   Moreover, we have the following sequence of maps induced by inclusions for all $\alpha$
%   \[ \cech^\delta S_\alpha\xrightarrow{k_\alpha} \rips^{2\delta} S_\alpha\xrightarrow{\ell_\alpha} \cech^{2\delta} S_\alpha. \]
%   So $j_\alpha = \ell_\alpha\circ k_\alpha$ and
%   \begin{align*}
%     i_\alpha &= \eta_\alpha^{2\delta}\circ j_\alpha\circ (\eta_\alpha^\delta)^{-1}\\
%       &= \eta_\alpha^{2\delta}\circ \ell_\alpha\circ k_\alpha\circ (\eta_\alpha^\delta)^{-1}.
%   \end{align*}
%   Letting $\Sigma\in\Hom(\S^\delta, \rips^{2\delta}\S)$ be defined as the family
%   \[\{\sigma_\alpha := k_\alpha\circ (\eta_\alpha^\delta)^{-1} : S_\alpha^\delta\to \rips^{2\delta} S_\alpha\}\]
%   and $\Upsilon\in\Hom(\rips^{2\delta}\S, \S^{2\delta})$ be defined
%   \[\{\upsilon_\alpha := \eta_\alpha^{2\delta}\circ \ell_\alpha : \rips^{2\delta} S_\alpha \to S^{2\delta}_\alpha\}\]
%   it follows that $M = M'\circ\Upsilon\circ \Sigma$ and $F' = \Upsilon\circ\Sigma\circ F$, and therefore that $\Gamma$ is $2c\delta$-interleaved with $\rips^{2\delta}\S$ by Lemma~\ref{lem:left}.
% \end{proof}

% % We have the following maps induced by inclusions
% % \[ \cech^\e\Lambda\in\Hom(\cech^\e\S,\cech^\e\T),\ \rips^\e\Lambda\in\Hom(\rips^\e\S,\rips^\e\T).\]
%
% % Let $\N^\e \in\Hom(\S^\e,\cech^\e\S)$ be the family of isomorphisms
% % \[\{\eta_\alpha^\e : \hom_k(P\subi{\omega-c\zeta,\alpha}^\e, \Q^\e)\xrightarrow{\cong}\hom_k(\cech^\e(P\subi{\omega-c\zeta,\alpha}),\cech^\e(\Q))\}\]
% % provided by the Nerve Theorem.
% % We have the following sequence of maps induced by inclusion for all $\e > 0$, $\alpha\in\R$
% % \[ \cech^\e S_\alpha\xrightarrow{R_\alpha^\e} \rips^{2\e} S_\alpha\xrightarrow{C_\alpha^{2\e}}\cech^{2\e} S_\alpha. \]
% % \[ \cech^\e S_\alpha\xrightarrow{R_\alpha^\e} \rips^{2\e} S_\alpha\xrightarrow{C_\alpha^{2\e}}\cech^{2\e} S_\alpha. \]
% By Lemma~\ref{lem:rel_pers_nerve} the following diagrams commutes for all $\alpha$ where the vertical arrows are isomorphisms and the horizontal arrows are maps induced by inclusion.
%
% \begin{equation}\begin{tikzcd}
%   S^\delta_\alpha\arrow{r}{a_\alpha}\arrow{d}{i_\alpha} &
%   S^{2\delta}_\alpha\arrow{r}{b_\alpha}\arrow{d}{j_\alpha} &
%   T^{\zeta}_\alpha\arrow{r}{c_\alpha}\arrow{d}{k_\alpha} &
%   T^{2\zeta}_\alpha\arrow{d}{\ell_\alpha}\\
%   %
%   \cech^\delta S_\alpha\arrow{r}{a_\alpha'} &
%   \cech^{2\delta} S_\alpha\arrow{r}{b_\alpha'} &
%   \cech^{\zeta} T_\alpha\arrow{r}{c_\alpha'} &
%   \cech^{2\zeta} T_\alpha &
% \end{tikzcd}\end{equation}
% % \begin{equation}\begin{tikzcd}
% %   S^\delta_\alpha\arrow{r}\arrow{d} &
% %   S^{2\delta}_\alpha\arrow{r}\arrow{d} &
% %   T^{\zeta}_\alpha\arrow{r}\arrow{d} &
% %   T^{2\zeta}_\alpha\arrow{d}\\
% %   %
% %   \cech^\delta S_\alpha\arrow{r}\arrow{dr} &
% %   \cech^{2\delta} S_\alpha\arrow{r} &
% %   \cech^{\zeta} T_\alpha\arrow{r}\arrow{d} &
% %   \cech^{2\zeta} T_\alpha\\
% %   %
% %   &\rips^{2\delta} S_\alpha\arrow{u}\arrow{r} &
% %   \rips^{2\zeta} T_\alpha\arrow{ur} &
% % \end{tikzcd}\end{equation}
%
%
%
% Noting that we have the following sequences of maps induced by inclusion
% \[ \cech^\e S_\alpha\xrightarrow{p_\alpha} \rips^{2\e} S_\alpha\xrightarrow{q_\alpha}\cech^{2\e} S_\alpha, \]
% \[ \cech^\e T_\alpha\xrightarrow{p_\alpha'} \rips^{2\e} T_\alpha\xrightarrow{q_\alpha}\cech^{2\e} T_\alpha. \]
% As all maps are induced by inclusion it follows that $a_\alpha' = q_\alpha\circ p_\alpha$, $c_\alpha' = q_\alpha'\circ p_\alpha'$, and $b_\alpha'
%
% % Let $\N^{\e}'\in\Hom(\T^\e,\cech^\e\T)$ be defined similarly.
%
% Let $\Sigma\in \Hom(\S^\delta, \rips^{2\delta}\S)$ be defined as the family linear maps
% \[\{\sigma_\alpha := R_\alpha^\delta\circ \eta_\alpha^\delta :S_\alpha^\delta\to\rips^{2\delta}S_\alpha\}\]
% and $\Upsilon\in \Hom(\rips^{2\delta}\S, \S^{2\delta})$ be defined as the family linear maps
% \[\{\upsilon_\alpha := (\eta_\alpha^{2\delta})^{-1}\circ C_\alpha^{2\delta} :\rips^{2\delta}S_\alpha\to S_\alpha^{2\delta}\}.\]
%
% Suppose $\Phi^\delta_{M^\delta}(F^\delta, G^\delta)\in\Hom^\delta(\im~\Gamma, \im~\Lambda^\delta)$ is a partial $\delta$-interleaving and $\Phi^\delta_{M^{2\delta}}(F^{2\delta}, G^{2\delta})\in\Hom^\delta(\im~\Gamma, \im~\Lambda^{2\delta})$ is a partial $2\delta$-interleaving where all maps are induced by inclusion.
% Because $\N^\e$ is an isomorphism for all $\delta\leq\e\leq2\delta$ and $R_\alpha^\e$, $C^\e_\alpha$ are induced by inclusion
% \begin{align*}
%   m_\alpha^\delta &= m_\alpha^{2\delta}\circ ((\eta_\alpha^{2\delta})^{-1}\circ C_\alpha^{2\delta})\circ (R_\alpha^\delta\circ \eta_\alpha^\delta)\\
%     &= m_\alpha\circ\upsilon_\alpha\circ\sigma_\alpha
% \end{align*}
% thus $M^\delta = M^{2\delta}\circ \Upsilon\circ \Sigma$.
% Because $\Gamma$ is left $\delta$ interleaved with  $\Lambda^\delta$ and left $2\delta$ interleaved with $\Lambda^{2\delta}$ it follows that $\Gamma$ is left $2\delta$-interleaved with $\rips^{2\delta}\Lambda$ by Lemma~\ref{lem:left}.

% By the Nerve Theorem\textbf{TODO} there exist isomorphisms $\N^\e \in \Hom(\S^\e\to \cech^\e\S$ and $\T^\e\to\cech^\e\T$ for all $\e$.
% Moreover, we have the following sequences of inclusions for all $\e > 0$, $\alpha\in\R$
% \[ \cech^\e S_\alpha\subseteq \rips^{2\e} S_\alpha \subseteq \cech^{2\e} S_\alpha \]
% \[ \cech^\e T_\alpha\subseteq \rips^{2\e} T_\alpha \subseteq \cech^{2\e} T_\alpha. \]
% It follows that we have the following sequence of maps between image modules
% \[ \im~\Lambda^\e\to \cech^\e\Lambda\to\rips^{2\e}\Lambda\to\cech^{2\e}\Lambda\to\im\Lambda^{2\e}.\]

% So we have the following homomorphisms of persistence modules.
% \begin{align*}
%   \Sigma&\in\Hom(\S^\delta, \rips^{2\delta}\S) & \Theta&\in\Hom(\T^\delta,\rips^{2\delta}\T)\\
%   \Upsilon&\in\Hom(\rips^{2\delta}\S, \S^{2\delta}) & \Xi&\in\Hom(\rips^{2\delta}\T, \T^{2\delta}).
% \end{align*}
% \textbf{TODO $S^{2\delta}_{\alpha-\delta}\subseteq T^\delta_{\alpha+\delta}$.}
%
% \begin{itemize}
%   \item All maps between Rips/Cech induced by inclusion: there is a partial 0-interleaving from $\cech^\delta\Lambda$ to $\rips^{2\delta}\Lambda$ and from from $\rips^{2\delta}\Lambda$ to $\cech^{2\delta}\Lambda$ by Lemmas~\ref{lem:left} and~\ref{lem:right}
%   \item If everything commutes with the nerve theorem isomorphism there is a partial 0-interleaving from $\Lambda^\delta$ to $\cech^\delta\Lambda$ (and therefore from $\Lambda^\delta$ to $\rips^{2\delta}\Lambda$) and a partial $2\delta$-interleaving from $\cech^{2\delta}\Lambda$ to $\Lambda^{2\delta}$ by Lemmas~\ref{lem:left} and~\ref{lem:right}.
%   \item So there is a partial $2\delta$ interleaving from $\im~\Gamma$ to $\im~\rips^{2\delta}\Lambda$ and from $\im~\rips^{2\delta}\Lambda$ to $\im~\Pi$: apply Theorem~\ref{thm:interleaving_main}.
% \end{itemize}

% the $k$th persistent homology modules of
% \[ \{(P^\delta\subi{\omega-2c\delta,\alpha},Q_{\omega-2c\delta}^\delta)\to (P^\delta\subi{\omega+c\delta,\alpha},Q^{2\delta}_{\omega+c\delta}^\delta)\} \]
% and
% \[ \{(P^{2\delta}\subi{\omega-2c\delta,\alpha},Q_{\omega-2c\delta}^{2\delta})\to (P^{2\delta}\subi{\omega+c\delta,\alpha},Q^{2\delta}_{\omega+c\delta}^\delta)\} \]
% are $c\delta$ and $2c\delta$ interleaved with the $k$th persistent homology module of $\{(D\subi{\omega,\alpha},B_\omega)\}$, respectively.
%
%
% We also define $\hat{\S}$ and $\hat{\T}$ to be the $k$th persistent homology modules of $\{(\ext{P^{2\delta}\subi{\omega-2c\of,\alpha}},\ext{Q_{\omega-2c\of}^{2\of}})\}$ and $\{(\ext{P^{2\of}\subi{\omega+c\of,\alpha}},\ext{\QQ^{2\of}})\}$ respectively.
% We also define the $k$th persistent homology modules of the corresponding Rips and \v{C}ech complexes as
% % \[ \cech^\delta\S := (\{\check{S_\alpha}:=\hom_k(\ext{\cech^\delta(P\subi{\omega-c\of,\alpha}},\ext{\cech^\delta(Q_{\omega-c\of})}))
%
% $\cech^\e$ and $\rips^\e$ takes pairs of finite point sets to simplicial complexes.
% \begin{align*}
%   K \in\Hom(\cech^\delta\S, \rips^{2\delta}\S),&& L \in \Hom(\rips^{2\delta}\S, \cech^{2\delta}\S)\\
%   K'\in\Hom(\cech^\delta\T, \rips^{2\delta}\T),&& L' \in\Hom(\rips^{2\delta}\T, \cech^{2\delta}\T)
% \end{align*}
% are all homomorphisms induced by inclusion.
% We also have the following isomorphisms provided by the nerve theorem.
% \begin{align*}
%   C \in\Hom(\S,\cech^\of\S),&& D \in\Hom(\cech^{2\of}\S,\hat{\S})\\
%   C'\in \Hom(\T,\cech^\of\T),&& D'\in \Hom(\cech^{2\of}\T,\hat{\T})
% \end{align*}
%
% For $F\in\Hom^{c\delta}(\U,\S)$, $\hat{M}\in\Hom^{2c\delta}(\hat{\S},\V)$, $G\in\Hom^{c\delta}(\V,\T)$, and $\hat{N}\in\Hom^{2c\delta}(\hat{\T},\W)$ we define the following homomorphisms of degree $c\of$
% \[ \tilde{F} := \{\tilde{f_\alpha}:= k_{\alpha+c\of}\circ c_{\alpha+c\of}\circ f_\alpha : \U\to\rips^{2\of}\S\}\]
% \[ \tilde{G} := \{\tilde{g_\alpha}:= k_{\alpha+c\of}'\circ c_{\alpha+c\of}'\circ g_\alpha : \V\to\rips^{2\of}\T\}\]
% and the following homomorphisms of degree $2c\of$
% \[ \tilde{M} := \{\tilde{m_\alpha}:= \hat{m_\alpha}\circ d_\alpha\circ \ell_\alpha :\rips^{2\of}\S\to\V\},\]
% \[ \tilde{N} := \{\tilde{n_\alpha}:=\hat{n_\alpha}\circ d_\alpha'\circ\ell_\alpha' : \rips^{2\of}\T\to\W\}.\]
%
% Let $\Gamma\in\Hom(\U, \V)$, $\Pi\in\Hom(\V,\W)$, and $\tilde{\Lambda}\in\Hom(\rips^{2\of}\S,\rips^{2\of}\T)$.
% If the following diagrams commute $\tilde{\Phi}(\tilde{F},\tilde{G}) : \im~\Gamma\to\im~\tilde{\Lambda}$ is an image module homomorphism of degree $c\delta$ and $\tilde{\Psi}(\tilde{M},\tilde{N}) : \im~\tilde{\Lambda}\to \im~\Pi$ is an image module homomorphism of degree $2c\of$.
%
% % \begin{minipage}{0.5\textwidth}
% \begin{equation}\label{dgm:shifted_homomorphism_rips1}
%   \begin{tikzcd}[column sep=large]
%     U_\alpha\arrow{rr}{v_\alpha^\beta\circ\gamma_\alpha}\arrow{d}{\tilde{f}_\alpha} &&
%     V_\beta\arrow{d}{\tilde{g}_\beta}\\
%     %
%     \rips^{2\of} S_{\alpha+c\delta}\arrow{rr}{\tilde{t}_{\alpha+c\delta}^{\beta+c\delta}\circ\tilde{\lambda}_{\alpha+c\delta}} &&
%     \rips^{2\of} T_{\beta +c\delta}
% \end{tikzcd}\end{equation}
% \begin{equation}\label{dgm:shifted_homomorphism_rips2}
%   \begin{tikzcd}[column sep=large]
%     \rips^{2\of} S_{\alpha}\arrow{rr}{\tilde{t}_{\alpha}^{\beta}\circ\tilde{\lambda}_{\alpha}}\arrow{d}{\tilde{m}_\alpha} & &
%     \rips^{2\of} T_{\beta}\arrow{d}{\tilde{n}_\alpha}\\
%     %
%     V_{\alpha+2c\of}\arrow{rr}{v_{\alpha+2c\delta}^{\beta+2c\delta}\circ\gamma_{\alpha+2c\delta}} &&
%     W_{\beta+2c\of}\\
% \end{tikzcd}\end{equation}
%
% If the following diagram commutes $\tilde{\Phi}_{\tilde{M}}$ and $\tilde{\Psi}_{\tilde{G}}$ are partial $2c\of$-interleavings of image modules.
%
% % \end{minipage}
% % \begin{minipage}{0.5\textwidth}
% % \begin{equation}\label{dgm:shifted_homomorphism_rips2}
% %   \begin{tikzcd}[column sep=large]
% %     \rips^{2\of} S_{\alpha}\arrow{r}{\tilde{t}_{\alpha}^{\beta}\circ\tilde{\lambda}_{\alpha}}\arrow{d}{\tilde{m}_\alpha} &
% %     \rips^{2\of} T_{\beta}\arrow{d}{\tilde{n}_\alpha}\\
% %     %
% %     V_{\alpha+2c\of}\arrow{r}{v_{\alpha+2c\delta}^{\beta+2c\delta}\circ\gamma_{\alpha+2c\delta}} &
% %     W_{\beta+2c\of}\\
% % \end{tikzcd}\end{equation}
% % \end{minipage}
% \begin{equation}\label{dgm:partial_interleaving_rips}
%   \begin{tikzcd}
%     U_{\alpha-3c\delta}\arrow{rr}{v_{\alpha-3c\delta}^{\alpha}\circ\gamma_{\alpha-c\delta}}\arrow{dr}{\tilde{f}_{\alpha-c\delta}} & &
%     V_{\alpha}\arrow{rr}{w_\alpha^{\alpha+3c\of}\circ\pi_\alpha}\arrow{dr}{\tilde{g_\alpha}} & &
%     W_{\alpha+3c\of}\\
%     %
%     & \rips^{2\of}S_{\alpha-2c\of}\arrow{ur}{\tilde{m}_{\alpha-2c\of}}\arrow{rr}{\tilde{t}_{\alpha-2c\of}^{\alpha+c\of}\circ\tilde{\lambda}_{\alpha-2c\of}} & &
%     \rips^{2\of}T_{\alpha+c\of}\arrow{ur}{\tilde{n}_{\alpha+c\of}} &
% \end{tikzcd}\end{equation}
