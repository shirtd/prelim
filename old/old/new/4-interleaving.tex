% !TeX root = ../new.tex

\begin{lemma}
  Let $\Omega = [\omega+2c\of, \infty)$.
  The $k$th persistent homology modules of
  \[ \{((\P)_\alpha, \Q^\delta)\to ((\P)_\alpha, \QQ^\delta)\}_{\alpha\in\Omega} \]
  and
  \[ \{(\P_\alpha, \Q^\delta)\to (\P_\alpha,\QQ^\delta)\}_{\alpha\in\Omega} \]
  are $c\of$-interleaved for all $k > 0$.
\end{lemma}
\begin{proof}
  Clearly $\{((\P)_\alpha, (\Q^\of, \QQ^\of))\}$ and $\{(\P_\alpha, (\Q^\of, \QQ^\of))\}$ are both image compatible filtrations over $\Omega$ as all maps are induced by inclusion with $\Q^\of, \QQ^\of\subseteq (\P)_\alpha\cap \P_{\alpha'}$ for all $\alpha,\alpha'\in\Omega$.
  By Lemma~\ref{lem:ps_inter} $(\P)_\alpha\sim_{c\of} \P_\alpha$ for all $\alpha\in\R$ so
  \[ ((\P)_\alpha, \Q^\delta)\sim_{c\of} (\P_\alpha, \Q^\delta)\text{ for } \alpha \geq \omega - 2c\of\]
  and
  \[ ((\P)_\alpha, \QQ^\delta)\sim_{c\of} (\P_\alpha, \QQ^\delta)\text{ for }\alpha \geq \omega + 2c\delta.\]
  Because $\Q^\of$ and $\QQ^\of$ are $0$-interleaved trivially with themselves for all $\alpha\in\Omega$ it follows that Diagrams~\ref{dgm:fabgcd} and~\ref{dgm:gcdfab} commute for all $\alpha\in\Omega$ so the image compatible filtrations are compatibly $\e$-interleaved over $\Omega$.
  The result therefore follows from Theorem~\ref{thm:short_pair_inter}.
  % Because all maps are induced by inclusion the following diagrams commute for all $\beta\geq\alpha\geq\in\Omega$.
  %
  % \begin{scriptsize}
  % \vspace{3ex}\begin{subequations}
  % \begin{minipage}{0.5\textwidth}\begin{equation}\label{dgm:fab}
  % \begin{tikzcd}[column sep=scriptsize]
  %   \hom_k((\P)_\alpha, \Q^\delta)\arrow{r} \arrow{d} &
  %   \hom_k((\P)_\alpha, \QQ^\delta)\arrow{d} \\
  %   %
  %   \hom_k((\P)_\beta, \Q^\delta)\arrow{r} &
  %   \hom_k(\P)_\beta, \QQ^\delta)
  % \end{tikzcd}\end{equation}\end{minipage}
  % \begin{minipage}{0.5\textwidth}\begin{equation}\label{dgm:gcd}
  % \begin{tikzcd}[column sep=scriptsize]
  %   \hom_k(\P_\alpha, \Q^\of)\arrow{r} \arrow{d} &
  %   \hom_k(\P_\alpha, \QQ^\of)\arrow{d} \\
  %   %
  %   \hom_k(\P_\beta, \Q^\of)\arrow{r} &
  %   \hom_k(\P_\beta, \QQ^\of)
  % \end{tikzcd}\end{equation}\end{minipage}
  % \end{subequations}\vspace{3ex}
  %
  % \vspace{3ex}\begin{subequations}
  % \begin{minipage}{0.5\textwidth}\begin{equation}\label{dgm:fabgcd}
  % \begin{tikzcd}[column sep=scriptsize]
  %   \hom_k((\P)_\alpha, \Q^\delta)\arrow{r}\arrow{d} &
  %   \hom_k((\P)_\alpha, \QQ^\delta)\arrow{d}\\
  %   %
  %   \hom_k(\P_{\alpha+c\delta}, \Q^\of)\arrow{r}&
  %   \hom_k(\P_{\alpha+c\delta}, \QQ^\of)
  % \end{tikzcd}\end{equation}\end{minipage}
  % \begin{minipage}{0.5\textwidth}\begin{equation}\label{dgm:gcdfab}
  % \begin{tikzcd}[column sep=scriptsize]
  %   \hom_k(\P_\alpha, \Q^\of)\arrow{r}\arrow{d} &
  %   \hom_k(\P_\alpha, \Q^\of)\arrow{d}\\
  %   %
  %   \hom_k((\P)_{\alpha+\e}, \Q^\delta)\arrow{r} &
  %   \hom_k((\P)_{\alpha+\e}, \QQ^\delta)
  % \end{tikzcd}\end{equation}\end{minipage}
  % \end{subequations}\vspace{3ex}
  % \end{scriptsize}
  %
  % The result therefore follows from Lemma~\ref{lem:short_pair_inter}.
\end{proof}


In the following proof, we will be dealing with parameterized pairs of spaces $(X_\alpha, A)$ with ill-defined homology for $\alpha\in\R$ such that $A\not\subseteq X_\alpha$.
We therefore introduce the following utility function in order to complete the filtration to have well-defined homology groups for all $\alpha\in\R$.

Let $f_\alpha : X_\alpha\to Y_\alpha$ be a function between two parameterized spaces $X_\alpha$ and $Y_\alpha$ for $\alpha\in [\omega,\infty)$.
We can \textbf{complete} the parameterization of the function to $\alpha\in\R$ as follows.
\[ T_\omega(f_\alpha) := \begin{cases} f_\alpha&\text{ if } \omega < \alpha\\ f_\omega&\text{ otherwise.}\end{cases} \]
Similarly, let $g^\beta_\alpha : X_\alpha\to X_\beta$ be a function parameter shift function defined on a parameterized space $X_\alpha$ for $\beta\geq\alpha\in [\omega,\infty)$.
Now, with a slight abuse of notation, we define the completion of $g_\alpha^\beta$ as
\[ T_\omega(g_\alpha^\beta) := \begin{cases} g_\alpha^\beta&\text{ if } \omega < \alpha\\ g_\omega^\beta&\text{ if } \alpha\leq \omega < \beta\\ \id_{X_\omega}&\text{ otherwise.}\end{cases} \]


\begin{lemma}
  Suppose $\b$ surrounds $D$ in $\X$ and $\Q^\of$ surrounds $\P$ in $D$ such that
  \begin{itemize}
    \item $D\setminus \B\subseteq \P$,
    \item $\b\cap \P \subseteq \Q^\of\subseteq \B$, and
    \item $\B\cap \P\subseteq \QQ^\of\subseteq \BB$.
  \end{itemize}

  If $\eta^k : \hom_k(\b)\to \hom_k(\BB)$ is surjective and $\im~\eta^k\cong \hom_k(\B)$ then the $k$th persistent homology modules of $\{\hom_k(B_\alpha, \B)\}_{\alpha < 2c\of}$ and $\{(\P_\alpha, \Q^\delta)\to (\P_\alpha,\QQ^\delta)\}_{\alpha < 2c\of}$ are $c\of$-interleaved for all $k$.
\end{lemma}
\begin{proof}
  % Let $\ell_\alpha : \hom_k(B_\alpha,\ext{\Q^\of})\to \hom_k(B_\alpha, \ext{\QQ^\of})$ for $\alpha\geq\omega+2c\of$ and $


  % By Corollary~\ref{cor:tcc_iso}


  % As
  % \[ \b\subseteq \ext{\Q^\of}\subseteq \B\subseteq \ext{\QQ^\of} \]
  % by Lemma~\ref{lem:surround_and_cover} we have the following commutative diagrams induced by inclusion for all $\alpha\in (\omega+c\of, \omega+2c\of)$.

  % So we have the following image compatible filtrations over $(\omega+c\of, \omega+2c\of)$.
  % \[ \{(B_\alpha, (\b, \B))\},\ \{(B_\alpha, (\B, \BB))\},\text{ and } \{(\ext{\P_\alpha}, (\ext{\Q^\of}, \ext{\QQ^\of}))\},\]
  % where all maps are induced by inclusion.
  % Note that these filtrations are not compatibly $c\of$ interleaved over $(\omega+c\of, \omega+2c\of)$.
  % However, the following diagrams do commute.

  The following diagrams commute with inclusions over $[\omega,\infty)$ and $[\omega+2c\of,\infty)$, respectively.

  \begin{scriptsize}\vspace{3ex}
  \begin{subequations}
  \begin{minipage}{0.5\textwidth}
    \begin{equation}\label{dgm:intr_tight1a}
    \begin{tikzcd}%[column sep=scriptsize]
      \hom_k(B_{\alpha}, \b)\arrow{r}{r_{\alpha}} \arrow{d}{f_{\alpha}^\beta} &
      \hom_k(B_{\alpha}, \B)\arrow{d}{g_\alpha^\beta} \\
      %
      \hom_k(B_\beta, \b)\arrow{r}{r_{\beta}} &
      \hom_k(B_\beta, \B)
    \end{tikzcd}\end{equation}
  \end{minipage}
  \begin{minipage}{0.5\textwidth}
    \begin{equation}\label{dgm:intr_tight1b}
    \begin{tikzcd}%[column sep=scriptsize]
      \hom_k(B_{\alpha}, \B)\arrow{r}{s_{\alpha}} \arrow{d}{g_{\alpha}^\beta} &
      \hom_k(B_{\alpha}, \BB)\arrow{d}{h_\alpha^\beta} \\
      %
      \hom_k(B_\beta, \B)\arrow{r}{s_{\beta}} &
      \hom_k(B_\beta, \BB)
    \end{tikzcd}\end{equation}
  \end{minipage}
  \end{subequations}
  \end{scriptsize}\vspace{3ex}

  Because $\Q^\delta$ surrounds $\P$ in $D$ and $D\setminus \B\subseteq \P$ we have
  \[ \hom_k((B_\alpha,\ext{\Q^\of})\to (B_\alpha, \ext{\QQ^\of}))\cong \hom_k((\P\cap B_\alpha, \Q^\of)\to (\P\cap B_\alpha, \QQ^\of)) \]
  for $\alpha\geq\omega+2c\of$ by Lemma~\ref{lem:excision}, where
  \[ B_{\alpha-c\of} = (\P\cup (D\setminus \P))\cap B_{\alpha -c\of}\subseteq \P_\alpha \cup (B_{\alpha - c\of}\cap (D\setminus \P)).\]
  Moreover, $B_{\alpha - c\of}\cap (D\setminus \P) = (D\setminus \P)$ for $\alpha\geq \omega+c\of$ so we define the extension $\ext{\P_\alpha} := \P_\alpha\cup (D\setminus \P)$ of $\P_\alpha$ for $\alpha\geq\omega+c\of$.
  The following diagrams are induced by inclusion and therefore commute over $[\omega+c\of, \infty)$ and $[\omega+2c\of, \infty)$, respectively.

  \begin{scriptsize}\vspace{3ex}
  \begin{subequations}
  \begin{minipage}{0.5\textwidth}
    \begin{equation}\label{dgm:intr_tight2a}
    \begin{tikzcd}%[column sep=scriptsize]
      \hom_k(\ext{\P_{\alpha}}, \ext{\Q^\delta})\arrow{r}{t_\alpha} \arrow{d}{a_\alpha^\beta} &
      \hom_k(\ext{\P_{\alpha}}, \ext{\QQ^\delta})\arrow{d}{b_\alpha^\beta} \\
      %
      \hom_k(\ext{\P_{\beta}}, \ext{\Q^\delta})\arrow{r}{t_\beta} &
      \hom_k(\ext{\P_{\beta}}, \ext{\QQ^\delta})
    \end{tikzcd}\end{equation}
  \end{minipage}
  \begin{minipage}{0.5\textwidth}
    \begin{equation}\label{dgm:intr_tight2b}
    \begin{tikzcd}%[column sep=scriptsize]
      \hom_k(B_\alpha, \ext{\Q^\delta})\arrow{r}{\ell_\alpha} \arrow{d}{d_\alpha^\beta} &
      \hom_k(B_\alpha, \ext{\QQ^\delta})\arrow{d}{e_\alpha^\beta} \\
      %
      \hom_k(B_\beta, \ext{\Q^\delta})\arrow{r}{\ell_\beta} &
      \hom_k(B_\beta, \ext{\QQ^\delta})
    \end{tikzcd}\end{equation}
  \end{minipage}
  \end{subequations}
  \end{scriptsize}\vspace{3ex}


  Now, by Lemma~\ref{lem:ps_inter}, $\ext{\P_{\alpha-c\of}}\subseteq B_\alpha\subseteq\ext{\P_{\alpha+c\of}}$ for all $\alpha\in\R$, so the following diagram commutes for $\alpha\geq\omega+2c\of$.
  \begin{equation}\label{dgm:intr_tight3}
  \begin{tikzcd}%[column sep=scriptsize]
    \hom_k(\ext{\P_{\alpha-c\of}}, \ext{\Q^\of})\arrow{r}{t_{\alpha-c\of}}\arrow{d}{i_{\alpha-c\of}} &
    \hom_k(\ext{\P_\alpha}, \ext{\QQ^\of})\arrow{d}{j_{\alpha-c\of}}\\
    %
    \hom_k(B_{\alpha}, \ext{\Q^\of})\arrow{r}{\ell_{\alpha-c\of}} \arrow{d}{p_{\alpha}} &
    \hom_k(B_{\alpha}, \ext{\QQ^\of})\arrow{d}{q_{\alpha}} \\
    %
    \hom_k(\ext{\P_{\alpha+c\of}}, \ext{\Q^\of})\arrow{r}{t_{\alpha+c\of}} &
    \hom_k(\ext{\P_{\alpha+c\of}}, \ext{\QQ^\of})
  \end{tikzcd}\end{equation}
  Moreover, by Corollary~\ref{cor:tcc_iso}, we have an isomorphism $\xi_\alpha : \im~\ell_\alpha\to \hom_k(B_\alpha, \B)$ for all $\alpha\geq\omega+2c\of$.

  Finally, we have the following commutative diagram induced by inclusion for $\alpha\geq\omega+c\of$.
  \begin{equation}\label{dgm:intr_tight4}
  \begin{tikzcd}%[column sep=scriptsize]
    \hom_k(B_{\alpha-c\of}, \b)\arrow{r}{r_{\alpha-c\of}} \arrow{d}{u_{\alpha-c\of}} &
    \hom_k(B_{\alpha-c\of}, \B)\arrow{d}{v_{\alpha-c\of}} \\
    %
    \hom_k(\ext{\P_\alpha}, \ext{\Q^\of})\arrow{r}{t_{\alpha}}\arrow{d}{m_{\alpha}} &
    \hom_k(\ext{\P_\alpha}, \ext{\QQ^\of})\arrow{d}{n_{\alpha}}\\
    %
    \hom_k(B_{\alpha+c\of}, \B)\arrow{r}{s_{\alpha+c\of}} &
    \hom_k(B_\alpha, \BB)
  \end{tikzcd}\end{equation}

  %
  %  so that the maps $u_{\alpha-c\of}$ and $v_{\alpha-c\of}$ below are induced by inclusion.
  % We also note that $\P_{\alpha-c\of}\subseteq B_\alpha$ for all $\alpha$ as
  % \[ \P_{\alpha-c\of}\subseteq \P\cap B_\alpha\subseteq B_\alpha.\]

  % \begin{scriptsize}\vspace{3ex}
  % \begin{subequations}
  % \begin{minipage}{0.5\textwidth}
  %   \begin{equation}\label{dgm:intr_tight2a}
  %   \begin{tikzcd}%[column sep=scriptsize]
  %     \hom_k(B_{\alpha-c\of}, \b)\arrow{r}{r_{\alpha-c\of}} \arrow{d}{u_{\alpha-c\of}} &
  %     \hom_k(B_{\alpha-c\of}, \B)\arrow{d}{v_{\alpha-c\of}} \\
  %     %
  %     \hom_k(\ext{\P_\alpha}, \ext{\Q^\of})\arrow{r}{\tau_{\alpha}} &
  %     \hom_k(\ext{\P_\alpha}, \ext{\QQ^\of})
  %   \end{tikzcd}\end{equation}
  % \end{minipage}
  % \begin{minipage}{0.5\textwidth}
  %   \begin{equation}\label{dgm:intr_tight2b}
  %   \begin{tikzcd}%[column sep=scriptsize]
  %     \hom_k(\ext{\P_{\alpha-c\of}}, \ext{\Q^\delta})\arrow{r}{t_{\alpha-c\of}} \arrow{d}{m_{\alpha-c\of}} &
  %     \hom_k(\ext{\P_{\alpha-c\of}}, \ext{\QQ^\delta})\arrow{d}{n_{\alpha-c\of}} \\
  %     %
  %     \hom_k(B_\alpha, \B)\arrow{r}{s_\alpha} &
  %     \hom_k(B_\alpha, \BB)
  %   \end{tikzcd}\end{equation}
  % \end{minipage}
  % \end{subequations}
  % \end{scriptsize}\vspace{3ex}

  Using these diagrams we define the following for all $\alpha\in\R$.
  \begin{align*}
    \Gamma_\alpha &:=\im~T_\omega(r_\alpha),&\  \Sigma_\alpha &:= \im~T_{\omega+2c\of}(s_\alpha)\\
    \Phi_\alpha &= \im~ T_{\omega+c\of}(t_\alpha),&\ \Lambda_\alpha &:=\im~T_{\omega+2c\of}(\ell_\alpha),\\
  \end{align*}
  and
  \[ \Psi_\alpha := \begin{cases} \hom_k(B_\alpha, \B)&\text{ if } \omega < \alpha\\ \hom_k(\B, \B)&\text{ otherwise.}\end{cases} \]
  with
  \begin{align*}
    \phi_\alpha^\beta &:= T_{\omega+c\of}(b_\alpha^\beta\rest_{\Phi_\alpha}),&\ \lambda_\alpha^\beta &:= T_{\omega+2c\of}(\ell_\alpha),\\
    \sigma_\alpha &= T_{\omega+c\of}(n_\alpha\rest_{\Phi_\alpha}),&\ \gamma_\alpha &:= T_\omega(v_\alpha\rest_{\Gamma_\alpha}).
  \end{align*}

  Because we have assumed $\im~\eta^k = \hom_k(B_\alpha, \B)$ for all $\alpha\geq\omega+2c\of$ and $\eta^k = s_\alpha\circ r_\alpha$, $r_\alpha$ must be surjective for all $\alpha\geq \omega$.
  As $\hom_k(B_\alpha, \B)$ is trivial for all $\alpha\leq\omega$ it follows that $\Gamma_\alpha = \Psi_\alpha$ for all $\alpha\in\R$.

  Finally, because $s_\alpha : \hom_k(B_\alpha, \B)\to \hom_k(B_\alpha, \BB)$ is an isomorphism for all $\alpha\geq\omega+2c\of$ and $\Sigma_\alpha = \im~s_{\omega+2c\of}$ is trivial for all $\alpha\leq \omega+2c\of$ the map $\pi_\alpha : \Psi_\alpha\to \Sigma_\alpha$ induced by inclusion is injective for all $\alpha\in(\omega,\omega+2c\of)$ and bijective for $\alpha\in (-\infty, \omega]\cup [\omega+2c\of,\infty)$.
  % \[ \pi_\alpha := \begin{cases} s^{-1}_\alpha&\text{ if } \alpha > \omega+2c\of\\ \mathbf{0}_{\Sigma_\alpha\to \Psi_\alpha}&\text{ otherwise.}\]


  We would like to show that the following diagrams commute for all $\beta\geq\alpha\in\R$.

  \begin{equation}\label{dgm:intr_tight2a}
  \begin{tikzcd}
    % Fa & & & & Fb
    \Phi_{\alpha-c\of}  \arrow[to=Fb]
                        \arrow[to=Ea]
                        \arrow[to=Ga]
                        \arrow[to=Ha]
    &
    & &
    & |[alias=Fb]|
      \Phi_{\beta+c\of} \arrow[from=Hb]
                        \arrow[from=Gb]\\
    |[alias=Ha]|
    \Sigma_\alpha\arrow[to=Ga]
    & |[alias=Ea]|
    \Lambda_{\alpha}  \arrow[to=Eb]
                      \arrow[to=Ga]
    & & |[alias=Eb]|
    \Lambda_{\beta} \arrow[to=Fb]
                    \arrow[from=Gb]
    & |[alias=Hb]|
    \Sigma_\beta\\
    % |[alias=Ha]|
    % \Sigma_\alpha\arrow[to=Ga]
    & |[alias=Ga]|
    \Psi_\alpha\arrow[to=Gb]
    & & |[alias=Gb]|
    \Psi_\beta\arrow[to=Hb]
    % & |[alias=Hb]|
    % \Sigma_\beta
  \end{tikzcd}
  \end{equation}

  %
  % \vspace{3ex}
  % \begin{scriptsize}
  % \begin{subequations}
  % \begin{minipage}{0.45\textwidth}
  % \begin{equation}\label{dgm:intr_tight2a}\begin{tikzcd}[column sep=scriptsize]
  %   % Fa & & & Fb
  %   \Phi\arrow[to=Fb, "b_{\alpha-c\of}^{\beta+c\of}\rest_{\im~\tau_{\alpha-c\of}}"]
  %                     \arrow[to=Ga, "n_{\alpha-c\of}\rest_{\im~\tau_{\alpha-c\of}}"]
  %   & & & |[alias=Fb]|
  %     \im~\tau_{\beta+c\of} \\
  %   % & Ga & Gb &
  %   & |[alias=Ga]|
  %   \im~\sigma_\alpha \arrow[to=Gb, "g_\alpha^\beta"]
  %   & |[alias=Gb]|
  %     \im~\rho_\beta \arrow[to=Fb, "\nu_\beta"] &
  % \end{tikzcd}\end{equation}
  % % \end{minipage} \begin{minipage}{0.45\textwidth}
  % \begin{equation}\label{dgm:intr_tight2b}\begin{tikzcd}
  %   % Fa & Fb &
  %   \im~\tau_{\alpha-c\of}  \arrow[to=Fb, "b_{\alpha-c\of}^{\beta-c\of}\rest_{\im~\tau_{\alpha-c\of}}"]
  %                     \arrow[to=Ga, "\mu_{\alpha-c\of}"]
  %   & |[alias=Fb]|
  %     \im~\tau_{\beta-c\of} \arrow[to=Gb, "n_{\beta -c\of}\rest_{\im~\tau_{\beta-c\of}}"] \\
  %   % & Ga & Gb
  %   & |[alias=Ga]|
  %   \im~\sigma_\alpha \arrow[to=Gb, "s_\alpha^\beta"]
  %   & |[alias=Gb]|
  %     \im~\sigma_\beta
  % \end{tikzcd}\end{equation}
  % \end{minipage}
  % % \end{subequations}
  % % \vspace{3ex}
  % \begin{minipage}{0.45\textwidth}
  % \begin{equation}\label{dgm:intr_tight2c}\begin{tikzcd}
  %   % Fa & & & Fb
  %   & |[alias=Fa]|
  %   \im~\tau_\alpha  \arrow[to=Fb, "t_\alpha^\beta"]
  %   & |[alias=Fb]|
  %     \im~\tau_\beta  \arrow[to=Gb, "\mu_\beta"] & \\
  %   % & Ga & Gb &
  %   \im~\rho_{\alpha-c\of}  \arrow[to=Gb, "q_{\alpha-c\of}^{\beta+c\of}"]
  %                     \arrow[to=Fa, "\nu_{\alpha-c\of}"]
  %   & & & |[alias=Gb]|
  %     \im~\sigma_{\beta + c\of}
  % \end{tikzcd}\end{equation}
  % \begin{equation}\label{dgm:intr_tight2d}\begin{tikzcd}
  %   % & Fa & Fb
  %   & |[alias=Fa]|
  %   \im~\tau_\alpha  \arrow[to=Fb, "t_\alpha^\beta"]
  %   & |[alias=Fb]|
  %     \im~\tau_\beta\\
  %   % Ga & Gb &
  %   \im~\rho_{\alpha-c\of}  \arrow[to=Gb, "p_{\alpha-c\of}^{\beta-c\of}"]
  %                     \arrow[to=Fa, "\nu_{\alpha-c\of}"]
  %   & |[alias=Gb]|
  %     \im~\rho_{\beta - c\of} \arrow[to=Fb, "\nu_{\beta-c\of}"]&
  % \end{tikzcd}\end{equation}
  % \end{minipage}
  % \end{subequations}
  % \end{scriptsize}
  % \vspace{3ex}

  % Diagram~\ref{dgm:intr_tight1a} is well defined for $\alpha\geq\omega$, Diagrams~\ref{dgm:intr_tight1c} and~\ref{dgm:intr_tight2a} for $\alpha\geq\omega+c\of$, and Diagrams~\ref{dgm:intr_tight1b},~\ref{dgm:intr_tight1d}, and~\ref{dgm:intr_tight2b} are well defined for $\alpha\geq\omega+2c\of$.
  % In order to construct Diagrams which commute for all $\alpha$ we introduce the following utility function.
  %
  % Let
  % \[ b_\alpha^\beta : \hom_k(\ext{\P_\alpha},\ext{\QQ^\of})\to \hom_k(\ext{\P_\beta}, \ext{\QQ^\of}),\]
  % \[ g_\alpha^\beta : \hom_k(B_\alpha,\B)\to\hom_k(B_\beta, \B),\]
  % and
  % \[ h_\alpha^\beta : \hom_k(B_\alpha,\BB)\to\hom_k(B_\beta, \BB)\]
  % be induced by inclusion for all $\beta\geq\alpha\in (\omega+c\delta, \omega+2c\delta)$.
  % Note that, because our filtrations are image compatible, the restrictions $b_\alpha^\beta\rest_{\im~\tau_\alpha}$, $g_\alpha^\beta\rest_{\im~\rho_\alpha}$, and $h_\alpha^\beta\rest_{\im~\sigma_\alpha}$ are well defined.
  %
  % % % Let
  % % % \[ \nu_\alpha := \begin{cases} v_\alpha\rest_{\im~\rho_\alpha}&\text{ if } \alpha\geq \omega+c\of\\ 0&\text{ otherwise.}\end{cases}\]
  % %
  % % Because $\sigma_\alpha$ is an isomorphism and $\rho_\alpha$ is surjective for all $\alpha\geq\omega+2c\of$ we also define
  % % \[ r_\alpha^\beta := g_\alpha^\beta\circ\sigma_\alpha^{-1} : \im~\sigma_\alpha\to \im~\rho_\beta \]
  % % for all $\alpha\geq\omega+2c\of$ as $\beta\geq\alpha$ implies $\rho_\beta$ is surjective, so $\im~\rho_\beta = \hom_k(B_\beta, \B)$.
  % %
  % % % Noting that $\rho_\alpha$ is surjective for all $\alpha\in\R$ and $\sigma_\alpha$ is an isomorphism for all $\alpha\geq\omega+2c\of$ we observe that $g_\alpha^\beta$ is a map from
  % % % $\hom_k(B_\alpha, \B)\cong \im~\sigma_\alpha = \hom_k(B_\alpha, \BB)$ for $\alpha\geq\omega+2c\of$ to $\hom_k(B_\beta, \B)= \im~\rho_\beta$.
  % % % We also define
  % % % \[ \xi_\alpha : \hom_k(\ext{\P_\alpha}, \ext{\Q^\of})\to \hom_k(\P_\alpha, \Q^\of),\ \xi_\alpha' : \hom_k(\ext{\P_\alpha}, \ext{\QQ^\of})\to \hom_k(\P_\alpha, \QQ^\of)\]
  % % % to be the isomorphisms give by excision, with inverses induced by inclusion.
  % % % \[ \zeta_\alpha : \hom_k(\P_\alpha, \Q^\of)\to \hom_k(\ext{\P_\alpha}, \ext{\Q^\of}),\  \zeta_\alpha' : \hom_k(\P_\alpha, \QQ^\of)\to \hom_k(\ext{\P_\alpha}, \ext{\QQ^\of})\]
  % % % induced by inclusion.
  %
  % Let $\Phi_\alpha := \im~\tau_\alpha$ and
  %
  % We would like to show the following diagrams commute.

  \vspace{3ex}
  \begin{scriptsize}
  \begin{subequations}
  \begin{minipage}{0.45\textwidth}
  \begin{equation}\label{dgm:intr_tight2a}\begin{tikzcd}[column sep=scriptsize]
    % Fa & & & Fb
    \im~\tau_{\alpha-c\of}  \arrow[to=Fb, "b_{\alpha-c\of}^{\beta+c\of}\rest_{\im~\tau_{\alpha-c\of}}"]
                      \arrow[to=Ga, "n_{\alpha-c\of}\rest_{\im~\tau_{\alpha-c\of}}"]
    & & & |[alias=Fb]|
      \im~\tau_{\beta+c\of} \\
    % & Ga & Gb &
    & |[alias=Ga]|
    \im~\sigma_\alpha \arrow[to=Gb, "g_\alpha^\beta"]
    & |[alias=Gb]|
      \im~\rho_\beta \arrow[to=Fb, "\nu_\beta"] &
  \end{tikzcd}\end{equation}
  % \end{minipage} \begin{minipage}{0.45\textwidth}
  \begin{equation}\label{dgm:intr_tight2b}\begin{tikzcd}
    % Fa & Fb &
    \im~\tau_{\alpha-c\of}  \arrow[to=Fb, "b_{\alpha-c\of}^{\beta-c\of}\rest_{\im~\tau_{\alpha-c\of}}"]
                      \arrow[to=Ga, "\mu_{\alpha-c\of}"]
    & |[alias=Fb]|
      \im~\tau_{\beta-c\of} \arrow[to=Gb, "n_{\beta -c\of}\rest_{\im~\tau_{\beta-c\of}}"] \\
    % & Ga & Gb
    & |[alias=Ga]|
    \im~\sigma_\alpha \arrow[to=Gb, "s_\alpha^\beta"]
    & |[alias=Gb]|
      \im~\sigma_\beta
  \end{tikzcd}\end{equation}
  \end{minipage}
  % \end{subequations}
  % \vspace{3ex}
  \begin{minipage}{0.45\textwidth}
  \begin{equation}\label{dgm:intr_tight2c}\begin{tikzcd}
    % Fa & & & Fb
    & |[alias=Fa]|
    \im~\tau_\alpha  \arrow[to=Fb, "t_\alpha^\beta"]
    & |[alias=Fb]|
      \im~\tau_\beta  \arrow[to=Gb, "\mu_\beta"] & \\
    % & Ga & Gb &
    \im~\rho_{\alpha-c\of}  \arrow[to=Gb, "q_{\alpha-c\of}^{\beta+c\of}"]
                      \arrow[to=Fa, "\nu_{\alpha-c\of}"]
    & & & |[alias=Gb]|
      \im~\sigma_{\beta + c\of}
  \end{tikzcd}\end{equation}
  \begin{equation}\label{dgm:intr_tight2d}\begin{tikzcd}
    % & Fa & Fb
    & |[alias=Fa]|
    \im~\tau_\alpha  \arrow[to=Fb, "t_\alpha^\beta"]
    & |[alias=Fb]|
      \im~\tau_\beta\\
    % Ga & Gb &
    \im~\rho_{\alpha-c\of}  \arrow[to=Gb, "p_{\alpha-c\of}^{\beta-c\of}"]
                      \arrow[to=Fa, "\nu_{\alpha-c\of}"]
    & |[alias=Gb]|
      \im~\rho_{\beta - c\of} \arrow[to=Fb, "\nu_{\beta-c\of}"]&
  \end{tikzcd}\end{equation}
  \end{minipage}
  \end{subequations}
  \end{scriptsize}
  \vspace{3ex}

  \begin{enumerate}[label=\Roman*.]
    \item Note that $\im~\tau_{\alpha-c\of} = \im~\hom_k((\P_{\alpha-c\of}, \Q^\of)\to (\P_{\alpha-c\of}, \QQ^\of))$ is trivial for $\alpha\leq \omega+2c\of$, as
    $\P_{\alpha-c\of}\subseteq \QQ^\of$ implies $\hom_k(\P_{\alpha-c\of}, \QQ^\of)$ is trivial and $\im~\tau_{\alpha-c\of}$ is a subspace of $\hom_k(\P_{\alpha-c\of}, \QQ^\of)$.
    Therefore $t_{\alpha-c\of}^{\beta+c\of}$ is the zero map so Diagram~\ref{dgm:intr_tight2a} commutes.
    \item Once again, $t_{\alpha-c\of}^{\beta-c\of}$ is the zero map for all $\alpha\leq\omega+2c\delta$ as $\im~\tau_{\alpha-c\of}$ is a subspace of $\hom_k(\P_{\alpha-c\of}, \QQ^\of)$ which is trivial for all $\alpha\leq\omega+2c\delta$.
    So Diagram~\ref{dgm:intr_tight2b} commutes.
    \item We first note that $\im~\rho_{\alpha-c\of} = \im~\hom_k((B_{\alpha-c\of}, \b)\to (B_{\alpha-c\of, \B}))$ is trivial for all $\alpha\leq \omega+c\of$.
    So it suffices to assume that $\alpha\in (\omega+c\of, \omega+2c\of)$.
    So for all $\beta\geq\alpha$ we have $\beta+c\of\geq\omega+2c\of$.
    So $q_{\alpha-c\of}^{\beta+c\of}$ factors through $\hom_k(\B, \B)$ as $\B\subseteq B_{\beta+c\of}$ for all $\beta\geq\alpha$.
    Because $\hom_k(\B, \B)$ is trivial $q_{\alpha-c\of}^{\beta+c\of}$ is the zero map, so Diagram~\ref{dgm:intr_tight2c} commutes.
    \item Once again, $\im~\rho_{\alpha-c\of}$ is trivial for $\alpha\leq\omega+c\of$ so it suffices to assume $\alpha\in (\omega+c\of, \omega+2c\of)$.
    Suppose there exists a homology class $[x]\in\hom_k(B_{\alpha-c\of}, \b)$ such that $\rho_{\alpha-c\of}[x]\neq 0$ and $p_{\alpha-c\of}^{\beta-c\of}\circ\rho_{\alpha-c\of}[x]\neq 0$.
    Because Diagram~\ref{dgm:intr_tight1a} commutes it follows that
    \[\rho_\beta\circ f_{\alpha-c\of}^{\beta-c\of} = g_{\alpha-c\of}^{\beta-c\of}\circ\rho_{\alpha-c\of} = p_{\alpha-c\of}^{\beta-c\of}\circ\rho_{\alpha-c\of}\]
    so $f_{\alpha - c\of}^{\beta-c\of}[x]\neq 0$.

    If $[x]\in\ker~u_{\alpha-c\of}$ then $f_{\alpha - c\of}^{\beta-c\of}[x]\in\ker~u_{\beta-c\of}$.
    Otherwise, if $u_{\alpha-c\of}[x]\neq 0$ we can observe that $b_\alpha^\beta\circ\tau_\alpha$ factors through $\hom_k(\P_\alpha, \QQ^\of)$ as Diagram~\ref{dgm:intr_tight1e} commutes.
    That is, if $\rho_{\alpha -c\of}[x]\in\ker~\nu_{\alpha-c\of}$ then $p_{\alpha-c\of}^{\beta-c\of}\circ\rho_{\alpha-c\of}[x]\in\ker~\nu_{\beta-c\of}$ for any $[x]\in \hom_k(B_{\alpha-c\of}, \b)$.
    It follows that Diagram~\ref{dgm:intr_tight2d} commutes.
  \end{enumerate}
\end{proof}

\begin{corollary}\label{cor:geo_inter}
  If $\eta^k : \hom_k(\b)\to \hom_k(\BB)$ is surjective and $\im~\eta^k\cong \hom_k(\bb)$ then the $k$th persistent homology modules of
  $\{(B_\alpha, \bb)\}_{\alpha\geq\oo}$ are $c\of$-interleaved with that of
  \[\{(P_\alpha^\of, \Q^\of)\to (P_\alpha^\of, \QQ^\of)\}_{\alpha\geq\oo}\]
  for $k > 0$.
\end{corollary}
