% !TeX root = ../new.tex

% Let $\D\subset\R^d$ and $f:\D\to\R$ be a $c$-Lipschitz function satisfying the following assumptions for some $\omega\in\R$, $\delta > 0$ and $\gamma > 3\delta$.
Let $\D\subset \R^d$ be a compact subset of $\R^d$ and $f:\D\to\R$ be a $c$-Lipschitz function on $\D$.
The requirement that $\D$ is a strict subset is in order to ensure that any set $B$ surrounding $\D$ separates $\R^d$ with \emph{non-empty} parts $(\D\setminus B, \R^d\setminus\D)$.
In the case of $\D = \R^d$ we can instead view $\D$ as a proper subset of the sphere $S^d\cong \R^d\cup\{\infty\}$.
Let $P\subset \D$ be a finite collection of points which will serve as the (unknown) locations of a collection of sensors with the following capabilities.

\vspace{3ex}
\begin{center}
\setlength{\fboxsep}{2ex}
\fbox{\parbox{\textwidth}{
\begin{small}
\textbf{Sensor Capabilities}
    \begin{itemize}
        \item[a.]\textbf{(Communication Radii)} detect the presence, but not location or distance, of sensors within distances $\delta > 0$ and $\gamma \geq 3\delta$, and discriminate between sensors within each scale,
        % \item[b.]\textbf{(Coverage Radius)} cover a radially symmetric subset of the domain with radius $\delta$,
        \item[c.]\textbf{(Measurement)} measure the scalar value $f(p)$.
    \end{itemize}
\end{small}
}}\end{center}\vspace{3ex}

The original TCC~\cite{desilva07coverage} and work on scalar fields~\cite{chazal09analysis} make strict assumptions about the geometry of the domain $\D$ and its boundary $\B$.
We re-interpreted the TCC by replacing the assumptions about the geometry of the domain with assumptions about its persistent homology~\cite{cavanna2017when}.
This amounted to a requirement that the zero-dimensional homology of the domain did not change as we shrunk the domain within a certain range.

In this work we would like to extend this idea from the zero-dimensional homology of a geometric domain to the homology of all dimensions of a domain defined as a sublevel-set of a function.
We can then re-interpret the geometric assumptions made in the work on scalar fields in a similar way which re-purposes the machinery used to confirm coverage.

Throughout, let $D_a = f^{-1}[a, \infty)$ and $B_a = f^{-1}(-\infty, a]$  denote the superlevel and sublevel sets of $f$.
We will make the following assumptions about the sublevel-sets of $f$ with respect to constants $\omega\in\R$ and $\gamma\geq \delta > 0$, as well as $2\delta\geq\e\leq \gamma - \delta$.

\vspace{2ex}
\begin{center}
\setlength{\fboxsep}{2ex}
\fbox{\parbox{\textwidth}{
% \vspace{1ex}\hspace{1ex}
\textbf{Geometric Assumptions}
\begin{small}
    \begin{enumerate}
    \setcounter{enumi}{0}
        \item\textbf{(Surrounds)} $B_{\omega-c\e}$ and $B_{\omega + c(\gamma-\delta)}$ are nonempty, compact subsets that surround $\D$ in $\R^d$.
        % \item \textbf{(Domain)} $(D_0, B_0)$ and $(D_1, B_1)$ are surrounding pairs of nonempty, compact subsets of $\R^d$ with $(D_0^{\delta+\gamma}, B_0^{\delta+\gamma}) \subset (D_1, B_1)$.
        \item \textbf{(Surjective Components)} $\hom_0(\D\setminus B_{\omega + c(\gamma-\delta)} \hookrightarrow \D\setminus B_{\omega-\e}^{2\delta})$ is surjective.
        \item \textbf{(Sufficiently Large $\gamma$)} $B_\omega\subseteq B_{\omega-\e}^{\gamma-\delta}\cap\D$.
        % \item \textbf{(Interleaving)} $\im~\hom_k((D_0, B_0)\hookrightarrow (D_1, B_1))\cong \hom_k(D_0^{2\delta}, B_0^{2\delta})$.
    \end{enumerate}
\end{small}
}}\end{center}\vspace{4ex}

% \vspace{2ex}
% \begin{center}
% \setlength{\fboxsep}{2ex}
% \fbox{\parbox{\textwidth}{
% % \vspace{1ex}\hspace{1ex}
% \textbf{Geometric Assumptions}
% \begin{small}
%     \begin{enumerate}
%     \setcounter{enumi}{0}
%         % \item \textbf{(Domain)} $\D$ is a bounded, compact subset of $\R^d$ and $\D_{\omega-2c\delta}$ is closed and surrounds $\D$.
%         \item \textbf{(Domain)} $(D_0, B_0)$ and $(D_1, B_1)$ are surrounding pairs of nonempty, compact subsets of $\R^d$ with $(D_0^{\delta+\gamma}, B_0^{\delta+\gamma}) \subset (D_1, B_1)$.
%         \item \textbf{(Boundary)} $\im~\hom_k(\D_{\omega-2c\delta}\hookrightarrow \D_{\omega+{2c\delta}})$ is isomorphic to $\hom_k(\D_{\omega-2c\delta}^{2\delta})$ for all $k\in\N$.
%         \item \textbf{(Retraction)} $\D_{\omega-c\delta}^\delta\to\D_{\omega-2c\delta}^{2\delta}$ is a deformation retraction.
%         \item \textbf{(Inclusion)} $\D_{\omega - 2c\delta}\subseteq \D_{\omega - 2c\delta}^{2\delta} \subseteq \D_{\omega - 2c\delta}^{\delta + \gamma} \subseteq \D_{\omega + 2c\delta}.$
%     \end{enumerate}
% \end{small}
% }}\end{center}\vspace{4ex}

In the following we restrict $P$ to $D_{\omega-c\e}$ and let $Q = P\cap B_{\omega - c(\e - \delta)}$.
For ease of notation let $B_0 = B_{\omega - c\e}$, $B_1 = B_{\omega + c(\gamma-\delta)}$ and $\B = B_{\omega-c\e}^{\gamma-\delta}\cap\D$.

% Assumption 1 replaces the requirement that $\B$ is the topological boundary of the domain $\D$.
% As we did in previous work the notion of a topological boundary is replaced with that of a subset which surrounds the domain.
% In addition to this we define this surrounding subset in terms of a scalar value $\omega$ which gives us a sublevel set that will serve as our boundary $\B$.
% In doing so we can state the following assumptions in terms of the persistent homology of the function itself instead of the persistent homology of a geometric domain.
%
% Assumption 2 is our main restriction on the persistent homology of the surrounding sub-level set and limits the size of features in terms of the offset parameters $\delta,\gamma$.
% We note that this assumption is stronger than those made in our initial work which only required that the map $\hom_0(\D\setminus\D_{\omega+2c\delta}\hookrightarrow\D\setminus\B^{2\delta})$ be surjective which is not sufficient for the analysis of scalar fields.
%
% Assumption 3 requires that the region containing the sampled boundary $Q^\delta$ resembles the extended boundary $\B^{2\delta}$.
% While this assumption would not be necessary if $Q$ was defined as the set of points within distance $\delta$ of $\B$ we are particularily interested in the case in which the points cannot detect any features of the domain, such as the presence of a boundary, and can only measure some scalar value.
%
% Finally, assumption 4 is our only additional restriction on the choice of the parameter $\gamma\geq 3\delta$ so that it does not extend beyond the sub-level set $\D_{\omega+2c\delta}$.
