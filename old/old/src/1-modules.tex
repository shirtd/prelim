% !TeX root = ../new.tex

\begin{definition}[Persistence Module]
  A \textbf{persistence module} $\S$ over $\R$ is an indexed family of vector spaces $\{S_\alpha\}$ and linear maps $\{s_\alpha^\beta : S_\alpha\to S_\beta\}$ such that $s^\gamma_\beta\circ s_\alpha^\beta = s_\alpha^\gamma$ whenever $\alpha\leq\beta\leq\gamma$ and $s_\alpha^\alpha$ is the identity on $S_\alpha$.
\end{definition}

For a persistence module $\S$ we will also use $S[\alpha]$ to denote the vector space $S_\alpha$ and $s[\alpha,\beta]$ to denote the linear maps $s_\alpha^\beta$ when additional notation is required.

\begin{definition}[Persistence Module Homomorphism]
  A \textbf{homomorphism} $\Lambda$ between two $\R$-persistence modules $\S, \T$ is a collection of linear maps $\{\lambda_\alpha : S_\alpha\to T_\alpha\}$ such that the following diagram commutes for all $\alpha\leq\beta$.
  \begin{equation}\label{dgm:homomorphism}
    \begin{tikzcd}
      S_\alpha\arrow{r}{s_\alpha^\beta}\arrow{d}{\lambda_\alpha} &
      S_\beta\arrow{d}{\lambda_\beta}\\
      %
      T_\alpha\arrow{r}{t_\alpha^\beta} &
      T_\beta
  \end{tikzcd}\end{equation}
  The space of homomorphisms from $\S$ to $\T$ will be denoted $\Hom(\S, \T)$.
\end{definition}

\begin{definition}[Shifted Homomorphism]
  A \textbf{homomorphism of degree $\delta$} is a collection $F$ of linear maps $f_\alpha : U_\alpha\to S_{\alpha+\delta}$ such that the following diagram commutes for all $\alpha\leq\beta$.

  \begin{equation}\label{dgm:shifted_homomorphism}
    \begin{tikzcd}
      U_\alpha\arrow{r}{u_\alpha^\beta}\arrow{d}{f_\alpha} &
      U_\beta\arrow{d}{f_\beta}\\
      %
      S_{\alpha+\delta}\arrow{r}{s_{\alpha+\delta}^{\beta+\delta}} &
      S_{\beta +\delta}
  \end{tikzcd}\end{equation}
  The space of homomorphisms of degree $\delta$ from $\U$ to $\S$ will be denoted $\Hom^\delta(\U, \S)$.
\end{definition}

Noting that $\Hom^\delta(\U,\V)\subseteq\Hom^{\delta'}(\U,\V)$ for all $0\leq\delta\leq\delta'$ we will define particular shifted homomorphisms with the assumption that $\Hom^\delta(\U,\V) = \Hom(\U,\V)$ for $\delta = 0$.
For $\Gamma\in\Hom(\U,\V)$ let $\Gamma[\delta]\in\Hom^\delta(\U,\V)$ denote the homomorphism of degree $\delta$ defined as the family of linear maps
\[\{\gamma_\alpha[\delta] := v_\alpha^{\alpha+\delta}\circ \gamma_\alpha : U_\alpha\to V_{\alpha+\delta}\}.\]

Using our alternative notation we will write
\[\gamma[\alpha;\delta] := v[\alpha,\alpha+\delta]\circ\gamma[\alpha] : U[\alpha]\to V[\alpha+\delta]\]
to denote a map $\gamma_\alpha[\delta]$ of $\Gamma[\delta]$.

\begin{lemma}\label{lem:trans_shift}
  If $F\in \Hom^\delta(\U,\S)$ and $F'\in\Hom^{\delta'}(\S,\S')$ then $F'\circ F\in\Hom^{\delta+\delta'}(\U,\S')$.
\end{lemma}
\begin{proof}
  Because $F'\in \Hom^{\delta'}(\S,\S')$ we have $f_{\beta}'\circ s_\alpha^\beta = {s_{\alpha+\delta'}^{\beta+\delta'}}'\circ f_{\alpha+\delta'}'$ for all $\alpha\leq\beta$.
  Because $F\in\Hom^\delta(\U,\S)$ we have $f_{\beta-\delta}\circ u_{\alpha-\delta}^{\beta-\delta} = s_\alpha^\beta\circ f_{\alpha-\delta}$ for all $\alpha\leq\beta$.
  So
  \begin{align*}
    f_{\beta}'\circ (f_{\beta-\delta}\circ u_{\alpha-\delta}^{\beta-\delta})
      &= (f_{\beta}'\circ s_\alpha^\beta)\circ f_{\alpha-\delta}\\
      &= {s_{\alpha+\delta'}^{\beta+\delta'}}'\circ f_{\alpha+\delta'}'\circ f_{\alpha-\delta}
  \end{align*}
  so $F'\circ F\in\Hom^{\delta+\delta'}(\U,\S')$.
\end{proof}

\begin{definition}[Interleaving]
  Two persistence modules $\U$ and $\S$ are \textbf{$\delta$-interleaved} if there exist homomorphisms $F\in\Hom^\delta(\U, \S)$ and $G \in\Hom^\delta(\S,\U)$ such that the following diagrams commute for all $\alpha$.

  \begin{minipage}{0.45\textwidth}
  \begin{equation}\label{dgm:interleaving1}
    \begin{tikzcd}
      U_{\alpha-\delta}\arrow{rr}{u_{\alpha-\delta}^{\alpha+\delta}}\arrow{dr}{f_{\alpha-\delta}} & &
      U_{\alpha+\delta}\\
      %
      & S_{\alpha}\arrow{ur}{g_\alpha} &
  \end{tikzcd}\end{equation}
  \end{minipage}
  \begin{minipage}{0.45\textwidth}
  \begin{equation}\label{dgm:interleaving2}
    \begin{tikzcd}
      & U_{\alpha}\arrow{dr}{f_\alpha} &\\
      %
      S_{\alpha-\delta}\arrow{rr}{s_{\alpha-\delta}^{\alpha+\delta}}\arrow{ur}{g_{\alpha-\delta}} & &
      S_{\alpha+\delta}
  \end{tikzcd}\end{equation}
  \end{minipage}
\end{definition}

\subsection{Weak Interleavings}

\begin{definition}[Weak Interleaving]
  For $I\in\Hom^{2\delta}(\U,\V)$ a pair $(F, M)\in \Hom^\delta(\U,\S)\times\Hom^\delta(\S,\V)$ is a \textbf{weak $\delta$-interleaving} of $I$ with $\S$ if $I = M\circ F$.
  If $J\in\Hom^{\delta'}(\S,\T)$ and $(F,N)\in\Hom^\delta(\U,\S)\times\Hom^\delta(\T,\V)$ is a weak $\delta$-interleaving of $I$ with $J$ if $I = N\circ J\circ F$.
\end{definition}

\begin{lemma}\label{lem:left}
  Let $I\in\Hom^{4\delta}(\U,\V)$, $J\in\Hom^{2\delta}(\S,\T)$ and suppose $(F, N)\in\Hom^{\delta}(\U,\S)\times\Hom^\delta(\T,\V)$ is a weak $\delta$-interleaving of $I$ with $J$.

  If $(F', M')\in\Hom^\delta(\S,\S')\times\Hom^\delta(\S',\T)$ is a weak $\delta$-interleaving of $J$ with $\S'$ then
  \[(F'\circ F, N\circ M')\in\Hom^{2\delta}(\U,\S')\times \Hom^{2\delta}(\S',\V)\]
  is a weak $2\delta$-interleaving of $I$ with $\S'$.
\end{lemma}
\begin{proof}
  By Lemma~\ref{lem:trans_shift} we have $F'\circ F\in\Hom^{2\delta}(\U,\S')$ and $N\circ M'\in \Hom^{2\delta}(\S',\V)$.
  If $(F', M')$ is a weak $\delta$-interleaving of $J$ with $\S'$ then $J = M'\circ F'$.
  By our hypothesis that $(F, N)$ is a weak $\delta$-interleaving of $I$ with $J$
  \[ I = N\circ J\circ F = (N\circ M')\circ (F'\circ F).\]
  We may therefore conclude that $(F'\circ F, N\circ M')\in\Hom^{2\delta}(\U,\S')\times \Hom^{2\delta}(\S',\V)$ is a weak $2\delta$-interleaving of $I$ with $\S'$.
\end{proof}

\begin{lemma}\label{lem:right}
  Let $I\in\Hom^{2\delta}(\U,\V)$, $I'\in\Hom^{4\delta}(\U',\V')$ and suppose $(F, M)\in\Hom^\delta(\U,\S)\times\Hom^\delta(\S,\V)$ is a weak $\delta$-interleaving of $I$ with $\S$.

  If $(F',N')\in\Hom^\delta(\U',\U)\times\Hom^\delta(\V,\V')$ is a weak $\delta$-interleaving of $I'$ with $I$ then
  \[(F\circ F', N'\circ M)\in\Hom^{2\delta}(\U',\S)\times\Hom^{2\delta}(\S,\V')\]
  is a weak $2\delta$-interleaving of $I'$ with $\S$.
\end{lemma}
\begin{proof}
  By Lemma~\ref{lem:trans_shift} we have $F\circ F'\in \Hom^{2\delta}(\U',\S)$ and $N'\circ M\in \Hom^{2\delta}(\S,\V')$.
  By our hypothesis that $(F, M)$ is a weak $\delta$-interleaving of $I$ with $\S$ we know $I = M\circ F$.
  If $(F',N')$ is a weak $\delta$-interleaving of $I'$ with $I$ then $I' = N'\circ I\circ F'$.
  \[ I' = N'\circ I\circ F' = (N'\circ M)\circ (F\circ F').\]
  We may therefore conclude that $(F\circ F', N'\circ M)\in\Hom^{2\delta}(\U',\S)\times\Hom^{2\delta}(\S,\V')$ is a weak $2\delta$-interleaving of $I'$ with $\S$.
\end{proof}

% A homomorphism $I\in\Hom^{2\delta}(\U,\V)$ of degree (at most) $2\delta$ are said to be \textbf{partially $\delta$-interleaved} with a persistence module $\S$ if there exist $F\in\Hom^\delta(\U,\S)$ and $M\in\Hom^\delta(\S,\V)$ such that the following diagram commutes for all $\alpha$.
%
% \begin{equation}\label{dgm:left_interleaving}
%   \begin{tikzcd}
%     U_{\alpha-\delta}\arrow{rr}{i_{\alpha-\delta}}\arrow{dr}{f_{\alpha-\delta}} & &
%     V_{\alpha+\delta}\\
%     %
%     & S_{\alpha}\arrow{ur}{m_\alpha} &
% \end{tikzcd}\end{equation}
% We say that $I$ is partially $\delta$-interleaved by the pair $(F, M)$.
%
% % The homomorphisms are said to be \textbf{right $delta$-interleaved} if there exist $M\in\Hom^\delta(\S,\V)$ and $G\in\Hom^\delta(\V,\T)$ such that the following diagram commutes for all $\alpha$.
% %
% % \begin{equation}\label{dgm:right_interleaving}
% %   \begin{tikzcd}
% %     & V_{\alpha}\arrow{dr}{g_\alpha} &\\
% %     %
% %     S_{\alpha-\delta}\arrow{rr}{j}\arrow{ur}{m_{\alpha-\delta}} & &
% %     T_{\alpha+\delta}
% % \end{tikzcd}\end{equation}
% % $I$ and $J$ are \textbf{weakly $\delta$-interleaved} if they are both left and right $\delta$-interleaved.
%
%
% % In the following let $I\in\Hom^{2\delta}(\U, \V)$, $J\in\Hom^{2\delta}(\S,\T)$, $\tilde{J}\in\Hom^{2\delta}(\tilde{\S},\tilde{\T})$, $J'\in\Hom^{2\delta}(\S',\T')$ be homomorphisms of degree (at most) $2\delta$.
%
% \begin{lemma}\label{lem:left}
%   Suppose $I\in\Hom^{2\delta}(\U,\V)$ is partially $\delta$-interleaved with $\S$ by a pair $(F, M)$ and $\S'$ by a pair $(F',M')$.
%
%   If there exist $\Sigma\in\Hom(\S, \tilde{S})$ and $\Upsilon\in\Hom(\tilde{\S},\S')$ such that $M = M'\circ\Upsilon\circ\Sigma$ and $F' = \Upsilon\circ\Sigma\circ F$ then $I$ is partially $\delta$-interleaved with $\tilde{S}$ by the pair $(\Sigma\circ F, M'\circ\Upsilon)$.
% \end{lemma}
% \begin{proof}
%   If $I$ is partially $\delta$-interleaved with $\S$ by a pair $(F, M)$ then $i_{\alpha-\delta} = m_\alpha\circ f_{\alpha-\delta}$ for all $\alpha$.
%   So, by our hypothesis that $M = M'\circ\Upsilon\circ\Sigma$ and $F' = \Upsilon\circ\Sigma\circ F$,
%   \begin{align*}
%     i_{\alpha-\delta} &= m_\alpha\circ f_{\alpha-\delta}\\
%       &= (m_\alpha'\circ \upsilon_\alpha)\circ (\sigma_\alpha\circ f_{\alpha-\delta}).
%   \end{align*}
%   as desired.
% \end{proof}
%
% \begin{lemma}\label{lem:right}
%   Suppose $J\in\Hom^{2\delta}(\S,\T)$ is partially $\delta$-interleaved with $\V$ by the pair $(M, G)$.
%
%   If there exist $\Upsilon\in\Hom(\tilde{\S},\S)$ and $\Theta\in\Hom(\T, \tilde{\T})$ such that $\tilde{J} := \Theta\circ J\circ \Upsilon\in\Hom^{2\delta}(\tilde{\S},\tilde{\T})$ then $\tilde{J}$ is $\delta$-interleaved with $\V$ by the pair $(M\circ \Upsilon, \Theta\circ G)$.
% \end{lemma}
% \begin{proof}
%   Because $J$ is partially $\delta$-interleaved with $\V$ we know $j_{\alpha-\delta} = g_\alpha\circ m_{\alpha-\delta}$.
%   By our hypothesis that $\tilde{J} = \Theta\circ J\circ \Upsilon$ is a homomorphism of degree $2\delta$ we have
%   \begin{align*}
%     \tilde{j}_{\alpha-\delta} &= \vartheta_{\alpha+\delta}\circ j_{\alpha-\delta}\circ \upsilon_{\alpha-\delta}\\
%       &= (\vartheta_{\alpha+\delta}\circ g_\alpha)\circ (m_{\alpha-\delta}\circ \upsilon_{\alpha-\delta})
%     \end{align*}
%     for all $\alpha$ as desired.
% \end{proof}
%
% % \begin{lemma}\label{lem:left}
% %   Suppose $I\in\Hom^{2\delta}(\U, \V)$ is left $\delta$-interleaved with $J\in\Hom^{2\delta}(\S,\T)$ and $J'\in\Hom^{2\delta}(\S',\T')$ by pairs $(F, M)$ and $(F', M')$, respectively.
% %
% %   If there exist $\Sigma\in\Hom(\S,\tilde{\S})$ and $\Upsilon\in\Hom(\tilde{\S}, \S')$ such that $M = M'\circ\Upsilon\circ\Sigma$ then $I$ and $\tilde{J}\in\Hom^{2\delta}(\tilde{\S},\tilde{\T})$ are left $\delta$-interleaved by the pair $(\Sigma\circ F, M'\circ\Upsilon)$.
% % \end{lemma}
% % \begin{proof}
% %   Because $I$ is left $\delta$-interleaved with $J$ Diagram~\ref{dgm:left_interleaving} commutes for all $\alpha$, so $i_{\alpha-\delta} = m_\alpha\circ f_{\alpha-\delta}$.
% %   By our hypothesis that $M = M'\circ\Upsilon\circ\Sigma$ it follows
% %   \[ i_{\alpha-\delta} = (m_\alpha'\circ\upsilon_\alpha)\circ(\sigma\circ f_{\alpha-\delta}) \]
% %   which implies that $I$ and $\tilde{J}$ are left $\delta$-interleaved by the pair $(\Sigma\circ F, M'\circ\Upsilon)$ as desired.
% %   % Similarly, because $\Gamma[2\delta]$ is left $\delta$-interleaved with $\Lambda'[2\delta]$ we have $\gamma[2\delta]_{\alpha-\delta}
% % \end{proof}
% %
% % \begin{lemma}\label{lem:right}
% %   Suppose $I\in\Hom^{2\delta}(\U, \V)$ is right $\delta$-interleaved with $L\in\Hom^{2\delta}(\S',\T)$ by a pair $(M', G)$.
% %
% %   If there exist $\Upsilon\in\Hom(\tilde{\S},\S')$ and $\Theta\in\Hom(\T,\tilde{\T})$ such that
% %   \[\tilde{j}_\alpha = \omega_{\alpha+2\delta}\circ \ell_\alpha\circ\upsilon_\alpha \]
% %   for all $\alpha$ then $I$ and $\tilde{J}\in\Hom^{2\delta}(\tilde{\S},\tilde{\T})$ are right $\delta$-interleaved by the pair $(M'\circ\Upsilon, \Theta\circ G)$.
% % \end{lemma}
% % \begin{proof}
% %   Because $I$ and $L$ are right $\delta$-interleaved by a pair $(M', G)$ Daigram~\ref{dgm:right_interleaving} commutes for all $\alpha$, so $\ell_{\alpha-\delta} = g_\alpha\circ m_{\alpha-\delta}'$.
% %   By our assumption that $\tilde{j}_\alpha = \vartheta_{\alpha+2\delta}\circ \ell_\alpha\circ\upsilon_\alpha$ it follows
% %   \[ \tilde{j}_\alpha = \vartheta_{\alpha+2\delta}\circ \ell_\alpha\circ\upsilon_\alpha = (\vartheta_{\alpha+\delta}\circ g_\alpha)\circ (m_{\alpha-\delta}\circ\upsilon_{\alpha-\delta}) \]
% %   which implies that $I$ and $\tilde{J}$ are right $\delta$-interleaved by the pair $(M'\circ \Upsilon, \Theta\circ G)$ as desired.
% % \end{proof}

\subsection{Image Persistence Modules}

\begin{definition}[Image Persistence Module]
  The \textbf{image persistence module} of a homomorphism $\Gamma\in\Hom(\U,\V)$ is the family of subspaces $\{\Gamma_\alpha :=\im~\gamma_\alpha\}$ in $\V$ along with linear maps $\{\gamma_\alpha^\beta := v_\alpha^\beta\rest_{\im~\gamma_\alpha} : \Gamma_\alpha\to\Gamma_\beta\}$ and will be denoted by $\im~\Gamma$.
\end{definition}

\begin{definition}[Image Module Homomorphism]
  Given $\Gamma\in\Hom(\U,\V)$ and $\Lambda\in\Hom(\S,\T)$ along with $(F,G)\in\Hom^\delta(\U,\S)\times\Hom^\delta(\V,\T)$ let $\Phi(F, G) : \im~\Gamma\to\im~\Lambda$ denote the family of linear maps $\{\phi_\alpha := g_\alpha\rest_{\Gamma_\alpha} : \Gamma_\alpha\to\Lambda_{\alpha+\delta}\}$.
  $\Phi(F, G)$ is an \textbf{image module homomorphism of degree $\delta$} if the following diagram commutes for all $\alpha\leq\beta$.

  \begin{equation}\label{dgm:image_homomorphism}
    \begin{tikzcd}[column sep=large]
      % U_\alpha\arrow{r}{v_\alpha^\beta\circ\gamma_\alpha}\arrow{d}{f_\alpha} &
        U_\alpha\arrow{r}{\gamma_\alpha[\beta-\alpha]}\arrow{d}{f_\alpha} &
      V_\beta\arrow{d}{g_\beta}\\
      %
      % S_{\alpha+\delta}\arrow{r}{t_{\alpha+\delta}^{\beta+\delta}\circ\lambda_{\alpha+\delta}} &
      S_{\alpha+\delta}\arrow{r}{\lambda_{\alpha+\delta}[\beta-\alpha]} &
      T_{\beta +\delta}
  \end{tikzcd}\end{equation}
  The space of image module homomorphisms of degree $\delta$ between $\im~\Gamma$ and $\im~\Lambda$ will be denoted $\Hom^\delta(\im~\Gamma,\im~\Lambda)$.
\end{definition}

Recall that $\gamma_\alpha[\beta-\alpha] = v_\alpha^\beta\circ\gamma_\alpha$ and $\lambda_\alpha[\beta-\alpha] = t_\alpha^\beta\circ\lambda_\alpha$.
Because $\im~\gamma_\alpha[\beta-\alpha] = \im~\gamma_\alpha^\beta$ and $\im~\lambda_{\alpha}[\beta-\alpha] = \im~\lambda_\alpha^\beta$ for all $\alpha\leq\beta$ the commutativity of Diagram~\ref{dgm:image_homomorphism} implies the following diagram of images commutes.
\begin{equation}\label{dgm:shifted_homomorphism}
  \begin{tikzcd}[column sep=large]
    \Gamma_\alpha\arrow{r}{\gamma_\alpha^\beta}\arrow{d}{\phi_\alpha} &
    \Gamma_\beta\arrow{d}{\phi_\beta}\\
    %
    \Lambda_{\alpha+\delta}\arrow{r}{\lambda_{\alpha+\delta}^{\beta+\delta}} &
    \Lambda_{\beta +\delta}
\end{tikzcd}\end{equation}

We note that whenever $\Gamma\in\Hom(\U,\V)$ and $\Lambda\in\Hom(\S,\T)$ the assumption that $\Phi(F, G)\in\Hom^\delta(\im~\Gamma, \im~\Lambda)$ implies that $F\in\Hom^\delta(\U,\S)$ and $G\in\Hom^\delta(\V,\T)$.

\begin{lemma}\label{lem:image_composition}
  Suppose $\Gamma\in\Hom(\U,\V)$, $\Lambda\in\Hom(\S,\T)$, and $\Lambda'\in\Hom(\S',\T')$.
  If $\Phi(F, G)\in\Hom^\delta(\im~\Gamma, \im~\Lambda)$ and $\Phi'(F', G')\in\Hom^{\delta'}(\im~\Lambda, \im~\Lambda')$ then $\Phi''(F'\circ F, G'\circ G)\in\Hom^{\delta+\delta'}(\im~\Gamma,\im~\Lambda')$.
\end{lemma}
\begin{proof}
  Because $\Phi(F, G)$ is an image module homomorphism of degree $\delta$
  \[ g_{\beta-\delta}\circ\gamma_{\alpha-\delta}[\beta-\alpha] = \lambda_\alpha[\beta-\alpha]\circ f_{\alpha-\delta}.\]
  Because $\Phi'(F', G')$ is an image module homomorphism of degree $\delta'$
  \[ g_{\beta}'\circ\lambda_{\alpha}[\beta-\alpha] = \lambda_{\alpha +\delta'}'[\beta-\alpha]\circ f_{\alpha}'.\]
  Therefore, for all $\alpha\leq\beta$,
  \begin{align*}
    g_\beta'\circ (g_{\beta-\delta}\circ \gamma_{\alpha-\delta}[\beta-\alpha]) &= (g_\beta'\circ \lambda_\alpha[\beta-\alpha])\circ f_{\alpha-\delta}\\
      &=\lambda_{\alpha+\delta'}[\beta-\alpha]\circ f_\alpha'\circ f_{\alpha-\delta}.
  \end{align*}
  So $\Phi''(F'\circ F, G'\circ G)\in\Hom^{\delta+\delta'}(\im~\Gamma,\im~\Lambda')$ as desired.
  % \[g_\beta\circ (v_\alpha^\beta\circ \gamma_\alpha) = (t_{\alpha+\delta}^{\beta+\delta}\circ\lambda_{\alpha+\delta})\circ f_\alpha\]
  % for all $\alpha$.
  % Because $\Psi(M, N)$ is an image module homomorphism of degree $\delta'$
  % \[n_\beta\circ (_\alpha^\beta\circ \gamma_\alpha) = (t_{\alpha+\delta}^{\beta+\delta}\circ\lambda_{\alpha+\delta})\circ f_\alpha\]

\end{proof}

\subsection{Partial Interleavings of Image Modules}

% Let $\Gamma[2\delta]\in\Hom^{2\delta}(\U,\V)$ be defined as the family of linear maps
% \[\{\gamma[2\delta]_\alpha := v_{\alpha}^{\alpha+2\delta}\circ\gamma_{\alpha} : U_\alpha\to V_{\alpha+2\delta}\}\]
% and let $\Lambda[2\delta]\in\Hom^{2\delta}(\S,\T)$ be defined as the family of linear maps
% \[\{\lambda[2\delta]_\alpha := t_{\alpha}^{\alpha+2\delta}\circ\lambda_\alpha : S_\alpha\to T_{\alpha+2\delta}\}.\]

% An image module homomorphism $\Phi(F,G)\in\Hom^\delta(\im~\Gamma,\im~\Lambda)$ of degree $\delta$ is said to be a \textbf{left $\delta$-interleaving of image modules} if there exists some $M\in\Hom^\delta(\S,\V)$ such that $\Gamma[2\delta]$ and $\Lambda[2\delta]$ are right $\delta$-interleaved by the pair $(F, M)$.
% If $\Gamma[2\delta]$ and $\Lambda[2\delta]$ are left $\delta$-interleaved by the pair $(M, G)$ then $\Phi(F, G)$ is said to be a \textbf{right $\delta$-interleaving of image modules}.

% An image module homomorphism $\Phi(F, G)\in\Hom^\delta(\im~\Gamma,\im~\Lambda)$ along with a map $M\in\Hom^\delta(\S,\V)$ is a \textbf{left $\delta$-interleaving of image modules} if $\Gamma[2\delta]$ is partially $\delta$-interleaved with $\S$ by the pair $(F, M)$.
% $\Phi(F, G)$ is a \textbf{right $\delta$-interleaving of image modules} if $\Lambda[2\delta]$ is partially $\delta$-interleaved with $\V$ by the pair $(M, G)$.
% $\Phi(F, G)$ is a \textbf{partial $\delta$-interleaving of image modules}, and denoted $\Phi_M(F, G)$ if it is is both a left and right $\delta$-interleaving of image modules.

\begin{definition}[Partial Interleaving of Image Modules]
  For $\Gamma\in\Hom(\U,\V)$ and $\Lambda\in\Hom(\S,\T)$ an image module homomorphism $\Phi(F, G)\in\Hom^\delta(\im~\Gamma,\im~\Lambda)$ along with $M\in\Hom^\delta(\S,\V)$ is a \textbf{left $\delta$-interleaving of image modules} if $(F, M)$ is a weak $\delta$-interleaving of $\Gamma[2\delta]$ with $\S$.
  $\Phi(F, G)$ is a \textbf{right $\delta$-interleaving of image modules} if $(M, G)$ is a weak $\delta$-interleaving of $\Lambda[2\delta]$ with $\V$.

  An image module homomorphism $\Phi(F, G)$ is a \textbf{partial $\delta$-interleaving of image modules}, and denoted $\Phi_M(F, G)$, if it is both a left and right $\delta$-interleaving of image modules.
\end{definition}

% A strong $\delta$-interleaving of image modules by pairs $(F, M)$ and $(M, G)$ will be denoted $\Phi_M(F, G)
% then $\Phi(F,G)$ is said to be a
% Let $\Gamma\in\Hom(\U,\V)$, $\Lambda\in\Hom(\S,\T)$ and $F\in\Hom^\delta(\U, \S)$, $G\in\Hom^\delta(\V, \T)$ be such that $\Phi(F,G) : \im~\Gamma\to\im~\Lambda$ is a image module homomorphism of degree $\delta$.
% $\Phi(F, G)$ is said to be a \textbf{partial $\delta$ interleaving of image modules} $\im~\Gamma$ and $\im~\Lambda$ if there exists some $M\in\Hom^\delta(\S,\V)$ such that the following diagrams commute for all $\alpha$.
%
% \begin{minipage}{0.45\textwidth}
% \begin{equation}\label{dgm:partial_interleaving1}
%   \begin{tikzcd}
%     U_{\alpha-\delta}\arrow{rr}{v_{\alpha-\delta}^{\alpha+\delta}\circ\gamma_{\alpha-\delta}}\arrow{dr}{f_{\alpha-\delta}} & &
%     V_{\alpha+\delta}\\
%     %
%     & S_{\alpha}\arrow{ur}{m_\alpha} &
% \end{tikzcd}\end{equation}
% \end{minipage}
% \begin{minipage}{0.45\textwidth}
% \begin{equation}\label{dgm:partial_interleaving2}
%   \begin{tikzcd}
%     & V_{\alpha}\arrow{dr}{g_\alpha} &\\
%     %
%     S_{\alpha-\delta}\arrow{rr}{t_{\alpha-\delta}^{\alpha+\delta}\circ\lambda_{\alpha-\delta}}\arrow{ur}{m_{\alpha-\delta}} & &
%     T_{\alpha+\delta}
% \end{tikzcd}\end{equation}
% \end{minipage}
%
% An image module homomorphism of degree $\delta$ that provides a partial $\delta$-interleaving will be denoted $\Phi_M(F, G)$.

% In the following let $\Gamma\in\Hom(\U, \V)$, $\Lambda\in\Hom(\S,\T)$, $\tilde{\Lambda}\in\Hom(\tilde{\S},\tilde{\T})$, $\Lambda'\in\Hom(\S',\T')$, and $\Pi\in\Hom(\U,\W)$ be homomorphisms of persistence modules.


% \begin{lemma}
%   Suppose $I(\Sigma, \Theta)\in\Hom(\im~\Lambda,\im~\tilde{\Lambda})$ is an image module homomorphism and $\Phi'(F', G')\in\Hom^{\delta'}(\im~\Gamma,\im~\Lambda')$ is a left $\delta'$-interleaving for some $M'\in\Hom^{\delta'}(\S',\V)$.
%
%   If there exists some $\Xi\in\Hom(\tilde{\S},\S')$ such that $\Phi(F, G)\in\Hom^\delta(\im~\Gamma,\im~\Lambda)$ is a left $\delta$-interleaving by the pair $(F, M'\circ\Xi\circ\Sigma)$ then
%   \[\tilde{\Phi}(\Sigma\circ F, G'\circ\Theta)\in\Hom^{\delta'}(\im~\Gamma, \im~\tilde{\Lambda})\]
%   is a left $\delta'$-interleaving by the pair $(\Sigma\circ F, M'\circ\Xi)$.
% \end{lemma}


% \begin{lemma}
%   Suppose $I(\Sigma, \Theta)\in\Hom(\im~\Lambda,\im~\tilde{\Lambda})$ is an image module homomorphism and $L\in\Hom^{2\delta'}(\S',\T)$.
%
%   If $\Gamma$ and $L$ are right $\delta'$-interleaved by a pair $(M', G)$ and there exists some $\Xi\in\Hom(\tilde{\S},\S')$ such that $\Phi(F, G)\in\Hom^\delta(\im~\Gamma,\im~\Lambda)$ is a left $\delta$-interleaving by the pair $(F, M'\circ\Xi\circ\Sigma)$ and
%   \[\tilde{\lambda}[2\delta']_\alpha = \omega_{\alpha+2\delta'}\circ \ell_\alpha\circ\xi_\alpha \]
%   then
%   \[\tilde{\Phi}(\Sigma\circ F, G'\circ\Theta)\in\Hom^{\delta'}(\im~\Gamma, \im~\tilde{\Lambda})\]
%   is a right $\delta'$-interleaving by the pair $(M'\circ\Xi, \Omega\circ G)$.
% \end{lemma}


\begin{lemma}\label{thm:interleaving_main}
  % For $\R$-persistence modules $\S, \T, \U, \V, \W$ let
  Suppose $\Gamma\in\Hom(\U,\V)$, $\Pi\in\Hom(\V,\W)$, and $\Lambda\in\Hom(\S, \T)$.
  % be homomorphisms of persistence modules.% and
  % \begin{align*}
  %   F\in\Hom^\delta(\U,\S)& & M\in\Hom^\delta(\S,\V)\\
  %   G\in\Hom^\delta(\V,\T)& & N\in\Hom^\delta(\T,\W)
  % \end{align*}
  % be homomorphisms of degree $\delta$.

  If $\Phi_M(F, G)\in\Hom^\delta(\im~\Gamma, \im~\Lambda)$ and $\Psi_G(M, N)\in\Hom^\delta(\im~\Lambda, \im~\Pi)$ are partial $\delta$-interleavings of image modules such that $\Gamma$ is a epimorphism and $\Pi$ is a monomorphism then $\im~\Lambda$ is $\delta$-interleaved with $\V$.
\end{lemma}
\begin{proof}
  For ease of notation let $\Phi$ denote $\Phi_M(F, G)$ and $\Psi$ denote $\Psi_G(M, N)$.

  If $\Gamma$ is an epimorphism $\gamma_\alpha$ is surjective so $\Gamma_\alpha = V_\alpha$ and $\phi_{\alpha} = g_{\alpha}\rest_{\Gamma_\alpha} = g_\alpha$ for all $\alpha$.
  So $\im~\Gamma = \V$ and $\Phi\in\Hom^\delta(\V,\im~\Lambda)$.

  If $\Pi$ is a monomorphism then $\pi_\alpha$ is injective so we can define a natural isomorphism $\pi_\alpha^{-1} : \Pi_\alpha\to V_\alpha$ for all $\alpha$.
  Let $\Psi^*$ be defined as the family of linear maps $\{\psi_\alpha^* := \pi^{-1}_\alpha \circ \psi_\alpha : \Lambda_\alpha\to V_{\alpha+\delta}\}$.
  Because $\Psi$ is a partial $\delta$-interleaving of image modules, $n_\alpha\circ\lambda_\alpha = \pi_{\alpha+\delta}\circ m_\alpha$.
  So, because $\psi_\alpha = n_\alpha\rest_{\Lambda_\alpha}$ for all $\alpha$,
  \begin{align*}
    \im~\psi_\alpha^* &= \im~\pi^{-1}_{\alpha+\delta}\circ\psi_\alpha\\
                      &= \im~\pi^{-1}\circ (n_\alpha\circ\lambda_\alpha)\\
                      &= \im~\pi^{-1}\circ (\pi_{\alpha+\delta}\circ m_\alpha)\\
                      &= \im~ m_\alpha.
  \end{align*}
  It follows that $\im~v_{\alpha+\delta}^{\beta+\delta}\circ\psi_\alpha^* = \im~v_{\alpha+\delta}^{\beta+\delta}\circ m_\alpha$

  % Now,
  % \begin{align*}
  %   v_{\alpha+\delta}^{\beta+\delta}\circ\psi_\alpha^* &= v_{\alpha+\delta}^{\beta+\delta}\circ (\pi^{-1}_\alpha \circ \psi_\alpha)\\
  %     &= v_{\alpha+\delta}^{\beta+\delta}\circ \pi^{-1}_\alpha \circ n_\alpha\rest_{\Lambda_\alpha}
  % \end{align*}
  % where, because $\Psi$ is a partial $\delta$-interleaving of image modules, $n_\alpha\circ\lambda_\alpha = \pi_{\alpha+\delta}\circ m_\alpha$.
  % So,
  % \begin{align*}
  %   \im~v_{\alpha+\delta}^{\beta+\delta}\circ\psi_\alpha^* &= \im~ v_{\alpha+\delta}^{\beta+\delta}\circ \pi^{-1}_\alpha \circ n_\alpha\rest_{\Lambda_\alpha}\\
  %     &= \im~v_{\alpha+\delta}^{\beta+\delta}\circ \pi^{-1}_{\alpha+\delta}\circ(\pi_{\alpha+\delta}\circ m_\alpha)\\
  %     &= \im~v_{\alpha+\delta}^{\beta+\delta}\circ m_\alpha.
  % \end{align*}

  Similarly, because $\Psi$ is a $\delta$-interleaving of image modules $n_\beta\circ t_\alpha^\beta\circ \lambda_\alpha = w_{\alpha+\delta}^{\beta+\delta}\circ\pi_{\alpha+\delta}\circ m_\alpha$.
  Moreover, because $\Pi$ is a homomorphism of persistence modules, $w_{\alpha+\delta}^{\beta+\delta}\circ\pi_{\alpha+\delta} = \pi_{\beta+\delta}\circ v_{\alpha+\delta}^{\beta+\delta}$, so
  \[ n_\beta\circ t_\alpha^\beta\circ \lambda_\alpha = \pi_{\beta+\delta}\circ v_{\alpha+\delta}^{\beta+\delta}\circ m_\alpha.\]
  As $\psi_\beta\circ\lambda_\alpha^\beta = n_\beta\circ\lambda_\alpha^\beta = n_\beta\circ t_\alpha^\beta\rest_{\Lambda_\alpha}$ it follows
  \begin{align*}
    % \im~\psi_\beta^*\circ\lambda_\alpha^\beta &= \im~\pi^{-1}_{\beta+\delta}\circ\psi_\beta\circ\lambda_\alpha^\beta\\
    \im~\psi_\beta^*\circ\lambda_\alpha^\beta &= \im~\pi^{-1}_{\beta+\delta}\circ (n_\beta\circ t_\alpha^\beta\circ\lambda_\alpha)\\
      % &= \im~\pi^{-1}_{\beta+\delta}\circ (w_{\alpha+\delta}^{\beta+\delta}\circ\pi_{\alpha+\delta})\circ m_\alpha\\
      &= \im~\pi^{-1}_{\beta+\delta}\circ (\pi_{\beta+\delta}\circ v_{\alpha+\delta}^{\beta+\delta})\circ m_\alpha\\
      &= \im~v_{\alpha+\delta}^{\beta+\delta}\circ m_\alpha\\
      &= \im~v_{\alpha+\delta}^{\beta+\delta}\circ\psi_\alpha^*.
  \end{align*}
  So we may conclude that $\Psi^*\in\Hom^\delta(\im~\Lambda,\V)$.

  % \begin{align*}
  %   \psi_{\beta}^*\circ\lambda_\alpha^\beta &=\pi^{-1}_{\beta+\delta}\circ\psi_\beta\circ\lambda_\alpha^\beta\\
  %     &= \pi^{-1}_{\beta+\delta}\circ (n_\beta\rest_{\Lambda_\beta})\circ\lambda_\alpha^\beta\\
  %     &= \pi^{-1}_{\beta+\delta}\circ n_\beta\circ (t_\alpha^\beta\rest_{\Lambda_\alpha}).
  % \end{align*}
  % Because $\Psi$ is a $\delta$-interleaving of image modules $n_\beta\circ t_\alpha^\beta\circ \lambda_\alpha = w_{\alpha+\delta}^{\beta+\delta}\circ\pi_{\alpha+\delta}\circ m_\alpha$ and, because $\Pi$ is a homomorphism of persistence modules, $w_{\alpha+\delta}^{\beta+\delta}\circ\pi_{\alpha+\delta} = \pi_{\beta+\delta}\circ v_{\alpha+\delta}^{\beta+\delta}$.
  % So,
  % \begin{align*}
  %   \im~\psi_{\beta}^*\circ\lambda_\alpha^\beta &= \im~\pi^{-1}_{\beta+\delta}\circ (n_\beta\circ t_\alpha^\beta\rest_{\Lambda_\alpha})\\
  %     &= \im~\pi^{-1}_{\beta+\delta}\circ (\pi_{\beta+\delta}\circ v_{\alpha+\delta}^{\beta+\delta}\circ m_\alpha)\\
  %     &= \im~v_{\alpha+\delta}^{\beta+\delta}\circ m_\alpha\\
  %     &= \im~v_{\alpha+\delta}^{\beta+\delta}\circ\psi_\alpha^*.
  % \end{align*}
  % As the images of $v_{\alpha+\delta}^{\beta+\delta}\circ\psi_\alpha^*, \psi_{\beta}^*\circ\lambda_\alpha^\beta : \Lambda_\alpha\to V_{\beta+\delta}$ are equal we may conclude that $v_{\alpha+\delta}^{\beta+\delta}\circ\psi_\alpha^* = \psi_{\beta}^*\circ\lambda_\alpha^\beta$ for all $\alpha\leq\beta$, so $\Psi^*\in\Hom^\delta(\im~\Lambda,\V)$.

  So $\Phi\in\Hom^\delta(\V,\im~\Lambda)$ and $\Psi_G^*\in\Hom^\delta(\im~\Lambda,\V)$.
  As we have shown, $\im~\psi_{\alpha-\delta}^* = \im~m_{\alpha-\delta}$ so $\im~\phi_\alpha\circ\psi_{\alpha-\delta}^* = \im~\phi_\alpha\circ m_{\alpha-\delta}$.
  Moreover, because $\gamma_\alpha$ is surjective $\phi_\alpha = g_\alpha$ and, because $\Phi$ is a partial $\delta$-interleaving of image modules, $g_\alpha\circ m_{\alpha-\delta} = t_{\alpha-\delta}^{\alpha+\delta}\circ \lambda_{\alpha-\delta}$.
  As $\lambda_{\alpha-\delta}^{\alpha+\delta} = t_{\alpha-\delta}^{\alpha+\delta}\rest_{\im~\lambda_{\alpha-\delta}}$ it follows that $\im~\phi_\alpha\circ\psi_{\alpha-\delta}^* = \im~\lambda_{\alpha-\delta}^{\alpha+\delta}$.

  Finally, $\psi_\alpha^*\circ\phi_\alpha = \pi_{\alpha+\delta}^{-1}\circ n_\alpha\circ g_{\alpha-\delta}$ where, because $\Psi$ is a partial $\delta$-interleaving of image modules, $n_\alpha\circ g_{\alpha-\delta} = w_{\alpha-\delta}^{\alpha+\delta}\circ\pi_{\alpha-\delta}$.
  Because $\Pi$ is a homomorphism of persistence modules $w_{\alpha-\delta}^{\alpha+\delta}\circ \pi_{\alpha-\delta} = \pi_{\alpha+\delta}\circ v_{\alpha-\delta}^{\alpha+\delta}$.
  Therefore,
  \begin{align*}
    \psi_\alpha^*\circ\phi_\alpha &= \pi_{\alpha+\delta}^{-1}\circ n_\alpha\circ g_{\alpha-\delta}\\
      &= \pi_{\alpha+\delta}^{-1}\circ (\pi_{\alpha+\delta}\circ v_{\alpha-\delta}^{\alpha+\delta})\\
      &= v_{\alpha-\delta}^{\alpha+\delta}
  \end{align*}
  which, along with $\phi_\alpha\circ\im~\psi_{\alpha-\delta}^* = \lambda_{\alpha-\delta}^{\alpha+\delta}$ implies Diagrams~\ref{dgm:interleaving1} and~\ref{dgm:interleaving2} commute with $\Phi\in\Hom^\delta(\V,\im~\Lambda)$ and $\Psi^*\in\Hom^\delta(\im~\Lambda, \V)$.
  We may therefore conclude that $\im~\Lambda$ and $\V$ are $\delta$-interleaved.

\end{proof}

% \begin{theorem}\label{thm:interleaving}
%   Suppose $(\{S_\alpha\},\{T_\alpha\})$, $(\{X_\alpha\}, \{Y_\alpha\})$, and $(\{Y_\alpha\}, \{Z_\alpha\})$ are all pairs of compatible filtrations such that the following diagram commutes for all $\alpha\in\R$.
%   \begin{equation}\label{dgm:interleaving}
%   \begin{tikzcd}
%     \hom_k(X_{\alpha-\e})\arrow{r}{u_{\alpha-\e}}\arrow{d}{f_{\alpha-\e}} &
%     \hom_k(S_\alpha)\arrow{r}{m_\alpha}\arrow{d}{g_\alpha} &
%     \hom_k(Y_{\alpha+\e})\arrow{d}{h_{\alpha+\e}}\\
%     %
%     \hom_k(Y_{\alpha-\e})\arrow{r}{v_{\alpha-\e}} &
%     \hom_k(T_\alpha)\arrow{r}{n_\alpha} &
%     \hom_k(Z_{\alpha+\e})
%   \end{tikzcd}\end{equation}
%
%   If $f_\alpha : \hom_k(X_\alpha)\to\hom_k(Y_\alpha)$ is surjective and $h_\alpha : \hom_k(Y_\alpha)\to \hom_k(Z_\alpha)$ is injective for all $\alpha\in\R$ then the $k$th persistent homology modules of $\{S_\alpha\to T_\alpha\}$ and $\{Y_\alpha\}$ are $\e$-interleaved.
% \end{theorem}
% \begin{proof}
%   Let $\Phi := (\{\Phi_\alpha\}, \{\phi_\alpha^\beta : \Phi_\alpha\to\Phi_\beta\})$ denote the $k$th persistent homology module of $\{S_\alpha\to T_\alpha\}$ where $\Phi_\alpha := \im~g_\alpha$  and $\phi_\alpha^\beta := t_\alpha^\beta\rest_{\im~g_\alpha}$ for $t_\alpha^\beta : \hom_k(T_\alpha)\to\hom_k(T_\beta)$.
%   Note that this map is well defined for all $\alpha\leq\beta$ as $\{S_\alpha\}$ and $\{T_\alpha\}$ are compatible filtrations.
%   Let $\Psi := (\{\Psi_\alpha\}, \{\psi_\alpha^\beta : \Psi_\alpha\to\Psi_\beta\})$ be the $k$th persistent homology module of $\{Y_\alpha\}$ where $\Psi_\alpha := \hom_k(Y_\alpha)$ and $\psi_\alpha^\beta$ induced by the map $Y_\alpha\to Y_\beta$.
%   Once again, because $(\{X_\alpha\},\{Y_\alpha\})$ and $(\{Y_\alpha\},\{Z_\alpha\})$ are compatible pairs of filtrations this map commutes with both $X_\alpha\to X_\beta$ and $Z_\alpha\to Z_\beta$ for all $\alpha\leq\beta$.
%
%   Because $h_\alpha$ is injective for all $\alpha$ we have an isomorphism $h_\alpha^{-1} : \im~h_\alpha\to \Psi_\alpha$.
%   Because the right square of Diagram~\ref{dgm:interleaving} commutes $n_\alpha\circ g_\alpha = h_\alpha\circ m_\alpha$, so $\im~n_\alpha\rest_{\Phi_\alpha}$ is a subspace of $\im~h_\alpha$.
%   We therefore define $\mu_\alpha : \Phi_\alpha\to\Psi_{\alpha+\e}$ as $\mu_\alpha := h^{-1}_{\alpha+\e}\circ n_\alpha\rest_{\Phi_\alpha}$.
%   Similarly, because the left square of Diagram~\ref{dgm:interleaving} commutes and $f_{\alpha-\e}$ is surjective $\im~v_{\alpha-\e}\circ f_{\alpha-\e} = \im~v_{\alpha-\e}$ is a subspace of $\Phi_\alpha$.
%   We may therefore define $\nu_{\alpha-\e} : \Psi_{\alpha-\e}\to \Phi_\alpha$ as $\nu_{\alpha-\e} = v_{\alpha-\e}$ for all $\alpha$.
%
%   \textbf{It follows that the following diagrams commute for all $\alpha\leq\beta$}
%
%   \begin{minipage}{0.5\textwidth}
%   \begin{equation}\label{dgm:intr1}
%   \begin{tikzcd}
%     \Phi_{\alpha-\e}\arrow{rr}{\phi_{\alpha-\e}^{\alpha+\e}}\arrow{dr}{\mu_{\alpha-\e}} & &
%     \Phi_{\alpha+\e}\\
%     %
%     & \Psi_{\alpha}\arrow{ur}{\nu_\alpha} &
%   \end{tikzcd}\end{equation}
%   \begin{equation}\label{dgm:intr3}
%   \begin{tikzcd}
%     \Phi_{\alpha-\e}\arrow{r}{\phi_{\alpha-\e}^{\beta-\e}}\arrow{dr}{\mu_{\alpha-\e}} &
%     \Phi_{\beta-\e}\arrow{dr}{\mu_{\beta-\e}} &\\
%     %
%     & \Psi_{\alpha}\arrow{r}{\psi_\alpha^\beta} &
%     \Psi_{\beta}
%   \end{tikzcd}\end{equation}
%   \end{minipage}
%   \begin{minipage}{0.5\textwidth}
%   \begin{equation}\label{dgm:intr2}
%   \begin{tikzcd}
%     & \Phi_{\alpha}\arrow{dr}{\mu_{\alpha}} &\\
%     %
%     \Psi_{\alpha-\e}\arrow{ur}{\nu_{\alpha-\e}}\arrow{rr}{\psi_{\alpha-\e}^{\alpha+\e}} & &
%     \Psi_{\alpha+\e}
%   \end{tikzcd}\end{equation}
%   \begin{equation}\label{dgm:intr4}
%   \begin{tikzcd}
%     & \Phi_{\alpha}\arrow{r}{\phi_{\alpha}^{\beta}} &
%     \Phi_{\beta}\\
%     %
%     \Psi_{\alpha-\e}\arrow{ur}{\nu_{\alpha-\e}}\arrow{r}{\psi_{\alpha-\e}^{\beta-\e}} &
%     \Psi_{\beta-\e}\arrow{ur}{\nu_{\beta-\e}}
%   \end{tikzcd}\end{equation}
%   \end{minipage}
%
%   % Because
%   % $\nu_\alpha\circ\mu_{\alpha-\e} = v_\alpha\circ (h^{-1}_{\alpha-\e}\circ n_\alpha\rest_{\Phi_\alpha})
% \end{proof}
%
