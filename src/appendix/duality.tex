% !TeX root = ../../main.tex

% \subsection{Duality}

For a pair $(A, B)$ in a topological space $X$ and any $R$ module $G$ let $\hom^k(A, B; G)$ denote the \textbf{singular cohomology} of $(A,B)$ (with coefficients in $G$) as a vector space.
Let $\hom^k_c(A, B; G)$ denote the corresponding \textbf{singular cohomology with compact support}, where $\hom^k_c(A, B; G)\cong \hom^k(A, B; G)$ for any compact pair $(A,B)$.

The following corollary follows from the Universal Coefficient Theorem for singular homology (and cohomology) as vector spaces over a field $\FF$, as the dual vector space $\Hom(\hom_k(A, B), \FF)$ is isomorphic to $\hom_k(A, B; \FF)$ for any finitely generated $\hom_k(A, B)$.\footnote{Reference/verify.}

\begin{corollary}\label{cor:univ_coef}
  For a topological pair $(A, B)$ and a field $\FF$ such that $\hom_0(A, B)$ is finitely generated there is a natural isomorphism
  \[\nu : \hom^0(A, B; \FF)\to \hom_0(A, B; \FF).\]
\end{corollary}

Let $\overline{\hom}^k(A, B; G)$ be the \textbf{Alexander-Spanier cohomology} of the pair $(A,B)$, defined as the limit of the direct system of neighborhoods $(U,V)$ of the pair $(A, B)$.
Let $\overline{\hom}^k_c(A, B; G)$ denote the corresponding \textbf{Alexander-Spanier cohomology with compact support} where $\overline{\hom}^k_c(A, B; G)\cong\overline{\hom}^k(A, B; G)$ for any compact pair $(A, B)$.

\begin{theorem}[\textbf{Alexander-Poincar\'e-Lefschetz Duality} (Spanier, Theorem 6.2.17)]\label{thm:alexander}
  Let $X$ be an orientable $d$-manifold and $(A, B)$ be a compact pair in $X$.
  Then for all $k$ and $R$ modules $G$ there is a (natural) isomorphism
  \[\lambda : \hom_k(X\setminus B, X\setminus A; G)\to \overline{\hom}^{d-k}(A, B; G).\]
\end{theorem}

A space $X$ is said to be \textbf{homologically locally connected in dimension $n$} if for every $x\in X$ and neighborhood $U$ of $x$ there exists a neighborhood $V$ of $x$ in $U$ such that $\tilde{\hom}_n(V)\to\tilde{\hom}_n(U)$ is trivial for $k\leq n$.

\begin{lemma}[Spanier p. 341, Corollary 6.9.6]\label{lem:alexander_iso}
  Let $A$ be a closed subset, homologically locally connected in dimension $n$, of a Hausdorff space $X$, homologically locally connected in dimension $n$.
  If $X$ has the property that every open subset is paracompact, $\mu : \overline{\hom}_c^k(X,A; G)\to \hom_c^k(X, A; G)$ is an isomorphism for $k\leq n$ and a monomorphism for $q = n+1$.
\end{lemma}

In the following we will assume homology (and cohomology) over a field $\FF$.

\begin{lemma}\label{cor:alexander_iso}
  Let $X$ be an orientable $d$-manifold and $(A,B)$ a compact pair of locally path connected subspaces in $X$.
  Then
  \[\xi : \hom_d(X\setminus B, X\setminus  A)\to \hom_0(A, B)\]
  is a natural isomorphism.
\end{lemma}
\begin{proof}
  Because $X$ is orientable and $(A,B)$ are compact $\lambda : \hom_d(X\setminus B, X\setminus A)\to \overline{\hom}^{0}(A, B)$ is an isomorphism by Theorem~\ref{thm:alexander}.
  Note that
  Moreover, because every subset of $X$ is (hereditarily) paracompact every open set in $A$, with the subspace topology, is paracompact.
  For any neighborhood $U$ of a point $x$ in a locally path connected space there must exist some neighborhood $V\subset U$ of $x$ that is path connected in the subspace topology.
  As $\tilde{\hom}_0(V) = 0$ for any nonempty, path connected topological space $V$ (see Spanier p. 175, Lemma 4.4.7) it follows that $A$ (resp. $B$) are homologically locally connected in dimension $0$.
  Because $(A,B)$ is a compact pair the singular and Alexander-spanier cohomology modules of $(A,B)$ with compact support are isomorphic to those without, thus $\mu:\overline{\hom}^{0}(A, B)\to \hom^0(A, B)$ is an isomorphism.
  By Corollary~\ref{cor:univ_coef} we have a natural isomorphism $\nu : \hom^0(A, B)\to\hom_0(A, B)$ thus the composition $\xi := \nu\circ\mu\circ\lambda : \hom_d(X\setminus B, X\setminus  A)\to \hom_0(A, B)$ is a natural isomorphism.
\end{proof}


% % \begin{theorem}[\textbf{Alexander-Poincar\'e Duality} (Julian et. al.~\cite{julian83alexander}, Theorem 5.1)]\label{thm:alexander}
% %   Let $K$ be an abstract simplicial complex that is a combinatorial oriented $d$-manifold.
% %   Let $L$ be a subcomplex of some refinement of $K$ and $M$ be a subcomplex of $L$.
% %   Let $\overline{L}$ and $\overline{M}$ denote the complements of $L$ and $M$ as subcomplexes of $K$ that do not share vertices with the original complexes.
% %   Then for all $k$ there is a natural isomorphism
% %   \[ \hom^k(L, M)\to \hom_{d-k}(\overline{M},\overline{L}). \]
% % \end{theorem}
%
% % \begin{corollary}
% %   Let $X$ be a topological space and $D$ be a compact subspace of $X$.
% %   Let $(U, V)$ be a topological pair of spaces in $D$ and suppose there exists a triangulation $\Delta X$ of $X$ such that there exists triangulation $\Delta U$ of $U\subset$ that is a subcomplex of some refinement of $\Delta X$ and a triangulation $\Delta V$ of $V$ that is a subcomplex of $\Delta U$.
% %   Then for all $k$ there is a natural isomorphism
% %   \[ \hom^k(U, V)\to\hom_{d-k}(D\setminus U, D\setminus V).\]
% % \end{corollary}
%
%
% \begin{corollary}\label{cor:univ_coef}
%   Let $(A, B)$ be a topological pair and $\FF$ be a field such that $\hom_k(A, B; \FF)$ is finitely generated.
%   Then there is a natural isomorphism
%   \[\hom^k(A, B; \FF)\to \hom_k(A, B; \FF).\]
% \end{corollary}
% \begin{proof}
%   As $\mathrm{Ext}(\Hom(A, B), \FF) = 0$ for any field $\FF$ the map
%   \[\hom^k(A, B; \FF)\to \Hom(\hom_k(A, B), \FF)\]
%   in the natural short exact sequence provided by Theorem~\ref{thm:univ_coef} is a natural isomorphism.
%   The result follows from the fact that $\hom_k(A, B; \FF)$ is finitely generated, and is therefore isomorphic to the dual vector space $\Hom(\hom_k(A, B), \FF)$.
% \end{proof}
%
% % \begin{theorem}[\textbf{Universal Coefficient Theorem} (Munkres p. 337, Corollary 56.4)]\label{thm:univ_coef}
% %   Let $(A,B)$ be a topological pair such that $\hom_k(A, B)$ is finitely generated for all $k$.
% %   Then for all $k$ and any abelian group $G$ there is a natural exact sequence
% %   \[ 0\to\mathrm{Ext}(\hom^{k+1}(A, B), G)\to \hom_k(A, B; G)\to \Hom(\hom^k(A, B), G)\to 0.\]
% %   This sequence splits, but not naturally.
% % \end{theorem}
%
% \begin{theorem}
%   Let $X$ be a topological space and $D$ be a compact subspace of $X$.
%   Suppose there exists a triangulation $\Delta X$ of $X$ that is a combinatorial oriented $d$-manifold.
%   Let $(U, V)$ be a topological pair of spaces in $D$ and $\FF$ be a field such that $\hom_k(U,V;\FF)$ is finitely generated for all $k$.
%
%   If there exists a pair of triangulations $(\Delta U, \Delta V)$ of the pair $(U, V)$ such that $\Delta U$ is a subcomplex of some refinement of $\Delta X$ then there is a natural isomorphism
%   \[ \hom_d(U, V; \FF)\to \hom_0(X\setminus V, X\setminus U; \FF).\]
% \end{theorem}
% \begin{proof}
%   By Theorem~\ref{thm:alexander} we have a natural isomorphism
%   \[ \hom^d(\Delta U, \Delta V; \FF)\to \hom_{0}(\overline{\Delta V}, \overline{\Delta U}; \FF) \]
%   where $\overline{\Delta V}$ and $\overline{\Delta U}$ denote the complements of $\Delta V$ and $\Delta U$ as subcomplexes of $\Delta X$ that do not share vertices with their respective original complexes.
%   \textbf{TODO}\footnote{$\hom^d(U, V;\FF)\cong \hom^d(\Delta U, \Delta V;\FF),\ \hom_{0}(\overline{\Delta V}, \overline{\Delta U}; \FF) \cong \hom_0(X\setminus V, X\setminus U; \FF).$}
%
%   Because $\hom_d(U, V; \FF)$ is finitely generated $\hom_d(U, V;\FF)\cong\hom^d(U, V; \FF)$ by Corollary~\ref{cor:univ_coef}.
%   It follows that the composition
%   \[\hom_d(U, V; \FF)\to \hom^d(U,V;\FF)\to\hom_0(X\setminus V, X\setminus U; \FF)\]
%   is a natural isomorphism as desired.
% \end{proof}
% % \begin{proof}
% %   \begin{itemize}
% %     \item By Theorem~\ref{thm:univ_coef} we have a short exact sequence
% %       \[ 0\to\mathrm{Ext}(\hom^{d+1}(\Delta U, \Delta V), G)\to \hom_d(\Delta U, \Delta V; G)\to \Hom(\hom^d(\Delta U, \Delta V), G)\to 0\]
% %       for any abelian group $G$.
% %       Because $\Delta U,\Delta V$ are subcomplexes of the combinatorial $d$-manifold $\Delta X$, $\hom^{d+1}(\Delta U, \Delta V) = 0$, so $\hom_d(\Delta U, \Delta V; G)\to \Hom(\hom^d(\Delta U, \Delta V), G)$ is an isomorphism.
% %     \item By Theorem~\ref{thm:alexander} we have a natural isomorphism $\hom^d(\Delta U, \Delta V)\to \hom_0(\overline{\Delta V},\overline{\Delta U})$.
% %       Therefore, because we have natural\footnote{\textbf{TODO} natural short exact $\implies$ natural isomorphism.} isomorphisms $\hom_d(\Delta U, \Delta V; G)\to \Hom(\hom^d(\Delta U, \Delta V), G)$ for all abelian groups $G$, we have a natural isomorphism (\textbf{TODO} prove it).
% %       \[\hom_d(\Delta U, \Delta V)\to \hom_0(\overline{\Delta V},\overline{\Delta U}).\]
% %     \item Because $\hom_d(U, V)\cong \hom_d(\Delta U, \Delta V)$ and $\hom_0(X\setminus V, X\setminus U)\cong \hom_0(\overline{\Delta V},\overline{\Delta U})$ (\textbf{TODO} prove it\footnote{$(X\setminus V, X\setminus U)$ or $(D\setminus V, D\setminus U)$?}), we have a natural isomorphism
% %     \[ \hom_d(U, V)\to \hom_0(X\setminus V, X\setminus U). \]
% %   \end{itemize}
% % \end{proof}
%
%
% %
% % \begin{theorem}[\textbf{Alexander Duality} (Spanier p. 296, Theorem 6.2.17)]
% %   Let $U$ be an orientation over $R$ of an $d$-manifold $X$ and let $(A, B)$ be a compact pair in $X$.
% %   Then for all $k$ and $R$ modules $G$ there is a natural isomorphism
% %   \[ \hom_k(X\setminus B, X\setminus A; G)\to\overline{\hom}^{d-k}(A, B; G).\]
% % \end{theorem}
% %
% % \begin{lemma}
% %   Let $U$ be an orientation over $R$ of an $d$-manifold $X$ and let $(A, B)$ be a compact pair in $X$ such that $\hom_k(A, B)$ is finitely generated for all $k$.
% %   Then for all $R$ modules $G$ there is a natural isomorphism
% %   \[ \hom_0(X\setminus B, X\setminus A; G)\to\hom_d(A, B; G). \]
% % \end{lemma}
% % \begin{proof}
% %   \textbf{TODO}
% % \end{proof}
