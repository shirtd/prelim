% !TeX root = ../../main.tex

\subsection{Surrounding and Extensions}

\begin{lemma}\label{lem:excision}
  If $(L, S)$ is a surrounding pair in a subspace $D$ of $X$ and $L$ is open in $D$ then
  \[ \hom_k(L\cap A, S) \cong \hom_k(A, \ext{S}) \]
  for all $k$ and any $A\subseteq D$ such that $\ext{S}\subset A$.
\end{lemma}
\begin{proof}
  Because $S$ surrounds $L$ in $D$, $(L\setminus S, D\setminus L)$ is a separation of $D\setminus S$, a subspace of $D$.
  So $\cl_D(L\setminus S)\setminus L = \cl_D(L\setminus S) \cap (D\setminus L) = \emptyset$ which implies $\cl_D(L\setminus S)\subseteq L = \intr_D(L)$ as $L$ is open in $D$.
  Therefore,
  \begin{align*}
    \cl_D(D\setminus L) &= D\setminus \intr_D(L)\\
                        &\subseteq D\setminus \cl_D(L\setminus S)\\
                        &= \intr_D(D\setminus (L\setminus S))\\
                        &= \intr_D(\ext{S}).
  \end{align*}
  so,
  \begin{align*}
    \hom_k(L\cap A, S) &= \hom_k(A\setminus (D\setminus L), \ext{S}\setminus (D\setminus L))\\
      &\cong \hom_k(A, \ext{S})
  \end{align*}
  for all $k$ and any $A\subseteq D$ such that $\ext{S}\subset A$ by Excision.
\end{proof}

% \begin{lemma}\label{lem:excision_commute}
%   Suppose $(L, S)$ is a surrounding pair in a subspace $D$ of $X$ and $L$ is open in $D$.
%
%   For all $A\subseteq A'\subseteq D$ such that $\ext{S}\subset A$ the isomorphisms induced by inclusions $(L\cap A, S)\hookrightarrow (A, \ext{S})$ and $(L\cap A', S)\hookrightarrow (A', \ext{S})$ commute with the inclusions $(L\cap A, S)\hookrightarrow (L\cap A', S)$ and $(A, \ext{S})\hookrightarrow (A', \ext{S})$ on the level of homology.
% \end{lemma}

\begin{lemma}\label{lem:surround_and_cover}
  Suppose $S$ surrounds $L$ in $D$ and $B\subseteq B'\subset D$.

  If $D\setminus B'\subseteq L$ and $L\cap B\subseteq S\subseteq B'$ then $B\subseteq \ext{S}\subseteq B'$.
\end{lemma}
\begin{proof}
  Note that $B\setminus (D\setminus L) = B\cap L\subseteq S$ implies $B\subseteq S\sqcup(D\setminus L) = \ext{S}$.
  Moreover, because $S\subseteq B'$ and $D\setminus B'\subseteq L$ implies $D\setminus L \subset D\setminus (D\setminus B') = B'$, we have
  \[ \ext{S} = S\sqcup (D\setminus L) \subseteq B'\cup (D\setminus L) = B'. \]
  So $B \subseteq \ext{S}\subseteq B'$ as desired.
\end{proof}

In the following let $X$ be a topological space and $\overline{A} := X\setminus U$ denote the complement of a subset $U$ of $X$.

\begin{lemma}\label{lem:coverage}
  Let $(D, B)$ be a surrounding pair in $X$ and $L\subseteq D$, $S\subseteq L\cap B$ be subsets so that $\ell: \hom_0(\overline{B}, \overline{D})\to \hom_0(\overline{S}, \overline{L})$ is induced by inclusion.

  If $\ell$ is injective then $D\setminus B\subseteq L$.
\end{lemma}
\begin{proof}
    Suppose, for the sake of contradiction, that $p$ is injective and there exists a point $x\in (D\setminus B)\setminus L$.
    Because $B$ surrounds $D$ in $X$ the pair $(D\setminus B, \overline{D})$ forms a separation of $\overline{B}$.
    Therefore, $\hom_0(\overline{B})\cong \hom_0(D\setminus B)\oplus \hom_0(\overline{D})$ so
    \[ \hom_0(\overline{B}, \overline{D})\cong \hom_0(D\setminus B). \]
    So $[x]$ is non-trivial in $\hom_0(\overline{B},\overline{D})\cong \hom_0(D\setminus B)$ as $x$ is in some connected component of $D\setminus B$.
    So we have the following sequence of maps induced by inclusions
    \[ \hom_0(\overline{B},\overline{D})\xrightarrow{f} \hom_0(\overline{B},\overline{D}\cup\{x\})\xrightarrow{g} \hom_0(\overline{S},\overline{L}).\]
    As $f[x]$ is trivial in $\hom_0(\overline{B},\overline{D}\cup\{x\})$ we have that $\ell[x] = (g\circ f)[x]$ is trivial, contradicting our hypothesis that $\ell$ is injective.
\end{proof}

\begin{lemma}\label{lem:cov_surrounds}
  Let $(D, B)$ be a surrounding pair in $X$ and $L\subseteq D$, $S\subseteq L\cap B$ be subsets so that $\ell: \hom_0(\overline{B}, \overline{D})\to \hom_0(\overline{S}, \overline{L})$ is induced by inclusion.

  If $\ell$ injective then $S$ surrounds $L$ in $D$.
\end{lemma}
\begin{proof}
  Suppose, for the sake of contradiction, that $S$ does not surround $L$ in $D$.
  Then there exists a path $\gamma : [0,1]\to\overline{S}$ with $\gamma(0)\in L\setminus S$ and $\gamma(1)\in D\setminus L$.
  By Lemma~\ref{lem:coverage} we know that $D\setminus B\subseteq L$, so $D\setminus B\subseteq L\setminus S$.

  Choose $x\in D\setminus B$ and $z\in \overline{D}$ such that there exist paths $\xi : [0,1]\to L\setminus S$ with $\xi(0) = x$, $\xi(1) = \gamma(0)$ and $\zeta : [0,1]\to \overline{D}\cup (D\setminus L)$ with $\zeta(0) = z$, $\zeta(1) = \gamma(1)$.
  $\xi, \gamma$ and $\zeta$ all generate chains in $C_1(\overline{S}, \overline{L})$ and $\xi + \gamma + \zeta = \gamma^*\in C_1(\overline{S}, \overline{L})$ with $\partial\gamma^* = x + z$.
  Moreover, $z$ generates a chain in $C_0(\overline{L})$ as $\overline{D}\subseteq\overline{L}$.
  So $x = \partial\gamma^* + z$ is a relative boundary in $C_0(\overline{S}, \overline{L})$, thus $\ell[x] = \ell[z]$ in $\hom_0(\overline{S}, \overline{L})$.
  However, because $B$ surrounds $D$, $[x]\neq [y]$ in $\hom_0(\overline{B}, \overline{D})$ contradicting our assumption that $\ell$ is injective.
\end{proof}
