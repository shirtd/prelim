% !TeX root = ../new.tex

\begin{definition}[Separation (Munkres~\cite{munkres00topology})]
  Let $X$ be a topological space. A \textbf{separation} of $X$ is a pair $U, V$ of disjoint, nonempty, open subsets of $X$ whose union is $X$.
  The space $X$ is said to be \textbf{connected} if there does not exist a separation of $X$.
\end{definition}

Note that the sets $U, V$ that form a separation of $X$ are both open and closed in $X$.
For a subspace $Y$ of $X$ we will denote the interior and closure of a set $U$ in $Y$ with $\intr_Y(U)$ and $\cl_Y(X)$.

\begin{lemma}[23.1 (Munkres~\cite{munkres00topology})]
  If $Y$ is a subspace of $X$, a separation of $Y$ is a pair of disjoint, nonempty sets $A, B$ whose union is $Y$, neither of which contains a limit point of the other.
  The space $Y$ is connected if there exists no separation of $Y$.
\end{lemma}

If $A, B$ is a separation of a subspace $Y$ of $X$ then $A, B$ are both open and closed in $Y$, but not necessarily $X$.
The condition that neither $A$ nor $B$ contains a limit point of the other requires that $\cl_X(A)\cap B = \emptyset$ and $A\cap \cl_X(B) =\emptyset$ where $\cl_Y(A) = A$ and $\cl_Y(B) = B$.

% \begin{definition}[Components (Munkres~\cite{munkres00topology})]
%   Given $X$, define an equivalence relation on $X$ by setting $x\sim y$ if there is a connected subspace of $X$ containing both $x$ and $y$.
%   The equivalence class are called the \textbf{components} (or ``connected components'') of $X$.
% \end{definition}

For a disconnected topological space $X$ let $X_1, X_2, \ldots$ denote it's path-connected components.
For $A\subseteq X$ let $A_i = A\cap X_i$ denote the component of $A$ in $X_i$.

\begin{definition}[Separating Set]
  Let $X$ be a (possibly disconnected) topological space and $S\subset X$.
  $S$ \textbf{separates $X$ with a pair $(U, V)$} if $(U_i, V_i)$ is a separation of $X_i\setminus S_i$ for all $i$.
\end{definition}

If $S$ separates $X$ with a pair $(U, V)$ then $X = U\sqcup S\sqcup V$.
Note that while $U$ and $V$ are both open and closed in $X\setminus S$, each component $X_i = U_i\sqcup S_i\sqcup V_i$ is connected.
Therefore, if $S$ separates $X$ with a pair $(U, V)$, we require that $\cl_X(U)\cap V = \emptyset$ and $U\cap \cl_X(V) = \emptyset$.
If $S$ is an open set in $X$ then $U$ and $V$ are closed in $X$, therefore $\cl_X(U)\cap V = \emptyset$ and $U\cap \cl_X(V) = \emptyset$.
Otherwise, if $S$ is closed in $X$, then $U$ and $V$ are open in $X$.

Throughout we will use $U, S,$ and $V$ to denote subsets of $X$ analogous to the interior, boundary, and complement of $S\sqcup U$ in $X$, respectively.
The following definition, while equivalent to that of a separating set, makes this distinction explicit by defining the set $S$ relative to the set $S\sqcup U$.

\begin{definition}[Surrounding]
  Given $B\subset D \subset X$ the set $B$ \textbf{surrounds $D$ in $X$} if $B$ separates $X$ with the pair $(D\setminus B, X\setminus D)$.
  We will refer to such a pair as a \textbf{surrounding pair in $X$}.
\end{definition}

Now, the set $D\setminus B$ corresponds to the interior of $D$ and $X\setminus D$ corresponds to the complement of $D$ in $X$.
This allows us to clearly state the extension of a surrounding pair in a subspace of $X$ to a surrounding pair in $X$.

\begin{definition}[Extension]
  If $S$ surrounds $L$ in a subspace $D$ of $X$ let $\ext{S} := S\sqcup (D\setminus L)$ denote the (disjoint) union of the separating set $S$ with the complement of $L$ in $D$.
  The \textbf{extension of $(L, S)$ in $D$} is the pair
  \[ (D, \ext{S}) = (L\sqcup (D\setminus L), S\sqcup (D\setminus L)).\]
\end{definition}

\begin{lemma}\label{lem:excision}
  If $(L, S)$ is a surrounding pair in a subspace $D$ of $X$ and $L$ is open in $D$ then
  \[ \hom_k(L\cap A, S) \cong \hom_k(A, \ext{S}) \]
  for all $k$ and any $A\subseteq D$ such that $\ext{S}\subset A$.
\end{lemma}
\begin{proof}
  Because $S$ surrounds $L$ in $D$, $(L\setminus S, D\setminus L)$ is a separation of $D\setminus S$, a subspace of $D$.
  So $\cl_D(L\setminus S)\setminus L = \cl_D(L\setminus S) \cap (D\setminus L) = \emptyset$ which implies $\cl_D(L\setminus S)\subseteq L = \intr_D(L)$ as $L$ is open in $D$.
  Therefore,
  \begin{align*}
    \cl_D(D\setminus L) &= D\setminus \intr_D(L)\\
                        &\subseteq D\setminus \cl_D(L\setminus S)\\
                        &= \intr_D(D\setminus (L\setminus S))\\
                        &= \intr_D(\ext{S}).
  \end{align*}
  so,
  \begin{align*}
    \hom_k(L\cap A, S) &= \hom_k(A\setminus (D\setminus L), \ext{S}\setminus (D\setminus L))\\
      &\cong \hom_k(A, \ext{S})
  \end{align*}
  for all $k$ and any $A\subseteq D$ such that $\ext{S}\subset A$ by Excision.
\end{proof}

\begin{lemma}\label{lem:excision_commute}
  Suppose $(L, S)$ is a surrounding pair in a subspace $D$ of $X$ and $L$ is open in $D$.

  For all $A\subseteq A'\subseteq D$ such that $\ext{S}\subset A$ the isomorphisms induced by inclusions $(L\cap A, S)\hookrightarrow (A, \ext{S})$ and $(L\cap A', S)\hookrightarrow (A', \ext{S})$ commute with the inclusions $(L\cap A, S)\hookrightarrow (L\cap A', S)$ and $(A, \ext{S})\hookrightarrow (A', \ext{S})$ on the level of homology.
\end{lemma}

\begin{lemma}\label{lem:surround_and_cover}
  Suppose $S$ surrounds $L$ in $D$ and $B\subseteq B'\subset D$.

  If $D\setminus B'\subseteq L$ and $L\cap B\subseteq S\subseteq B'$ then $B\subseteq \ext{S}\subseteq B'$.
\end{lemma}
\begin{proof}
  Note that $B\setminus (D\setminus L) = B\cap L\subseteq S$ implies $B\subseteq S\sqcup(D\setminus L) = \ext{S}$.
  Moreover, because $S\subseteq B'$ and $D\setminus B'\subseteq L$ implies $D\setminus L \subset D\setminus (D\setminus B') = B'$, we have
  \[ \ext{S} = S\sqcup (D\setminus L) \subseteq B'\cup (D\setminus L) = B'. \]
  So $B \subseteq \ext{S}\subseteq B'$ as desired.
\end{proof}

In the following let $X$ be a topological space and $\overline{A} := X\setminus U$ denote the complement of a subset $U$ of $X$.

\begin{lemma}\label{lem:coverage}
  Let $(D, B)$ be a surrounding pair in $X$ and $L\subseteq D$, $S\subseteq L\cap B$ be subsets so that $\ell: \hom_0(\overline{B}, \overline{D})\to \hom_0(\overline{S}, \overline{L})$ is induced by inclusion.

  If $\ell$ is injective then $D\setminus B\subseteq L$.
\end{lemma}
\begin{proof}
    Suppose, for the sake of contradiction, that $p$ is injective and there exists a point $x\in (D\setminus B)\setminus L$.
    Because $B$ surrounds $D$ in $X$ the pair $(D\setminus B, \overline{D})$ forms a separation of $\overline{B}$.
    Therefore, $\hom_0(\overline{B})\cong \hom_0(D\setminus B)\oplus \hom_0(\overline{D})$ so
    \[ \hom_0(\overline{B}, \overline{D})\cong \hom_0(D\setminus B). \]
    So $[x]$ is non-trivial in $\hom_0(\overline{B},\overline{D})\cong \hom_0(D\setminus B)$ as $x$ is in some connected component of $D\setminus B$.
    So we have the following sequence of maps induced by inclusions
    \[ \hom_0(\overline{B},\overline{D})\xrightarrow{f} \hom_0(\overline{B},\overline{D}\cup\{x\})\xrightarrow{g} \hom_0(\overline{S},\overline{L}).\]
    As $f[x]$ is trivial in $\hom_0(\overline{B},\overline{D}\cup\{x\})$ we have that $\ell[x] = (g\circ f)[x]$ is trivial, contradicting our hypothesis that $\ell$ is injective.
\end{proof}

\begin{lemma}\label{lem:cov_surrounds}
  Let $(D, B)$ be a surrounding pair in $X$ and $L\subseteq D$, $S\subseteq L\cap B$ be subsets so that $\ell: \hom_0(\overline{B}, \overline{D})\to \hom_0(\overline{S}, \overline{L})$ is induced by inclusion.

  If $\ell$ injective then $S$ surrounds $L$ in $D$.
\end{lemma}
\begin{proof}
  Suppose, for the sake of contradiction, that $S$ does not surround $L$ in $D$.
  Then there exists a path $\gamma : [0,1]\to\overline{S}$ with $\gamma(0)\in L\setminus S$ and $\gamma(1)\in D\setminus L$.
  By Lemma~\ref{lem:coverage} we know that $D\setminus B\subseteq L$, so $D\setminus B\subseteq L\setminus S$.

  Choose $x\in D\setminus B$ and $z\in \overline{D}$ such that there exist paths $\xi : [0,1]\to L\setminus S$ with $\xi(0) = x$, $\xi(1) = \gamma(0)$ and $\zeta : [0,1]\to \overline{D}\cup (D\setminus L)$ with $\zeta(0) = z$, $\zeta(1) = \gamma(1)$.
  $\xi, \gamma$ and $\zeta$ all generate chains in $C_1(\overline{S}, \overline{L})$ and $\xi + \gamma + \zeta = \gamma^*\in C_1(\overline{S}, \overline{L})$ with $\partial\gamma^* = x + z$.
  Moreover, $z$ generates a chain in $C_0(\overline{L})$ as $\overline{D}\subseteq\overline{L}$.
  So $x = \partial\gamma^* + z$ is a relative boundary in $C_0(\overline{S}, \overline{L})$, thus $\ell[x] = \ell[z]$ in $\hom_0(\overline{S}, \overline{L})$.
  However, because $B$ surrounds $D$, $[x]\neq [y]$ in $\hom_0(\overline{B}, \overline{D})$ contradicting our assumption that $\ell$ is injective.
\end{proof}
