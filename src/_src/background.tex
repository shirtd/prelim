% !TeX root = ../main.tex

\begin{definition}[Persistence Module]
  A \textbf{persistence module} $\S$ over $\R$ is an indexed family of vector spaces $\{S_\alpha\}$ and linear maps $\{s_\alpha^\beta : S_\alpha\to S_\beta\}$ such that $s^\gamma_\beta\circ s_\alpha^\beta = s_\alpha^\gamma$ whenever $\alpha\leq\beta\leq\gamma$ and $s_\alpha^\alpha$ is the identity on $S_\alpha$.
\end{definition}

% For a persistence module $\S$ we will also use $S[\alpha]$ to denote the vector space $S_\alpha$ and $s[\alpha,\beta]$ to denote the linear maps $s_\alpha^\beta$ when additional notation is required.

\begin{definition}[Persistence Module Homomorphism]
  A \textbf{homomorphism} $\Lambda$ between two $\R$-persistence modules $\S, \T$ is a collection of linear maps $\{\lambda_\alpha : S_\alpha\to T_\alpha\}$ such that the following diagram commutes for all $\alpha\leq\beta$.
  \begin{equation}\label{dgm:homomorphism}
    \begin{tikzcd}
      S_\alpha\arrow{r}{s_\alpha^\beta}\arrow{d}{\lambda_\alpha} &
      S_\beta\arrow{d}{\lambda_\beta}\\
      %
      T_\alpha\arrow{r}{t_\alpha^\beta} &
      T_\beta
  \end{tikzcd}\end{equation}
  The space of homomorphisms from $\S$ to $\T$ will be denoted $\Hom(\S, \T)$.
\end{definition}

\begin{definition}[Shifted Homomorphism]
  A \textbf{homomorphism of degree $\delta$} is a collection $F$ of linear maps $f_\alpha : U_\alpha\to S_{\alpha+\delta}$ such that the following diagram commutes for all $\alpha\leq\beta$.

  \begin{equation}\label{dgm:shifted_homomorphism}
    \begin{tikzcd}
      U_\alpha\arrow{r}{u_\alpha^\beta}\arrow{d}{f_\alpha} &
      U_\beta\arrow{d}{f_\beta}\\
      %
      S_{\alpha+\delta}\arrow{r}{s_{\alpha+\delta}^{\beta+\delta}} &
      S_{\beta +\delta}
  \end{tikzcd}\end{equation}
  The space of homomorphisms of degree $\delta$ from $\U$ to $\S$ will be denoted $\Hom^\delta(\U, \S)$.
\end{definition}

Noting that $\Hom^\delta(\U,\V)\subseteq\Hom^{\delta'}(\U,\V)$ for all $0\leq\delta\leq\delta'$ we will define particular shifted homomorphisms with the assumption that $\Hom^\delta(\U,\V) = \Hom(\U,\V)$ for $\delta = 0$.
For $\Gamma\in\Hom(\U,\V)$ let $\Gamma[\delta]\in\Hom^\delta(\U,\V)$ denote the homomorphism of degree $\delta$ defined as the family of linear maps
\[\{\gamma_\alpha[\delta] := v_\alpha^{\alpha+\delta}\circ \gamma_\alpha : U_\alpha\to V_{\alpha+\delta}\}.\]

% Using our alternative notation we will write
% \[\gamma[\alpha;\delta] := v[\alpha,\alpha+\delta]\circ\gamma[\alpha] : U[\alpha]\to V[\alpha+\delta]\]
% to denote a map $\gamma_\alpha[\delta]$ of $\Gamma[\delta]$.

% \begin{lemma}\label{lem:trans_shift}
%   If $F\in \Hom^\delta(\U,\S)$ and $F'\in\Hom^{\delta'}(\S,\S')$ then $F'\circ F\in\Hom^{\delta+\delta'}(\U,\S')$.
% \end{lemma}
% \begin{proof}
%   Because $F'\in \Hom^{\delta'}(\S,\S')$ we have $f_{\beta}'\circ s_\alpha^\beta = {s_{\alpha+\delta'}^{\beta+\delta'}}'\circ f_{\alpha+\delta'}'$ for all $\alpha\leq\beta$.
%   Because $F\in\Hom^\delta(\U,\S)$ we have $f_{\beta-\delta}\circ u_{\alpha-\delta}^{\beta-\delta} = s_\alpha^\beta\circ f_{\alpha-\delta}$ for all $\alpha\leq\beta$.
%   So
%   \begin{align*}
%     f_{\beta}'\circ (f_{\beta-\delta}\circ u_{\alpha-\delta}^{\beta-\delta})
%       &= (f_{\beta}'\circ s_\alpha^\beta)\circ f_{\alpha-\delta}\\
%       &= {s_{\alpha+\delta'}^{\beta+\delta'}}'\circ f_{\alpha+\delta'}'\circ f_{\alpha-\delta}
%   \end{align*}
%   so $F'\circ F\in\Hom^{\delta+\delta'}(\U,\S')$.
% \end{proof}

\begin{definition}[Interleaving]
  Two persistence modules $\U$ and $\S$ are \textbf{$\delta$-interleaved} if there exist homomorphisms $F\in\Hom^\delta(\U, \S)$ and $G \in\Hom^\delta(\S,\U)$ such that the following diagrams commute for all $\alpha$.

  \begin{minipage}{0.45\textwidth}
  \begin{equation}\label{dgm:interleaving1}
    \begin{tikzcd}
      U_{\alpha-\delta}\arrow{rr}{u_{\alpha-\delta}^{\alpha+\delta}}\arrow{dr}{f_{\alpha-\delta}} & &
      U_{\alpha+\delta}\\
      %
      & S_{\alpha}\arrow{ur}{g_\alpha} &
  \end{tikzcd}\end{equation}
  \end{minipage}
  \begin{minipage}{0.45\textwidth}
  \begin{equation}\label{dgm:interleaving2}
    \begin{tikzcd}
      & U_{\alpha}\arrow{dr}{f_\alpha} &\\
      %
      S_{\alpha-\delta}\arrow{rr}{s_{\alpha-\delta}^{\alpha+\delta}}\arrow{ur}{g_{\alpha-\delta}} & &
      S_{\alpha+\delta}
  \end{tikzcd}\end{equation}
  \end{minipage}
\end{definition}

\subsubsection{Interval Modules}

For an interval $I = [s,t)\subseteq \R$ let $I_+ := [t,\infty)$ and $I_- := (-\infty, s]$.
For $\alpha\leq\beta\in\R$ let
\[ F_\alpha^I := \begin{cases} \FF&\text{ if } \alpha\in I\\ 0 &\text{otherwise,}\end{cases}\ \text{ and }\ \ f_{\alpha,\beta}^I := \begin{cases} \id_\FF&\text{ if } \alpha,\beta\in I\\ 0&\text{otherwise.}\end{cases}.\]
An \textbf{interval module} is a persistence module $\FF^I$ defined to be the family of vector spaces $\{F_\alpha^I\}_{\alpha\in\R}$ along with linear maps $\{f_{\alpha,\beta}^I : F_\alpha^I\to F_\beta^I\}_{\alpha\leq\beta}$.

A \textbf{interval decomposition} of a persistence module $\S$ a collection $\I$ of interval $I\subseteq\R$ such that
\[ \S = \bigoplus_{I\in \I} \FF^I. \]
If such a decomposition exists $\S$ is said to be \textbf{decomposable}.

% \subsection{}
%
% \begin{definition}[Weak Interleaving]
%   For $I\in\Hom^{2\delta}(\U,\V)$ a pair $(F, M)\in \Hom^\delta(\U,\S)\times\Hom^\delta(\S,\V)$ is a \textbf{weak $\delta$-interleaving} of $I$ with $\S$ if $I = M\circ F$.
%   If $J\in\Hom^{\delta'}(\S,\T)$ and $(F,N)\in\Hom^\delta(\U,\S)\times\Hom^\delta(\T,\V)$ is a weak $\delta$-interleaving of $I$ with $J$ if $I = N\circ J\circ F$.
% \end{definition}
%
% \begin{lemma}\label{lem:left}
%   Let $I\in\Hom^{4\delta}(\U,\V)$, $J\in\Hom^{2\delta}(\S,\T)$ and suppose $(F, N)\in\Hom^{\delta}(\U,\S)\times\Hom^\delta(\T,\V)$ is a weak $\delta$-interleaving of $I$ with $J$.
%
%   If $(F', M')\in\Hom^\delta(\S,\S')\times\Hom^\delta(\S',\T)$ is a weak $\delta$-interleaving of $J$ with $\S'$ then
%   \[(F'\circ F, N\circ M')\in\Hom^{2\delta}(\U,\S')\times \Hom^{2\delta}(\S',\V)\]
%   is a weak $2\delta$-interleaving of $I$ with $\S'$.
% \end{lemma}
% \begin{proof}
%   By Lemma~\ref{lem:trans_shift} we have $F'\circ F\in\Hom^{2\delta}(\U,\S')$ and $N\circ M'\in \Hom^{2\delta}(\S',\V)$.
%   If $(F', M')$ is a weak $\delta$-interleaving of $J$ with $\S'$ then $J = M'\circ F'$.
%   By our hypothesis that $(F, N)$ is a weak $\delta$-interleaving of $I$ with $J$
%   \[ I = N\circ J\circ F = (N\circ M')\circ (F'\circ F).\]
%   We may therefore conclude that $(F'\circ F, N\circ M')\in\Hom^{2\delta}(\U,\S')\times \Hom^{2\delta}(\S',\V)$ is a weak $2\delta$-interleaving of $I$ with $\S'$.
% \end{proof}
%
% \begin{lemma}\label{lem:right}
%   Let $I\in\Hom^{2\delta}(\U,\V)$, $I'\in\Hom^{4\delta}(\U',\V')$ and suppose $(F, M)\in\Hom^\delta(\U,\S)\times\Hom^\delta(\S,\V)$ is a weak $\delta$-interleaving of $I$ with $\S$.
%
%   If $(F',N')\in\Hom^\delta(\U',\U)\times\Hom^\delta(\V,\V')$ is a weak $\delta$-interleaving of $I'$ with $I$ then
%   \[(F\circ F', N'\circ M)\in\Hom^{2\delta}(\U',\S)\times\Hom^{2\delta}(\S,\V')\]
%   is a weak $2\delta$-interleaving of $I'$ with $\S$.
% \end{lemma}
% \begin{proof}
%   By Lemma~\ref{lem:trans_shift} we have $F\circ F'\in \Hom^{2\delta}(\U',\S)$ and $N'\circ M\in \Hom^{2\delta}(\S,\V')$.
%   By our hypothesis that $(F, M)$ is a weak $\delta$-interleaving of $I$ with $\S$ we know $I = M\circ F$.
%   If $(F',N')$ is a weak $\delta$-interleaving of $I'$ with $I$ then $I' = N'\circ I\circ F'$.
%   \[ I' = N'\circ I\circ F' = (N'\circ M)\circ (F\circ F').\]
%   We may therefore conclude that $(F\circ F', N'\circ M)\in\Hom^{2\delta}(\U',\S)\times\Hom^{2\delta}(\S,\V')$ is a weak $2\delta$-interleaving of $I'$ with $\S$.
% \end{proof}
%
