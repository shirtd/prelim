% !TeX root = ../../main.tex

Our re-statement of the TCC in terms of a surrounding sub-levelset $B_\omega$ of a $c$-Lipschitz function $f : D\to \R$ sets us up with (most of) the machinery we need to approximate the persistent homology of the function $f$.
In this section we will shift our focus from confirming coverage to approximating the persistent homology of $f$ by a sample that satisfies the TCC.
In particular, we would like to extend the TCC to a condition which verifies when a sample can approximate the persistent homology of a function modulo a \emph{static} sub-levelset.

While persistent relative homology~\cite{todo} has been studied, interleaving relative modules requires interleaving pairs by pairs of shifted homomorphisms which, ideally, are induced by inclusions.
However, taking the persistent homology relative to a \emph{static} sub-levelset $B_\omega$ without asserting that the corresponding approximation is homotopy equivalent.
Moreover, the TCC only confirms coverage of a subset $D\setminus B_\omega$ so we cannot even assume we have coverage of some subset of $B_\omega$.

We will first introduce the notion of an extension which will provide us with maps on relative homology induced by inclusion via excision.
However, even then, a map that factors through our pair $(D, B_\omega)$ is not enough to prove an interleaving of persistence modules by inclusion directly.
To address this we impose conditions on sub-levelsets near $B_\omega$ which generalize the assumptions made in the TCC on maps induced by the inclusions
\[ D\setminus B_{\omega+c(\delta+\zeta)}\hookrightarrow D\setminus B_\omega\hookrightarrow D\setminus B_{\omega-c(\delta+\zeta)}\]
on $0$-dimensional homology, to assumptions on maps induced by the corresponding inclusions
\[ B_{\omega-c(\delta+\zeta)}\hookrightarrow B_\omega\hookrightarrow B_{\omega+c(\delta+\zeta)}\]
on homology in all dimensions $k$.
We will then introduce image modules and partial interleavings before proving the interleaving in general.
Finally, in the next section we set up notation and prove the interleaving in the geometric context introduced in section~\ref{sec:geometric}.
