% !TeX root = ../main.tex

Throughout, let $\X$ denote an orientable $d$-manifold and $D\subset\X$ a compact subspace.
For a $c$-Lipschitz functon $f : D\to \R$ let $B_\alpha := f^{-1}((-\infty,\alpha])$ denote the $\alpha$-sublevel set of $f$ for all $\alpha\in\R$.
We will select a sublevel set $B_\omega$ to serve as our boundary.
Specifically, we require that $B_\omega$ \emph{surrounds} $D$, where the notion of a surrounding set is defined formally in Section~\ref{sec:tcc}.
This distinction allows us to generalize the standard proof of the TCC to properties of surrounding pairs.

Our sample will be denoted $P$, and the subset of points sampling $B_\alpha$ will be denoted $Q_\alpha := P\cap B_\alpha$.
As a minimal assumption we require that every component of $D\setminus B_\omega$ contains a point in $P$.
We also make additional technical assumptions on $P$ with respect to the pair $(D, B_\omega)$ (see Lemma~\ref{lem:duality_apply}, Appendix~\ref{apx:duality}).
% We make the following assumptions on our sample $P$ with respect to the pair $(D,B_\omega)$ for a constant $\delta < \varrho_D/4$:
% \begin{itemize}
%   \item Each component of $D\setminus B_\omega$ contains a point in $P$,
%   \item $P^\delta\subset \intr_\X(D)$,
%   \item $D\setminus Q_{\omega-2c\delta}^\delta$, $D\setminus Q_{\omega+c\delta}^\delta$, and $D\setminus P^\delta$ are locally path connected.
% \end{itemize}
% This last assumption is a technicality that is required to apply Alexander Duality (see Lemma~\ref{lem:duality_apply}, Appendix~\ref{apx:duality}).
%
Theorem~\ref{thm:algo_tcc} states that for any sublevel set $B_\omega$ that surrounds $D$ such that
\begin{enumerate}
  \item $\hom_0(D\setminus B_{\omega+5c\delta}\hookrightarrow D\setminus B_\omega)$ is \emph{surjective},
  \item $\hom_0(D\setminus B_\omega\hookrightarrow D\setminus B_{\omega-3c\delta})$ is \emph{injective}.
\end{enumerate}
the condition
\[ \rk~\hom_d(\rips^\delta(P, Q_{\omega - 2c\delta})\hookrightarrow \rips^{2\delta}(P, Q_{\omega+c\delta}))\geq \hom_0(\rips^\delta(P\setminus Q_{\omega-2c\delta})) \]
implies not only that $D\setminus B_\omega\subseteq P^\delta$, but also that $Q_{\omega-2c\delta}^\delta$ surrounds $P^\delta$ in $D$.
\footnote{We state this result using constants that will be used to prove the interleaving.
  The statement of Theorem~\ref{thm:algo_tcc} parameterizes the region around $\omega$ in terms of $\zeta\geq\delta$ as $[\omega-c(\delta+\zeta),\omega+c(\delta+\zeta)]$.}

In other words, this formulation of the TCC states that our approximation by a nested pairs of Rips complexes captures the homology of the pair $(D,B_\omega)$ in a specific way.
% We can use this fact to construct an interleaving between an approximation of the function $f$ by Rips complexes and filtration $\{(D_\alpha\cup B_\omega, B_\omega)\}_{\alpha\in\R}$.
% The resulting (relative) persistent homology modules capture the persistent homology of the sublevel set filtration relative to a specific sublevel set $B_\omega$.
% This sublevel set, along with its approximation by the inclusion $Q_{\omega - 2c\delta}^\delta\hookrightarrow Q_{\omega+c\delta}^\delta$, are \emph{static} throughout the filtration.
% To deal with the challenges of interleaving static elements we generalize the regularity assumptions on the zero dimensional homology of the superlevel sets $D\setminus B_{\omega+5c\delta}$ and $D\setminus B_{\omega-3c\delta}$ to assumptions about the corresponding \emph{sublevel} sets $B_{\omega+5c\delta}$ and $B_{\omega-3c\delta}$.
We can use this fact to prove an interleaving with the filtration $\{(D_\alpha\cup B_\omega, B_\omega)\}_{\alpha\in\R}$.
This is done by generalizing our regularity assumptions near $D\setminus B_\omega$ in a way that allows us to interleave persistence modules relative to a static submodule.
Specifically, given a sample $P$ and constants $\omega\in\R$, $\delta < \varrho/4$ that satisfy the TCC, Theorem~\ref{thm:interleaving_main_2} states that whenever
\begin{enumerate}
  \item $\hom_k(B_{\omega-3c\delta}\hookrightarrow B_\omega)$ is \emph{surjective} and
  \item $\hom_k(B_\omega\hookrightarrow B_{\omega+5c\delta})$ is an \emph{isomorphism}
\end{enumerate}
for all $k$ the persistent homology modules of
\[ \{\rips^{2\delta}(P_\alpha\cup Q_{\omega-2c\delta}, Q_{\omega-2c\delta})\hookrightarrow \rips^{4\delta}(P_\alpha\cup Q_{\omega+c\delta}, Q_{\omega+c\delta})\}_{\alpha\in\R}\]
and $\{(D_\alpha\cup B_\omega, B_\omega)\}_{\alpha\in\R}$ are $4c\delta$ interleaved.

The main challenges we face come from the fact that the sublevel set $B_\omega$ and our approximation by the inclusion $\rips^{2\delta}(Q_{\omega-2c\delta})\hookrightarrow \rips^{4\delta}(Q_{\omega+c\delta})$ remain \emph{static} throughout.
Using the fact that $Q_{\omega-2c\delta}^\delta$ surrounds $P^\delta$ in $D$ we define an \emph{extension} $(D,\E Q_{\omega-2c\delta}^\delta)$ of the pair $(P^\delta, Q_{\omega-2c\delta}^\delta)$ that has isomorphic relative homology by excision.
These extensions give us the following sequence of inclusion maps
\[ B_{\omega-3c\delta}\hookrightarrow \E Q_{\omega-2c\delta}^{2\delta}\hookrightarrow B_\omega\hookrightarrow \E Q_{\omega+c\delta}^{4\delta}\hookrightarrow B_{\omega+5c\delta}\]
that can be used along with our regularity assumptions in terms to prove the interleaving using so-called \emph{partial interleavings} of image persistence modules.

\paragraph{Outline of Sections~\ref{sec:tcc} and~\ref{sec:middle}}

We will begin with our reformulation of the TCC in Section~\ref{sec:tcc}.
This requires introducing the notion of a surrounding set before proving the Geometric TCC, and the computable Algorithmic TCC in Theorem~\ref{thm:algo_tcc}.
Section~\ref{sec:middle} formally introduces extensions, image persistence modules, and partial interleavings which will be used in the proof of the interleaving in Theorem~\ref{thm:interleaving_main_2}.
% In Section~\ref{sec:truncations} we address the meaning of the persistent homology relative to a sublevel set and relate it to the sublevel set filtration as a \emph{truncation}.
% Finally, Section~\ref{sec:experiments} investigates the relationships between the relative, truncated, and restricted persistence diagrams through a number of experiments.

% Specifically, we will select constants $\delta\leq\zeta < \varrho_D/2$ such that, granted the assumptions detailed~\textbf{TODO}, coverage of  can be confirmed by verifying
%
% Theorem~\ref{thm:algo_tcc} uses this condition, along with the established properties of surrounding pairs, to confirm that $Q_{\omega-c\zeta}^\delta$ surrounds $P^\delta$ in $D$ in addition to coverage.
% This allows us to apply excision to \emph{extend} the pair $(P^\delta, Q_{\omega-c\zeta}^\delta)$ to the pair $(D, \E Q_{\omega-c\zeta}^\delta)$ surrounding $D$ in $\X$.
% The resulting inclusion maps facilitate our interleaving of the persistent (relative) homology of $f$ on the pair $(D, B_\omega)$ to that of our sample.
%
% Section~\ref{sec:middle} formally establishes the notion of an extension of a surrounding pair before introducing image modules, and \emph{partial interleavings}.
% These partial interleavings are required due to the static natto relate our approximated boundary \textbf{fuck}%$Q_{\omega-}
%
% At this point we have a set of points $P$ that covers a subset (a \emph{superlevel}-set of $f$) $D\setminus B_\omega$ at scale $\delta$ and a subcover provided by a set of points $Q_{\omega-c\zeta}$ that \emph{surrounds} $P^\delta$ in $D$ at scale $\delta$.
% We can show that the fact that $Q_{\omega-c\zeta}^\delta$ surrounds $P^\delta$ implies that we can apply excision in a way that provides inclusion maps $(D,B_{\omega-c(\delta-\zeta)})\hookrightarrow (D, \E Q_{\omega-c\zeta})\hookrightarrow (D, B_\omega)$ such that the homology of $(D, \E Q_{\omega-c\zeta})$ is isomorphic to that of $(P^\delta, Q_{\omega-c\zeta}^\delta)$\textbf{fuck}
