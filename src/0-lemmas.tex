% !TeX root = ../main.tex

% \begin{lemma}[Splitting Lemma (Hatcher p. 147)]\label{lem:splitting}
%   For a short exact sequence \[0\to A\xrightarrow{i} B\xrightarrow{j} C\to 0\] of abelian groups the following statements are equivalent
%   \begin{enumerate}
%     \item There is a homomorphism $p: B\to A$ such that $p\circ i = \mathbf{Id}_A$.
%     \item There is a homomorphism $s: C\to B$ such that $j\circ s = \mathbf{Id}_C$.
%     \item There is an isomorphism $B\cong A\oplus C$ making the commutative diagram below, where the maps in the lower row are the obvious ones $a\mapsto (a, 0)$ and $(a,c)\mapsto c$.
%
%     \[\begin{tikzcd}[column sep=small, row sep=small]
%               &                       & B\ar[dd, "\cong"]\ar[dr,"j"]  &         & \\
%       0\ar[r] & A\ar[ur, "i"]\ar[dr]  &                               & C\ar[r] & 0\\
%               &                       & A\oplus C\ar[ur]              &         &
%     \end{tikzcd}\]
%   \end{enumerate}
% \end{lemma}

\begin{lemma}[\textbf{The Five-Lemma} (Hatcher p. 129)]\label{lem:five}
  In a commutative diagram of abelian groups as below, if the two rows are exact and $\alpha,\beta,\delta$, and $\e$ are isomorphisms then $\gamma$ is an isomorphism.
  \[\begin{tikzcd}
      A\ar[r, "i"]\ar[d, "\alpha"]
    & B\ar[r, "j"]\ar[d, "\beta"]
    & C\ar[r, "k"]\ar[d, "\gamma"]
    & D\ar[r, "\ell"]\ar[d, "\delta"]
    & E\ar[d, "\e"]\\
    %
      A'\ar[r, "i'"]
    & B'\ar[r, "j'"]
    & C'\ar[r, "k'"]
    & D'\ar[r, "\ell'"]
    & E'\\
  \end{tikzcd}\]

  \begin{itemize}
    \item If $\beta$ and $\delta$ are surjective and $\e$ is injective then $\gamma$ is surjective.
    \item If $\beta$ and $\delta$ are injective and $\alpha$ is surjective then $\gamma$ is injective.
  \end{itemize}
\end{lemma}

\subsection{For Algorithmic TCC}

For a pair $(A, B)$ in a topological space $X$ and any $R$ module $G$ let $\hom^k(A, B; G)$ denote the \textbf{singular cohomology} of $(A,B)$ (with coefficients in $G$) as a vector space.
Let $\hom^k_c(A, B; G)$ denote the corresponding \textbf{singular cohomology with compact support}, where $\hom^k_c(A, B; G)\cong \hom^k(A, B; G)$ for any compact pair $(A,B)$.

\begin{theorem}[\textbf{Universal Coefficient Theorem} (Munkres p. 323, Corollary 53.2)]\label{thm:univ_coef}
  Let $(A,B)$ be a topological pair.
  Then for all $k$ and any abelian group $G$ there is a natural exact sequence
  \[ 0\to\mathrm{Ext}(\hom_{k-1}(A, B), G)\to \hom^k(A, B; G)\xrightarrow{\nu} \Hom(\hom_k(A, B), G)\to 0.\]
  This sequence splits, but not naturally.
\end{theorem}

The following corollary follows from the Universal Coefficient Theorem for singular homology (and cohomology) as vector spaces over a field $\FF$, as the dual vector space $\Hom(\hom_k(A, B), \FF)$ is isomorphic to $\hom_k(A, B; \FF)$ for any finitely generated $\hom_k(A, B)$.\footnote{Reference/verify.}

\begin{corollary}\label{cor:univ_coef}
  For a topological pair $(A, B)$ and a field $\FF$ such that $\hom_0(A, B)$ is finitely generated there is a natural isomorphism
  \[\nu : \hom^0(A, B; \FF)\to \hom_0(A, B; \FF).\]
\end{corollary}

Let $\overline{\hom}^k(A, B; G)$ be the \textbf{Alexander-Spanier cohomology} of the pair $(A,B)$, defined as the limit of the direct system of neighborhoods $(U,V)$ of the pair $(A, B)$.
Let $\overline{\hom}^k_c(A, B; G)$ denote the corresponding \textbf{Alexander-Spanier cohomology with compact support} where $\overline{\hom}^k_c(A, B; G)\cong\overline{\hom}^k(A, B; G)$ for any compact pair $(A, B)$.

\begin{theorem}[\textbf{Alexander-Poincar\'e-Lefschetz Duality} (Spanier, Theorem 6.2.17)]\label{thm:alexander}
  Let $X$ be an orientable $d$-manifold and $(A, B)$ be a compact pair in $X$.
  Then for all $k$ and $R$ modules $G$ there is a (natural) isomorphism
  \[\lambda : \hom_k(X\setminus B, X\setminus A; G)\to \overline{\hom}^{d-k}(A, B; G).\]
\end{theorem}

A space $X$ is said to be \textbf{homologically locally connected in dimension $n$} if for every $x\in X$ and neighborhood $U$ of $x$ there exists a neighborhood $V$ of $x$ in $U$ such that $\tilde{\hom}_n(V)\to\tilde{\hom}_n(U)$ is trivial for $k\leq n$.

\begin{lemma}[Spanier p. 341, Corollary 6.9.6]\label{lem:alexander_iso}
  Let $A$ be a closed subset, homologically locally connected in dimension $n$, of a Hausdorff space $X$, homologically locally connected in dimension $n$.
  If $X$ has the property that every open subset is paracompact, $\mu : \overline{\hom}_c^k(X,A; G)\to \hom_c^k(X, A; G)$ is an isomorphism for $k\leq n$ and a monomorphism for $q = n+1$.
\end{lemma}

In the following we will assume homology (and cohomology) over a field $\FF$.

\begin{lemma}\label{cor:alexander_iso}
  Let $X$ be an orientable $d$-manifold and $(A,B)$ a compact pair of locally path connected subspaces in $X$.
  Then
  \[\xi : \hom_d(X\setminus B, X\setminus  A)\to \hom_0(A, B)\]
  is a natural isomorphism.
\end{lemma}
\begin{proof}
  Because $X$ is orientable and $(A,B)$ are compact $\lambda : \hom_d(X\setminus B, X\setminus A)\to \overline{\hom}^{0}(A, B)$ is an isomorphism by Theorem~\ref{thm:alexander}.
  Note that
  Moreover, because every subset of $X$ is (hereditarily) paracompact every open set in $A$, with the subspace topology, is paracompact.
  For any neighborhood $U$ of a point $x$ in a locally path connected space there must exist some neighborhood $V\subset U$ of $x$ that is path connected in the subspace topology.
  As $\tilde{\hom}_0(V) = 0$ for any nonempty, path connected topological space $V$ (see Spanier p. 175, Lemma 4.4.7) it follows that $A$ (resp. $B$) are homologically locally connected in dimension $0$.
  Because $(A,B)$ is a compact pair the singular and Alexander-spanier cohomology modules of $(A,B)$ with compact support are isomorphic to those without, thus $\mu:\overline{\hom}^{0}(A, B)\to \hom^0(A, B)$ is an isomorphism.
  By Corollary~\ref{cor:univ_coef} we have a natural isomorphism $\nu : \hom^0(A, B)\to\hom_0(A, B)$ thus the composition $\xi := \nu\circ\mu\circ\lambda : \hom_d(X\setminus B, X\setminus  A)\to \hom_0(A, B)$ is a natural isomorphism.
\end{proof}


% % \begin{theorem}[\textbf{Alexander-Poincar\'e Duality} (Julian et. al.~\cite{julian83alexander}, Theorem 5.1)]\label{thm:alexander}
% %   Let $K$ be an abstract simplicial complex that is a combinatorial oriented $d$-manifold.
% %   Let $L$ be a subcomplex of some refinement of $K$ and $M$ be a subcomplex of $L$.
% %   Let $\overline{L}$ and $\overline{M}$ denote the complements of $L$ and $M$ as subcomplexes of $K$ that do not share vertices with the original complexes.
% %   Then for all $k$ there is a natural isomorphism
% %   \[ \hom^k(L, M)\to \hom_{d-k}(\overline{M},\overline{L}). \]
% % \end{theorem}
%
% % \begin{corollary}
% %   Let $X$ be a topological space and $D$ be a compact subspace of $X$.
% %   Let $(U, V)$ be a topological pair of spaces in $D$ and suppose there exists a triangulation $\Delta X$ of $X$ such that there exists triangulation $\Delta U$ of $U\subset$ that is a subcomplex of some refinement of $\Delta X$ and a triangulation $\Delta V$ of $V$ that is a subcomplex of $\Delta U$.
% %   Then for all $k$ there is a natural isomorphism
% %   \[ \hom^k(U, V)\to\hom_{d-k}(D\setminus U, D\setminus V).\]
% % \end{corollary}
%
%
% \begin{corollary}\label{cor:univ_coef}
%   Let $(A, B)$ be a topological pair and $\FF$ be a field such that $\hom_k(A, B; \FF)$ is finitely generated.
%   Then there is a natural isomorphism
%   \[\hom^k(A, B; \FF)\to \hom_k(A, B; \FF).\]
% \end{corollary}
% \begin{proof}
%   As $\mathrm{Ext}(\Hom(A, B), \FF) = 0$ for any field $\FF$ the map
%   \[\hom^k(A, B; \FF)\to \Hom(\hom_k(A, B), \FF)\]
%   in the natural short exact sequence provided by Theorem~\ref{thm:univ_coef} is a natural isomorphism.
%   The result follows from the fact that $\hom_k(A, B; \FF)$ is finitely generated, and is therefore isomorphic to the dual vector space $\Hom(\hom_k(A, B), \FF)$.
% \end{proof}
%
% % \begin{theorem}[\textbf{Universal Coefficient Theorem} (Munkres p. 337, Corollary 56.4)]\label{thm:univ_coef}
% %   Let $(A,B)$ be a topological pair such that $\hom_k(A, B)$ is finitely generated for all $k$.
% %   Then for all $k$ and any abelian group $G$ there is a natural exact sequence
% %   \[ 0\to\mathrm{Ext}(\hom^{k+1}(A, B), G)\to \hom_k(A, B; G)\to \Hom(\hom^k(A, B), G)\to 0.\]
% %   This sequence splits, but not naturally.
% % \end{theorem}
%
% \begin{theorem}
%   Let $X$ be a topological space and $D$ be a compact subspace of $X$.
%   Suppose there exists a triangulation $\Delta X$ of $X$ that is a combinatorial oriented $d$-manifold.
%   Let $(U, V)$ be a topological pair of spaces in $D$ and $\FF$ be a field such that $\hom_k(U,V;\FF)$ is finitely generated for all $k$.
%
%   If there exists a pair of triangulations $(\Delta U, \Delta V)$ of the pair $(U, V)$ such that $\Delta U$ is a subcomplex of some refinement of $\Delta X$ then there is a natural isomorphism
%   \[ \hom_d(U, V; \FF)\to \hom_0(X\setminus V, X\setminus U; \FF).\]
% \end{theorem}
% \begin{proof}
%   By Theorem~\ref{thm:alexander} we have a natural isomorphism
%   \[ \hom^d(\Delta U, \Delta V; \FF)\to \hom_{0}(\overline{\Delta V}, \overline{\Delta U}; \FF) \]
%   where $\overline{\Delta V}$ and $\overline{\Delta U}$ denote the complements of $\Delta V$ and $\Delta U$ as subcomplexes of $\Delta X$ that do not share vertices with their respective original complexes.
%   \textbf{TODO}\footnote{$\hom^d(U, V;\FF)\cong \hom^d(\Delta U, \Delta V;\FF),\ \hom_{0}(\overline{\Delta V}, \overline{\Delta U}; \FF) \cong \hom_0(X\setminus V, X\setminus U; \FF).$}
%
%   Because $\hom_d(U, V; \FF)$ is finitely generated $\hom_d(U, V;\FF)\cong\hom^d(U, V; \FF)$ by Corollary~\ref{cor:univ_coef}.
%   It follows that the composition
%   \[\hom_d(U, V; \FF)\to \hom^d(U,V;\FF)\to\hom_0(X\setminus V, X\setminus U; \FF)\]
%   is a natural isomorphism as desired.
% \end{proof}
% % \begin{proof}
% %   \begin{itemize}
% %     \item By Theorem~\ref{thm:univ_coef} we have a short exact sequence
% %       \[ 0\to\mathrm{Ext}(\hom^{d+1}(\Delta U, \Delta V), G)\to \hom_d(\Delta U, \Delta V; G)\to \Hom(\hom^d(\Delta U, \Delta V), G)\to 0\]
% %       for any abelian group $G$.
% %       Because $\Delta U,\Delta V$ are subcomplexes of the combinatorial $d$-manifold $\Delta X$, $\hom^{d+1}(\Delta U, \Delta V) = 0$, so $\hom_d(\Delta U, \Delta V; G)\to \Hom(\hom^d(\Delta U, \Delta V), G)$ is an isomorphism.
% %     \item By Theorem~\ref{thm:alexander} we have a natural isomorphism $\hom^d(\Delta U, \Delta V)\to \hom_0(\overline{\Delta V},\overline{\Delta U})$.
% %       Therefore, because we have natural\footnote{\textbf{TODO} natural short exact $\implies$ natural isomorphism.} isomorphisms $\hom_d(\Delta U, \Delta V; G)\to \Hom(\hom^d(\Delta U, \Delta V), G)$ for all abelian groups $G$, we have a natural isomorphism (\textbf{TODO} prove it).
% %       \[\hom_d(\Delta U, \Delta V)\to \hom_0(\overline{\Delta V},\overline{\Delta U}).\]
% %     \item Because $\hom_d(U, V)\cong \hom_d(\Delta U, \Delta V)$ and $\hom_0(X\setminus V, X\setminus U)\cong \hom_0(\overline{\Delta V},\overline{\Delta U})$ (\textbf{TODO} prove it\footnote{$(X\setminus V, X\setminus U)$ or $(D\setminus V, D\setminus U)$?}), we have a natural isomorphism
% %     \[ \hom_d(U, V)\to \hom_0(X\setminus V, X\setminus U). \]
% %   \end{itemize}
% % \end{proof}
%
%
% %
% % \begin{theorem}[\textbf{Alexander Duality} (Spanier p. 296, Theorem 6.2.17)]
% %   Let $U$ be an orientation over $R$ of an $d$-manifold $X$ and let $(A, B)$ be a compact pair in $X$.
% %   Then for all $k$ and $R$ modules $G$ there is a natural isomorphism
% %   \[ \hom_k(X\setminus B, X\setminus A; G)\to\overline{\hom}^{d-k}(A, B; G).\]
% % \end{theorem}
% %
% % \begin{lemma}
% %   Let $U$ be an orientation over $R$ of an $d$-manifold $X$ and let $(A, B)$ be a compact pair in $X$ such that $\hom_k(A, B)$ is finitely generated for all $k$.
% %   Then for all $R$ modules $G$ there is a natural isomorphism
% %   \[ \hom_0(X\setminus B, X\setminus A; G)\to\hom_d(A, B; G). \]
% % \end{lemma}
% % \begin{proof}
% %   \textbf{TODO}
% % \end{proof}

\subsection{For Relative Persistent Nerve Lemma}

\begin{lemma}[\textbf{Persistent Nerve Lemma} (Chazal et. al.~\cite{chazal08towards}, Lemma 3.4)]\label{lem:pers_nerve}
  Let $X\subseteq X'$ be two paracompact spaces, and let $\cU = \{U_i\}_{i\in I}$ and $\mathcal{U}' = \{U_i'\}_{i\in I}$ be good open covers of $X$ and $X'$, respectively, based on some finite parameter set $I$, such that $U_i\subseteq U_i'$ for all $i\in I$.
  Then there exist homotopy equivalences of pairs $\N\cU\to X$ and $\N\cU'\to X'$ that commute with the canonical inclusions $X \hookrightarrow X'$ and $\N\cU\hookrightarrow \N\cU'$ at the homology and homotopy levels.
\end{lemma}

The following lemma is a straightforward application of the Persistent Nerve Lemma to a filtration.

\begin{lemma}\label{lem:pers_nerve_filt}
  Let $X$ be a paracompact space and $\cU = \{U_i\}_{i\in I}$ be a good open covers of $X$ based on some finite parameter set $I$.
  Let $\F = \{F_\alpha\subseteq X\}_{\alpha\in\R}$ be a filtration in $X$ and define $\cU_\alpha := \{X_\alpha\cap U_i\}_{i\in I}$ for all $\alpha\in\R$.
  Then for all $\alpha\leq\beta$ there exist homotopy equivalences of pairs $\N\cU_\alpha\to F_\alpha$ and $\N\cU_\beta\to F_\beta$ that commute with the canonical inclusions $F_\alpha \hookrightarrow F_\beta$ and $\N\cU_\alpha\hookrightarrow \N\cU_\beta$ at the homology and homotopy levels.
\end{lemma}
\begin{proof}
  Because $\cU$ is a good open cover of $X$ we know that $\cU_\alpha$ is a good open cover of $F_\alpha$ for all $\alpha\in\R$.\footnote{\textbf{TODO} prove it.}
  Moreover, because $\F$ is a filtration $F_\alpha\subseteq F_\beta$ for all $\alpha\leq\beta$.
  So $F_\alpha\cap U_i\subseteq X_\beta\cap U_i$ for all $i\in I$, $\alpha\leq\beta$.
  The result therefore follows from Lemma~\ref{lem:pers_nerve}.
\end{proof}

\begin{definition}[Compatible Filtrations (Skraba et. al.~\cite{skraba14approximating})]
  Two filtrations $\A = \{A_\alpha\}$ and $\F = \{F_\alpha\}$ are said to be \textbf{compatible} if for all $\alpha\leq\beta$ the following diagram commutes
  \begin{equation}\label{dgm:compatible}
    \begin{tikzcd}
      A_\alpha\arrow{r}\arrow{d} &
      F_\alpha\arrow{d}\\
      %
      A_\beta\arrow{r} &
      F_\beta.
    \end{tikzcd}
  \end{equation}
  In order to specify an order we will refer to a pair of filtrations $(\F, \A)$ as a \textbf{compatible pair of filtrations}.
\end{definition}

\begin{theorem}[Skraba et. al.~\cite{skraba14approximating}, Theorem 1]\label{thm:rel_interleave}
  If $(\F, \A)$ and $(\F', \A')$ are compatible pairs of filtrations such that the $k$th persistent homology modules of $\F$ and $\F'$ (resp. $\A$ and $\A'$) are $\e$-interleaved then the corresponding relative modules of $\{(F_\alpha, A_\alpha)\}$ and $\{(X_\alpha', Y_\alpha')\}$ are $\e$-interleaved.
\end{theorem}

Because two isomorphic persistence modules are $0$-interleaved we have the following corollary of Theorem~\ref{thm:rel_interleave}.

\begin{corollary}\label{cor:rel_interleave_iso}
  If $(\F, \A)$ and $(\F', \A')$ are compatible pairs of filtrations such that the $k$th persistent homology modules of $\F$ and $\F'$ (resp. $\A$ and $\A'$) are isomorphic then the relative modules $\{(F_\alpha, A_\alpha)\}$ and $\{(X_\alpha', Y_\alpha')\}$ are isomorphic.
\end{corollary}

\begin{lemma}[\textbf{Relative Persistent Nerve Lemma}]
  Let $X$ be a paracompact space and $Y\subseteq X$.
  Let $\cU = \{U_i\}_{i\in I}$ and $\cV = \{V_i\}_{i\in I}$ be good open covers of $X$ and $Y$, respectively, based on some finite parameter set $I$, such that $V_i\subseteq U_i$ for all $i\in I$.
  Let $\F = \{F_\alpha\}$ be a filtration in $X$ and $A_\alpha := Y\cap F_\alpha$ so that $\A = \{A_\alpha\}$ is a filtration in $Y$.
  Let $\cU_\alpha := \{X_\alpha\cap U_i\}$ and $\cV_\alpha := \{A_\alpha\cap V_i\}$.
  Then the $k$th (relative) persistent homology modules of $\{(F_\alpha, A_\alpha)\}$ and $\{(\N\cU_\alpha, \N\cV_\alpha)\}$ are isomorphic.
\end{lemma}
\begin{proof}
  Because
  \[\N\cU_\alpha = \left\{\bigcap_{i\in S} F_\alpha\cap U_i\mid S\subseteq I\right\} = \{\sigma\cap F_\alpha\}_{\sigma\in \N\cU}\]
  and $F_\alpha\subseteq F_\beta$ for all $\alpha\leq\beta$ we have inclusions $\N\cU_\alpha\hookrightarrow\N\cU_\beta$ for all $\alpha\leq\beta$.
  Similarly, because
  \[\N\cV_\alpha = \{\tau\cap A_\alpha\}_{\tau\in \N\cV} = \{\sigma\cap (Y\cap F_\alpha)\}_{\sigma\in\N\cU}\]
  we have inclusions $\N\cV_\alpha\hookrightarrow\N\cV_\beta$ for all $\alpha\leq\beta$ and, because $V_i\subseteq U_i$ for all $i\in I$, we have inclusions $\N\cV_\alpha\hookrightarrow \N\cU_\alpha$ for all $\alpha\in\R$.
  Letting $\N\cU_\F := \{\N\cU_\alpha\}$ and $\N\cV_\A := \{\N\cV_\alpha\}$ it follows that $(\N\cU_\F, \N\cV_\A)$ is a compatible pair of filtrations.
  Moreover, by Lemma~\ref{lem:pers_nerve_filt}, the following diagrams commute for all $\alpha\leq\beta$.
  \[\begin{tikzcd}
      \N\cU_\alpha \arrow{r}\arrow{d} &
      \N\cU_\beta \arrow{d}\\
      %
      X_\alpha\arrow{r} &
      X_\beta,
    \end{tikzcd}\hspace{10ex}
    \begin{tikzcd}
      \N\cV_\alpha \arrow{r}\arrow{d} &
      \N\cV_\beta \arrow{d}\\
      %
      Y_\alpha\arrow{r} &
      Y_\beta
    \end{tikzcd}\]
  where the horizontal maps are canonical inclusions and vertical maps are homotopy equivalences.
  Letting $\N\U_\F$, $\X_\F$ and $\N\V_\A$, $\Y_\A$ denote the $k$th persistent homology modules of $\N\cU_\F$, $\F$ and $\N\cV_\A$, $\A$, respectively, the homotopy equivalences induce isomorphisms on homology for all $\alpha\in\R$, which constitute isomorphisms of persistence modules.
  % the following isomorphisms of persistence modules
  % \[ \N_\F\in\Hom(\N\U_\F, \X_\F),\text{ and } \N_\A\in\Hom(\N\V_\A,\Y_\A).\]
  % \[ \N_\F := \{n_{\cU_\alpha} : \hom_k(\N\cU_\alpha)\to \hom_k(F_\alpha)\}\in\Hom(\N\U_\F, \X_\F),\text{ and }\]
  % \[ \N_\A : = \{n_{\cV_\alpha} : \hom_k(\N\cV_\alpha)\to \hom_k(A_\alpha)\}\in\Hom(\N\V_\A,\Y_\A).\]
  The result therefore follows from Corollary~\ref{cor:rel_interleave_iso}.
\end{proof}

% \begin{lemma}[Persistent Nerve Lemma]\label{lem:rel_pers_nerve}% (Lemma 3.4, Chazal~\cite{chazal08towards})]\label{lem:rel_pers_nerve}
%   Let $X\subseteq X'$ be two paracompact spaces and $Y\subseteq Y'$ be two subspaces $Y\subseteq X$, $Y'\subseteq X'$.
%   Let $\mathcal{U} = \{U_a\}_{a\in A}$ and $\mathcal{U}' = \{U_a'\}_{a\in A}$ be good open covers of $X$ and $X'$, respectively, such that $U_a\subseteq U_a'$ for all $a\in A$.
%   Let $\cV = \{V_a\}_{a\in A} \subseteq \mathcal{U}$ and $\cV' = \{V_a'\}_{a\in A}\subseteq \mathcal{U}'$ be good open subcovers of $Y$ and $Y'$, respectively, such that $V\subseteq V'$ for all $a\in A$.
%   Then there exist homotopy equivalences of pairs $(\N\mathcal{U}, \N\cV)\to (X, Y)$ and $(\N\mathcal{U}', \N\cV')$ that commute with the canonical inclusions of pairs $(X, Y)\hookrightarrow (X', Y')$ and $(\N\mathcal{U}, \N\cV)\hookrightarrow (\N\mathcal{U}', \N\cV')$ at the homology and homotopy levels.
% \end{lemma}

% \begin{theorem}[Alexander Duality]\label{thm:alexander}
%   If $D$ is a compact, locally contractible, nonempty, proper subspace of $S^d$ then for all $k$ there is an isomorphism
%   \[ \Gamma_D^k : \tilde{\hom}_k(D)\to \tilde{\hom}^{d-k-1}(S^d\setminus D). \]
%
%   If $(D, B)$ is a pair of such subspaces of $S^d$ then for all $k$ there is an isomorphism
%   \[ \Gamma_{(D,B)}^k : \tilde{\hom}_k(D, B)\to \tilde{\hom}^{d-k}(S^d\setminus B, S^d\setminus D). \]
% \end{theorem}

% \begin{lemma}[Lemma 3.2 from~\cite{chazal08towards}]\label{lem:sandwich}
%     Given a sequence $A\to B\to C\to D\to E\to F$ of homomorphisms between finite-dimensional vector spaces, if $\rk(A\to F) = \rk(C\to D)$ then this quantity also equals the rank of $B\to E$.
%     Similarly, if $A\to B\to C\to E\to F$ is a sequence of homomorphisms such that $\rk(A\to F) = \dim~C$ then $\rk(B\to E) = \dim~C$.
% \end{lemma}

\subsection{What we need from $X$, $D$, and $P$}

Let $\X$ be a compact Riemannian $d$-manifold and let $D$ be a compact subset of $\X$ and let $\dist(x, y) = \|x - y\|$ denote the distance between points $x,y\in D$ as a subspace of $\X$.
For $A\subset D$ and $x\in D$ let
\[\dist_A(x) = \min_{a\in A}\dist(x, a)\]
denote the distance from $x$ to the set $A$.
We will use open metric balls restricted to $D$ with the subspace topology
\[\ball_D^\e(x) = \{y\in D\mid \dist(x, y) < \e\}\]
and offsets
\[A^\e = \{x\in D\mid \dist_A(x) < \e\}.\]

A ball $\ball_D^\e(x)$ is said to be \textbf{strongly convex} if for each pair of points $y,z$ in the closure of $\ball_D^\e(x)$ there exists a unique shortest path in $D$ between $y$ and $z$, and the interior of this path is included in $\ball_D^\e(x)$.
Let $\varrho_D(x)$ be the supremum of the radii such that $\ball_D^\e(x)$ is strongly convex.
The \textbf{strong convexity radius} of $D$ is defined
\[ \varrho_D := \inf_{x\in D} \varrho_D(x).\]
Note that this value is positive for compact $D$.

We will require that, for some $\zeta > 0$, $D$ is a compact subspace of $\X$ with strong convexity radius $\varrho_D > 2\zeta$.
For a finite point set $P$ such that $P^{2\delta}\subset \intr_\X(D)$ and any subset $Q\subseteq P$ we will assume that $D\setminus Q^\delta,$ and $D\setminus P^\delta$ are locally path connected for some $\delta > 0$.

% \begin{lemma}
%   If $Q\subseteq R\subseteq P$ then
%   \[ \rk~\hom_d((P^\delta, Q^\delta)\hookrightarrow (P^{2\delta}, R^{\delta})) = \rk~\hom_0((D\setminus R^{\delta}, D\setminus P^{\delta})\hookrightarrow(D\setminus Q^\delta, D\setminus P^\delta)).\]
% \end{lemma}

\begin{lemma}
  For any $Q\subseteq R\subseteq P$
  \[\rk~\hom_d(\rips^\delta(P, Q)\hookrightarrow \rips^{2\delta}(P, R))\leq \rk~\hom_0((D\setminus R^{\delta}, D\setminus P^{\delta})\hookrightarrow (D\setminus Q^\delta, D\setminus P^\delta)).\]
\end{lemma}
