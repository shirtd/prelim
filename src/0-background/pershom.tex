% !TeX root = ../../main.tex

\paragraph*{\textbf{Functions and Filtrations}}

% \figblock{%
% \begin{figure}[htbp]
% \centering
%     \includegraphics[scale=0.45]{figures/fgrid.pdf}
%     % \includegraphics[scale=0.45]{figures/fsample.pdf}
%     \includegraphics[scale=0.55]{figures/fcomplex.pdf}
%     \caption{(Left) Continuous function on the plane.
%             % (Middle) Function values on random sample.
%             (Right) Function values on a random sample extended to a continuous piecewise linear approximation on a 2-dimension simplicial as the $z$-axis.}
%     \label{fig:function}
% \end{figure}}

Suppose we have some real valued function $f:\D\to\R$ and endow our sensors with the ability to measure this function at their points $P$.
If we know our network covers the domain we may construct a simplicial complex which captures its topology in a discrete structure.
We may therefore integrate the sparse measurements provided by the sensors throughout the simplicial complex to capture the topology of the function itself.
Not only do we now have a discrete approximation of the function but an ordering on the simplices of the complex which may be used to explore the evolution of the topological structure of the function.

This leads to a more general notion defined for a sequence of topological spaces, which we may take as sublevel sets of the function on the domain.
\begin{definition}
    Let $f:D\to\R$ be a function on a topological domain $D$ and let $I_0, I_1,\ldots$ be a sequence of intervals $I_i = [a_i, b_i)$ that covers the image of $f$ on $D$.
    Let and $X_i = \{x\in D\mid a_i\leq f(x) < b_i\}$ be topological subspaces connected by inclusion maps $X_i\to X_{i+1}$.
    A \textbf{filtration} is the resulting sequence of topological spaces
    \[X_0\to X_1\to\ldots X_i\to X_{i+1}\to\ldots .\]
\end{definition}
A filtration $\{K_i\}_{i=1,\ldots,n}$ may also be interpreted as a sequence of simplicial maps, each an inclusion $K_i\to K_{i+1}$.
This induces an algebraic sequence of homomorphisms on homology by functoriality, for all $k$:
\[ H_k(X_0)\to H_k(X_1)\to\ldots\to H_k(X_i)\to H_k(X_{i+1})\to\ldots . \]
This sequence encodes the local topological changes that occur at each step of the filtration.
Global information is encoded in terms of the \textbf{birth} and \textbf{death} of homology classes, represented as a \textbf{persistence diagram} or \textbf{barcode}.
