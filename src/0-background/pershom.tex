% !TeX root = ../../main.tex

Given a topological space $X$ a \textbf{filtration} of $X$ is a sequence of subspaces $X_0\hookrightarrow X_1\hookrightarrow\ldots$.
The \textbf{$k$th persistent homology module} of a filtration $\{X_i\}_{i\in I}$ over some indexing set $I$ is the persistence module consisting of vector spaces $\hom_k(X_i)$ and linear maps $\hom_k(X_i)\to \hom_k(X_j)$ induced by inclusion for $i\leq j$.
Similarly, given a pair $(X,Y)$ and filtrations $\{X_i\}_{i\in I}$ and $\{Y_i\}_{i\in I}$ such that $Y_i\subseteq X_i$ for all $i\in I$ the $k$th persistent relative homology module is the persistence module of vector spaces $\hom_k(X_i, Y_i)$ and linear maps $\hom_k(X_i, Y_i)\to\hom_k(X_j, Y_j)$ induced by inclusions for all $i\leq j$.
In particular, given a function $f: X\to \R$ we are interested in the persistent homology of the \textbf{sub-levelset filtration} $\{f^{-1}((-\infty, \alpha])\}_{\alpha\in\R}$..\footnote{Similarly, the super-levelset filtration is defined $\{f^{-1}([\alpha,\infty))\}_{\alpha\in\R}$.}

The persistent homology of a filtration encodes the changes in the topology of a space as it changes monotonically in some way.
In the case of the sub-levelset filtration, the homological structure of the subspace $f^{-1}((-\infty,\alpha]) = \{x\in X\mid f(x) \leq \alpha\}$ of $X$ as $\alpha$ increases.
These changes are usually referred to either the birth and death of a \emph{feature} in some dimension---homology classes that \emph{persist} over an interval $[b, d)\subset\R$.
This information can be summarized by a \textbf{persistence diagram} or \textbf{barcode} which visualizes these features either as pairs of points $(b, d)\in\R^2$, or as intervals $[b, d)\subset\R$.
