% !TeX root = ../main.tex

In the following let $D\subi{w,a} := B_w\cup f\rest_{\comp{B_w}}^{-1}(-\infty,a]$ and $P\subi{w,a} := P\cap D\subi{w,a}$.

\begin{lemma}\label{lem:pt_interleaving}
  If $\hom_k(\b\to\B)$ is surjective and $\hom_k(\B)\cong \hom_k(\BB)$ for all $k$ then for all $k$ and $\alpha\leq\beta$
  \[\hom_k\left((D\subi{\omega-c(\delta+\zeta),\alpha},B_{\omega-c(\delta+\zeta)})\hookrightarrow (D\subi{\omega,\beta},B_{\omega})\right)\]
  is surjective and
  \[ \hom_k\left( (D\subi{\omega,\alpha},B_\omega)\hookrightarrow (D\subi{\omega+c(\delta+\zeta),\beta},B_{\omega+c(\delta+\zeta)}) \right)\]
  is an isomorphism.
\end{lemma}
\begin{proof}
  By applying Lemma~\ref{lem:five} to the long exact sequences of the pairs $(D\subi{\omega-c(\delta+\zeta),\alpha},\b)$ and $(D\subi{\omega,\alpha},\B)$ our assumption that $\hom_k(\b\to\B)$ is surjective for all $k$ implies $\hom_k((D\subi{\omega-c(\delta+\zeta),\alpha},\b)\hookrightarrow (D\subi{\omega,\alpha},\B))$ is surjective for all $\alpha\in\R$.
  Similarly, the assumption that $\hom_k(\B)\cong \hom_k(\BB)$ implies $\hom_k((D\subi{\omega,\alpha},\B)\hookrightarrow (D\subi{\omega+c(\delta+\zeta),\alpha},\BB))$ is an isomorphism by applying Lemma~\ref{lem:five} to the long exact sequences of the pairs $(D\subi{\omega,\alpha},\B)$ and $(D\subi{\omega+c(\delta+\zeta),\alpha},\BB)$.
\end{proof}

\begin{lemma}\label{lem:extension_apply}
  If $Q_w^\e$ surrounds $P^\e$ in $D$ then for all $w,a\in\R$ and $\ext{P\subi{w,a}^\e} = P\subi{w,a}^\e \cup (D\setminus P^\e)$
  \[\hom_k(P\subi{w,a}^\e, Q_w^\e)\cong \hom_k(\ext{P\subi{w,a}^\e}, \ext{Q^\e_a}).\]
\end{lemma}
\begin{proof}
  Because $P\subi{w,a} := P\cap D\subi{w,a}$ and $B_w\subseteq D\subi{w,a}$ we know $Q_w = P\cap B_w \subseteq P\subi{w,a}$ for all $w,a\in\R$.
  So
  \[\ext{Q^\e_a} = Q^\e_a\cup (D\setminus P^\e) \subseteq P\subi{w,a}^\e \cup (D\setminus P^\e) = \ext{P\subi{w,a}^\e}.\]
  As $(P^\e, Q_w^\e)$ is a surrounding pair in $D$, $P^\e$ is open in $D$ and $\ext{P\subi{w,a}^\e}\subseteq D$ is such that $\ext{Q^\e_a}\subseteq \ext{P\subi{w,a}^\e}$ it follows that
  \[\hom_k(P\subi{w,a}^\e, Q^\e_a) = \hom_k(P^\e\cap \ext{P\subi{w,a}^\e}, Q^\e_a) \cong\hom_k(\ext{P\subi{w,a}^\e}, \ext{Q^\e_a})\]
  by Lemma~\ref{lem:excision}
\end{proof}

So we define the extensions
\[\ext{P^\e\subi{\omega-c\zeta, \alpha}} := P^\e\subi{\omega-c\delta, \alpha}\cup (D\setminus P^\of)\]
and
\[\ext{P^\e\subi{\omega+c\delta, \alpha}} := P^\e\subi{\omega+c\delta, \alpha}\cup (D\setminus P^\of)\]
of $P^\e_\alpha$ for all $\alpha\in\R$ and $\delta\leq\e\leq\zeta$.

\begin{lemma}\label{lem:p_interleave}
 If $Q_w^\e$ surrounds $P^\e$ in $D$ and $D\setminus B_{w + \e}\subseteq P^\e$ then
 \[ D\subi{w-c\e, a-c\e} \subseteq \ext{P\subi{w, a}^\e}\subseteq D\subi{w+c\e,a+c\e}.\]
\end{lemma}
\begin{proof}
  Suppose $x\in (P^\e\cap B\subi{w-c\e, a-c\e})\setminus B_{w+\e}$.
  Because $B_{w-\e}\subset B_{w+\e}$ we know $x\notin B_{w-\e}$ so $w+c\e < f(x)\leq a-c\e$ and there exists some $p\in P$ such that $\dist(x, p) < \e$.
  Because $f$ is $c$-Lipschitz it follows
  \[ f(p)\leq f(x) + c\dist(x, p) < a - c\e + c\e = a\]
  and
  \[ f(p)\geq f(x) - c\dist(x, p) > w+c\e-c\e = w.\]
  So $x\in P\subi{w,a}^\e$.

  Now, suppose $x\in P\subi{w,a}^\e\setminus B_{w+c\e}$.
  So $w+c\e < f(x)$ and there exists some $p\in P\subi{w,a}$ such that $\dist(x,p) < \e$.
  Because $f$ is $c$-Lipschitz it follows
  \[ f(x) \leq f(p) + c\dist(x,p) < a + c\e.\]
  So $x\in B\subi{w+c\e, a+c\e}\setminus B_{w+c\e}$.

  Because $D\setminus B_{w+c\e}\subseteq P^\e$ we know that $D\setminus P^\e \subseteq B_{w+c\e}$, so
  \[D\subi{w-c\e, a-c\e}\setminus B_{w+c\e} \subseteq P\subi{w,a}^\e\setminus B_{w+c\e}\subseteq D\subi{w+c\e, a+c\e}\setminus B_{w+c\e}\]
  implies
  \[ D\subi{w-c\e, a-c\e}\subseteq P\subi{w,a}^\e\cup (D\setminus P^\e) = \ext{P\subi{w,a}^\e} \subseteq D\subi{w+c\e, a+c\e} \]
  as desired.
\end{proof}
