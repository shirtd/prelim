% !TeX root = ../main.tex

For an interval $I = [s,t)\subseteq \R$ let $I_+ := [t,\infty)$ and $I_- := (-\infty, s]$.
For $\alpha\leq\beta\in\R$ let
\[ F_\alpha^I := \begin{cases} \FF&\text{ if } \alpha\in I\\ 0 &\text{otherwise,}\end{cases}\ \text{ and }\ \ f_{\alpha,\beta}^I := \begin{cases} \id_\FF&\text{ if } \alpha,\beta\in I\\ 0&\text{otherwise.}\end{cases}.\]
An \textbf{interval module} is a persistence module $\FF^I$ defined to be the family of vector spaces $\{F_\alpha^I\}_{\alpha\in\R}$ along with linear maps $\{f_{\alpha,\beta}^I : F_\alpha^I\to F_\beta^I\}_{\alpha\leq\beta}$.

For $\omega\in\R$ let $\FF_\omega^I$ denote the interval module consisting of vector spaces $\{F_{\alpha\mid\omega}^I\}_{\alpha\in\R}$ and linear maps $\{f_{\alpha,\beta\mid \omega}^I : F_{\alpha\mid\omega}^I\to F_{\beta\mid\omega}^I\}_{\alpha\leq\beta}$ where
\[ F_{\alpha\mid \omega}^I := \begin{cases} F_\alpha^I&\text{ if } \omega\in I_-\\ 0&\text{ otherwise,}\end{cases}\ \text{ and }\ \ f_{\alpha,\beta\mid\omega}^I := \begin{cases} f_{\alpha,\beta}^I&\text{ if } \omega\in I_-\\ 0&\text{ otherwise.}\end{cases}\]

\subsubsection{The Persistent Homology of a Function Modulo a Sublevel-set}

In the following we will assume that the homology groups $\hom_k(B_\omega)$, $\hom_k(D\subi{\omega,\alpha})$, and $\hom_k(D\subi{\omega,\alpha}, B_\omega)$ are finite dimensional vector spaces for all $k$ and $\alpha\in\R$.
Let $\AA^k$ denote the $k$th persistent homology module of $\{B_\alpha\}_{\alpha\in\R}$ and $\DD{\omega}^k$ denote the $k$th persistent (relative) homology module of $\{(D\subi{\omega,\alpha},B_\omega)\}_{\alpha\in\R}$, taking homology with coefficients in a field $\FF$.

Because all homology groups are finite dimensional we can decompose the modules $\AA^k$ and $\DD{\omega}^k$ into direct sums of interval modules
% That is, for all $k$ there exist collections $\I^k$ and $\J^k$ of intervals $I,J\subseteq \R$ such that
\[ \AA^k = \bigoplus_{I\in\I^k} \FF^I\ \text{ and }\ \DD{\omega}^k = \bigoplus_{J\in\J^k} \FF^J\]
for collections $\I^k$ and $\J^k$ of intervals $I,J\subseteq \R$ for all $k$.
Moreover, for all $\alpha\in\R$ we have
\[ \hom_k(B_\alpha) = \bigoplus_{I\in \I^k}F_\alpha^I\ \text{ and }\ \hom_k(D\subi{\omega,\alpha}, B_\omega) = \bigoplus_{J\in\J^k} F_\alpha^J.\]

Consider the long exact sequence of the pair $\hom_k(D\subi{\omega,\alpha}, B_\omega)$
\[ \ldots\to \hom_k(B_\omega)\xrightarrow{p_\alpha^k} \hom_k(B_\alpha)\xrightarrow{q_\alpha^k}\hom_k(D\subi{\omega,\alpha}, B_\omega)\xrightarrow{r_\alpha^k} \hom_{k-1}(B_\omega)\xrightarrow{p_\alpha^{k-1}}\hom_{k-1}\hom_{k-1}(B_\alpha)\to\ldots\]
where $p_\alpha^k = \displaystyle\bigoplus_{I\in\I^k} f_{\omega,\alpha}^I$.
By exactness %$\ker~r_\alpha = \im~q_\alpha^k \cong \hom_k(B_\alpha) /\ker~q_\alpha^k$, where
\[\ker~p_\alpha^k = \im~p_\alpha^k = \bigoplus_{I\in\I^k}\im~f_{\omega,\alpha}^I = \bigoplus_{I\in\I^k \mid \omega\in I} F_\alpha^I.\]
Thus,
\begin{align*}
  \ker~r_\alpha^k &\cong \hom_k(B_\alpha) / \ker~q_\alpha^k\\
    % &= \frac{\bigoplus_{I\in \I^k}F_\alpha^I}{\bigoplus_{I\in\I^k \mid \omega\in I} F_\alpha^I}\\
    &= \bigoplus_{I\in \I^k\mid \omega\notin I} F_\alpha^I\\
    &= \bigoplus_{I\in\I^k} F_{\alpha\mid \omega}^I.
\end{align*}
Similarly,
\begin{align*} \im~r_\alpha^k &= \ker~p_\alpha^{k-1}\\
  &= \bigoplus_{I\in\I^{k-1}\mid \alpha\notin I} F_\omega^I\\ %= \bigoplus_{I\in\I^{k-1}\mid \omega\in I} F_\alpha^{I_+}\\
  &= \bigoplus_{I\in\I^{k-1}} F_{\alpha\mid\omega}^{I_+}.
\end{align*}

Note that we have the following split exact sequence associated with the connecting homomorphism $r_\alpha^k$
\[ 0\to \ker~r_\alpha^k\xrightarrow{\phi_\alpha^k}\bigoplus_{J\in\J^k} F_\alpha^J\xrightarrow{\psi_\alpha^k}\im~r_\alpha^k\to 0.\]
So for all $\alpha\in\R$
% \[ \bigoplus_{J\in\J^k} F_\alpha^J\cong \ker~r_\alpha^k\oplus \im~r_\alpha^k.\]
% Thus
\begin{align*}
  \bigoplus_{J\in\J^k} F_\alpha^J &\cong \ker~r_\alpha^k\oplus \im~r_\alpha^k\\
    &\cong\left(\bigoplus_{I\in\I^k} F_{\alpha\mid \omega}^I\right)\oplus\left(\bigoplus_{I\in\I^{k-1}} F_{\alpha\mid\omega}^{I_+}\right).
\end{align*}
Letting $\AA_\omega^k := \displaystyle\bigoplus_{I\in\I^k} \FF_\omega^I$ and $\BB_\omega^k := \displaystyle\bigoplus_{I\in\I^k} \FF_\omega^{I_+}$ for all $k$ we have
\[ \DD{\omega}^k \cong \AA_\omega^k\oplus \BB_\omega^{k-1}.\]% = \left(\bigoplus_{I\in\I^k} \FF_\omega^I\right)\oplus\left(\bigoplus_{I\in\I^{k-1}} \FF_\omega^{I_+} \right).\]

%$\hom_k(B_\alpha) = \displaystyle\bigoplus_{I\in \I^k}F_\alpha^I$
