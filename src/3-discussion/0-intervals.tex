% !TeX root = ../../main.tex

In this section we consider the meaning of the $k$th persistent (relative) homology of a function $f : D\to \R$ modulo a fixed sub-levelset $B_\omega := f^{-1}((-\infty,\omega])$.
Unlike previous work~\cite{cohen09extending} we do not consider the persistent relative homology of $D$ modulo the sub-levelset filtration $\{B_\alpha\}_{\alpha\in\R}$.
Instead, we are interested in the role of a specific sub-levelset $B_\omega$ in the context of the long exact sequences of $f$ modulo $B_\omega$ throughout the sub-levelset filtration.

We find that the $k$th persistent (relative) homology of a function restricted to fixed sub-levelset modulo that sub-levelset $B_\omega$ is equal to the submodule of features born after $\omega$ with additional infinite $k$-dimensional features which are paired with $(k-1)$-dimensonal features that are born before $\omega$ and die after $\omega$ in the full diagram.
Unlike the persistent homology of the restriction $f\rest_{D\setminus B_\omega}$ this approach leaves features of the full diagram that are born after $\omega$ unchanged.

For lack of a better analogy, this has the effect of ``quarantining'' the persistent homology of the function below $\omega$.
Our hypothesis is that, given the persistent homology of the function up to $\omega$, one can recover the full diagram by pairing specific infinite $(k-1)$-dimensional features of $f\rest_{B_\omega}$ with specific infinite $k$-dimensional features of $f\rest_{D\setminus B_\omega}$ modulo $B_\omega$ via the long exact sequence(s) of pairs $(D\subi{\omega}{\alpha}, B_\omega)$.

\paragraph{Restricted Interval Modules}

For an interval $I = [s,t)\subseteq \R$ let $I_+ := [t,\infty)$ and $I_- := (-\infty, s]$.
For a collection $\I$ of intervals let $\I_+ = \{ I_+ \}_{I\in\I}$ and $\I_- = \{ I_- \}_{I\in\I}$.

For $\omega\in\R$ let $\FF\subi{\cdot}{\omega}^I$ denote the interval module consisting of vector spaces $\{F\subi{\omega}{\alpha}^I\}_{\alpha\in\R}$ and linear maps $\{f_{\alpha,\beta\mid \omega}^I : F\subi{\omega}{\alpha}^I\to F_{\beta\mid\omega}^I\}_{\alpha\leq\beta}$ where
\[ F\subi{\omega}{\alpha}^I := \begin{cases} F_\alpha^I&\text{ if } \omega\in I_-\\ 0&\text{ otherwise,}\end{cases}\ \text{ and }\ \ f_{\alpha,\beta\mid\omega}^I := \begin{cases} f_{\alpha,\beta}^I&\text{ if } \omega\in I_-\\ 0&\text{ otherwise.}\end{cases}\]

\subsection{The Persistent Homology of a Function Modulo a Sub-levelset}

In the following we will assume that, for fixed $\omega\in\R$ and taking homology in a field $\FF$, the homology groups $\hom_k(B_\alpha)$ and $\hom_k(D\subi{\omega}{\alpha}, B_\omega)$ are finite dimensional vector spaces for all $k$ and $\alpha\in\R$.
Let $\LL^k$ denote the $k$th persistent homology module of the sub-levelset filtration $\{B_\alpha\}_{\alpha\in\R}$ of $f$.
As in the previous section, let $\DD{\omega}^k$ denote the $k$th persistent (relative) homology module of $\{(D\subi{\omega}{\alpha},B_\omega)\}_{\alpha\in\R}$, the sub-levelset filtration of $f$ modulo $B_\omega$.
Here, we are taking homology with coefficients in a field $\FF$.

Because all homology groups are finite dimensional we can decompose the modules $\LL^k$ and $\DD{\omega}^k$ into direct sums of interval modules
\[ \LL^k = \bigoplus_{I\in\I^k} \FF^I\ \text{ and }\ \DD{\omega}^k = \bigoplus_{J\in\J^k} \FF^J\]
for collections $\I^k$ and $\J^k$ of intervals $I,J\subseteq \R$ for all $k$.
Moreover, for all $\alpha\in\R$ we have
\[ \hom_k(B_\alpha) = \bigoplus_{I\in \I^k}F_\alpha^I\ \text{ and }\ \hom_k(D\subi{\omega}{\alpha}, B_\omega) = \bigoplus_{J\in\J^k} F_\alpha^J.\]

\begin{lemma}
  \[\DD{\omega}^k = \bigoplus_{I\in\I^k \cup \I^{k-1}_+} \FF\subi{\omega}{\cdot}^I = \LL\subi{\omega}{\cdot}^k \oplus \bigoplus_{I\in \I^{k-1}} \FF\subi{\omega}{\cdot}^{I_+}.\]
\end{lemma}

\begin{proof}
  Consider the long exact sequence of the pair $\hom_k(D\subi{\omega}{\alpha}, B_\omega)$
  \[ \ldots\to \hom_k(B_\omega)\xrightarrow{p_\alpha^k} \hom_k(B_\alpha)\xrightarrow{q_\alpha^k}\hom_k(D\subi{\omega}{\alpha}, B_\omega)\xrightarrow{r_\alpha^k} \hom_{k-1}(B_\omega)\xrightarrow{p_\alpha^{k-1}}\hom_{k-1}\hom_{k-1}(B_\alpha)\to\ldots\]
  where $p_\alpha^k = \displaystyle\bigoplus_{I\in\I^k} f_{\omega,\alpha}^I$.
  By exactness
  \[\ker~p_\alpha^k = \im~p_\alpha^k = \bigoplus_{I\in\I^k}\im~f_{\omega,\alpha}^I = \bigoplus_{I\in\I^k \mid \omega\in I} F_\alpha^I.\]
  Thus,
  \begin{align*}
    \ker~r_\alpha^k \cong \hom_k(B_\alpha) / \ker~q_\alpha^k = \bigoplus_{I\in \I^k\mid \omega\notin I} F_\alpha^I = \bigoplus_{I\in\I^k} F\subi{\omega}{\alpha}^I.
  \end{align*}
  Similarly,
  \begin{align*} \im~r_\alpha^k = \ker~p_\alpha^{k-1} = \bigoplus_{I\in\I^{k-1}\mid \alpha\notin I} F_\omega^I = \bigoplus_{I\in\I^{k-1}} F\subi{\omega}{\alpha}^{I_+}.
  \end{align*}

  Note that we have the following split exact sequence associated with the connecting homomorphism $r_\alpha^k$
  \[ 0\to \ker~r_\alpha^k\xrightarrow{\phi_\alpha^k}\bigoplus_{J\in\J^k} F_\alpha^J\xrightarrow{\psi_\alpha^k}\im~r_\alpha^k\to 0.\]
  So for all $\alpha\in\R$
  \[ \bigoplus_{J\in\J^k} F_\alpha^J \cong \ker~r_\alpha^k\oplus \im~r_\alpha^k
    \cong\left(\bigoplus_{I\in\I^k} F\subi{\omega}{\alpha}^I\right)\oplus\left(\bigoplus_{I\in\I^{k-1}} F\subi{\omega}{\alpha}^{I_+}\right).\]
  \end{proof}
% Letting $\AA_\omega^k := \displaystyle\bigoplus_{I\in\I^k} \FF_\omega^I$ and $\BB_\omega^k := \displaystyle\bigoplus_{I\in\I^k} \FF_\omega^{I_+}$ for all $k$ we have
% \[ \DD{\omega}^k \cong \AA_\omega^k\oplus \BB_\omega^{k-1}.\]

\begin{theorem}
  Let $\X$ be an orientable $d$-manifold and let $D$ be a compact subset of $\X$ with strong convexity radius $\varrho_D > \delta$.
  Let $f : D\to\R$ be $c$-Lipschitz function and let $\omega\in\R$ and $2\delta\leq\zeta < \varrho_D/2$ be constants such that $B_{\omega - c(\zeta +\delta)}$ surrounds $D$ in $\X$.
  Let $P\subset \intr_\X(D)$ and suppose $P^\delta$, $Q_{\omega-c\zeta}^\delta$, and $Q_{\omega+c\delta}^\delta$ satisfy the assumptions of Lemma~\ref{lem:duality_apply}.
  Suppose $\hom_k(B_{\omega-c(\delta+\zeta)}\hookrightarrow B_\omega)$ and $\hom_k(B_\omega)\cong\hom_k(B_{\omega+c(\delta+2\zeta)})$ for all $k$.

  If
  \[\rk~\hom_d(\rips^\delta(P, Q_{\omega -c\zeta})\hookrightarrow \rips^{2\delta}(P, Q_{\omega+c\delta})) \geq \dim~\hom_0(\rips^\delta(P\setminus Q_{\omega-c\zeta}))\]
  then the image module
  \[ \im~(\RPP{\omega-c\zeta}{2\delta, k}\to \RPP{\omega+c\delta}{2\zeta, k})\]
  is $2c\zeta$-interleaved with
  \[ \LL\subi{\omega}{\cdot}^k \oplus \bigoplus_{I\in \I^{k-1}} \FF\subi{\omega}{\cdot}^{I_+}.\]
\end{theorem}
