% !TeX root = ../main.tex

We have extended the Topological Coverage Criterion to the setting of Topological Scalar Field Analysis.
By defining the boundary in terms of a sublevel set of a scalar field we provide an interpretation of the TCC that applies more naturally to data coverage.
We then showed how the assumptions and machinery of the TCC can be used to approximate the persistent homology of the scalar field relative to a static sublevel set.
This relative persistent homology is shown to be related to a truncation of that of the scalar field as whole, and therefore provides a way to approximate a part of its persistence diagram in the presence of un-verified data.

There are a number of unanswered questions and directions for future work.
Our theoretical results were limited by our understanding of duality.
Importantly, a more rigorous treatment of duality would formally link the regularity assumptions made in the TCC and our interleaving.
This would allow us to merge the assumptions made in these two statements as our main theorem.
It would also simplify some of the assumptions made on our sample in the statement of the TCC.
Moreover, as duality plays a central role in the TCC it is natural to investigate its role in the persistent homology of scalar fields as in~\cite{edelsbrunner12alexander}.
Our hope is to be able to provide a rigorous comparison between the relative approach and the persistent homology of the superlevel set filtration, and explore connections with Extended Persistence~\cite{cohen09extending}.
% This would not only allow us to apply duality to persistent homology~\cite{desilva11duality}, but also allow us to provide a rigorous comparison between the relative approach and the persistent homology of the superlevel set filtration and explore connections with Extended Persistence~\cite{cohen09extending}.

From a computational perspective, we are particularly interested in the matching problem discussed in Section~\ref{sec:experiments} that can be used to recover the full diagram.
Our statements in terms of sublevel sets can also be generalized to disjoint unions of sub and superlevel sets, where coverage is confirmed in an \emph{interlevel} set.
This, along with a better understanding of the duality between sub and superlevel sets could lead to an iterative approach in which the persistent homology of a scalar field is constructed as data becomes available.
% We also note that computing the diagram of a nested pair of relative Rips complexes that vary in both function values and scale is nontrivial.
% We are interested in finding efficient ways to compute these image diagrams.
% We are also interested in finding efficient ways to compute the image persistent (relative) homology that vary in both scalar and scale.

% The problem of relaxing our assumptions on the boundary can be approached from both a theoretical and computational perspective.
% Ways to avoid the isomorphism we require could be investigated in theory, and the interaction of relative persistent homology and the Persistent Nerve Lemma may be used tighten our assumptions.
% We would also like to conduct a more rigorous investigation on the effect of these assumptions in practice.
