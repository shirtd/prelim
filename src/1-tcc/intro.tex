% !TeX root = ../../main.tex

% A collection of sensors can be verified as covering a domain if
% \begin{enumerate}
%     \item[a.] the boundary of the domain is adequately covered,
%     \item[b.] the interior of the
% \end{enumerate}
% Condition (b) relies on condition (a) in order to provide a topological condition that is necessary but not sufficient.
% Given (a) we can confirm coverage by checking if the balloon has been punctured simply by checking the dimension of the top-dimensional relative homology of the sample.
%
% Adequate coverage of the boundary can be broken into two parts.
% First, we require that the sampled boundary is sufficiently simple in order to ensure our condition cannot produce false positives.
% This is achieved by using what we refer to as \emph{short-filtrations}: applying one step of persistence in order to de-noise the data.
% By testing our network at two scales we can ensure no spurious features are present in the boundary which may contribute to false positives.
% % We also note that these short-filtrations are employed in the analysis of scalar fields as well.
%
% Secondly, we require that the so-called ``sampled boundary'' surrounds the interior of the domain.
% Otherwise, we may cover the domain but see what looks like a punctured ball as the ball when in fact the ball was never formed.
% In the TCC this situation is not handled explicitly.
% Instead it is stated as a condition for coverage that is necessary but not sufficient.
% That is, it can verify \emph{coverage} without false positives but may produce false negatives.
% In fact, the TCC tests a more specific problem: whether we have a reliable representation of the boundary \emph{and} a reliable representation of the interior.
% \footnote{\textbf{TODO} discuss how this is still not a sufficient condition.}

% Given this observation we considered how best to use \emph{all} the information given by the TCC in a way that re-uses the machinery used to compute it.
% In the following sections we consider the relative persistent homology of a function modulo a sublevel-set as an extension of the TCC.
% In this section we re-cast the TCC for a domain surrounded by sub-levelset in order to ensure that a given sample can provide an adequate approximation.
% First, we will provide some definitions and preliminary lemmas which will formalize the notion of a surrounding sub-levelset and its properties.
% We will then modify the analysis of scalar fields in order to give an approximation of the \emph{relative} persistent homology of a sample.
% Finally, we consider classes of functions which satisfy the assumptions made.
% Namely, we consider functions with multiple sub-level sets which may serve as a boundary for this procedure and show how they can be integrated to give a more robust signature for the function.

The TCC provides a computable condition for coverage which uses the top-dimensional relative homology group of a space $D$ with respect to its boundary $B$.
Under certain conditions it can be shown that the rank of this group is equal to the number of connected components of its complement in some larger space $\X$ by Alexander Duality.
Now, suppose we are given a collection of subsets of the space, such as a collection of metric balls centered at points in some set $P$.
One can now check if the union of the collection covers the space by comparing the number of connected components of its complement to that of the space---holes in the cover will appear as additional components in the complement space.

Clearly, we cannot compute this quantity directly from the complement, but we can use duality to recover it from the top dimensional relative homology provided we have a subset $Q$ of our cover to serve as the boundary.
That is, a positive result indicates that we not only have coverage, but also that we have a pair of spaces that reflects the pair $(D, B)$ topologically.
We call such a pair a \emph{surrounding pair} defined in terms of separating sets.
It has been shown~\ref{cavanna2017when} that the TCC can be stated in terms of these surrounding pairs, which allows us enough flexibility to define our surrounding set as a sub-levelset of a $c$-Lipschitz function $f$.

The initial assumptions required for the TCC were in terms of the smoothness of the boundary.
Computing the TCC requires factoring the cover through a pair of Vietoris-Rips complexes of varying scale.
To account for the resulting error assumptions must be made in order to ensure the number of connected components of the complement of $D$ does not change close to the boundary.
Specifically, the assumptions ensure that the condition does not produce a false positive, and allows for changes that will not affect the determination.
That is, if we were to expand or contract the space by an amount dependent on the coverage radius no components would appear or disappear, respectively\footnote{\textbf{double check}}.
On the other hand, components disappearing with expansion and appearing with contraction are permitted\footnote{\textbf{double check}}.

The original TCC required the boundary to be sufficiently smooth, and relied only on the geometry of the boundary~\ref{desilva09coverage}.
Later work made these assumptions directly in terms of the \emph{zero dimensional} persistent homology of the domain close to the boundary~\ref{cavanna2017when}.
This perspective is particularly well suited to our situation where we define our ``boundary'' as a sub-levelset of a function.
That is, we can now state our assumptions in terms of the \emph{zero dimensional} persistent homology of the function itself.

This opens the door to a wide array of applications in which we have either a small portion or a very rough approximation of the persistence diagram of a function.
We can use then either identify or verify a suitable threshold $\omega$ for our sub-levelset to serve as our boundary and confirm coverage.
As we will see these same assumptions will appear as assumptions for \emph{all dimensions} in our proof of the interleaving.
It is in this sense that the approximation of a truncated diagram can be seen as a generalization of the TCC to the topological analysis of scalar fields with guarantees.
