% !TeX root = ../../main.tex

The TCC uses the top-dimensional relative homology group of a space $D$ with respect to its boundary $B$ to provide a computable condition for coverage.
Under certain conditions it can be shown that the rank of this group is equal to the number of connected components of its complement in some larger space $\X$ by Alexander Duality.
Now, suppose we are given a collection of subsets of the space, such as a collection of metric balls centered at points in some set $P$.
One can now check if the union of the collection covers the space by comparing the number of connected components of its complement to that of the space---holes in the cover will appear as additional components in the complement space.

As we cannot compute the number components of the complement space from a sample directly, the TCC uses duality to recover it from the top dimensional relative homology.
However, this requires that we have a subset $Q$ of our cover to serve as the boundary.
That is, a positive result indicates that we not only have coverage, but also that we have a pair of spaces that reflects the pair $(D, B)$ topologically.
We call such a pair a \emph{surrounding pair} defined in terms of separating sets.
It has been shown that the TCC can be stated in terms of these surrounding pairs~\cite{cavanna2017when}, which allows us enough flexibility to define our surrounding set as a sub-levelset of a $c$-Lipschitz function $f$.

Computing the TCC requires factoring the cover through a pair of Vietoris-Rips complexes of varying scale.
To account for the resulting error assumptions must be made in order to ensure the number of connected components of the complement of $D$ does not change close to the boundary.
Specifically, these assumptions ensure that the condition does not produce a false positive, and allows for changes that will not affect the determination.
% That is, if we were to expand or contract the space by an amount dependent on the coverage radius no components would appear or disappear, respectively\footnote{\textbf{double check}}.
% On the other hand, components disappearing with expansion and appearing with contraction are permitted\footnote{\textbf{double check}}.

The original TCC required the boundary to be sufficiently smooth, and relied only on the geometry of the boundary~\cite{desilva07coverage}.
Later work made these assumptions directly in terms of the \emph{zero dimensional} persistent homology of the domain close to the boundary~\cite{cavanna2017when}.
This perspective is particularly well suited to our situation where we define our ``boundary'' as a sub-levelset of a function.
% That is, we can now state our assumptions in terms of the \emph{zero dimensional} persistent homology of the function itself.
That is, we can now state our assumptions in terms of the \emph{zero dimensional} persistent homology of the function itself.
This replaces the convoluted assumptions made in terms various ``restricted domains'' with statements about the persistent homology of a monotonically increasing filtration with respect to a constant $\omega$.

This opens the door to a wide array of applications in which we have either a small portion or a very rough approximation of the persistence diagram of a function.
We can use then either identify or verify a threshold $\omega$ for our sub-levelset that satisfies our assumptions.
As we will see these same assumptions will appear as statements about persistent homology in \emph{all dimensions} in our proof of the interleaving.
It is in this sense that the approximation of a truncated diagram can be seen as a generalization of the TCC to the topological analysis of scalar fields with guarantees.

Before our proof of the TCC we will introduce the notion of a surrounding pair.
We will require a sub-levelset $B_\omega$ to serve as our boundary in terms of a surrounding pair $(D, B_\omega)$.
We also show that the TCC not only checks for coverage, but also checks if our cover near the boundary surrounds our cover.
In the next section we will use the resulting augmented TCC to provide inclusion maps between pairs that will be used in our interleaving.
