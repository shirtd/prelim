% !TeX root = ../../main.tex

% In the following let $\Lambda\in\Hom(\ext{\PP{w}{\e}}, \ext{\PP{w}{2\e}})$, $\cech\Lambda\in\Hom(\CPP{w}{\e},\CPP{w}{2\e})$, and $\rips\Lambda\in\Hom(\RPP{w}{\e},\CPP{w}{2\e})$ be induced by inclusions for all $w\in\R$, $\e > 0$.
For $w\leq z$ and $\e\leq\eta < \varrho_D$ let
\[\Sigma_w^\e := \I_w^\e\circ (\E\N_w^\e)^{-1}\in \Hom(\PP{w}{\e},\RPP{w}{2\e})\ \text{ and }\ \Upsilon_w^\e := \E\N_w^{2\e}\circ \J_w^{\e}\in \Hom(\RPP{w}{\e},\RPP{w}{2\e})\]
and let
\[ \Lambda\in\Hom(\ext{\PP{w}{\e}}, \ext{\PP{w}{2\e}}),\ \rips\Lambda\in\Hom(\RPP{w}{\e},\CPP{w}{2\e}),\text{ and } \Lambda'\in\Hom(\ext{\PP{w}{2\e}},\ext{\PP{z}{2\eta}})\]
be induced by inclusion. The proofs that
\[ \tilde{\Phi}(\Sigma_w^\e,\Sigma_z^\eta)\in\Hom(\im~\Lambda,\im~\rips\Lambda)\ \text{ and }\ \tilde{\Psi}(\Upsilon_w^{2\e},\Upsilon_z^{2\eta})\in\Hom(\im~\rips\Lambda,\im~\Lambda')\]
are image module homomorphisms are straightforward, and can be found in the appendix.

% \begin{lemma}\label{lem:rips_homomorphism_left}
%   For any $w\leq z$, $\e\leq\eta < \varrho_D$ let $\Lambda\in\Hom(\ext{\PP{w}{\e}}, \ext{\PP{w}{2\e}})$ and $\rips\Lambda\in\Hom(\RPP{w}{\e},\CPP{w}{2\e})$ be induced by inclusions.
%   Then $\tilde{\Phi}(\Sigma_w^\e,\Sigma_z^\eta)$ is an image module homomorphism.
% \end{lemma}
% \begin{proof}
%   By Lemma~\ref{cor:excisive_nerve} we have $\cech\Lambda\circ (\E\N_w^\e)^{-1} = (\E\N_z^\eta)^{-1}\circ \Lambda$ for $\cech\Lambda\in\Hom(\CPP{w}{\e},\CPP{z}{\eta})$ induced by inclusions.
%   As $\rips\Lambda\circ\I_w^\e = \I_z^\eta\circ\cech\Lambda$
%   \[ \rips\Lambda\circ \I_w^\e\circ(\E\N_w^\e)^{-1} = \I_z^\eta\circ\cech\Lambda\circ (\E\N_w^\e)^{-1} = \I_z^\eta\circ (\E\N_z^\eta)^{-1}\circ\Lambda.\]
%   It follows that $\rips\Lambda\circ\Sigma_w^\e = \Sigma_z^\eta\circ\Lambda$ by the definition of $\Sigma$.
%   So Diagram~\ref{dgm:image_homomorphism} commutes and we may therefore conclude that $\tilde{\Phi}(\Sigma_w^\e,\Sigma_z^\eta)$ is an image module homomorphism.
% \end{proof}
%
% \begin{lemma}\label{lem:rips_homomorphism_right}
%   For any $w\leq z$, $\e\leq\eta$ let $\rips\Lambda\in\Hom(\RPP{w}{\e},\RPP{w}{\eta})$ and $\Lambda'\in\Hom(\ext{\PP{w}{2\e}},\ext{\PP{z}{2\eta}})$ be induced by inclusions.
%   Then $\tilde{\Psi}(\Upsilon_w^{2\e},\Upsilon_z^{2\eta})$ is an image module homomorphism.
% \end{lemma}
% \begin{proof}
%   The proof is similar to Lemma~\ref{lem:rips_homomorphism_left}.
%   By Lemma~\ref{cor:excisive_nerve} we have $\E\N_z^{2\eta} \circ\cech\Lambda'  = \cech \Lambda\circ \E\N_w^{2\e}$ for $\cech\Lambda'\in\Hom(\CPP{w}{2\e},\CPP{z}{2\eta})$ induced by inclusions.
%   As $\J_z^\eta\circ \rips\Lambda = \cech\Lambda'\circ\J_w^\e$
%   \[ \E\N_z^{2\eta}\circ \J_z^\eta\circ \rips\Lambda = \E\N_z^{2\eta}\circ\cech\Lambda'\circ\J_w^\e = \cech \Lambda\circ \E\N_w^{2\e}\circ\J_w^\e.\]
%   Once again, Diagram~\ref{dgm:image_homomorphism} commutes by the definition of $\Upsilon$, so $\tilde{\Psi}(\Upsilon_w^{2\e},\Upsilon_z^{2\eta})$ is an image module homomorphism.
% \end{proof}
%
\begin{lemma}\label{lem:weak_rips_left}
  $(\Sigma_w^\e, \Upsilon_w^{2\e})$ factors $\Lambda$ through $\RPP{w}{2\e}$.
\end{lemma}
\begin{proof}
  % Let $\cech\Lambda\in\Hom(\CPP{w}{\e},\CPP{w}{2\e})$ be induced by inclusion.
  % Because $\I_w^\e$ and $\J_w^{2\e}$ are induced by inclusions $\cech\Lambda = \J_w^{2\e}\circ \I_w^\e$.
  % Let
  % \[ \Sigma_w^\e := \I_w^\e\circ (\E\N_w^\e)^{-1}\text{and}\ \Upsilon_w^{2\e} := \E\N_w^{2\e}\circ \J_w^{2\e}.\]
  Because $\I_w^\e$ and $\J_w^{2\e}$ are induced by inclusions $\Lambda = \E\N_w^{2\e}\circ (\J_w^{2\e})\circ \I_w^\e)\circ \E\N_w^\e)^{-1}$ by Lemma~\ref{cor:excisive_nerve}.
  Therefore, by the definitions of $\Sigma_w^\e$ and $\Upsilon_w^{2\e}$, the pair $(\Sigma_w^\e, \Upsilon_w^{2\e})$ factors $\Lambda$ through $\RPP{w}{2\e}$.
\end{proof}

% \begin{lemma}\label{lem:weak_rips_left}
%   Let $\Lambda\in\Hom(\ext{\PP{w}{\e}}, \ext{\PP{w}{2\e}})$ be induced by inclusions.
%   Then there exists a weak interleaving
%   \[ (\Sigma_w^\e, \Upsilon_w^{2\e})\in \Hom(\ext{\PP{w}{\e}}, \RPP{w}{2\e})\times \Hom(\RPP{w}{2\e},\ext{\PP{w}{2\e}})\]
%   of $\Lambda$ with $\RPP{w}{2\e}$.
% \end{lemma}
% \begin{proof}
%   Let $\cech\Lambda\in\Hom(\CPP{w}{\e},\CPP{w}{2\e})$ be induced by inclusion.
%   Because $\I_w^\e$ and $\J_w^{2\e}$ are induced by inclusions $\cech\Lambda = \J_w^{2\e}\circ \I_w^\e$.
%   Let
%   \[ \Sigma_w^\e := \I_w^\e\circ (\E\N_w^\e)^{-1}\text{and}\ \Upsilon_w^{2\e} := \E\N_w^{2\e}\circ \J_w^{2\e}.\]
%   By Corollary~\ref{cor:excisive_nerve} we have
%   \begin{align*}
%     \Lambda &= \E\N_w^{2\e}\circ \cech\Lambda\circ (\E\N_w^\e)^{-1}\\
%       &= (\E\N_w^{2\e}\circ \J_w^{2\e})\circ (\I_w^\e\circ (\E\N_w^\e)^{-1})\\
%       &= \Upsilon_w^{2\e}\circ \Sigma_w^\e
%   \end{align*}
%   so $(\Sigma_w^\e, \Upsilon_w^{2\e})$ is a weak interleaving of $\Lambda$ with $\RPP{w}{2\e}$.
% \end{proof}

% In the following let $(\Sigma_w^\e, \Upsilon_w^{2\e}) := (\E\N_w^\e)^{-1}, \E\N_w^{2\e}\circ \J_w^{2\e})$ denote the pair provided by Lemma~\ref{lem:weak_rips_left} for any $w\in\R$, $\e\geq 0$.

% \begin{lemma}\label{lem:rips_homomorphisms}
%   For any $w\leq z$ and $\e\leq\eta$ let $\Lambda\in\Hom(\ext{\PP{w}{\e}}, \ext{\PP{z}{\eta}})$, $\Lambda'\in\Hom(\ext{\PP{w}{2\e}},\ext{\PP{z}{2\eta}})$, and $\rips\Lambda\in\Hom(\RPP{w}{\e},\RPP{w}{\eta})$ be induced by inclusions.
%   Then
%   \[ \tilde{\Phi}(\Sigma_w^\e,\Sigma_z^\eta)\in\Hom(\im~\Lambda,\im~\rips\Lambda)\]
%   and
%   \[ \tilde{\Psi}(\Upsilon_w^{2\e},\Upsilon_z^{2\eta})\in\Hom(\im~\rips\Lambda,\im~\Lambda')\]
%   are image module homomorphisms.
% \end{lemma}
% \begin{proof}
%   Because $\I_w^\e$, $\I_z^\eta$, and $\rips\Lambda$ are induced by inclusions, and letting $\cech\Lambda\in\Hom(\CPP{w}{\e},\CPP{z}{\eta})$ be induced by inclusion,
%   \[ \rips\Lambda\circ\I_w^\e = \I_z^\eta\circ\cech\Lambda.\]
%   Moreover, because $\cech\Lambda$ is induced by inclusions
%   \[\cech\Lambda\circ(\E\N_w^\e)^{-1} = (\E\N_z^\eta)^{-1}\circ\Lambda\]
%   by Lemma~\ref{lem:excisive_nerve}.
%   We therefore have the following for all $\alpha\leq\beta$ by the definition of $\Sigma_w^\e$.
%   \begin{align*}
%     \rips\lambda[\alpha;\beta-\alpha]\circ\sigma_w^\e[\alpha] &= (\rips\lambda[\alpha;\beta-\alpha]\circ\I_w^\e[\alpha])\circ (\E\N_w^\e)^{-1}[\alpha]\\
%       &= \I_z^\eta[\beta]\circ(\cech\lambda[\alpha;\beta-\alpha]\circ (\E\N_w^\e)^{-1}[\alpha])\\
%       &= \I_z^\eta[\beta]\circ (\E\N_z^\eta)^{-1}[\beta]\circ\lambda[\alpha;\beta-\alpha]\\
%       &= \sigma_z^\eta[\beta]\circ\lambda[\alpha;\beta-\alpha]
%   \end{align*}
%   so Diagram~\ref{dgm:image_homomorphism} commutes, and we may therefore conclude that $\tilde{\Phi}(\Sigma_w^\e,\Sigma_z^\eta)$ is an image module homomorphism.
%
%   Because $\Lambda'$ is induced by inclusions and letting $\cech\Lambda'\in\Hom(\CPP{w}{2\e},\CPP{w}{2\eta})$ be induced by inclusions
%   \[\Lambda'\circ\E\N_w^{2\e} = \E\N_z^{2\eta}\circ \cech\Lambda'\]
%   by Lemma~\ref{lem:excisive_nerve}.
%   Because $\rips\Lambda$, $\J_w^\e$ and $\J_z^\eta$ are induced by inclusions
%   \[ \cech\Lambda'\circ \J_w^{2\e} = \J_z^{2\eta}\circ\rips\Lambda.\]
%   We therefore have the following for all $\alpha\leq\beta$ by the definition of $\Upsilon_w^\e$.
%   \begin{align*}
%     \lambda'[\alpha;\beta-\alpha]\circ \upsilon_w^{2\e}[\alpha] &= (\lambda'[\alpha;\beta-\alpha]\circ \E\N_w^{2\e}[\alpha])\circ \J_w^{2\e}[\alpha]\\
%       &=\E\N_z^{2\eta}[\beta]\circ(\cech\lambda'[\alpha;\beta-\alpha]\circ\J_w^{2\e}[\alpha])\\
%       &=(\E\N_z^{2\eta}[\beta]\circ\J_z^{2\e}[\beta])\circ\rips\lambda[\alpha;\beta-\alpha]\\
%       &=\upsilon_z^{2\eta}[\beta]\circ\rips\lambda[\alpha;\beta-\alpha]
%   \end{align*}
%   so Diagram~\ref{dgm:image_homomorphism} commutes, and we may therefore conclude that $\tilde{\Psi}(\Upsilon_w^{2\e},\Upsilon_z^{2\eta})$ is an image module homomorphism.
% \end{proof}

% \subsubsection{Rips-Function Interleaving}

For $w\in\R$ and $k\in\Z$ let
\[ \DD{w}^k := \left(\left\{\D{w}{\alpha}^k := \hom_k(D\subi{w}{\alpha},B_w)\right\}_{\alpha\in\R},\left\{d\subi{w}{\alpha,\beta}^k : \D{w}{\alpha}\to\D{w}{\beta}\right\}_{\alpha\leq\beta}\right)\]
denote the $k$th persistent homology module of the sub-levelset filtration modulo $B_w$, $\{(D\subi{w}{\alpha},B_w)\}$.
Once again, we will omit the dimension $k$ and write $\DD{w}$ if a statement holds for all dimensions.

The proof of the following lemma can be found in the appendix.

\begin{lemma}\label{lem:p_interleave}
 If $Q_w^\e$ surrounds $P^\e$ in $D$ and $D\setminus B_{w + \e}\subseteq P^\e$ then we have the following sequence of homomorphisms of degree $c\e$ induced by inclusions
 \[\DD{w-c\e}\xrightarrow{F}\E\PP{w}{\e}\xrightarrow{M}\DD{w+c\e}.\]
 % $F\in\Hom^{c\e}(\DD{w-c\e}, \E\PP{w}{\e})$ and $M\in\Hom^{c\e}(\E\PP{w}{\e}, \DD{w+c\e})$ indced by inclusions.
 % \[ D\subi{w-c\e}{a-c\e} \subseteq \ext{P\subi{w}{a}^\e}\subseteq D\subi{w+c\e}{a+c\e}.\]
\end{lemma}
% \begin{proof}
%   Suppose $x\in (P^\e\cap B\subi{w-c\e}{a-c\e})\setminus B_{w+\e}$.
%   Because $B_{w-\e}\subset B_{w+\e}$ we know $x\notin B_{w-\e}$ so $w+c\e < f(x)\leq a-c\e$ and there exists some $p\in P$ such that $\dist(x, p) < \e$.
%   Because $f$ is $c$-Lipschitz it follows
%   \[ f(p)\leq f(x) + c\dist(x, p) < a - c\e + c\e = a\]
%   and
%   \[ f(p)\geq f(x) - c\dist(x, p) > w+c\e-c\e = w.\]
%   So $x\in P\subi{w}{a}^\e$.
%
%   Now, suppose $x\in P\subi{w}{a}^\e\setminus B_{w+c\e}$.
%   So $w+c\e < f(x)$ and there exists some $p\in P\subi{w}{a}$ such that $\dist(x,p) < \e$.
%   Because $f$ is $c$-Lipschitz it follows
%   \[ f(x) \leq f(p) + c\dist(x,p) < a + c\e.\]
%   So $x\in B\subi{w+c\e}{a+c\e}\setminus B_{w+c\e}$.
%
%   Because $D\setminus B_{w+c\e}\subseteq P^\e$ we know that $D\setminus P^\e \subseteq B_{w+c\e}$, so
%   \[D\subi{w-c\e}{a-c\e}\setminus B_{w+c\e} \subseteq P\subi{w}{a}^\e\setminus B_{w+c\e}\subseteq D\subi{w+c\e}{a+c\e}\setminus B_{w+c\e}\]
%   implies
%   \[ D\subi{w-c\e}{a-c\e}\subseteq P\subi{w}{a}^\e\cup (D\setminus P^\e) = \ext{P\subi{w}{a}^\e} \subseteq D\subi{w+c\e}{a+c\e} \]
%   as desired.
% \end{proof}

% Let $\zeta\geq 2\delta$ and suppose $Q_{\omega-c\zeta}$ surrounds $P^\delta$ in $D$ and $D\setminus B_\omega\subseteq P^\delta$.
% Then, because $f$ is $c$-Lipschitz, $B_{\omega-c(\delta+\zeta)}\cap P^\delta\subseteq Q_{\omega-c\zeta}^\delta$ and $B_\omega\cap P^\delta\subseteq Q_{\omega+c\delta}^\zeta$.
% Similarly, $Q_{\omega-c\zeta}^{2\delta}\subseteq B_\omega$ and $Q_{\omega+c\delta}^{2\zeta}\subseteq B_{\omega+c{\delta+2\zeta}}$.
% Therefore, by Lemma~\ref{lem:surround_and_cover}
% \[ B_{\omega-c(\delta+\zeta)}\subseteq \E Q_{\omega-c\zeta}^\delta\subseteq\E Q_{\omega-c\zeta}^{2\delta}\subseteq B_\omega
%   \subseteq \E Q_{\omega+c\delta}^\zeta\subseteq \E Q_{\omega+c\delta}^{2\zeta}\subseteq B_{\omega+c{\delta+2\zeta}}.\]

% and
% \[ B_\omega\subseteq \E Q_{\omega+c\delta}^\zeta\subseteq \E Q_{\omega+c\delta}^{2\zeta}\subseteq B_{\omega+c{\delta+2\zeta}}.\]
