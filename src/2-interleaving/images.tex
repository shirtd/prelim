% !TeX root = ../../main.tex

In the TCC a nested pair of spaces is used in order to filter out noise introduced by the sample.
This same technique is used in the analysis of scalar fields~\cite{chazal09analysis} to interleave the persistent homology of a sequence of subspaces with that of a function.
These subspaces are simply the images of homomorphisms between homology groups induced by inclusion, and we refer to the resulting persistence module as an image persistence module.

\begin{definition}[Image Persistence Module]
  The \textbf{image persistence module} $\im~\Gamma$ of a homomorphism $\Gamma\in\Hom(\UU,\VV)$ is the family of subspaces $\{\Gamma_\alpha :=\im~\gamma_\alpha\}$ in $\VV$ along with linear maps $\{\gamma_\alpha^\beta := v_\alpha^\beta\rest_{\im~\gamma_\alpha} : \Gamma_\alpha\to\Gamma_\beta\}$.
\end{definition}

While we will primarily work with homomorphisms of persistence modules induced by inclusions defining homomorphisms between images simply as subspaces of the codomain is not sufficient in general.
Instead, we require that homomorphisms between image modules commute not only with shifts in scale, but also with the functions themselves.

\begin{definition}[Image Module Homomorphism]
  Given $\Gamma\in\Hom(\UU,\VV)$ and $\Lambda\in\Hom(\S,\T)$ along with $(F,G)\in\Hom^\delta(\UU,\S)\times\Hom^\delta(\VV,\T)$ let $\Phi(F, G) : \im~\Gamma\to\im~\Lambda$ denote the family of linear maps $\{\phi_\alpha := g_\alpha\rest_{\Gamma_\alpha} : \Gamma_\alpha\to\Lambda_{\alpha+\delta}\}$.

  $\Phi(F, G)$ is an \textbf{image module homomorphism of degree $\delta$} if the following diagram commutes for all $\alpha\leq\beta$.\footnote{Recall that $\gamma_\alpha[\beta-\alpha] = v_\alpha^\beta\circ\gamma_\alpha$ and $\lambda_\alpha[\beta-\alpha] = t_\alpha^\beta\circ\lambda_\alpha$.}

  \begin{equation}\label{dgm:image_homomorphism}
    \begin{tikzcd}[column sep=large]
        U_\alpha\arrow{r}{\gamma_\alpha[\beta-\alpha]}\arrow{d}{f_\alpha} &
      V_\beta\arrow{d}{g_\beta}\\
      %
      S_{\alpha+\delta}\arrow{r}{\lambda_{\alpha+\delta}[\beta-\alpha]} &
      T_{\beta +\delta}
  \end{tikzcd}\end{equation}
  The space of image module homomorphisms of degree $\delta$ between $\im~\Gamma$ and $\im~\Lambda$ will be denoted $\Hom^\delta(\im~\Gamma,\im~\Lambda)$.
\end{definition}

Note that the commutativity of Diagram~\ref{dgm:image_homomorphism} implies the following diagram of images commutes
\begin{equation}
  \begin{tikzcd}[column sep=large]
    \Gamma_\alpha\arrow{r}{\gamma_\alpha^\beta}\arrow{d}{\phi_\alpha} &
    \Gamma_\beta\arrow{d}{\phi_\beta}\\
    %
    \Lambda_{\alpha+\delta}\arrow{r}{\lambda_{\alpha+\delta}^{\beta+\delta}} &
    \Lambda_{\beta +\delta}
\end{tikzcd}\end{equation}
but the converse does not hold in general.
%
In the following the existence of an image module homomorphism $\Phi(F, G)\in\Hom^\delta(\im~\Gamma, \im~\Lambda)$ where $\Gamma\in\Hom(\UU,\VV)$ and $\Lambda\in\Hom(\S,\T)$  will imply that $(F,G)\in\Hom^\delta(\UU,\S)\times \Hom^\delta(\VV,\T)$.

% The proof of the following theorem may be found in the appendix.

\begin{lemma}\label{lem:image_composition}
  Suppose $\Gamma\in\Hom(\UU,\VV)$, $\Lambda\in\Hom(\S,\T)$, and $\Lambda'\in\Hom(\S',\T')$.
  If $\Phi(F, G)\in\Hom^\delta(\im~\Gamma, \im~\Lambda)$ and $\Phi'(F', G')\in\Hom^{\delta'}(\im~\Lambda, \im~\Lambda')$ then $\Phi''(F'\circ F, G'\circ G) := \Phi'\circ\Phi\in\Hom^{\delta+\delta'}(\im~\Gamma,\im~\Lambda')$.
\end{lemma}
\begin{proof}
  Because $\Phi(F, G)$ is an image module homomorphism of degree $\delta$ we have $g_{\beta-\delta}\circ\gamma_{\alpha-\delta}[\beta-\alpha] = \lambda_\alpha[\beta-\alpha]\circ f_{\alpha-\delta}$.
  Similarly, $g_{\beta}'\circ\lambda_{\alpha}[\beta-\alpha] = \lambda_{\alpha +\delta'}'[\beta-\alpha]\circ f_{\alpha}'$.
  So $\Phi''(F'\circ F, G'\circ G)\in\Hom^{\delta+\delta'}(\im~\Gamma,\im~\Lambda')$ as
  \[ g_\beta'\circ (g_{\beta-\delta}\circ \gamma_{\alpha-\delta}[\beta-\alpha]) = (g_\beta'\circ \lambda_\alpha[\beta-\alpha])\circ f_{\alpha-\delta} =\lambda_{\alpha+\delta'}[\beta-\alpha]\circ f_\alpha'\circ f_{\alpha-\delta}\]
  for all $\alpha\leq\beta$.
\end{proof}

Here, the notation $\Phi'\circ \Phi$ denotes the composition of pairs $(F'\circ F, G'\circ G)$.

\paragraph{Partial Interleavings of Image Modules}

Image module homomorphisms introduce a direction to the traditional notion of interleaving.
That is, given $\Gamma\in\Hom(\UU,\VV)$ and $\Lambda\in\Hom(\S,\T)$ and $\Phi(F, G)\in\Hom^\delta(\im~\Gamma, \im~\Lambda)$ we consider the case in which there is only a map $\S\to\VV$ that commutes.
As we will see, our interleaving via Lemma~\ref{thm:interleaving_main} involves partially interleaving an image module to two other image modules whose composition is isomorphic to our target.

\begin{definition}[Partial Interleaving of Image Modules]
  For homomorphisms $\Gamma\in\Hom(\UU,\VV)$ and $\Lambda\in\Hom(\S,\T)$ an image module homomorphism $\Phi(F, G)\in\Hom^\delta(\im~\Gamma,\im~\Lambda)$ is said to be a \textbf{left $\delta$-interleaving of image modules} if there exists some $M\in\Hom^\delta(\S,\VV)$ such that $\Gamma[2\delta] = M\circ F$.
  If $\Lambda[2\delta] = G\circ M$ then $\Phi(F, G)$ is a \textbf{right $\delta$-interleaving of image modules}.

  An image module homomorphism $\Phi(F, G)$ is a \textbf{partial $\delta$-interleaving of image modules}, and denoted $\Phi_M(F, G)$, if it is both a left and right $\delta$-interleaving of image modules.
\end{definition}

% Proof of the following lemma can be found in the appendix.
The following Lemma can be seen as the primary tool for the proof of our interleaving.
It uses partial interleavings surrounding a module $\VV$ to prove an interleaving of an image module with $\VV$.
When applied, the hypothesis of this Lemma will be satisfied by assumptions on our sublevel set similar to those made in the TCC.

\begin{lemma}\label{thm:interleaving_main}
  Suppose $\Gamma\in\Hom(\UU,\VV)$, $\Pi\in\Hom(\VV,\W)$, and $\Lambda\in\Hom(\S, \T)$.

  If $\Phi_M(F, G)\in\Hom^\delta(\im~\Gamma, \im~\Lambda)$ and $\Psi_G(M, N)\in\Hom^\delta(\im~\Lambda, \im~\Pi)$ are partial $\delta$-interleavings of image modules such that $\Gamma$ is a epimorphism and $\Pi$ is a monomorphism then $\im~\Lambda$ is $\delta$-interleaved with $\VV$.
\end{lemma}
\begin{proof}
  % For ease of notation let $\Phi$ denote $\Phi_M(F, G)$ and $\Psi$ denote $\Psi_G(M, N)$.
  %
  If $\Gamma$ is an epimorphism $\gamma_\alpha$ is surjective so $\Gamma_\alpha = V_\alpha$ and $\phi_{\alpha} = g_{\alpha}\rest_{\Gamma_\alpha} = g_\alpha$ for all $\alpha$.
  So $\im~\Gamma = \VV$ and $\Phi\in\Hom^\delta(\VV,\im~\Lambda)$.

  If $\Pi$ is a monomorphism then $\pi_\alpha$ is injective so we can define an isomorphism $\pi_\alpha^{-1} : \Pi_\alpha\to V_\alpha$ for all $\alpha$.
  Let $\Psi^*$ be defined as the family of linear maps $\{\psi_\alpha^* := \pi^{-1}_\alpha \circ \psi_\alpha : \Lambda_\alpha\to V_{\alpha+\delta}\}$.
  Because $\Psi$ is a partial $\delta$-interleaving of image modules, $n_\alpha\circ\lambda_\alpha = \pi_{\alpha+\delta}\circ m_\alpha$.
  So, because $\psi_\alpha = n_\alpha\rest_{\Lambda_\alpha}$ for all $\alpha$,
  \begin{align*}
    \im~\psi_\alpha^* = \im~\pi^{-1}_{\alpha+\delta}\circ\psi_\alpha = \im~\pi^{-1}_{\alpha+\delta}\circ (n_\alpha\circ\lambda_\alpha) = \im~\pi^{-1}_{\alpha+\delta}\circ (\pi_{\alpha+\delta}\circ m_\alpha) = \im~ m_\alpha.
  \end{align*}
  It follows that $\im~v_{\alpha+\delta}^{\beta+\delta}\circ\psi_\alpha^* = \im~v_{\alpha+\delta}^{\beta+\delta}\circ m_\alpha$

  Similarly, because $\Psi$ is a $\delta$-interleaving of image modules $n_\beta\circ t_\alpha^\beta\circ \lambda_\alpha = w_{\alpha+\delta}^{\beta+\delta}\circ\pi_{\alpha+\delta}\circ m_\alpha$.
  Moreover, because $\Pi$ is a homomorphism of persistence modules, $w_{\alpha+\delta}^{\beta+\delta}\circ\pi_{\alpha+\delta} = \pi_{\beta+\delta}\circ v_{\alpha+\delta}^{\beta+\delta}$, so $n_\beta\circ t_\alpha^\beta\circ \lambda_\alpha = \pi_{\beta+\delta}\circ v_{\alpha+\delta}^{\beta+\delta}\circ m_\alpha.$
  As $\psi_\beta\circ\lambda_\alpha^\beta = n_\beta\circ\lambda_\alpha^\beta = n_\beta\circ t_\alpha^\beta\rest_{\Lambda_\alpha}$ it follows
  \begin{align*}
    \im~\psi_\beta^*\circ\lambda_\alpha^\beta &= \im~\pi^{-1}_{\beta+\delta}\circ (n_\beta\circ t_\alpha^\beta\circ\lambda_\alpha)\\
      &= \im~\pi^{-1}_{\beta+\delta}\circ (\pi_{\beta+\delta}\circ v_{\alpha+\delta}^{\beta+\delta})\circ m_\alpha\\
      &= \im~v_{\alpha+\delta}^{\beta+\delta}\circ m_\alpha\\
      &= \im~v_{\alpha+\delta}^{\beta+\delta}\circ\psi_\alpha^*.
  \end{align*}
  So we may conclude that $\Psi^*\in\Hom^\delta(\im~\Lambda,\VV)$.

  So $\Phi\in\Hom^\delta(\VV,\im~\Lambda)$ and $\Psi_G^*\in\Hom^\delta(\im~\Lambda,\VV)$.
  As we have shown, $\im~\psi_{\alpha-\delta}^* = \im~m_{\alpha-\delta}$ so $\im~\phi_\alpha\circ\psi_{\alpha-\delta}^* = \im~\phi_\alpha\circ m_{\alpha-\delta}$.
  Moreover, because $\gamma_\alpha$ is surjective $\phi_\alpha = g_\alpha$ and, because $\Phi$ is a partial $\delta$-interleaving of image modules, $g_\alpha\circ m_{\alpha-\delta} = t_{\alpha-\delta}^{\alpha+\delta}\circ \lambda_{\alpha-\delta}$.
  As $\lambda_{\alpha-\delta}^{\alpha+\delta} = t_{\alpha-\delta}^{\alpha+\delta}\rest_{\im~\lambda_{\alpha-\delta}}$ it follows that the following diagram commutes as $\im~\phi_\alpha\circ\psi_{\alpha-\delta}^* = \im~\lambda_{\alpha-\delta}^{\alpha+\delta}$:
  \begin{equation}\label{dgm:interleaving1_1}
    \begin{tikzcd}
      & V_{\alpha}\arrow{dr}{\phi_\alpha} &\\
      %
      \Lambda_{\alpha-\delta}\arrow{rr}{\lambda_{\alpha-\delta}^{\alpha+\delta}}\arrow{ur}{\psi_{\alpha-\delta}^*} & &
      \Lambda_{\alpha+\delta}.
  \end{tikzcd}\end{equation}

  Finally, $\psi_\alpha^*\circ\phi_\alpha = \pi_{\alpha+\delta}^{-1}\circ n_\alpha\circ g_{\alpha-\delta}$ where, because $\Psi$ is a partial $\delta$-interleaving of image modules, $n_\alpha\circ g_{\alpha-\delta} = w_{\alpha-\delta}^{\alpha+\delta}\circ\pi_{\alpha-\delta}$.
  Because $\Pi$ is a homomorphism of persistence modules $w_{\alpha-\delta}^{\alpha+\delta}\circ \pi_{\alpha-\delta} = \pi_{\alpha+\delta}\circ v_{\alpha-\delta}^{\alpha+\delta}$.
  Therefore,
  \begin{align*}
    \psi_\alpha^*\circ\phi_{\alpha-\delta} = \pi_{\alpha+\delta}^{-1}\circ n_\alpha\circ g_{\alpha-\delta} = \pi_{\alpha+\delta}^{-1}\circ (\pi_{\alpha+\delta}\circ v_{\alpha-\delta}^{\alpha+\delta}) = v_{\alpha-\delta}^{\alpha+\delta}
  \end{align*}
  so the following diagram commutes
  \begin{equation}\label{dgm:interleaving2_1}
    \begin{tikzcd}
      V_{\alpha-\delta}\arrow{rr}{v_{\alpha-\delta}^{\alpha+\delta}}\arrow{dr}{\phi_\alpha} & &
      V_{\alpha+\delta}.\\
      %
      & \Lambda_{\alpha}\arrow{ur}{\psi_\alpha^*} &
  \end{tikzcd}\end{equation}

  Because $\Phi\in\Hom^\delta(\VV,\im~\Lambda)$, $\Psi^*\in\Hom^\delta(\im~\Lambda, \VV)$, and Diagrams~\ref{dgm:interleaving1_1} and~\ref{dgm:interleaving2_1} commute we may conclude that $\im~\Lambda$ and $\VV$ are $\delta$-interleaved.

\end{proof}
