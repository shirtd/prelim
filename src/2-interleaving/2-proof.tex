% !TeX root = ../../main.tex

% Let $\zeta\geq 2\delta$ and suppose $Q_{\omega-c\zeta}$ surrounds $P^\delta$ in $D$ and $D\setminus B_\omega\subseteq P^\delta$.
% Then, because $f$ is $c$-Lipschitz, $B_{\omega-c(\delta+\zeta)}\cap P^\delta\subseteq Q_{\omega-c\zeta}^\delta$ and $B_\omega\cap P^\delta\subseteq Q_{\omega+c\delta}^\zeta$.
% Similarly, $Q_{\omega-c\zeta}^{2\delta}\subseteq B_\omega$ and $Q_{\omega+c\delta}^{2\zeta}\subseteq B_{\omega+c{\delta+2\zeta}}$.
% Therefore, by Lemma~\ref{lem:surround_and_cover}
% \[ B_{\omega-c(\delta+\zeta)}\subseteq \E Q_{\omega-c\zeta}^\delta\subseteq\E Q_{\omega-c\zeta}^{2\delta}\subseteq B_\omega
%   \subseteq \E Q_{\omega+c\delta}^\zeta\subseteq \E Q_{\omega+c\delta}^{2\zeta}\subseteq B_{\omega+c{\delta+2\zeta}}.\]

% and
% \[ B_\omega\subseteq \E Q_{\omega+c\delta}^\zeta\subseteq \E Q_{\omega+c\delta}^{2\zeta}\subseteq B_{\omega+c{\delta+2\zeta}}.\]

Let $\zeta\geq 2\delta$ and suppose $Q_{\omega-c\zeta}$ surrounds $P^\delta$ in $D$ and $D\setminus B_\omega\subseteq P^\delta$.
Then, because $f$ is $c$-Lipschitz, $B_{\omega-c(\delta+\zeta)}\cap P^\delta\subseteq Q_{\omega-c\zeta}^\delta$ and $B_\omega\cap P^\delta\subseteq Q_{\omega+c\delta}^\zeta$.
Similarly, $Q_{\omega-c\zeta}^{2\delta}\subseteq B_\omega$ and $Q_{\omega+c\delta}^{2\zeta}\subseteq B_{\omega+c{\delta+2\zeta}}$.
Therefore, by Lemma~\ref{lem:surround_and_cover}
\[ B_{\omega-c(\delta+\zeta)}\subseteq \E Q_{\omega-c\zeta}^\delta\subseteq\E Q_{\omega-c\zeta}^{2\delta}\subseteq B_\omega
  \subseteq \E Q_{\omega+c\delta}^\zeta\subseteq \E Q_{\omega+c\delta}^{2\zeta}\subseteq B_{\omega+c{\delta+2\zeta}}.\]

% \subsubsection{Rips-Function Interleaving}

For $w\in\R$ and $k\in\Z$ let
\[ \DD{w}^k := \left(\left\{\D{w}{\alpha}^k := \hom_k(D\subi{w}{\alpha},B_w)\right\}_{\alpha\in\R},\left\{d\subi{w}{\alpha,\beta}^k : \D{w}{\alpha}^k\to\D{w}{\beta}^k\right\}_{\alpha\leq\beta}\right)\]
denote the $k$th persistent homology module of the sub-levelset filtration modulo $B_w$, $\{(D\subi{w}{\alpha},B_w)\}$.
Once again, we will omit the dimension $k$ and write $\DD{w}$ if a statement holds for all dimensions.

The proof of the following lemma can be found in the appendix.

\begin{lemma}\label{lem:p_interleave}
 If $Q_w^\e$ surrounds $P^\e$ in $D$ and $D\setminus B_{w + \e}\subseteq P^\e$ then we have the following sequence of homomorphisms of degree $c\e$ induced by inclusions
 \[\DD{w-c\e}\xrightarrow{F}\E\PP{w}{\e}\xrightarrow{M}\DD{w+c\e}.\]
 % $F\in\Hom^{c\e}(\DD{w-c\e}, \E\PP{w}{\e})$ and $M\in\Hom^{c\e}(\E\PP{w}{\e}, \DD{w+c\e})$ indced by inclusions.
 % \[ D\subi{w-c\e}{a-c\e} \subseteq \ext{P\subi{w}{a}^\e}\subseteq D\subi{w+c\e}{a+c\e}.\]
\end{lemma}
% \begin{proof}
\proofatend
  Suppose $x\in (P^\e\cap B\subi{w-c\e}{\alpha-c\e})\setminus B_{w+\e}$.
  Because $B_{w-\e}\subset B_{w+\e}$ we know $x\notin B_{w-\e}$ so $w+c\e < f(x)\leq \alpha-c\e$ and there exists some $p\in P$ such that $\dist(x, p) < \e$.
  Because $f$ is $c$-Lipschitz it follows
  \[ f(p)\leq f(x) + c\dist(x, p) < \alpha - c\e + c\e = \alpha\]
  and
  \[ f(p)\geq f(x) - c\dist(x, p) > w+c\e-c\e = w.\]
  So $x\in P\subi{w}{\alpha}^\e$.

  Now, suppose $x\in P\subi{w}{\alpha}^\e\setminus B_{w+c\e}$.
  So $w+c\e < f(x)$ and there exists some $p\in P\subi{w}{\alpha}$ such that $\dist(x,p) < \e$.
  Because $f$ is $c$-Lipschitz it follows
  \[ f(x) \leq f(p) + c\dist(x,p) < a + c\e.\]
  So $x\in B\subi{w+c\e}{\alpha+c\e}\setminus B_{w+c\e}$.

  Because $D\setminus B_{w+c\e}\subseteq P^\e$ we know that $D\setminus P^\e \subseteq B_{w+c\e}$, so
  \[D\subi{w-c\e}{\alpha-c\e}\setminus B_{w+c\e} \subseteq P\subi{w}{\alpha}^\e\setminus B_{w+c\e}\subseteq D\subi{w+c\e}{\alpha+c\e}\setminus B_{w+c\e}\]
  implies
  \[ D\subi{w-c\e}{\alpha-c\e}\subseteq P\subi{w}{\alpha}^\e\cup (D\setminus P^\e) = \ext{P\subi{w}{\alpha}^\e} \subseteq D\subi{w+c\e}{\alpha+c\e} \]
  as desired.

  % Let $\zeta\geq 2\delta$ and suppose $Q_{\omega-c\zeta}$ surrounds $P^\delta$ in $D$ and $D\setminus B_\omega\subseteq P^\delta$.
  Because $f$ is $c$-Lipschitz, $B_{w-c\e}\cap P^\delta\subseteq Q_{w}^\e$ so $B_{w-c\e} \subseteq \E Q_w^\e\subseteq B_{w+c\e}$ by Lemma~\ref{lem:surround_and_cover}.
  It follows that we have homomorphisms $F\in \Hom^{c\e}(\DD{w-c\e}, \E\PP{w}{\e})$ and $M\in\Hom^{c\e}(\E\PP{w}{\e}, \DD{w+c\e})$ induced by inclusions.

  %  and $B_w\cap P^\delta\subseteq Q_{\omega+c\delta}^\zeta$.
  % Similarly, $Q_{\omega-c\zeta}^{2\delta}\subseteq B_\omega$ and $Q_{\omega+c\delta}^{2\zeta}\subseteq B_{\omega+c{\delta+2\zeta}}$.
  % Therefore, by Lemma~\ref{lem:surround_and_cover}
  % \[ B_{\omega-c(\delta+\zeta)}\subseteq \E Q_{\omega-c\zeta}^\delta\subseteq\E Q_{\omega-c\zeta}^{2\delta}\subseteq B_\omega
  %   \subseteq \E Q_{\omega+c\delta}^\zeta\subseteq \E Q_{\omega+c\delta}^{2\zeta}\subseteq B_{\omega+c{\delta+2\zeta}}.\]
\endproofatend
%\end{proof}

So we have the following commutative diagrams of persistence modules where all maps are induced by inclusions.
\[\begin{tikzcd}
    \DD{\omega-c(\delta+\zeta)} \arrow{r}{\Gamma}\arrow{d}{F} &
    \DD{\omega} \arrow{d}{G}\\
    %
    \E\PP{\omega-c\zeta}{\delta}\arrow{r}{\Lambda} &
    \E\PP{\omega+c\delta}{\zeta}
  \end{tikzcd}\hspace{10ex}
  \begin{tikzcd}
    \E\PP{\omega-c\zeta}{2\delta} \arrow{r}{\Lambda'}\arrow{d}{M} &
    \E\PP{\omega+c\delta}{2\zeta}\arrow{d}{N}\\
    %
    \DD{\omega} \arrow{r}{\Pi} &
    \DD{\omega+c(\delta+2\zeta)}.
  \end{tikzcd}\]

In the following let $\rips\Lambda\in\Hom(\RPP{\omega-c\zeta}{2\delta},\RPP{\omega+c\delta}{2\zeta})$ be induced by inclusion.
Clearly, $\Phi(F, G)$ is an image module homomorphism of degree $c\zeta$ and $\Psi(M, N)$ is an image module homomorphism of degree $2c\zeta$.
By Lemma~\ref{lem:rips_homomorphism_left} we have image module homomorphisms $\tilde{\Phi}(\Sigma_{\omega-c\zeta}^\delta, \Sigma_{\omega+c\delta}^\zeta)$ and $\tilde{\Psi}(\Upsilon_{\omega-c\zeta}^{2\delta}, \Upsilon_{\omega+c\delta}^{2\zeta})$.
% Having defined image module homomorphisms $\tilde{\Phi}(\Sigma_{\omega-c\zeta}^\delta,\Sigma_{\omega+c\delta}^\zeta)$ and $\tilde{\Psi}(\Upsilon_{\omega-c\zeta}^{2\delta},\Upsilon_{\omega+c\delta}^{2\zeta})$ let $\tilde{\Phi}\circ\Phi$ and $\Psi\circ\tilde{\Psi}$ denote the composition of pairs $(\Sigma_{\omega-c\zeta}^\delta\circ F,\Sigma_{\omega+c\delta}^\zeta\circ G)$ and $(M\circ \Upsilon_{\omega-c\zeta}^{2\delta},N\circ\Upsilon_{\omega+c\delta}^{2\zeta})$, respectively.
% By Lemmas~\ref{lem:rips_homomorphism_left} and~\ref{lem:rips_homomorphism_right} we have image module homomorphisms $\tilde{\Phi}(\Sigma_{\omega-c\zeta}^\delta, \Sigma_{\omega+c\delta}^\zeta)\in\Hom(\im~\Lambda, \im~\rips\Lambda)$ and $\tilde{\Psi}(\Upsilon_{\omega-c\zeta}^{2\delta}, \Upsilon_{\omega+c\delta}^{2\zeta})\in\Hom(\im~\rips\Lambda,\im~\Lambda')$.
Therefore, by Lemma~\ref{lem:image_composition} we have image module homomorphisms
\[ \rips\Phi := \tilde{\Phi}\circ\Phi\in\Hom^{c\zeta}(\im~\Gamma,\im~\rips\Lambda)\ \text{ and }\ \rips\Psi :=\Psi\circ\tilde{\Psi}\in\Hom^{2c\zeta}(\im~\rips\Lambda, \im~\Pi)\] given by the compositions
\[ \rips\Phi(\rips F, \rips G) := (\Sigma_{\omega-c\zeta}^\delta\circ F, \Sigma_{\omega+c\delta}^\zeta\circ G)\]
and
\[ \rips\Psi(\rips M, \rips N) := (M\circ \Upsilon_{\omega-c\zeta}^{2\delta}, N\circ\Upsilon_{\omega+c\delta}^{2\zeta}).\]

\begin{lemma}\label{lem:rips_factor_mid}
  The pair $(\rips M, \rips G)$ factors $\rips\Lambda[2c\delta+c\zeta]$ through $\DD{\omega}$.
\end{lemma}
\begin{proof}
  Let $\Theta\in\Hom(\ext{\PP{\omega-c\zeta}{2\delta}},\ext{\PP{\omega+c\delta}{\zeta}})$ and $\cech\Theta\in\Hom(\CPP{\omega-c\zeta}{2\delta}, \CPP{\omega+c\delta}{\zeta})$ be induced by inclusions so that $\Theta[2c\delta+c\zeta] = G\circ M$ and $\rips\Lambda = \I_{\omega+c\delta}^\zeta\circ\cech\Theta\circ\J_{\omega-c\zeta}^{2\delta}$.
  So $\cech\Theta$ factors through $\Theta$ with the pair $(\E\N_{\omega-c\zeta}^{2\delta}, (\E\N_{\omega+c\delta}^\zeta)^{-1})$ by Lemma~\ref{cor:excisive_nerve}.
  That is,
  \begin{align*}
    \rips\Lambda &= \I_{\omega+c\delta}^\zeta\circ\cech\Theta\circ\J_{\omega-c\zeta}^{2\delta}\\
      &= (\I_{\omega+c\delta}^\zeta\circ (\E\N_{\omega+c\delta}^\zeta)^{-1})\circ \Theta\circ (\E\N_{\omega-c\zeta}^{2\delta}\circ \J_{\omega-c\zeta}^{2\delta})\\
      &= \Sigma_{\omega+c\delta}^\zeta\circ \Theta\circ \Upsilon_{\omega-c\zeta}^{2\delta}\\
  \end{align*}
  As $\Theta[2c\delta+c\zeta] = G\circ M$ the result follows from the definition
  \[ \rips\Lambda[2c\delta+c\zeta] = (\Sigma_{\omega+c\delta}^\zeta\circ G)\circ (M\circ \Upsilon_{\omega-c\zeta}^{2\delta}) = \rips G\circ \rips M.\]
\end{proof}

\begin{corollary}\label{cor:rips_inter_left}
  $\rips \Phi_{\rips M} := \tilde{\Phi}\circ \Phi\in\Hom^{2c\delta}(\im~\Gamma,\im~\rips\Lambda)$ is a partial $c\zeta$-interleaving of image modules.
\end{corollary}
\begin{proof}
  Because $F,M$ are induced by inclusions and $\Upsilon_{\omega-c\zeta}^{2\delta}\circ \Sigma_{\omega-c\zeta}^{\delta}$ commutes with inclusion it follows that
  \[\Gamma[3c\delta] = M\circ (\Upsilon_{\omega-c\zeta}^{2\delta}\circ \Sigma_{\omega-c\zeta}^{\delta})\circ F = \rips M\circ \rips F.\]
  So $\rips\Phi$ with $\rips M$ is a left $2c\delta$-interleaving of image modules.
  As Lemma~\ref{lem:rips_factor_mid} implies $\rips \Phi$ (with $\rips M$) is a right $c\zeta$-interleaving of image modules it follows that $\rips \Phi_{\rips M}$ is a partial $c\zeta$-interleaving of image modules.
\end{proof}

\begin{corollary}\label{cor:rips_inter_right}
  $\rips \Psi_{\rips G} := \Psi\circ\tilde{\Psi}\in\Hom^{2c\zeta}(\im~\rips\Lambda, \im~\Pi)$ is a partial $2c\zeta$-interleaving of image modules.
\end{corollary}
\begin{proof}
  This proof is identical to that of Corollary~\ref{cor:rips_inter_left}.
  Because $G,N$ are induced by inclusions and $\Upsilon_{\omega+c\delta}^{2\zeta}\circ \Sigma_{\omega+c\delta}^{\zeta}$ commutes with inclusion
  \[\Pi[3c\zeta] = N\circ (\Upsilon_{\omega+c\delta}^{2\zeta}\circ \Sigma_{\omega+c\delta}^{\zeta})\circ G = \rips N\circ \rips G.\]
  So $\rips\Psi$ with $\rips G$ is a right $2c\zeta$-interleaving of image modules.
  As Lemma~\ref{lem:rips_factor_mid} implies $\rips \Psi$ (with $\rips G$) is a left $c\zeta$-interleaving of image modules it follows that $\rips \Psi_{\rips G}$ is a partial $2c\zeta$-interleaving of image modules.
\end{proof}
% \begin{proof}
%   Because $\Pi\in\Hom(\DD{\omega},\DD{\omega+c(\delta+2\zeta)})$, $G\in\Hom^{c\zeta}(\DD{\omega}, \ext{\PP{\omega+c\delta}{\zeta}})$, and $N\in\Hom^{2c\zeta}(\ext{\PP{\omega+c\delta}{2\zeta}})$ are induced by inclusion $\Pi[3c\zeta] = N\circ \mathcal{T}\circ G$, so $(G, N)$ is a weak $2c\zeta$-interleaving of $\Pi$ with $\mathcal{T}$.
%   Once again, by Lemma~\ref{lem:weak_rips_left}, $(\Sigma_{\omega+c\delta}^{\zeta},\Upsilon_{\omega+c\delta}^{2\zeta})$ is a weak $2c\zeta$-interleaving of $\mathcal{T}$ with $\RPP{\omega+c\delta}{2\zeta}$.
%   So $(\Sigma_{\omega+c\delta}^{\zeta}\circ G, N\circ \Upsilon_{\omega+c\delta}^{2\zeta}) = (\rips G, \rips N)$ is a weak $2c\zeta$-interleaving of $\Pi[3c\zeta]$ with $\RPP{\omega+c\delta}{2\zeta}$ by Lemma~\ref{lem:left}.
% \end{proof}
%
% \begin{proof}
%   Let $\mathcal{T}\in\Hom(\ext{\PP{\omega+c\delta}{\zeta}},\ext{\PP{\omega+c\delta}{2\zeta}})$ be induced by inclusions.
%   Because $G,N$ are induced by inclusions it follows that $\Pi[3c\zeta] = N\circ\mathcal{T}\circ G$.
%   So $(G, N)$ factors $\Gamma[3c\zeta]$ through $\mathcal{T}$.
%   Moreover,
%   \[ \mathcal{T} = (\E\N_{\omega+c\delta}^{2\zeta}\circ\J_{\omega+c\zeta}^{2\zeta})\circ(\I_{\omega+c\delta}^\zeta\circ (\E\N_{\omega+c\delta}^\zeta)^{-1}) = \Upsilon_{\omega+c\delta}^{2\zeta}\circ \Sigma_{\omega+c\delta}^\zeta\] by Lemma~\ref{cor:excisive_nerve}.
%   It follows that the pair $(\rips G,\rips N) = (\Sigma_{\omega+c\delta}^\zeta\circ G, N\circ \Upsilon_{\omega+c\delta}^{2\zeta})$ factors $\Pi[3c\zeta]$ through $\RPP{\omega+c\delta}{2\zeta}$.
%
%   So $\rips\Psi$ with $\rips M$ is a left $2c\delta$-interleaving of image modules and.
%   As Lemma~\ref{lem:rips_factor_mid} implies $\rips\Phi$ is a right $c\zeta$-interleaving of image modules it follows that $\rips\Phi_{\rips M}(\rips F, \rips G)$ is a partial $c\zeta$-interleaving of image modules.
% \end{proof}

\begin{theorem}
  Let $D\subset\X$ and $f : D\to\R$ be a $c$-Lipschitz function.
  Let $\omega\in\R$, $2\delta\leq\zeta\leq\varrho_D/2$ be constants such that $B_{\omega-c(\delta+\zeta)}$ surrounds $D$ in $\X$.
  Let $P\subset D$ be a finite subset and $Q_w := P\cap B_w$.
  Suppose $\hom_k(B_{\omega-c(\delta+\zeta)}\hookrightarrow B_\omega)$ is surjective and $\hom_k(B_\omega)\cong\hom_k(B_{\omega+c(\delta+2\zeta)})$ for all $k$.

  If $D\setminus B_\omega\subseteq P^\delta$ and $Q_{\omega-c\zeta}^\delta$ surrounds $P^\delta$ in $D$ then the image module $\im~(\RPP{\omega-c\zeta}{2\delta}\to \RPP{\omega+c\delta}{2\zeta})$ is $2c\zeta$-interleaved with $\DD{\omega}$.
\end{theorem}
\begin{proof}
  Because $D\setminus B_\omega\subseteq P^\delta$ and $Q_{\omega-c\zeta}^\delta$ surrounds $P^\delta$ in $D$ Diagrams~\ref{TODO} and~\ref{TODO} commute as all maps are induced by inclusions.
  Moreover, because $\zeta < \varrho_D/2$ the isomorphisms provided by the Nerve Theorem commute with inclusions by Lemma~\ref{cor:excisive_nerve}.

  Let $\rips\Lambda \in\Hom(\RPP{\omega-c\zeta}{2c\delta}, \RPP{\omega+c\delta}{2c\zeta})$ be induced by inclusions.
  By Corollary~\ref{cor:rips_inter_left} $\rips \Phi_{\rips M}(\rips F, \rips G)\in\Hom^{2c\delta}(\im~\Gamma,\im~\rips\Lambda)$ is a partial $c\zeta$-interleaving of image modules.
  Similarly, by Corollary~\ref{cor:rips_inter_right} $\rips \Psi_{\rips G} (\rips M,\rips N)\in\Hom^{2c\zeta}(\im~\rips\Lambda, \im~\Pi)$ is a partial $2c\zeta$-interleaving of image modules.

  As we have assumed that $\hom_k(B_{\omega-c(\delta+\zeta)}\hookrightarrow B_\omega)$ is surjective and $\hom_k(B_\omega)\cong\hom_k(B_{\omega+c(\delta+2\zeta)})$ the five-lemma implies $\gamma_\alpha$ is surjective and $\pi_\alpha$ is an isomorphism (and therefore injective) for all $\alpha$.
  So $\Gamma$ is an epimorphism and $\Pi$ is a monomorphism, thus $\im~\rips\Lambda$ is $2c\zeta$-interleaved with $\DD{\omega}$ by Lemma~\ref{thm:interleaving_main} as desired.
\end{proof}

% \begin{theorem}
%   Let $D\subset\X$ and $f : D\to\R$ be a $c$-Lipschitz function.
%   Let $\omega\in\R$, $\zeta\geq 2\delta\geq 0$ be constants such that
%   \begin{enumerate}[label=\Roman*.]
%     \item $B_{\omega-c(\delta+\zeta)}$ surrounds $D$ in $\X$,
%     \item $\hom_k(B_{\omega-c(\delta+\zeta)}\hookrightarrow B_\omega)$ is surjective, and
%     \item $\hom_k(B_\omega)\cong\hom_k(B_{\omega+c(\delta+2\zeta)})$
%   \end{enumerate}
%   for all $k$.
%   Let $P\subset D$ be a finite subset and $Q_w := P\cap B_w$.
%
%   If $D\setminus B_\omega\subseteq P^\delta$ and $Q_{\omega-c\zeta}^\delta$ surrounds $P^\delta$ in $D$ then the $k$th persistent homology modules of $\{(D\subi{\omega,\alpha}, B_\omega)\}$ and
%   \[
%     \{(\rips^{2\delta}(P\subi{\omega-c\zeta,\alpha}), \rips^{2\delta}(Q_{\omega-c\zeta})) \hookrightarrow
%       (\rips^{2\zeta}(P\subi{\omega+c\delta,\alpha}), \rips^{2\zeta}(Q_{\omega+c\delta}))\}
%   \]
%   are $2c\zeta$-interleaved.
% \end{theorem}
% \begin{proof}
%   By Lemma~\ref{lem:surround_and_cover} and Lemma~\ref{lem:p_interleave}
%   \[ (D\subi{\omega-c(\delta+\zeta),\alpha-c\delta},\b)\subseteq (\ext{P^\e\subi{\omega-c\zeta,\alpha}},\ext{\Q^\e})\subseteq (D\subi{\omega,\alpha+c\e},\B)\]
%   and
%   \[ (D\subi{\omega,\alpha-c\delta},\B)\subseteq (\ext{P^\e\subi{\omega+c\delta,\alpha}},\ext{\QQ^\e})\subseteq (D\subi{\omega+c(\delta+\zeta),\alpha+c\e},\BB)\]
%   for all $\alpha\in\R$ and $\delta\leq\e\leq\zeta$.
%
%   Let $\Gamma\in\Hom(\DD{\omega-c(\delta+\zeta)},\DD{\omega})$ and $\Lambda\in\Hom(\ext{\PP{\omega-c\zeta}{\delta}},\ext{\PP{\omega+c\delta}{\zeta}})$.
%   As all maps are induced by inclusion we have $\Phi(F, G)\in\Hom^{c\zeta}(\im~\Gamma, \im~\Lambda).$
%   Similarly, for $\Pi\in\Hom(\DD{w},\DD{\omega+c(\delta+2\zeta)})$ and $\Lambda\in\Hom(\ext{\PP{\omega-c\zeta}{2\delta}},\ext{\PP{\omega+c\delta}{2\zeta}})$ we have $\Psi(M, N)\in\Hom^{2c\zeta}(\im~\Lambda',\im~\Pi).$
%
%   By Lemma~\ref{lem:rips_homomorphisms} we have
%   \[ \tilde{\Phi}(\Sigma_{\omega-c\zeta}^\delta,\Sigma_{\omega+c\delta}^\zeta)\in\Hom(\im~\Lambda,\im~\rips\Lambda)\]
%   and
%   \[ \tilde{\Psi}(\Upsilon_{\omega-c\zeta}^{2\delta},\Upsilon_{\omega+c\delta}^{2\zeta})\in\Hom(\im~\rips\Lambda,\im~\Lambda')\]
%   therefore, letting
%   \[ (\rips F, \rips G) := (\Sigma_{\omega-c\zeta}^\delta\circ F, \Sigma_{\omega+c\delta}^\zeta\circ G)\]
%   and
%   \[ (\rips F, \rips G) := (M\circ \Upsilon_{\omega-c\zeta}^{2\delta}, N\circ\Upsilon_{\omega+c\delta}^{2\zeta})\]
%   it follows from Lemma~\ref{lem:image_composition} that $\rips \Phi (\rips F, \rips G)\in\Hom^{c\zeta}(\im~\Gamma,\im~\rips\Lambda)$ is an image module homomorphism of degree $c\zeta$ and $\rips\Psi(\rips M, \rips N)\in\Hom^{2c\zeta}(\im~\rips\Lambda,\im~\Pi)$ is an image module homomorphism of degree $2c\zeta$.
%
%
%   Let $\mathcal{S}\in\Hom(\ext{\PP{\omega-c\zeta}{\delta}},\ext{\PP{\omega-c\zeta}{2\delta}})$, $\Theta\in\Hom(\ext{\PP{\omega-c\zeta}{2\delta}},\ext{\PP{\omega+c\delta}{\zeta}})$, and $\mathcal{T}\in\Hom(\ext{\PP{\omega+c\delta}{\zeta}},\ext{\PP{\omega+c\delta}{2\zeta}})$ be induced by inclusions so that $\Lambda = \Theta\circ \mathcal{S}$ and $\Lambda' = \Theta\circ \mathcal{T}$.
%
%   Because $\Gamma\in\Hom(\DD{\omega-c(\delta+\zeta)},\DD{\omega})$, $F\in\Hom^{c\delta}(\DD{\omega-c(\delta+\zeta)}, \ext{\PP{\omega-c\zeta}{\delta}})$, and $M\in\Hom^{2c\delta}(\ext{\PP{\omega-c\zeta}{2\delta}}, \DD{\omega})$ are induced by inclusions $\Gamma[3c\delta] = M\circ\mathcal{S}\circ F$.
%   So $(F, M)$ is a weak $2c\delta$-interleaving of $\Gamma[3c\delta]$ with $\mathcal{S}$.
%   By Lemma~\ref{lem:weak_rips_left} $(\Sigma_{\omega-c\zeta}^{\delta},\Upsilon_{\omega-c\zeta}^{2\delta})$ is a weak interleaving of $\mathcal{S}$ with $\RPP{\omega-c\zeta}{2\delta}$.
%   So $(\Sigma_{\omega-c\zeta}^{\delta}\circ F, M\circ \Upsilon_{\omega-c\zeta}^{2\delta}) = (\rips F, \rips M)$ is a weak $2c\delta$-interleaving of $\Gamma[3c\delta]$ with $\RPP{\omega-c\zeta}{2\delta}$ by Lemma~\ref{lem:left}.
%
%   Similarly, because $\Pi\in\Hom(\DD{\omega},\DD{\omega+c(\delta+2\zeta)})$, $G\in\Hom^{c\zeta}(\DD{\omega}, \ext{\PP{\omega+c\delta}{\zeta}})$, and $N\in\Hom^{2c\zeta}(\ext{\PP{\omega+c\delta}{2\zeta}})$ are induced by inclusion $\Pi[3c\zeta] = N\circ \mathcal{T}\circ G$, so $(G, N)$ is a weak $2c\zeta$-interleaving of $\Pi$ with $\mathcal{T}$.
%   Once again, by Lemma~\ref{lem:weak_rips_left}, $(\Sigma_{\omega+c\delta}^{\zeta},\Upsilon_{\omega+c\delta}^{2\zeta})$ is a weak $2c\zeta$-interleaving of $\mathcal{T}$ with $\RPP{\omega+c\delta}{2\zeta}$.
%   So $(\Sigma_{\omega+c\delta}^{\zeta}\circ G, N\circ \Upsilon_{\omega+c\delta}^{2\zeta}) = (\rips G, \rips N)$ is a weak $2c\zeta$-interleaving of $\Pi[3c\zeta]$ with $\RPP{\omega+c\delta}{2\zeta}$ by Lemma~\ref{lem:left}.
%
%   By Corollary~\ref{cor:excisive_nerve} we know that $\cech\Theta := (\E\N_{\omega+c\delta}^\zeta)^{-1}\circ \Theta\circ \E\N_{\omega-c\zeta}^{2\delta}$
%   where, because all maps are induced by inclusions,
%   \begin{align*}
%     \rips\Lambda &= \I_{\omega+c\delta}^\zeta\circ\cech\Theta\circ\J_{\omega-c\zeta}^{2\delta}\\
%       &= (\I_{\omega+c\delta}^\zeta\circ (\E\N_{\omega+c\delta}^\zeta)^{-1})\circ \Theta\circ (\E\N_{\omega-c\zeta}^{2\delta}\circ \J_{\omega-c\zeta}^{2\delta})\\
%       &= \Sigma_{\omega+c\delta}^\zeta\circ \Theta\circ \Upsilon_{\omega-c\zeta}^{2\delta}\\
%   \end{align*}
%   Because $\Theta[2c\delta+c\zeta] = G\circ M$ the pair $(M, G)$ is a weak $c\zeta$-interleaving of $\Theta[2c\delta+c\zeta]$ with $\DD{\omega}$.
%   Therefore,
%   \begin{align*}
%     \rips\Lambda[2c\delta+c\zeta] &= \Sigma_{\omega+c\delta}^\zeta\circ \Theta[2c\delta+c\zeta]\circ \Upsilon_{\omega-c\zeta}^{2\delta}\\
%       &= (\Sigma_{\omega+c\delta}^\zeta\circ G)\circ (M\circ \Upsilon_{\omega-c\zeta}^{2\delta})\\
%       &= \rips G\circ \rips M.
%   \end{align*}
%   So $(\rips M, \rips G)$ is a weak $c\zeta$-interleaving of $\rips\Lambda$ with $\DD{\omega}$.
%
%   Because $(\rips F,\rips M)$ is a weak $2c\delta$-interleaving $\rips \Phi$ is a left $2c\delta$-interleaving of image modules.
%   Because $(\rips G, \rips N)$ is a weak $2c\zeta$-interleaving $\rips \Psi$ is a right $2c\zeta$-interleaving of image modules.
%   Finally, because $(\rips M, \rips G)$ is a weak $c\zeta$-interleaving $\rips \Phi$ is a right $c\zeta$-interleaving of image modules and $\rips \Psi$ is a left $c\zeta$-interleaving of image modules.
%   So $\rips \Phi_{\rips M}(\rips F, \rips G)$ is a partial $c\zeta$-interleaving of image modules and $\rips \Psi_{\rips G} (\rips M,\rips N)$ is a partial $2c\zeta$-interleaving of image modules.
%
%   As we have assumed that $\hom_k(B_{\omega-c(\delta+\zeta)}\hookrightarrow B_\omega)$ is surjective and $\hom_k(B_\omega)\cong\hom_k(B_{\omega+c(\delta+2\zeta)})$ Lemma~\ref{lem:pt_interleaving} implies $\gamma_\alpha$ is surjective and $\pi_\alpha$ is an isomorphism (and therefore injective) for all $\alpha$.
%   So $\Gamma$ is an epimorphism and $\Pi$ is a monomorphism, thus $\im~\rips\Lambda$ is $2c\zeta$-interleaved with $\DD{\omega}$ by Lemma~\ref{thm:interleaving_main} as desired.
% \end{proof}
