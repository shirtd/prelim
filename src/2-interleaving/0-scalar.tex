% !TeX root = ../../main.tex

Our re-statement of the TCC in terms of a surrounding sub-levelset $B_\omega$ of a $c$-Lipschitz function $f : D\to \R$ sets us up with (most of) the machinery we need to approximate the persistent homology of the function $f$.
In this section we will shift our focus from confirming coverage to approximating the persistent homology of $f$ by a sample that satisfies the TCC.

As we would like to analyze the persistent homology of $f$ in a way that extends the TCC we will not approximate the persistent homology of $f$ as in previous work~\cite{chazal09analysis}.
Assuming we have a sample that satisfies the TCC we do not know that we cover the entire domain, but only the \emph{super-levelset} $D\setminus B_\omega$.
Indeed, we could compute the persistent homology of the sub-levelset filtration restricted to this super-levelset, or that of the super-levelset filtration down to $\omega$.
Instead, we approximate the persistent homology of the sub-levelset filtration \emph{modulo} the sub-levelset $B_\omega$.
In the next section we explore the meaning of the resulting diagram in the context of the full diagram and compare it with those of the restricted sub and super-levelset filtrations.

As in the previous section let $\X$ be an oriented $d$-manifold and let $D$ be a compact subset.
Let $f: D\to\R$ be a $c$-lipschitz function and $B_w := f^{-1}((-\infty,a])$ denote a sub-levelset of $f$ at scale $w\in\R$.
Let $P$ be a finite collection of points in $D$ and $Q_w := P\cap B_w$.

Note that $\{B_\alpha\}_{\alpha\in\R}$ is precisely the sub-levelset filtration of $f$.
Because we will not assume coverage below some $\omega\in\R$ the persistent homology of $\{Q_\alpha^\delta\}_{\alpha\in\R}$ or even $\{Q_\alpha^\delta\setminus B_\omega\}_{\alpha\in\R}$ cannot be trusted as a reliable approximation.
In fact, because of the nature of homology as a global property we cannot assume either of these filtrations capture anything meaningful even for $\alpha >> \omega$.

We introduce the following notation to distinguish $B_w$ and $Q_w$ as static sub-levelsets with which we will compute take persistent homology with respect to.

\paragraph{Notation}

For $w,\alpha\in\R$ let $D\subi{w}{\alpha} := B_w\cup B_\alpha$ and $P\subi{w}{\alpha} := P\cap D\subi{w}{\alpha}$.
% \[ D\subi{w}{\alpha} := B_w\cup B_\alpha\ \text{ and }\ \ P\subi{w}{\alpha} := P\cap D\subi{w}{\alpha}.\]
% Now, the pairs $(D\subi{w}{\alpha}, B_w)$ and $(P\subi{w}{\alpha}, Q_w)$ are well defined for all $\alpha\in\R$.
%
Let
% $\DD{w}^k$

\[ \DD{w}^k := \left(\left\{\D{w}{\alpha}^k := \hom_k(D\subi{w}{\alpha},B_w)\right\}_{\alpha\in\R},\left\{d\subi{w}{\alpha,\beta}^k : \D{w}{\alpha}^k\to\D{w}{\beta}^k\right\}_{\alpha\leq\beta}\right)\]
denote the $k$th persistent homology module of the (relative) sub-levelset filtration $\{(D\subi{w}{\alpha},B_w)\}$ modulo $B_w$, and let
%$\PP{w}{\e,k}$

\[\PP{w}{\e,k} := \left(\left\{\P{w}{\e,k}{\alpha} := \hom_k(P\subi{w}{\alpha}^\e,Q_w^\e)\right\}_{\alpha\in\R}, \left\{p\subi{w}{\alpha,\beta}^{\e,k} : \P{w}{\e}{\alpha}\to\P{w}{\e}{\beta}\right\}_{\alpha\leq\beta}\right)\]
denote the $k$th persistent homology module of $\{(P\subi{w}{\alpha}^\e,Q_w^\e)\}$.
Similarly, let $\CPP{w}{\e,k}$ and $\RPP{w}{\e,k}$ denote the corresponding \Cech and Vietoris-Rips filtrations, respectively.
We will omit the dimension $k$ and write $\PP{w}{\e}$ if a statement holds for all dimensions.

\paragraph{Extensions}

If $Q_w^\e$ surrounds $P^\e$ in $D$ we can define a filtration of extensions $\{(\ext{P\subi{w}{\alpha}^\e},\ext{Q_w^\e})\}$ and let $\ext{\PP{w}{\e}}$ denote its $k$th persistent homology module.

\begin{lemma}\label{lem:extension_apply}
  If $Q_w^\e$ surrounds $P^\e$ in $D$ then for $w\in\R$ and $\ext{P\subi{w}{a}^\e} = P\subi{w}{a}^\e \cup (D\setminus P^\e)$ then there is an isomorphism $\E\subi{w}{\cdot}^\e \in \Hom(\PP{w}{\e},\ext{\PP{w}{\e}})$.
\end{lemma}\begin{proof}
  (See Appendix~\ref{apx:omit})
\end{proof}
\proofatend
  Because $P\subi{w}{a} := P\cap D\subi{w}{a}$ and $B_w\subseteq D\subi{w}{a}$ we know $Q_w = P\cap B_w \subseteq P\subi{w}{a}$ for all $a\in\R$.
  So
  \[\ext{Q^\e_a} = Q^\e_a\cup (D\setminus P^\e) \subseteq P\subi{w}{a}^\e \cup (D\setminus P^\e) = \ext{P\subi{w}{a}^\e}.\]
  As $(P^\e, Q_w^\e)$ is a surrounding pair in $D$, $P^\e$ is open in $D$ and $\ext{P\subi{w}{a}^\e}\subseteq D$ is such that $\ext{Q^\e_a}\subseteq \ext{P\subi{w}{a}^\e}$ it follows that
  \[\hom_k(P\subi{w}{a}^\e, Q^\e_a) = \hom_k(P^\e\cap \ext{P\subi{w}{a}^\e}, Q^\e_a) \cong\hom_k(\ext{P\subi{w}{a}^\e}, \ext{Q^\e_a})\]
  by Lemma~\ref{lem:excision}.

  Because these isomorphisms commute with inclusions we have an isomorphism $\E\subi{w}{\cdot}^\e \in \Hom(\PP{w}{\e},\ext{\PP{w}{\e}})$ defined to be the family $\{\E\subi{w}{\alpha}^\e : \P{w}{\e}{a}\to \E\P{w}{\e}{a}\}$.
\endproofatend

\begin{lemma}\label{lem:p_interleave}
 If $Q_w^\e$ surrounds $P^\e$ in $D$ and $D\setminus B_{w + \e}\subseteq P^\e$ then we have the following sequence of homomorphisms of degree $c\e$ induced by inclusions
 \[\DD{w-c\e}\xrightarrow{F}\E\PP{w}{\e}\xrightarrow{M}\DD{w+c\e}.\]
\end{lemma}
\begin{proof}
  (See Appendix~\ref{apx:omit})
\end{proof}
\proofatend
  Suppose $x\in (P^\e\cap B\subi{w-c\e}{\alpha-c\e})\setminus B_{w+\e}$.
  Because $B_{w-\e}\subset B_{w+\e}$ we know $x\notin B_{w-\e}$ so $w+c\e < f(x)\leq \alpha-c\e$ and there exists some $p\in P$ such that $\dist(x, p) < \e$.
  Because $f$ is $c$-Lipschitz it follows
  \[ f(p)\leq f(x) + c\dist(x, p) < \alpha - c\e + c\e = \alpha\]
  and
  \[ f(p)\geq f(x) - c\dist(x, p) > w+c\e-c\e = w.\]
  So $x\in P\subi{w}{\alpha}^\e$.

  Now, suppose $x\in P\subi{w}{\alpha}^\e\setminus B_{w+c\e}$.
  So $w+c\e < f(x)$ and there exists some $p\in P\subi{w}{\alpha}$ such that $\dist(x,p) < \e$.
  Because $f$ is $c$-Lipschitz it follows
  \[ f(x) \leq f(p) + c\dist(x,p) < a + c\e.\]
  So $x\in B\subi{w+c\e}{\alpha+c\e}\setminus B_{w+c\e}$.

  Because $D\setminus B_{w+c\e}\subseteq P^\e$ we know that $D\setminus P^\e \subseteq B_{w+c\e}$, so
  \[D\subi{w-c\e}{\alpha-c\e}\setminus B_{w+c\e} \subseteq P\subi{w}{\alpha}^\e\setminus B_{w+c\e}\subseteq D\subi{w+c\e}{\alpha+c\e}\setminus B_{w+c\e}\]
  implies
  \[ D\subi{w-c\e}{\alpha-c\e}\subseteq P\subi{w}{\alpha}^\e\cup (D\setminus P^\e) = \ext{P\subi{w}{\alpha}^\e} \subseteq D\subi{w+c\e}{\alpha+c\e} \]
  as desired.

  Because $f$ is $c$-Lipschitz, $B_{w-c\e}\cap P^\delta\subseteq Q_{w}^\e$ so $B_{w-c\e} \subseteq \E Q_w^\e\subseteq B_{w+c\e}$ by Lemma~\ref{lem:surround_and_cover}.
  It follows that we have homomorphisms $F\in \Hom^{c\e}(\DD{w-c\e}, \E\PP{w}{\e})$ and $M\in\Hom^{c\e}(\E\PP{w}{\e}, \DD{w+c\e})$ induced by inclusions.
\endproofatend

\paragraph{Rips-\Cech Interleaving}

% For any $w\in\R$ and $\e\geq 0$ let the $k$th persistent homology module of the Rips filtration $\{\rips^\e(P\subi{w}{\alpha}, Q_w)\}$ be denoted
% \[\RPP{w}{\e} := \left(\left\{\RP{w}{\e}{\alpha} := \hom_k(\rips^\e(P\subi{w}{\alpha}, Q_w))\right\}_{\alpha\in\R}, \left\{\rips p\subi{w}{\alpha,\beta}^\e : \RP{w}{\e}{\alpha}\to\RP{w}{\e}{\beta}\right\}_{\alpha\leq\beta}\right).\]
Let $\I_w^\e\in\Hom(\CPP{w}{\e}, \RPP{w}{2\e})$ and $\J_w^\e\in\Hom(\RPP{w}{\e},\CPP{w}{\e})$ be induced by inclusions
\[ \cech^\e(P_\alpha, Q_w)\subseteq \rips^{2\e}(P_\alpha,Q_w)\subseteq \cech^{2\e}(P_\alpha, Q_w).\]

\paragraph{Excisive Nerves}

For $\varrho_D > \e$ the isomorphisms $\N_w^\e\in\Hom(\CPP{w}{\e}, \PP{w}{\e})$ provided by Lemma~\ref{lem:rel_pers_nerve} commute with maps induced by inclusion.

\begin{lemma}\label{cor:excisive_nerve}
  For any $w\leq z$, $\e\leq \eta < \varrho_D$ let $\Lambda\in\Hom(\E\PP{w}{\e},\E\PP{z}{\eta})$ and $\cech\Lambda\in\Hom(\CPP{w}{\e},\CPP{z}{\eta})$ be induced by inclusions.
  Then $\E\N_w^\e$ and $\E\N_z^\eta$ are isomorphisms such that $\Lambda = \E\N_{z}^{\eta}\circ \cech\Lambda\circ (\E\N_w^\e)^{-1}$ and $\cech\Lambda = (\E\N_{z}^{\eta})^{-1}\circ \Lambda\circ \E\N_w^\e.$
\end{lemma}
\begin{proof}\textbf{TODO}
  (See Appendix~\ref{apx:omit})
\end{proof}
\proofatend
  \textbf{TODO}
\endproofatend
