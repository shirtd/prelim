% !TeX root = ../main.tex

TDA is concerned with the ``shape of data,'' and has found applications in
neuroscience~\cite{saggar2018towards, sizemore2019importance},
structural biology~\cite{gameiro2015topological,kovacev2016using,dey2018protein},
medical imaging~\cite{carlsson2008local,robins2011theory,bendich2016persistent},
geometry processing~\cite{skraba2010persistence,poulenard2018topological,bruel2020topology},
and machine learning~\cite{clough2019explicit,chen2019topological,carlsson2020topological,gabrielsson2020topology}, to name a few.
A number of approaches can be considered topological in nature, and \textbf{todo}.
Among these are \textbf{todo}.
However, much of the theoretical work over the last decade has been focused on \emph{topological peristence}.

Persistent homology was first introduced by Edelsbrunner, Letscher, and Zomorodian in 2002\cite{edelsbrunner02simplification}.
This work introduced the standard \emph{reduction algorithm} for computing the persistent homology of simplicial complexes in $\R^3$ with $\Z_2$ coefficients.
The output is a \emph{persistence diagram} or \emph{barcode} that encodes the ``birth'' and ``death'' of homological features of the complex.
Later work by Zomorodian and Carlsson provided a more general form of the reduction algortithm for filtered spaces with coefficients in a field\cite{zomorodian05computing}.
This work also introduced the notion of an \emph{persistence modules} and persistence barcodes as \emph{interval decompositions} of these modules.
\textbf{First instance of the structure theorem?}
This set the stage for a series of work that established stability results for general persistence modules.

Initial stability results by Cohen-Steiner, Edelesbrunner, and Harer were proved for the persistent homology of real-valued functions~\cite{cohensteiner07stability}.
This work was later extended to persistence modules in general by Chazal et al. as the \emph{algebraic stability theorem}~\cite{chazal09proximity}.
In this work the notion of an \emph{$\delta$-interleaving} of persistence modules is introduced.
The algebraic stability theorem roughly states that the existence of a $\delta$-interleaving indicates a $\delta$-matching between their persistence barcodes.
Later work on the \emph{isometry theorem} showed that the converse is also true~\cite{todo}.
Importantly, the isometry theorem established the relationship between interleavings of persistence and the \emph{bottleneck distance} between their barcodes, providing a computable metric on the space of persistence diagrams that is stable with respect to the Gromov-Hausdorff distance between spaces.

Additional work has been done on persistence modules in general including multidimensional persistence, categorical interpretations, \textbf{todo}.
For persistent homology, this robust foundation allowed researchers to establish important inference and approximation results in the more general language of persistence modules.
Among these is relevant work by Chazal et al. that proved stability results for the persistent homology of scalar-valued functions approximated by Vietoris-Rips complexes~\cite{chazal09analysis}.
This work showed that, given a sample $P$ of a scalar valued function that \emph{covers} the domain of the function at scale $\delta$, the persistent homology of a nested sequence of Rips complexes filtered by function values on its vertices is $\delta$-interleaved with that of the scalar field itself.
Later work applied this theory to approximate the persistent homology of probability density function~\cite{chazal2013persistence}.

This body of work comprises much of the relevant theoretical foundations for persistent homology for data analysis in general.
The input is a simplicial complex that represents a discretization of some unknown domain.
The simplicial complex is then \emph{filtered} by a scalar valued function on its vertices, representing some underlying process on the domain.
The standard reduction algorithm~\cite{edelsbrunner02simplification,zomorodian05computing} can then be used to compute the persistent homology of this filtered simplicial complex.
Stability of results state that, provided an interleaving of the persistence module computed in theory, the corresponding persistence barcode is an approximation of that of the function itself~\cite{cohensteiner07stability,chazal09proximity}.

A number of these steps have since been improved, and a number of useful tools from algebraic topology have been leveraged to offer different perspectives on persistent homology in general.
Importantly, the original reduction algorithm has been improved in a number of ways, including \textbf{todo distibuted, parallel, row/column operations, chunking, twisting, bopping}.
A key observation in computing persistent homology efficiently is to compute \emph{persistent cohomology}.
As vector spaces, cohomology is dual to homology.
In the late 20th century the theory of algebraic topology established the relationship between homology and cohomology for general space~\cite{spanier66algebraic,munkres84elements,bredon93,hatcher01}.
However, for reasonable spaces it is known that the persistence barcodes produced by persistent homology and cohomology are the same.
This allows one to compute persistent cohomology instead of persistent homology, and this is often more efficient.
In fact, the reduction algorithm is shown to be the same for persistent homology an cohomology, simply acting in opposite directions~\cite{desilva11duality}.
However, while the barcodes produced are the same, the information captured by homology and cohomology is not, and interesting applications specifically require persistent cohomology~\cite{desilva11circular}.
This work makes use of \emph{representative cocycles} that are produced by the reduction algorithm.
These representatives, and thei dual representative cycles in homology, are often considered a by-product of the homology computation.
In fact, many recent modifications to the reduction algorithm explicitly abandon these representatives for efficiency~\cite{todo}.
However, duality in persistent homology lives in these representatives, and how \textbf{todo}.

Perhaps the first example of duality in persistent homology is the Topological Coverage Criterion~\cite{desilva07coverage}.
While subtle, this work utilized Alexander Duality in their proof to relate the top-dimensional relative homology of a space modulo its boundary to the zero dimensional homology of the interior.
This allows their proof to be greatly simplified by reasoning about zero dimensional homology, connectedness, instead of connectedness in arbitrary dimension.
% Later work attempted to make this approach more explicit, and integrated\textbf{todo}~\cite{cavanna2017when}.
While this is an interesting example of how duality can be applied as a method of proof, the use of Alexander Duality in this case is lacking.
In fact, much of the work on duality in persistent homology lacks rigor
A fascinating subset of algebraic topology in general works towards general duality theorems that establish the relationship between

\begin{itemize}
  \item Other work on duality in persistent homology
  \item Other work in homological sensor networks
  \item Duality, complements of spaces, and relative persistence
  \item Relationship to scalar fields and level sets
  \item Extended persistence (and local homology?)
\end{itemize}
