% !TeX root = ../main.tex

\paragraph{Algorithms and Cohomology}

From a computational perspective, a great deal of progress has been made to improve the original reduction algorithm in a number of ways.
While the running time of the reduction is cubic in the number of simplices it has been shown that this can be improved to matrix multiplication time using Zigzag persistence~\cite{milosavljevic2011zigzag}.
However, a number of simple modifications improve on this worst-case bound~\cite{chen2011persistent,bauer2014clear} and additional work has been done on computing persistence in parallel~\cite{lewis15parallel} and in a distributed setting~\cite{tahbaz10distributed,bauer2014distributed}.
\emph{The overlap in assumptions made by the TCC and SFA can be used to modify the standard reduction algorithm to efficiently verify coverage during the persistence computation.}

One key observation is that it is often more efficient to compute persistent \emph{cohomology}.
As vector spaces, cohomology is dual to homology, and for reasonable spaces it is known that the persistence barcodes produced by persistent homology and cohomology are the same~\cite{todo}.
% This allows one to compute persistent cohomology instead of persistent homology, and this is often more efficient.
In fact, the reduction algorithm is shown to be the same for persistent homology an cohomology, simply acting in opposite directions~\cite{desilva11duality}.
However, while the barcodes produced are the same, the information captured by homology and cohomology is not.
While homology operates on \emph{cycles} representing \emph{holes} cohomology operates on \emph{cocycles} representing \emph{obstructions}.
Each feature of a barcode is associated with a representative (co)cycle that is often considered a by-product of the persistence computation.
In fact, many recent modifications to the reduction algorithm explicitly abandon these representatives for efficiency~\cite{todo}.
However, these representative encode useful information, and some interesting applications specifically require cocycle representatives~\cite{desilva11circular}.
\emph{Representative (co)cycles can be used to merge the relative persistence diagrams provided by the proposed approach in order to recover the full diagram.}
