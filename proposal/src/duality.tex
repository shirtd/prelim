% !TeX root = ../main.tex

\paragraph{Duality and Coverage}

The duality between homology and (co)homology is a powerful tool that has seen limited use in modern TDA.
Much of the theory of algebraic topology developed in the late 20th century focused on this duality for general spaces~\cite{spanier66algebraic,munkres84elements,bredon93,hatcher01}.
Poincar\'e duality states that the cap product with fundamental class provides a natural isomorphism between the $k$th cohomology and the $d-k$th homology of a closed, orientable $d$-manifold.
There are a number of variants to Poincar\'e duality such as Lefschetz duality for manifolds with boundary, and Alexander duality for compact subspaces of the sphere.

Perhaps the first application of duality in TDA was the use of Alexander duality in the TCC.
Here, Alexander duality relates the top-dimensional relative homology of a space modulo its boundary to the zero dimensional homology of its complement.
This greatly simplified their analysis by allowing them to reason about zero dimensional homology instead of connectedness in arbitrary dimension.
Similarly, in SFA, duality can be used to relate the persistent homology and cohomology of sub and super levelset filtrations of a real-valued function.
This fact is used  in work by Cohen-Steiner, Edelsbrunner, and Harer on Extended Persistence~\cite{cohen09extending} in which Poincar\'e and Lefschetz duality are used explicitly to construct an extended persistence diagram.
In both of these cases the existence of an isomorphism provided by duality is sufficient.
Little work has been done to apply duality in persistent homology as a general tool in TDA.
\emph{The cap product with the fundamental class provided by the TCC can be used to compute duals of representative cocycles.
The standard algorithm can then be modified to develop efficient algorithms that can exploit the duality between homology and cohomology within the persistence computation.}
