% !TeX root = ../main.tex

I propose adapting the TCC as a way to provide verified scalar field analysis.
% This is to develop a theoretical foundation for how topological methods can be used to \emph{verify} data with respect to prior information about the topology of the underlying space.
This work will re-cast the TCC as a way to check that a sample is \emph{topologically representative} of the underlying space with assumptions that are more natural to samples of functions.
I will also show how a generalization of these assumptions can be used to prove stability results of the resulting persistence diagram.
This gives a bound on the quality of the approximation that is proportional to the resolution of the sample.
It will also extend existing work on SFA to instances of \emph{partial coverage}.
The goal is to provide a theorem that unifies the assumptions made by the TCC and SFA to assert when that the quality of the diagram approximated by an arbitrary sample can be determined by the diagram itself.

Additional work is motivated by an application involving Lipschitz extensions that requires computing \emph{topological duals} within the persistence computation.
This will extend the existing theory of persistent (co)homology to support duality via the cap product with a fundamental class.
How this can be used with the observation that the TCC provides this fundamental class is currently an open research question.
This work also leads to a new characterization of \emph{pair groups} in persistent (co)homology in terms of long exact sequences.
The theory can then be applied to modify the standard reduction algorithm for persistent (co)homology to pair (co)cycles with dual (co)boundaries.
