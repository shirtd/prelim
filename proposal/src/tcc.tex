% !TeX root = ../main.tex

\paragraph{Homological Sensor Networks and Short Filtrations}

The application of homology to coordinate-free sensor networks is a novel application of homology in that it is minimally persistent.
De Silva and Ghrist first introduced homological sensor networks (HSNs) as a collection of problems that applied homology to the problem of coverage testing without coordinates~~\cite{desilva06coordinate,desilva07homological}.
The underlying idea requires no persistence, and simply observes that, for a bounded $d$-dimensional space, the $d-1$ dimensional homology of a cover contains the information needed to confirm coverage.
Provided a sub-cover of the boundary this information is captured in the $d$ dimensional relative homology of the cover modulo the sampled boundary.
Work on HSNs culminated in the Topological Coverage Criterion~\cite{desilva07coverage} which computes this relative homology using a short filtration of (Vietoris-)Rips complexes.
% The short filtration is factored through the \v Cech complex which is known to be homotopy equivalent to the cover by the Nerve Theorem~\cite{hatcher01}.
There has been a great deal of additional work on homological sensor networks including mobile networks and evasion paths, \textbf{todo}.
However, we will focus on this foundation, and how the use of relative homology and short filtrations relates to SFA and the broader theory of persistent homology.
% the Rips-\v Cech factoring process, which we refer to as an example of a \emph{short filtration}, relates to SFA and the broader theory of persistent homology.

While the use of short filtrations can be considered persistent homology their purpose is generally as a de-noising technique.
Taken instead as the image of homomorphisms between persistence modules, themselves persistence modules, short filtrations can be used with persistence without the need for multidimensional persistence~\cite{todo}.
A canonical example that motivates the proposed work is the use of short filtrations in SFA.
\emph{By modifying the TCC to bounding subspaces defined by a scalar valued function we can use the TCC to verify the quality of an approximation provided by SFA.}
% In SFA, the Persistent Nerve Theorem~\cite{chazal08towards} is used with a nested sequences of Rips complexes to prove stability results in a similar way.

As a note, the images, kernels, and cokernels of maps in homology are particularly useful due to their use in \emph{exact sequences}.
Cohen-Steiner, Edelsbrunner, Harer, and Morozov extend persistent homology to images, as well as kernels and cokernels, of persistence modules, and provide an algorithm for computing them in general~\cite{cohen09persistent}.
Much of the proposed work is inspired by the use of relative homology in the TCC and how persistence module homomorphisms can be interpreted in terms of the long exact (co)homology sequences of pairs and triples.
\emph{Prior work has identified a new formulation of pair groups in terms of the long exact sequences of pairs and triples.}
