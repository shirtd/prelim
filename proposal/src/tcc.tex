% !TeX root = ../main.tex

% While there is extensive literature on homological coverage testing for sensor networks the statement of the criterion as it stands is not conducive to data coverage.
In essence, the criterion counts the number of generators in the top-dimensional relative homology of a domain modulo its boundary.
This number is known to be equal to the number of connected components of the domain.
Coverage can therefore be confirmed by comparing generators in the top-dimensional relative homology of a sample, modulo a \emph{sampled boundary} to the number of connected components of the domain.
This sampled boundary consists of points that are geometrically close to the boundary of the domain.
The TCC must therefore require that sample points are labeled to indicate which points are close to the boundary.
% Specifically, the TCC requires that sample points are labeled to indicate which points are close to the boundary.
This requirement is perhaps the main reason why the TCC can so rarely be applied in practice.
% In fact, the criterion only requires that this ``sampled boundary'' resembles

It can be shown that this assumption can be removed by requiring that our input points sample some unknown $c$-Lipschitz function.
This requirement is natural for applications to data analysis, where data represents local measurements of some underlying process.
Given a threshold on function values that can be used to define a bounding subset of the domain we can define our sampled boundary by their function values alone.
Moreover, the TCC requires additional \emph{regularity assumptions} that are required in order to establish the minimum resolution of the sample.
Traditionally, these assumptions required information about the geometry of the domain, and were made in terms of the \emph{smoothness} of its boundary.
Now, these assumptions can be stated directly in terms of the persistent homology of the function itself.

From this perspective, computing the TCC and approximating the persistent homology of a scalar field overlap in a number of ways.
In fact, the coverage condition can be extracted directly from the persistence diagram, and the quality of the approximation depends coverage.
It is in this way that one can verify that a persistence diagram is representative of the scalar field by leveraging topological priors similar to those required by the TCC.
Not only is the criterion encoded in this diagram, but its computation produces a \emph{fundamental class} for the domain that can be used to compute topological duals.


\textbf{Actual problem statement. This is just background.}
