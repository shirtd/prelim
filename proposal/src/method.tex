% !TeX root = ../main.tex

\begin{itemize}
  \item Duality has been used to relate sub and super level set persistence via relative homology.
  \item This connection can be applied to the scalar field analysis TDA pipeline above.
  \item This method requires coverage, which can be confirmed using TCC.
  \begin{itemize}
    \item The assumptions of the TCC can be re-worked in terms of sub and super levelsets
    \begin{itemize}
      \item a ``surrounding'' sub or super levelset can be made the boundary of a submanifold by excision.
      \item coverage of the interior \emph{super} levelset can then be confirmed.
    \end{itemize}
    \item A way to do verified persistent homology of a truncated diagram
  \end{itemize}
  \item the TCC also provides a fundamental class that can be used to compute duals
  \begin{itemize}
    \item the duality of sub and super levelset homology can be computed explicitly
    \item potential for algorithms that pass between homology and cohomology of sub and super levelsets.
  \end{itemize}
  \item Motivating example of a case in which we would like to compute the peristent homology of the homological image of one filtration into another but we can only compute the sub levelset of one and the super levelset of the other
  \begin{itemize}
    \item can utilize duality to compute the image explicitly
    \item potential for a modification of the persistence algorithm that pairs cycles with cocycles.
  \end{itemize}
\end{itemize}
