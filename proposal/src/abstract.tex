% !TeX root = ../main.tex

\begin{abstract}
  % In modern data analysis data is often interpreted as a sample of some unknown process, or space.
  % A fundamental requirement for any meaningful analysis is that the sample preserves the properties of shape that characterize the underlying space.
  % Methods from Topological Data Analysis (TDA) can be used to study these properties, formalized as \emph{topological invariants}, in data.
  % While these methods can offer novel insight on the underlying structure of data it is important to ensure that this structure is representative of the underlying space, and not a product of insufficient sampling.
  % % Otherwise, we cannot reliably attribute the observed structure of \emph{local} data to the underlying \emph{global} process that generates it.

  Methods from Topological Data Analysis (TDA) can be used to offer novel insight on the structure of data.
  Like all meaningful analysis, these techniques require \emph{data coverage} in order to ensure this structure is representative of the underlying space, and not a product of insufficient sampling.
  The topological coverage criterion (TCC) is a topological approach that can be used to test whether an underlying space is sufficiently well sampled by a given data set.
  Given a sufficiently dense sample, topological scalar field analysis (SFA) can give a summary of the shape of a real-valued function on its domain.
  Both of these techniques function in a coordinate-free setting, and the \emph{persistent homology} computation required in each overlap in a number of ways.
  The proposed work will re-cast the TCC as a way to check that a sample of a real-valued function is sufficient for SFA.
  This unifies the two techniques as a way to conduct verified scalar field analysis, simultaneously confirming coverage while computing a summary.
  It also extends SFA to instances of \emph{partial coverage}.
  Additional work explores a novel interpretation of the TCC that can be used to apply duality within the persistence computation.
\end{abstract}
