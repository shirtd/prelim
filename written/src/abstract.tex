% !TeX root = ../main.tex

\begin{abstract}
  The topological coverage criterion (TCC) can be used to test whether an underlying space is sufficiently well covered by a given data set.
  Given a sufficiently dense sample, topological scalar field analysis (SFA) can give a summary of the shape of a real-valued function on its domain.
  The goal of this paper is to adapt the TCC so that one can test for coverage while computing a summary with SFA.
  % The goal of this paper is to put these theories together so that one can test coverage with the TCC while computing a summary with SFA.
  The challenge is that the TCC requires a well-defined boundary that is not generally available in the setting of SFA.
  To overcome this, we show how the scalar field itself can be used to define a boundary that can be used to confirm coverage.
  % This requires a generalization of the TCC proof and resolves one of the major barriers to wider use of the TCC.
  This requires an interpretation of the TCC that resolves one of the major barriers to wider use.
  % It also extends SFA methods to a wider class of spaces.
  % It also extends SFA methods to the setting in which coverage is only confirmed in a subset of the domain. %a space surrounded by a sub-levelset.
  % We show how the intersection of these two theories can be used to approximate the persistent homology relative to a static sublevel set.
  % We then discuss how this persistent relative homology relates to that of the scalar field as a whole.
\end{abstract}
