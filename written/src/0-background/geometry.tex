% !TeX root = ../../main.tex

% \subsection{Topology and Geometry}

\paragraph{Separation}

Let $X$ be a topological space.
A \textbf{separation} of $X$ is a pair $A, B$ of disjoint, nonempty, open subsets of $X$ whose union is $X$.
The space $X$ is said to be \textbf{connected} if there does not exist a separation of $X$.

Note that the sets $A, B$ that form a separation of $X$ are both open and closed in $X$.
For a subspace $Y$ of $X$ we will denote the interior and closure of a set $U$ in $Y$ with $\intr_Y(U)$ and $\cl_Y(U)$.
If $Y$ is a subspace of $X$, a separation of $Y$ is a pair of disjoint, nonempty sets $A, B$ whose union is $Y$, neither of which contains a limit point of the other.
The space $Y$ is connected if there exists no separation of $Y$ (Munkres~\cite{munkres00topology}, Lemma 23.1).

If $A, B$ is a separation of a subspace $Y$ of $X$ then $A, B$ are both open and closed in $Y$, but not necessarily $X$.
The condition that neither $A$ nor $B$ contains a limit point of the other requires that $\cl_X(A)\cap B = \emptyset$ and $A\cap \cl_X(B) =\emptyset$ where $\cl_Y(A) = A$ and $\cl_Y(B) = B$.

% \begin{definition}[Components (Munkres~\cite{munkres00topology})]
%   Given $X$, define an equivalence relation on $X$ by setting $x\sim y$ if there is a connected subspace of $X$ containing both $x$ and $y$.
%   The equivalence class are called the \textbf{components} (or ``connected components'') of $X$.
% \end{definition}

For a disconnected topological space $X$ let $X_1, X_2, \ldots$ denote its path-connected components.
For $A\subseteq X$ let $A_i = A\cap X_i$ denote the component of $A$ in $X_i$.

\begin{definition}[Separating Set]
  Let $X$ be a (possibly disconnected) topological space and $V\subset X$.
  $V$ \textbf{separates $X$ with a pair $(A, B)$} if $(A_i, B_i)$ is a separation of $X_i\setminus V_i$ for all $i$.
\end{definition}

If $V$ separates $X$ with a pair $(A, B)$ then $X = A\sqcup V\sqcup B$.
Note that while $A$ and $B$ are both open and closed in $X\setminus V$, each component $X_i = A_i\sqcup V_i\sqcup B_i$ is connected.
Therefore, if $V$ separates $X$ with a pair $(A, B)$, we require that $\cl_X(A)\cap B = \emptyset$ and $A\cap \cl_X(B) = \emptyset$.
If $V$ is an open set in $X$ then $A$ and $B$ are closed in $X$, therefore $\cl_X(A)\cap B = \emptyset$ and $A\cap \cl_X(B) = \emptyset$.
% Otherwise, if $V$ is closed in $X$, then $A$ and $B$ are open in $X$.

For $A\subseteq X$ let $\overline{A} := X\setminus A$ denote the complement of $A$ in $X$.

\paragraph{Metric Spaces}

Let $(X,\dist)$ be a metric space where $\dist(x, y)$ denotes the distance between points $x,y\in X$.
For $P\subset X$ and $x\in X$ let $\dist_P(x) = \displaystyle\min_{p\in P}\dist(p, x)$ denote the distance from $x$ to the set $P$.
We will use open metric balls $\ball^\e(x) = \{y\in X\mid \dist(x, y) < \e\}$ and offsets $P^\e = \{x\in X\mid \dist_P(x) < \e\}.$

A metric ball is said to be \textbf{strongly convex} if for each pair of points $y,z$ in its closure there exists a unique shortest path in $X$ between $y$ and $z$ whose interior is contained in the metric ball.
For $x\in X$ let $\varrho_X(x)$ be the supremum of the radii $\e$ such that $\ball^\e(x)$ is strongly convex.
The \textbf{strong convexity radius} of $X$ is defined as
\[ \varrho_X := \inf_{x\in X} \varrho_X(x).\]
Note that this value is positive for compact $X$.
Importantly, strongly convex sets are contractible, and intersections of stronly convex sets are also strongly convex~\cite{chazal09analysis}.
Given a subspace $D\subseteq X$ let $\ball_D^\e(x)$ denote the open ball $D\cap \ball_X^\e(x)$ in the subspace topology and $\varrho_D$ denote the strong convexity radius of $D$.

A real-valued function $f$ on $X$ is $c$-Lipschitz if for all $x,y\in X$ we have
\[
  |f(x) - f(y)| \le c \dist(x,y).
\]
