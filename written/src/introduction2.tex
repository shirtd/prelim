% !TeX root = ../main.tex

Topology is the study of shape.
Topological data analysis (TDA) studies the hape of data.
This qualatative and quantitative information can provide insight on the structure of high dimensional data.
A key component of TDA is the analysis of scalar fields by persistent homology.
Homology is a tool from algebraic topology can identifies topological invariants, commonly referred to as ``holes,'' in arbitrary dimension.
Persistent homology tracks the evolution of these invariants over time.
The persistent homology of a scalar valued function provides a signature for the structure of the domain that maps to increasing sublevel sets.
Given a collection of points that sample the function, one can approximate this signature with some guarantees.
Key to this is the requirement that the sample covers the domain of the function geometrically.
The topological coverage criterion uses these same homological and combinatorial tools to do just this---provide a computatible condition for coverage requiring only pariwise connectivity information.
Unlike previous work that requires the sample points can ``detect'' the presence of the true topological boundary we re-cast the TCC in a way that embraces the sample as that of a function.
That is, we will use the function values to identify a boundary as a sublevel set of the function, resolving one of the major barriers to wider use.
Moreover, this allows us to extend topological scalar field analysis to instances with partial coverage.

% TOPOLOGY IS THE STUDY OF SHAPE FUCK
% FUCKING HOLES AND YOU CANT CUT OR GLUE
% CONTINUOUS DEFORMATIONS AND SHIT
% TOPOLOGICAL DATA ANALYSIS USES TOOLS FROM ALGEBRAIC TOPOLOGY TO STUDY THE SHAPE OF DATA
% IN PARTICULAR, MODERN PROBLEMS IN DATA ANALYSIS AND MACHINE LEARNING NEED TOPOLOGY CUZ THEY'RE FUCKING DUMB
% TOPOLOGY MAKES THEM SMART FUCK
% BUT WE NEED TO KNOW THAT THEY SAW ALL THE SHIT
% THAT IS, THAT OUR SAMPLE IS SUFFICIENT
% IN ORDER TO PROVIDE A TOPOLOGICAL SIGNATURE WITH GUARANTEES FUCK

The topological coverage criterion (TCC) can be used to test whether an underlying space is sufficiently well covered by a given data set.
Given a sufficiently dense sample, topological scalar field analysis (SFA) can give a summary of the shape of a real-valued function on its domain.
The calculation and use of this summary is the subject of Topological Data Analysis (TDA) which is concerned with the ``shape'' of data.
The goal of this paper is to adapt the TCC so that one can test for coverage while computing a summary with SFA.
This provides a pipeline for verified scalar field analysis that simultaneously computes a topological signature of a function while testing that the underlying space is sufficiently well-sampled.

% WHY DO WE CARE ABOUT THE TOPOLOGICAL SIGNATURE OF A FUNCTION FUCK I DONT CARE

We will give a computable condition that ensures a set of sample points can be used to analyze the structure of a scalar field.
In other words, a way to verify that a given sample can provide a signature that captures both qualitative and quantitative shape information from a set of points endowed with a metric and a real-valued function.
% The only requirement is that these points are endowed with limited pairwise connectivity information.
In comparison to previous work our approach embraces the sample as that of a scalar-valued function, resolving one of the major barriers to wider use.

Topological scalar field analysis is concerned with the properties of a space through the lens of a real-valued function.
We consider a collection of points along with only
