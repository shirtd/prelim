% !TeX root = ../main.tex

\begin{frame}
  \frametitle{Homology and Simplicial Complexes}

  \begin{itemize}
    \item The idea of Homology,
    \item Simplicial complexes,
    \item Nerves of Covers and the \v Cech Complex
    \item Clique Complexes of graphs and the (Vietoris-)Rips complex.
    \item The Rips-\v Cech Interleaving
  \end{itemize}
\end{frame}

\begin{frame}
  \frametitle{The Idea of Homology}

  % \begin{itemize}
  %   \item $d$-dimensional holes: algebraic tool
  %   \item $d$-dimension simplices: combinatorial/geometric tool
  %   \item betti numbers and representative cycles.
  % \end{itemize}

  \begin{textblock*}{11cm}(1cm,2cm)
    \only<1>{{\small
      \begin{itemize}
        \item $\hom_0$: connected components,
        \item $\hom_1$: loops,
        \item $\hom_2$: voids.
      \end{itemize}}}
    \only<2>{Betti numbers $\beta_k := \mathbf{dim}~\hom_k$}
    \only<3,4>{{\small $\hom_k := k\mathrm{-cycles} / k\mathrm{-boundaries}$.\\\vspace{2ex}
              Elements are equivalence classes of \emph{homologous} cycles.}}
  \end{textblock*}

  \begin{textblock*}{3cm}(1cm,3cm)
    \only<2>{{\small \[\beta_0 = 1\]}}
  \end{textblock*}
  \begin{textblock*}{3cm}(5.75cm,3cm)
    \only<2>{{\small $\beta_0 = 1$\\$\beta_1 = 1$}}
  \end{textblock*}
  \begin{textblock*}{3cm}(9.5cm,3cm)
    \only<2>{{\small $\beta_0 = 1$\\$\beta_1 = 0$\\$\beta_2 = 1$}}
  \end{textblock*}

  \begin{textblock*}{12cm}(1cm,4.5cm)
    \includegraphics<1,2>[trim=100 100 100 100, clip, width=0.3\textwidth]{figures/component}
    \includegraphics<1,2>[trim=100 100 100 100, clip, width=0.3\textwidth]{figures/loop}
    \includegraphics<1,2>[trim=100 100 100 100, clip, width=0.3\textwidth]{figures/void}
  \end{textblock*}
  \begin{textblock*}{8cm}(1.5cm,4.25cm)
    \includegraphics<3>[trim=200 400 200 400, clip, width=0.7\textwidth]{figures/torus1}
    \includegraphics<4>[trim=200 400 200 400, clip, width=0.7\textwidth]{figures/torus2}
  \end{textblock*}
  \begin{textblock*}{4cm}(8.5cm,6cm)
    \only<3,4>{{\small $\beta_0 = 1$\\$\beta_1 = 2$\\$\beta_2 = 1$}}
  \end{textblock*}

\end{frame}

\begin{frame}
  \frametitle{Simplicial Complexes}

  % \begin{itemize}
  %   \item definition
  %   \item simplicial homology?
  % \end{itemize}
  \begin{textblock*}{11cm}(1cm,2cm)
    \only<1,2>{{\small Vertex set $V$.\\ A \emph{$k$-simplex} is a subset $\sigma\subseteq V$ with $|\sigma| = k+1$.\\}}
    \only<2>{{\small A \emph{simplicial complex} $K$ is a collection of simplices that is closed under taking subsets.}}
    \only<3,4>{{\small The \emph{boundary} of a $k$-simplex is a $(k-1)$ cycle.\\}}
    \only<4>{{\small $k$ cycles in $K$ that are not boundaries of $(k+1)$ \emph{chains} belong to non-trivial homology classes in $\hom_k(K)$.}}
  \end{textblock*}

  \begin{textblock*}{11cm}(1cm,4.5cm)
    \includegraphics<1,2>[trim=500 500 500 500, clip, width=0.3\textwidth]{figures/component}
    \includegraphics<1,2>[trim=0 0 -200 0, clip, width=0.3\textwidth]{figures/tri}
    \includegraphics<1,2>[trim=-200 0 0 0, clip, width=0.3\textwidth]{figures/tet}
    \includegraphics<3,4>[trim=500 500 500 500, clip, width=0.3\textwidth]{figures/component}
    \includegraphics<3,4>[trim=0 0 -200 0, clip, width=0.3\textwidth]{figures/tri_loop}
    \includegraphics<3,4>[trim=-200 0 0 0, clip, width=0.3\textwidth]{figures/tet_void}
  \end{textblock*}
\end{frame}

\begin{frame}
  \frametitle{{\small Nerves of Covers and the \v Cech Complex}}

  % \begin{itemize}
  %   \item covers
  %   \item nerves of covers and the nerve theorem
  %   \item covers by metric balls and the \v Cech complex
  % \end{itemize}

  \begin{textblock*}{12cm}(1cm,4.5cm)
    \includegraphics<1>[trim=75 400 75 500, clip, width=0.5\textwidth]{figures/nerves/cover}
    \includegraphics<2>[trim=75 400 75 500, clip, width=0.5\textwidth]{figures/nerves/cover-points}
    \includegraphics<3>[trim=75 400 75 500, clip, width=0.5\textwidth]{figures/nerves/two-edges-points}
    \includegraphics<4>[trim=75 400 75 500, clip, width=0.5\textwidth]{figures/nerves/three-tri}
    \includegraphics<5>[trim=75 400 75 500, clip, width=0.5\textwidth]{figures/nerves/full}
    \includegraphics<6>[trim=75 400 75 500, clip, width=0.5\textwidth]{figures/cech/points}
    \includegraphics<7>[trim=75 400 75 500, clip, width=0.5\textwidth]{figures/cech/cover-points}
    \includegraphics<8>[trim=75 400 75 500, clip, width=0.5\textwidth]{figures/cech/full}
  \end{textblock*}

\end{frame}

\begin{frame}
  \frametitle{The (Vietoris-)Rips complex}

  \begin{itemize}
    \item connectivity only
    \item 1-skeletons and clique complexes
    \item metric neighborhood graph and the Rips complex.
  \end{itemize}
\end{frame}

\begin{frame}
  \frametitle{The Rips-\v Cech Interleaving}

  \begin{itemize}
    \item statement
    \item in homology
    \item how to ``approximate'' homology with Rips.
  \end{itemize}
\end{frame}
