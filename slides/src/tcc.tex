% !TeX root = ../main.tex

\begin{frame}
  \frametitle{Overview: The Geometric Case}

  \begin{textblock*}{11cm}(1cm,2cm)
    Unknown $c$-Lipschitz function $f : D\to \R$.\vspace{1ex}

    \only<2,3>{Finite sample $P\subset D$ of $f$.\vspace{1ex}}

    \only<3>{Coverage radius $\delta > 0$, covered region $P^\delta$.}
  \end{textblock*}

  \begin{textblock*}{12cm}(0.5cm,5cm)
    \includegraphics<1>[trim=50 200 50 200, clip, width=0.45\textwidth]{figures/nbhd/D}%
    \includegraphics<2>[trim=50 200 50 200, clip, width=0.45\textwidth]{figures/nbhd/P}%
    \includegraphics<3>[trim=50 200 50 200, clip, width=0.45\textwidth]{figures/nbhd/CP}
  \end{textblock*}
\end{frame}

\begin{frame}
  \frametitle{Overview: The Geometric Case}

  \begin{textblock*}{11cm}(1cm,2cm)
    Sublevel set $B_\omega$ that surrounds $D$.\vspace{1ex}

    \only<2,3>{Sample $Q$ such that $Q^\delta\subseteq B_\omega$\vspace{1ex}}

    \only<3>{Set $Q = P\cap B_{\omega-c\delta}$}
  \end{textblock*}

  \begin{textblock*}{12cm}(0.5cm,5cm)
    \includegraphics<1>[trim=50 200 50 200, clip, width=0.45\textwidth]{figures/nbhd/B0}%
    \includegraphics<2>[trim=50 200 50 200, clip, width=0.45\textwidth]{figures/nbhd/Q0}%
    \includegraphics<3>[trim=50 200 50 200, clip, width=0.45\textwidth]{figures/nbhd/CQ0}
  \end{textblock*}
\end{frame}
