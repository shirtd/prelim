% !TeX root = ../main.tex

\begin{frame}
  \begin{textblock*}{11cm}(0cm,1cm)
    \begin{small}
    \begin{description}
      \item[Goal:] Show $\ell : \hom_0(\overline{B_\omega}, \overline{D})\to \hom_0(\overline{Q^\delta},\overline{P^\delta})$ is injective.
      \item[Method:] Check
      % \only<1>{Check $\mathbf{dim}~\hom_0(D\setminus B_\omega)\leq \mathbf{dim}~\hom_0(\overline{Q^\delta}, \overline{P^\delta})$.}
      \[{\color<2>{red} \mathbf{rk}~\hom_d((P^\delta, Q_0^\delta)\hookrightarrow (P^\delta, Q_1^\delta))} \geq \mathbf{dim}~\hom_0(D\setminus B_\omega)).\]
      % \item[Method:] Check $\mathbf{dim}~\hom_0(D\setminus B_\omega)\leq \mathbf{dim}~\hom_0(\overline{Q^\delta}, \overline{P^\delta})$
      \item[Problems:]
    \end{description}
    \end{small}
  \end{textblock*}

  \begin{textblock*}{12cm}(1cm,4.75cm)
    \begin{small}
    \begin{enumerate}[a]
      \item $\mathbf{dim}~\hom_0(\overline{Q^\delta}, \overline{P^\delta})\geq \mathbf{dim}~\hom_0(D\setminus B_\omega)\nRightarrow \ell$ injective,
      \item Cannot compute homology of offsets,
      \item $\mathbf{dim}~\hom_0(D\setminus B_\omega)$ unknown
    \end{enumerate}
    \end{small}
  \end{textblock*}
\end{frame}

\begin{frame}
  \frametitle{Overview: Computing the TCC}

  \begin{textblock*}{11cm}(1cm,2cm)
    Unknown $c$-Lipschitz function $f : D\to \R$.\vspace{1ex}

    \only<2>{Sublevel sets $B_\omega\subseteq B_1$ that \emph{surround} $D$.}
  \end{textblock*}

  \begin{textblock*}{12cm}(0.5cm,5cm)
    \includegraphics<1>[trim=100 500 100 700, clip, width=0.4\textwidth]{figures/rips_dense1_2/surf}%
    \includegraphics<2>[trim=100 500 100 700, clip, width=0.4\textwidth]{figures/rips_dense1_2/Bonly}\hspace{6ex}%
    \includegraphics<2>[trim=100 500 100 700, clip, width=0.4\textwidth]{figures/rips_dense2_2/Bonly}%
  \end{textblock*}

  % \begin{textblock*}{12cm}(0.5cm,4.5cm)
  %   \includegraphics<1>[trim=50 200 50 200, clip, width=0.45\textwidth]{figures/nbhd/D}%
  %   \includegraphics<2>[trim=50 200 50 200, clip, width=0.45\textwidth]{figures/nbhd/P}%
  %   \includegraphics<3>[trim=50 200 50 200, clip, width=0.45\textwidth]{figures/nbhd/NP0}%
  %   \includegraphics<3>[trim=50 200 50 200, clip, width=0.45\textwidth]{figures/nbhd/NP1}
  % \end{textblock*}
\end{frame}

\begin{frame}
  \frametitle{Overview: Computing the TCC}

  \begin{textblock*}{11cm}(1cm,2cm)
    Finite sample $P\subset D$ of $f$.\vspace{1ex}

    \only<2>{Sub-samples $Q_0, Q_1\subset P$ between $B_\omega,B_1$}

  \end{textblock*}

  \begin{textblock*}{12cm}(0.5cm,5cm)
    \includegraphics<1>[trim=100 500 100 700, clip, width=0.4\textwidth]{figures/rips_dense1_2/samples}%
    \includegraphics<2>[trim=100 500 100 700, clip, width=0.4\textwidth]{figures/rips_dense1_2/Qsample}\hspace{6ex}%
    \includegraphics<2>[trim=100 500 100 700, clip, width=0.4\textwidth]{figures/rips_dense2_2/Qsample}%
  \end{textblock*}

  % \begin{textblock*}{12cm}(0.5cm,4.5cm)
  %   \includegraphics<1>[trim=50 200 50 200, clip, width=0.45\textwidth]{figures/nbhd/D}%
  %   \includegraphics<2>[trim=50 200 50 200, clip, width=0.45\textwidth]{figures/nbhd/P}%
  %   \includegraphics<3>[trim=50 200 50 200, clip, width=0.45\textwidth]{figures/nbhd/NP0}%
  %   \includegraphics<3>[trim=50 200 50 200, clip, width=0.45\textwidth]{figures/nbhd/NP1}
  % \end{textblock*}
\end{frame}

\begin{frame}
  \frametitle{Overview: Computing the TCC}

  \begin{textblock*}{11cm}(1cm,2cm)
    \begin{small}
      Want to compute $\mathbf{rk}~\hom_d((P^\delta, Q_0^\delta)\hookrightarrow (P^\delta, Q_1^\delta))$\vspace{1ex}

      \hspace{2ex} $\to$ equal to $\mathbf{rk}~\hom_d(\cech^\delta(P, Q_0)\hookrightarrow \cech^\delta(P,Q_1))$.

    \end{small}

    % \only<3-4>{Rips tells us about $(P^\delta, Q_0^\delta)\hookrightarrow (P^\delta, Q_1^\delta)$}
  \end{textblock*}

  \begin{textblock*}{12cm}(0.5cm,5cm)
    \includegraphics<1>[trim=100 500 100 700, clip, width=0.4\textwidth]{figures/rips_dense1_2/PQcover}\hspace{6ex}%
    \includegraphics<1>[trim=100 500 100 700, clip, width=0.4\textwidth]{figures/rips_dense3/PQcover}%
  \end{textblock*}

  % \begin{textblock*}{12cm}(0.5cm,4.5cm)
  %   \includegraphics<1>[trim=50 200 50 200, clip, width=0.45\textwidth]{figures/nbhd/B0}%
  %   \includegraphics<2>[trim=50 200 50 200, clip, width=0.45\textwidth]{figures/nbhd/NQ0}%
  %   \includegraphics<2>[trim=50 200 50 200, clip, width=0.45\textwidth]{figures/nbhd/NQ1}%
  %   \includegraphics<3>[trim=50 200 50 200, clip, width=0.45\textwidth]{figures/nbhd/PQ0}%
  %   \includegraphics<3>[trim=50 200 50 200, clip, width=0.45\textwidth]{figures/nbhd/PQ1}
  % \end{textblock*}
\end{frame}

\begin{frame}
  \frametitle{Overview: Computing the TCC}

  \begin{textblock*}{11cm}(0cm,2cm)
    \begin{description}
      \item[Input] Pair of neighborhood graphs on $P$
    \end{description}
  \end{textblock*}

  \begin{textblock*}{11cm}(1cm,3cm)
  \begin{small}
    \only<2-4>{Subgraphs on $Q_0, Q_1$.}%

    \only<3>{\[\rips^\delta(P, Q_0)\hookrightarrow \cech^\delta(P, Q_0)\hookrightarrow \cech^\delta(P,Q_1)\hookrightarrow \rips^{2\delta}(P,Q_1)\]}%
    \only<4>{\vspace{-2ex}\[\mathbf{rk}~\hom_d((P^\delta, Q_0^\delta)\hookrightarrow (P^\delta, Q_1^\delta))\geq\mathbf{rk}~\hom_d(\rips^\delta(P, Q_0)\hookrightarrow \rips^{2\delta}(P,Q_1))\]}
  \end{small}
  \end{textblock*}

  \begin{textblock*}{12cm}(0.5cm,5cm)
    \includegraphics<1>[trim=100 500 100 700, clip, width=0.4\textwidth]{figures/rips_dense1_2/skeleton}%
    \includegraphics<2>[trim=100 500 100 700, clip, width=0.4\textwidth]{figures/rips_dense1_2/Qskeleton}%
    \includegraphics<3,4>[trim=100 500 100 700, clip, width=0.4\textwidth]{figures/rips_dense1_Q/complex_nosurf}\hspace{6ex}%
    \includegraphics<1>[trim=100 500 100 700, clip, width=0.4\textwidth]{figures/rips_dense2_2/skeleton}%
    \includegraphics<2>[trim=100 500 100 700, clip, width=0.4\textwidth]{figures/rips_dense2_2/Qskeleton}%
    \includegraphics<3,4>[trim=100 500 100 700, clip, width=0.4\textwidth]{figures/rips_dense2_Q/complex_nosurf}
    % \includegraphics<2>[trim=100 500 100 700, clip, width=0.45\textwidth]{figures/rips_dense1_2/PQcover}%
    % \includegraphics<2>[trim=100 500 100 700, clip, width=0.45\textwidth]{figures/rips_dense3/PQcover}%
    % \includegraphics<3>[trim=100 500 100 700, clip, width=0.45\textwidth]{figures/rips_dense1_2/complex_nosurf}%
    % \includegraphics<3>[trim=100 500 100 700, clip, width=0.45\textwidth]{figures/rips_dense2_2/complex_nosurf}
  \end{textblock*}

  % \begin{textblock*}{12cm}(0.5cm,4.5cm)
  %   \includegraphics<1>[trim=50 200 50 200, clip, width=0.45\textwidth]{figures/nbhd/B0}%
  %   \includegraphics<2>[trim=50 200 50 200, clip, width=0.45\textwidth]{figures/nbhd/NQ0}%
  %   \includegraphics<2>[trim=50 200 50 200, clip, width=0.45\textwidth]{figures/nbhd/NQ1}%
  %   \includegraphics<3>[trim=50 200 50 200, clip, width=0.45\textwidth]{figures/nbhd/PQ0}%
  %   \includegraphics<3>[trim=50 200 50 200, clip, width=0.45\textwidth]{figures/nbhd/PQ1}
  % \end{textblock*}
\end{frame}

% \begin{frame}
%   \frametitle{Overview: Computing the TCC}
%
%   \begin{textblock*}{11cm}(1cm,2cm)
%     \begin{small}
%       Want to compute $\mathbf{rk}~\hom_d((P^\delta, Q_0^\delta)\hookrightarrow (P^\delta, Q_1^\delta))$\vspace{1ex}
%
%       \hspace{2ex} $\to$ equal to $\mathbf{rk}~\hom_d(\cech^\delta(P, Q_0)\hookrightarrow \cech^\delta(P,Q_1))$\vspace{1ex}
%
%       \only<2>{\[\rips^\delta(P, Q_0)\hookrightarrow \cech^\delta(P, Q_0)\hookrightarrow \cech^\delta(P,Q_1)\hookrightarrow \rips^{2\delta}(P,Q_1)\]}
%
%     \end{small}
%
%     % \only<3-4>{Rips tells us about $(P^\delta, Q_0^\delta)\hookrightarrow (P^\delta, Q_1^\delta)$}
%   \end{textblock*}
%
%   \begin{textblock*}{12cm}(0.5cm,4.5cm)
%     \includegraphics<1>[trim=100 500 100 700, clip, width=0.45\textwidth]{figures/rips_dense1_2/PQcover}%
%     \includegraphics<1>[trim=100 500 100 700, clip, width=0.45\textwidth]{figures/rips_dense3/PQcover}%
%     \includegraphics<2>[trim=100 500 100 700, clip, width=0.45\textwidth]{figures/rips_dense1_2/complex_nosurf}%
%     \includegraphics<2>[trim=100 500 100 700, clip, width=0.45\textwidth]{figures/rips_dense2_2/complex_nosurf}
%   \end{textblock*}
%
%   % \begin{textblock*}{12cm}(0.5cm,4.5cm)
%   %   \includegraphics<1>[trim=50 200 50 200, clip, width=0.45\textwidth]{figures/nbhd/B0}%
%   %   \includegraphics<2>[trim=50 200 50 200, clip, width=0.45\textwidth]{figures/nbhd/NQ0}%
%   %   \includegraphics<2>[trim=50 200 50 200, clip, width=0.45\textwidth]{figures/nbhd/NQ1}%
%   %   \includegraphics<3>[trim=50 200 50 200, clip, width=0.45\textwidth]{figures/nbhd/PQ0}%
%   %   \includegraphics<3>[trim=50 200 50 200, clip, width=0.45\textwidth]{figures/nbhd/PQ1}
%   % \end{textblock*}
% \end{frame}
