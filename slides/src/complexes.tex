% !TeX root = ../main.tex

\begin{frame}
  \frametitle{Simplicial Complexes}

  % \begin{itemize}
  %   \item definition
  %   \item simplicial homology?
  % \end{itemize}
  \begin{textblock*}{11cm}(1cm,2cm)
    \only<1,2>{{\small Vertex set $V$.\\ A \emph{$k$-simplex} is a subset $\sigma\subseteq V$ with $|\sigma| = k+1$.\\}}
    \only<2>{{\small A \emph{simplicial complex} $K$ is a collection of simplices that is closed under taking subsets.}}
    \only<3,4>{{\small The \emph{boundary} of a $k$-simplex is a $(k-1)$ cycle.\\}}
    \only<4>{{\small $k$ cycles in $K$ that are not boundaries of $(k+1)$ \emph{chains} belong to non-trivial homology classes in $\hom_k(K)$.}}
  \end{textblock*}

  \begin{textblock*}{11cm}(1cm,4.5cm)
    \includegraphics<1,2>[trim=0 0 -400 0, clip, width=0.3\textwidth]{figures/edge}
    \includegraphics<1,2>[trim=0 0 -200 0, clip, width=0.3\textwidth]{figures/tri}
    \includegraphics<1,2>[trim=-200 0 0 0, clip, width=0.3\textwidth]{figures/tet}
    \includegraphics<3,4>[trim=0 0 -400 0, clip, width=0.3\textwidth]{figures/edge_bdy}
    \includegraphics<3,4>[trim=0 0 -200 0, clip, width=0.3\textwidth]{figures/tri_loop}
    \includegraphics<3,4>[trim=-200 0 0 0, clip, width=0.3\textwidth]{figures/tet_void}
  \end{textblock*}
\end{frame}

\begin{frame}
  \frametitle{{\small Nerves of Covers and the \v Cech Complex}}

  \begin{textblock*}{11cm}(1cm,2cm)
    \begin{small}
      \only<1,2,3,4>{A collection $\U = \{U_i\}_{i\in I}$ is a \emph{cover} of $D$ if $\bigcup_{i\in I} U_i = D$.\vspace{1ex}}

      \only<2,3,4>{The \emph{nerve} of $\U$ is a simplicial complex that associates $k$-simplices with $k$-wise intersections.}

      \only<5>{The \emph{Nerve Theorem} states that the nerve of a \emph{good} open cover has the homotopy type of its union.}
    \end{small}
  \end{textblock*}

  \begin{textblock*}{12cm}(1cm,4.5cm)
    \includegraphics<1>[trim=75 400 75 500, clip, width=0.5\textwidth]{figures/nerves/cover}
    \includegraphics<2>[trim=75 400 75 500, clip, width=0.5\textwidth]{figures/nerves/cover-points}
    \includegraphics<3>[trim=75 400 75 500, clip, width=0.5\textwidth]{figures/nerves/two-edges-points}
    \includegraphics<4>[trim=75 400 75 500, clip, width=0.5\textwidth]{figures/nerves/three-tri}
    \includegraphics<5>[trim=75 400 75 500, clip, width=0.5\textwidth]{figures/nerves/full}
  \end{textblock*}

\end{frame}

\begin{frame}
  \frametitle{{\small Nerves of Covers and the \v Cech Complex}}

  \begin{textblock*}{11cm}(1cm,2cm)
    \begin{small}
      For a finite $P\subset D$ and a cover of open metric balls with radius $\delta$.\vspace{1ex}

      \only<2,3>{The nerve of this cover is known as the \emph{\v Cech complex} and is denoted $\cech^\delta(P)$.}
    \end{small}
  \end{textblock*}

  \begin{textblock*}{12cm}(1cm,4.5cm)
    \includegraphics<1>[trim=75 400 75 500, clip, width=0.5\textwidth]{figures/cech/points}
    \includegraphics<2>[trim=75 400 75 500, clip, width=0.5\textwidth]{figures/cech/cover-points}
    \includegraphics<3>[trim=75 400 75 500, clip, width=0.5\textwidth]{figures/cech/full}
  \end{textblock*}

\end{frame}

\begin{frame}
  \frametitle{The (Vietoris-)Rips complex}

  \begin{textblock*}{11cm}(1cm,2cm)
    \begin{small}
      \only<1,2>{Consider a graph $G$ on $P$.\vspace{1ex}}

      \only<2>{The \emph{clique complex} of this graph is a simplicial complex with $k$-simplices for each $k$-clique.\vspace{1ex}}

      \only<3>{The clique complex of a $\delta$-neighborhood graph is known as the (Vietoris)-Rips complex, denoted $\rips^\delta(P)$.}
    \end{small}
  \end{textblock*}

  \begin{textblock*}{12cm}(1cm,4.5cm)
    \includegraphics<1>[trim=75 400 75 500, clip, width=0.5\textwidth]{figures/rips/graph2}
    \includegraphics<2,3>[trim=75 400 75 500, clip, width=0.5\textwidth]{figures/rips/rips2}
    % \includegraphics<3>[trim=75 400 75 500, clip, width=0.5\textwidth]{figures/rips/cech}
  \end{textblock*}

\end{frame}

\begin{frame}
  \frametitle{The Rips-\v Cech Interleaving}

  \begin{textblock*}{11cm}(1cm,2cm)
    \begin{small}
      The Rips complex can be used to approximate the \v Cech.\vspace{2ex}

      \only<2,3,4>{$\cech^{\delta/2}(P)\only<3,4>{\subseteq \rips^\delta(P)}\only<4>{\subset\cech^\delta(P)\subset\rips^{2\delta}(P)\subset\ldots}$}
    \end{small}
  \end{textblock*}

  \begin{textblock*}{12cm}(1cm,4.5cm)
    \includegraphics<1>[trim=75 400 75 500, clip, width=0.5\textwidth]{figures/rips/graph}
    \includegraphics<2>[trim=75 400 75 500, clip, width=0.5\textwidth]{figures/rips/cech}
    % \includegraphics<3>[trim=75 400 75 500, clip, width=0.5\textwidth]{figures/rips/rips}
    \includegraphics<3,4>[trim=75 400 75 500, clip, width=0.5\textwidth]{figures/rips/rips2}
  \end{textblock*}
\end{frame}
