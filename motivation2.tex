
\subsection{What do we have? How can we use it?}

A collection of sensors can be verified as covering a domain if
\begin{enumerate}
    \item[a.] the boundary of the domain is adequately covered,
    \item[b.] the balloon has not been ``punctured.''
\end{enumerate}
Condition (b) relies on condition (a) in order to provide a topological condition that is necessary but not sufficient.
Given (a) we can confirm coverage by checking if the balloon has been punctured simply by checking the dimension of the top-dimensional relative homology of the sample.
Adequate coverage of the boundary can be broken into two parts.
First, we require that the sampled boundary is in some sense simple in order to ensure our condition cannot produce false positives.
This is achieved by using what we refer to as \emph{short-filtrations}: applying one step of persistence in order to de-noise the data.
By testing our network at two scales we can ensure no spurious features are present in the boundary which may contribute to false positives.
We also note that these short-filtrations are employed in the analysis of scalar fields as well.

Secondly, we require that the so-called ``sampled boundary'' surrounds the interior of the domain.
Otherwise, we may cover the domain but see what looks like a punctured ball as the ball when in fact the ball was never formed.
In the TCC this situation is not handled explicitly.
Instead it is stated as a condition for coverage that is necessary but not sufficient.
That is, it can verify \emph{coverage} without false positives but may produce false negatives.
In fact, the TCC tests a more specific problem: whether we have a reliable representation of the boundary \emph{and} a reliable representation of the interior.

Given this observation we considered how best two use \emph{all} the information given by the TCC in a way that re-uses the machinery used to compute it.
We therefore consider the relative persistent homology of a domain modulo a sublevel-set.
That is, we re-cast the TCC for a domain defined as a domain surrounded by sub-level set of a function on that domain in order to ensure that a given sample can adequately approximate the relative homology of the domain modulo a sub-level set.
We then modify the analysis of scalar fields in order to give an approximation of the \emph{relative} persistent homology of a sample.
Finally, we consider classes of functions which satisfy the assumptions made.
Namely, we consider functions with multiple sub-level sets which may serve as a boundary for this procedure and show how they can be integrated to give a more robust signature for the function.
