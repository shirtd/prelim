% !TeX root = ../new.tex



For two filtrations $\{F_\alpha\}_{\alpha\in\R}$ and $\{G_\alpha\}_{\alpha\in\R}$ we will write $F_\alpha\sim_\e G_\alpha$ to indicate that the persistence modules of the filtrations $\{F_\alpha\}$ and $\{G_\alpha\}$ are $\e$-interleaved.

\begin{definition}[Compatible Filtrations (Skraba~\cite{skraba14approximating})]
  Two filtrations $\{A_\alpha\}$ and $\{F_\alpha\}$ are \textbf{compatible} if the following diagram commutes for all $\alpha\leq\beta$
  \begin{equation}\begin{tikzcd}
    A_\alpha\arrow{r}\arrow{d} &
    F_\alpha\arrow{d} \\
    %
    A_\beta\arrow{r} &
    F_\beta
  \end{tikzcd}\end{equation}
\end{definition}

\begin{theorem}[Skraba~\cite{skraba14approximating}]
  Let $(\{F_\alpha\}, \{A_\alpha\})$ and $(\{G_\alpha\}, \{B_\alpha\})$ be pairs of compatible filtrations.
  If $F_\alpha\sim_{\e_1} G_\alpha$ and $A_\alpha\sim_{\e_2} B_\alpha$ then the relative modules $\{(F_\alpha, A_\alpha)\}$ and $\{(G_\alpha, B_\alpha)\}$ are $\max(\e_1, \e_2)$-interleaved.
\end{theorem}

Now, suppose $\{A_\alpha\}$, $\{B_\alpha\}$ are compatible with $\{F_\alpha\}$ and $\{C_\alpha\}$, $\{D_\alpha\}$ are compatible with $\{G_\alpha\}$.
If $A_\alpha\sim_{\e_1} C_\alpha$, $B_\alpha\sim_{\e_2} D_\alpha$, and $F_\alpha\sim_{\e_3} G_\alpha$ then
\[ (F_\alpha, A_\alpha)\sim_{\max(\e_1,\e_3)} (G_\alpha, C_\alpha)\]
and
\[ (F_\alpha, B_\alpha)\sim_{\max(\e_2,\e_3)} (G_\alpha, D_\alpha).\]
Let $\e = \max(\e_1, \e_2, \e_3)$ so both pairs of relative persistence modules are $\e$-interleaved.

% Consider the following commutative diagrams for $\alpha\leq\beta$


% \vspace{3ex}\begin{subequations}
% \begin{minipage}{0.5\textwidth}\begin{equation}\label{dgm:fab}\begin{tikzcd}
%   \hom_k(F_{\alpha-, A_\alpha)\arrow{r}{f_\alpha}\arrow{d}{a_\alpha^\beta} &
%   \hom_k(F_\alpha, B_\alpha)\arrow{d}{b_\alpha^\beta}\\
%   %
%   \hom_k(G_\beta, A_\beta)\arrow{r}{f_\beta} &
%   \hom_k(F_\beta, B_\beta)
% \end{tikzcd}\end{equation}\end{minipage}
% \begin{minipage}{0.5\textwidth}\begin{equation}\label{dgm:gcd}\begin{tikzcd}
%   \hom_k(G_\alpha, C_\alpha)\arrow{r}{g_\alpha}\arrow{d}{c_\alpha^\beta} &
%   \hom_k(G_\alpha, D_\alpha)\arrow{d}{d_\alpha^\beta}\\
%   %
%   \hom_k(G_\beta, C_\beta)\arrow{r}{g_\beta} &
%   \hom_k(G_\beta, D_\beta)
% \end{tikzcd}\end{equation}\end{minipage}
% \end{subequations}\vspace{3ex}

\begin{lemma}\label{lem:short_pair_inter}
  Suppose $(F_\alpha, A_\alpha)\sim_{\e} (G_\alpha, C_\alpha)$, $(F_\alpha, B_\alpha)\sim_\e (G_\alpha, D_\alpha)$, and the following diagrams commute for all $\beta\geq \alpha\in \Omega$.

  \vspace{3ex}\begin{subequations}
  \begin{minipage}{0.5\textwidth}\begin{equation}\label{dgm:fab}
  \begin{tikzcd}[column sep=scriptsize]
    \hom_k(F_\alpha, A_\alpha)\arrow{r}{f_\alpha}\arrow{d}{a_\alpha^\beta} &
    \hom_k(F_\alpha, B_\alpha)\arrow{d}{b_\alpha^\beta}\\
    %
    \hom_k(F_\beta, A_\beta)\arrow{r}{f_\beta} &
    \hom_k(F_\beta, B_\beta)
  \end{tikzcd}\end{equation}\end{minipage}
  \begin{minipage}{0.5\textwidth}\begin{equation}\label{dgm:gcd}
  \begin{tikzcd}[column sep=scriptsize]
    \hom_k(G_\alpha, C_\alpha)\arrow{r}{g_\alpha}\arrow{d}{c_\alpha^\beta} &
    \hom_k(G_\alpha, D_\alpha)\arrow{d}{d_\alpha^\beta}\\
    %
    \hom_k(G_\beta, C_\beta)\arrow{r}{g_\beta} &
    \hom_k(G_\beta, D_\beta)
  \end{tikzcd}\end{equation}\end{minipage}
  \end{subequations}\vspace{3ex}

  \vspace{3ex}\begin{subequations}
  \begin{minipage}{0.5\textwidth}\begin{equation}\label{dgm:fabgcd}
  \begin{tikzcd}[column sep=scriptsize]
    \hom_k(F_\alpha, A_\alpha)\arrow{r}{f_\alpha}\arrow{d}{m_{\alpha}} &
    \hom_k(F_\alpha, B_\alpha)\arrow{d}{n_\alpha}\\
    %
    \hom_k(G_{\alpha+\e}, C_{\alpha+\e})\arrow{r}{g_{\alpha+\e}} &
    \hom_k(G_{\alpha+\e}, D_{\alpha+\e})
  \end{tikzcd}\end{equation}\end{minipage}
  \begin{minipage}{0.5\textwidth}\begin{equation}\label{dgm:gcdfab}
  \begin{tikzcd}[column sep=scriptsize]
    \hom_k(G_\alpha, C_\alpha)\arrow{r}{g_\alpha}\arrow{d}{u_\alpha} &
    \hom_k(G_\alpha, D_\alpha)\arrow{d}{v_\alpha}\\
    %
    \hom_k(F_{\alpha+\e}, A_{\alpha+\e})\arrow{r}{f_{\alpha+\e}} &
    \hom_k(F_{\alpha+\e}, B_{\alpha+\e})
  \end{tikzcd}\end{equation}\end{minipage}
  \end{subequations}\vspace{3ex}

  \noindent Then
  \[ \{(F_\alpha, A_\alpha)\to (F_\alpha, B_\alpha)\}\]
  is $\e$-interleaved with
  \[\{(G_\alpha, C_\alpha)\to (G_\alpha, D_\alpha)\}.\]

\end{lemma}
\begin{proof}
%   \begin{tiny}
%   \vspace{3ex}\begin{subequations}
%   \begin{minipage}{0.5\textwidth}
%   \begin{equation}\label{dgm:fagc1}\begin{tikzcd}[column sep=tiny]
%     % Fa & & & Fb
%     \hom_k(F_{\alpha-\e}, A_{\alpha-\e})  \arrow[to=Fb, "a_{\alpha-\e}^{\beta+\e}"]
%                                           \arrow[to=Ga, "m_{\alpha-\e}"]
%     & & & |[alias=Fb]|
%       \hom_k(F_{\beta+e}, A_{\beta+\e}) \\
%     % & Ga & Gb &
%     & |[alias=Ga]|
%     \hom_k(G_\alpha, C_\alpha) \arrow[to=Gb, "c_\alpha^\beta"]
%     & |[alias=Gb]|
%       \hom_k(G_\beta, C_\beta) \arrow[to=Fb, "u_\beta"] &
%   \end{tikzcd}\end{equation}\end{minipage}
%   \begin{minipage}{0.5\textwidth}
%   \begin{equation}\label{dgm:fagc1}\begin{tikzcd}[column sep=tiny]
%     % & Fa & Fb &
%     & |[alias=Fa]|
%     \hom_k(F_\alpha, A_\alpha)  \arrow[to=Fb, "a_\alpha^\beta"]
%     & |[alias=Fb]|
%       \hom_k(F_\beta, A_\beta)  \arrow[to=Gb, "m_\beta"] & \\
%     % & Ga & Gb &
%     \hom_k(G_{\alpha-\e}, C_{\alpha-\e})  \arrow[to=Fa, "u_{\alpha-\e}"]
%                                           \arrow[to=Gb, "c_{\alpha-\e}^{\beta+\e}"]
%     & & & |[alias=Gb]|
%       \hom_k(G_{\beta+\e}, C_{\beta+\e})
%   \end{tikzcd}\end{equation}\end{minipage}
%   \end{subequations}\vspace{3ex}
% \end{tiny}

  % and $(F_\alpha, B_\alpha)\sim_\e (G_\alpha, D_\alpha)$ we know that $a_{\alpha-\e}

  Let $\Phi_\alpha = \im~f_\alpha$ and $\Psi_\alpha = \im~g_\alpha$ for all $\alpha\in\Omega$.
  We will show that the following four diagrams commute

  \begin{subequations}
  \begin{minipage}{0.45\textwidth}
  \begin{equation}\label{dgm:intr1}\begin{tikzcd}[column sep=scriptsize]
    % Fa & & & Fb
    \Phi_{\alpha-\e}  \arrow[to=Fb, "\phi_{\alpha-\e}^{\beta+\e}"]
                      \arrow[to=Ga, "\mu_{\alpha-\e}"]
    & & & |[alias=Fb]|
      \Phi_{\beta+\e} \\
    % & Ga & Gb &
    & |[alias=Ga]|
    \Psi_\alpha \arrow[to=Gb, "\psi_\alpha^\beta"]
    & |[alias=Gb]|
      \Psi_\beta \arrow[to=Fb, "\nu_\beta"] &
  \end{tikzcd}\end{equation}
  \begin{equation}\label{dgm:intr3}\begin{tikzcd}
    % Fa & Fb &
    \Phi_{\alpha-\e}  \arrow[to=Fb, "\phi_{\alpha-\e}^{\beta-\e}"]
                      \arrow[to=Ga, "\mu_{\alpha-\e}"]
    & |[alias=Fb]|
      \Phi_{\beta+\e} \arrow[to=Gb, "\mu_{\beta -\e}"] \\
    % & Ga & Gb
    & |[alias=Ga]|
    \Psi_\alpha \arrow[to=Gb, "\psi_\alpha^\beta"]
    & |[alias=Gb]|
      \Psi_\beta
  \end{tikzcd}\end{equation}
  \end{minipage} \begin{minipage}{0.45\textwidth}
  \begin{equation}\label{dgm:intr2}\begin{tikzcd}
    % Fa & & & Fb
    & |[alias=Fa]|
    \Phi_\alpha  \arrow[to=Fb, "\phi_\alpha^\beta"]
    & |[alias=Fb]|
      \Phi_\beta  \arrow[to=Gb, "\mu_\beta"] & \\
    % & Ga & Gb &
    \Psi_{\alpha-\e}  \arrow[to=Gb, "\psi_{\alpha-\e}^{\beta+\e}"]
                      \arrow[to=Fa, "\nu_{\alpha-\e}"]
    & & & |[alias=Gb]|
      \Psi_{\beta + \e}
  \end{tikzcd}\end{equation}
  \begin{equation}\label{dgm:intr4}\begin{tikzcd}
    % & Fa & Fb
    & |[alias=Fa]|
    \Phi_\alpha  \arrow[to=Fb, "\phi_\alpha^\beta"]
    & |[alias=Fb]|
      \Phi_\beta\\
    % Ga & Gb &
    \Psi_{\alpha-\e}  \arrow[to=Gb, "\psi_{\alpha-\e}^{\beta-\e}"]
                      \arrow[to=Fa, "\nu_{\alpha-\e}"]
    & |[alias=Gb]|
      \Psi_{\beta + \e} \arrow[to=Fb, "\nu_{\beta-\e}"]&
  \end{tikzcd}\end{equation}
  \end{minipage}
  \end{subequations}

  Where
  \[ \psi_\alpha^\beta = d_\alpha^\beta\rest_{\Psi_\alpha},\ \phi_\alpha^\beta = b_\alpha^\beta\rest_{\Phi_\alpha},\]
  \[ \mu_\alpha = n_\alpha\rest_{\Phi_\alpha},\text{ and } \nu_\alpha = v_\alpha\rest_{\Psi_\alpha}. \]

  We first make the following observations for any $\alpha\leq\beta$.
  \begin{itemize}
    \item Because Diagram~\ref{dgm:gcd} commutes $\im~\psi_\alpha^\beta = \im~d_\alpha^\beta\rest_{\im~g_\alpha}$ is a subspace of $\im~g_\beta = \Psi_\beta$, so
      \[ \nu_\beta\circ\psi_\alpha^\beta = v_\beta\rest_{\Psi_\beta}\circ d_\alpha^\beta\rest_{\Psi_\alpha} = v_\beta\circ d_\alpha^\beta\rest_{\Psi_\alpha}. \]
    \item Because Diagram~\ref{dgm:fabgcd} commutes $\im~\mu_{\alpha-\e} = \im~n_{\alpha-\e}\rest_{\Phi_{\alpha-\e}}$ is a subspace of $\im~g_\alpha = \Psi_\alpha$, so
      \[ \phi_\alpha^\beta\circ\mu_{\alpha-\e} = d_\alpha^\beta\rest_{\Psi_\alpha}\circ n_{\alpha-\e}\rest_{\Phi_{\alpha-\e}} = d_\alpha^\beta\circ n_{\alpha-\e}\rest_{\Phi_{\alpha-\e}}.\]
    \item Because Diagram~\ref{dgm:fab} commutes $\im~\phi_\alpha^\beta = \im~b_\alpha^\beta\rest_{\im~f_\alpha}$ is a subspace of $\im~f_\beta = \Phi_\beta$, so
      \[\phi_\alpha^\beta\circ \nu_{\alpha-\e} = b_\alpha^\beta\rest_{\Phi_\alpha}\circ v_{\alpha-\e}\rest_{\Psi_{\alpha-\e}} = b_\alpha^\beta\circ v_{\alpha-\e}\rest_{\Psi_{\alpha-\e}}.\]
    \item Because Diagram~\ref{dgm:gcdfab} commutes $\im~\nu_{\alpha-\e} = \im~v_{\alpha-\e}\rest_{\im~g_{\alpha-\e}}$ is a subspace of $\im~f_\alpha = \Phi_\alpha$, so
      \[ \mu_\beta\circ \phi_\alpha^\beta = n_\beta\rest_{\Phi_\beta}\circ b_\alpha^\beta\rest_{\Phi_\alpha} = n_\beta\circ b_\alpha^\beta\rest_{\Phi_\alpha}. \]
  \end{itemize}

  \begin{enumerate}[label=\Roman*.]
    \item $\im~\phi_\alpha^\beta\circ\mu_{\alpha-\e}$ is a subspace of $\im~g_\beta = \Psi_\beta$
      % $\im~\psi_\alpha^\beta = \im~d_\alpha^\beta\rest_{\im~g_\alpha}$ is a subspace of $\im~g_\beta = \Psi_\beta$ as Diagram~\ref{dgm:gcd} commutes for all $\beta\geq\alpha$, so
      % \[ \nu_\beta\circ\psi_\alpha^\beta = v_\beta\rest_{\Psi_\beta}\circ d_\alpha^\beta\rest_{\Psi_\alpha} = v_\beta\circ d_\alpha^\beta\rest_{\Psi_\alpha}. \]
      % Similarly, because Diagram~\ref{dgm:fabgcd} commutes $\im~\mu_{\alpha-\e} = \im~n_{\alpha-\e}\rest_{\Phi_{\alpha-\e}}$ is a subspace of $\im~g_\alpha = \Psi_\alpha$, therefore
      % \[ \phi_\alpha^\beta\circ\mu_{\alpha-\e} = d_\alpha^\beta\rest_{\Psi_\alpha}\circ n_{\alpha-\e}\rest_{\Phi_{\alpha-\e}} = d_\alpha^\beta\circ n_{\alpha-\e}\rest_{\Phi_{\alpha-\e}}.\]
      % So $\im~\phi_\alpha^\beta\circ\mu_{\alpha-\e}$ is a subspace of $\im~g_\beta = \Psi_\beta$ so
      \[ \nu_\beta\circ\psi_\alpha^\beta\circ\mu_{\alpha-\e} = v_\beta\circ d_\alpha^\beta\circ n_{\alpha-\e}\rest_{\Phi_{\alpha-\e}}.\]
      Because $(F_\alpha, B_\alpha)\sim_\e (G_\alpha, D_\alpha)$, $b_{\alpha-\e}^{\beta+\e} = v_\beta\circ d_\alpha^\beta\circ n_{\alpha-\e}$ for all $\beta\geq \alpha$.
      Therefore,
      \begin{align*}
        \phi_{\alpha-\e}^{\beta+\e} &= b_{\alpha-\e}^{\beta+\e}\rest_{\Phi_{\alpha-\e}}\\
          &= v_\beta\circ d_\alpha^\beta\circ n_{\alpha-\e}\rest_{\Phi_{\alpha-e}}\\
          &= \nu_\beta\circ\psi_\alpha^\beta\circ\mu_{\alpha-\e}
      \end{align*}
      so Diagram~\ref{dgm:intr1} commutes.

    \item $\phi_\alpha^\beta\circ \nu_{\alpha-\e}$ is a subspace of $\im~f_\beta = \Phi_\beta$ so
      % Because Diagrams~\ref{dgm:fab} and~\ref{dgm:gcdfab} commute for all $\alpha\leq\beta$ we have that $\im~\phi_\alpha^\beta = \im~b_\alpha^\beta\rest_{\im~f_\alpha}$ is a subspace of $\im~f_\beta = \Phi_\beta$ and $\im~\nu_{\alpha-\e} = \im~v_{\alpha-\e}\rest_{\im~g_{\alpha-\e}}$ is a subspace of $\im~f_\alpha = \Phi_\alpha$, respectively.
      % Therefore,
      \[ \mu_\beta\circ \phi_\alpha^\beta\circ \nu_{\alpha-\e} = n_\beta\circ b_\alpha^\beta\circ v_{\alpha-\e}\rest_{\Psi_{\alpha-\e}}. \]
      Because $(F_\alpha, B_\alpha)\sim_\e (G_\alpha, D_\alpha)$, $d_{\alpha-\e}^{\beta+\e} = n_\beta\circ b_\alpha^\beta\circ v_{\alpha-\e}$ so
      \begin{align*}
        \psi_{\alpha-\e}^{\alpha+\e} &= d_{\alpha-\e}^{\beta+\e}\rest_{g_{\alpha-\e}}\\
          &=n_\beta\circ b_\alpha^\beta\circ v_{\alpha-\e}\rest_{g_{\alpha-\e}}\\
          &= \mu_\beta\circ \phi_\alpha^\beta\circ \nu_{\alpha-\e}
      \end{align*}
      so Diagram~\ref{dgm:intr2} commutes.
    \item Because
      \[ \mu_{\beta - \e}\circ \phi_{\alpha-\e}^{\beta-\e} = n_{\beta-\e}\circ b_{\alpha-\e}^{\beta-\e}\rest_{\Phi_{\alpha-\e}}, \]
      \[ \psi_\alpha^\beta\circ\mu_{\alpha-\e} = d_\alpha^\beta\circ n_{\alpha-\e}\rest_{\Phi_{\alpha-\e}},\]
      and $(F_\alpha, B_\alpha)\sim_\e (G_\alpha, D_\alpha)$ imples $d_\alpha^\beta\circ n_{\alpha-\e} = n_{\beta-\e}\circ b_{\alpha-\e}$, we have
      \begin{align*}
        \mu_{\beta - \e}\circ \phi_{\alpha-\e}^{\beta-\e} &= n_{\beta-\e}\circ b_{\alpha-\e}^{\beta-\e}\rest_{\Phi_{\alpha-\e}}\\
          &= d_\alpha^\beta\circ n_{\alpha-\e}\rest_{\Phi_{\alpha-\e}}\\
          &= \psi_\alpha^\beta\circ\mu_{\alpha-\e}.
      \end{align*}
      So Diagram~\ref{dgm:intr3} commutes.
    \item Because
      \[\nu_{\beta-\e}\circ \psi_{\alpha-\e}^{\beta-\e} = v_{\beta-\e}\circ d_{\alpha-\e}^{\beta-\e}\rest_{\Psi_{\alpha-\e}},\]
      \[\phi_\alpha^\beta\circ \nu_{\alpha-\e} = b_\alpha^\beta\circ v_{\alpha-\e}\rest_{\Psi_{\alpha-\e}},\]
      and $(F_\alpha, B_\alpha)\sim_\e (G_\alpha, D_\alpha)$ implies $b_\alpha^\beta\circ v_{\alpha-\e} = v_{\beta-\e}\circ d_{\alpha-\e}^{\beta-\e}$ we have
      \begin{align*}
        \phi_\alpha^\beta\circ \nu_{\alpha-\e} &= b_\alpha^\beta\circ v_{\alpha-\e}\rest_{\Psi_{\alpha-\e}}\\
          &= v_{\beta-\e}\circ d_{\alpha-\e}^{\beta-\e}\rest_{\Psi_{\alpha-\e}}\\
          &= \nu_{\beta-\e}\circ \psi_{\alpha-\e}^{\beta-\e}.
      \end{align*}
      So Diagram~\ref{dgm:intr4} commutes.
    \end{enumerate}
\end{proof}

\begin{lemma}
  The $k$th persistent homology modules of
  \[ \{(\P)_\alpha, \Q^\delta)\to ((\P)_\alpha, \QQ^\delta)\}_{\alpha\geq\omega+2c\of} \]
  are $c\of$-interleaved with those of
  \[ \{(\P_\alpha, \Q^\delta)\to (\P_\alpha,\QQ^\delta)\}_{\alpha\geq \omega+2c\of} \]
  for all $k > 0$.
\end{lemma}
\begin{proof}
  By Lemma~\ref{lem:ps_inter} $(\P)_\alpha\sim_{c\of} \P_\alpha$ for all $\alpha\in\R$ and therefore
  \[ ((\P)_\alpha, \Q^\delta)\sim_{c\of} (\P_\alpha, \Q^\delta)\text{ for } \alpha \geq \omega - 2c\of\]
  and
  \[ ((\P)_\alpha, \QQ^\delta)\sim_{c\of} (\P_\alpha, \QQ^\delta)\text{ for }\alpha \geq \omega + 2c\delta.\]
  Because all maps are induced by inclusion the following diagrams commute for all $\beta\geq\alpha\geq\omega+2c\delta$.

  \begin{scriptsize}
  \vspace{3ex}\begin{subequations}
  \begin{minipage}{0.5\textwidth}\begin{equation}\label{dgm:fab}
  \begin{tikzcd}[column sep=scriptsize]
    \hom_k((\P)_\alpha, \Q^\delta)\arrow{r} \arrow{d} &
    \hom_k((\P)_\alpha, \QQ^\delta)\arrow{d} \\
    %
    \hom_k((\P)_\beta, \Q^\delta)\arrow{r} &
    \hom_k(\P)_\beta, \QQ^\delta)
  \end{tikzcd}\end{equation}\end{minipage}
  \begin{minipage}{0.5\textwidth}\begin{equation}\label{dgm:gcd}
  \begin{tikzcd}[column sep=scriptsize]
    \hom_k(\P_\alpha, \Q^\of)\arrow{r} \arrow{d} &
    \hom_k(\P_\alpha, \QQ^\of)\arrow{d} \\
    %
    \hom_k(\P_\beta, \Q^\of)\arrow{r} &
    \hom_k(\P_\beta, \QQ^\of)
  \end{tikzcd}\end{equation}\end{minipage}
  \end{subequations}\vspace{3ex}

  \vspace{3ex}\begin{subequations}
  \begin{minipage}{0.5\textwidth}\begin{equation}\label{dgm:fabgcd}
  \begin{tikzcd}[column sep=scriptsize]
    \hom_k((\P)_\alpha, \Q^\delta)\arrow{r}\arrow{d} &
    \hom_k((\P)_\alpha, \QQ^\delta)\arrow{d}\\
    %
    \hom_k(\P_{\alpha+c\delta}, \Q^\of)\arrow{r}&
    \hom_k(\P_{\alpha+c\delta}, \QQ^\of)
  \end{tikzcd}\end{equation}\end{minipage}
  \begin{minipage}{0.5\textwidth}\begin{equation}\label{dgm:gcdfab}
  \begin{tikzcd}[column sep=scriptsize]
    \hom_k(\P_\alpha, \Q^\of)\arrow{r}\arrow{d} &
    \hom_k(\P_\alpha, \Q^\of)\arrow{d}\\
    %
    \hom_k((\P)_{\alpha+\e}, \Q^\delta)\arrow{r} &
    \hom_k((\P)_{\alpha+\e}, \QQ^\delta)
  \end{tikzcd}\end{equation}\end{minipage}
  \end{subequations}\vspace{3ex}
  \end{scriptsize}

  The result therefore follows from Lemma~\ref{lem:short_pair_inter}.
\end{proof}

\begin{corollary}\label{cor:geo_inter}
  If $\eta^k : \hom_k(\b)\to \hom_k(\BB)$ is surjective and $\im~\eta^k\cong \hom_k(\bb)$ then the $k$th persistent homology modules of
  $\{(B_\alpha, \bb)\}_{\alpha\geq\oo}$ are $c\of$-interleaved with that of
  \[\{(P_\alpha^\of, \Q^\of)\to (P_\alpha^\of, \QQ^\of)\}_{\alpha\geq\oo}\]
  for $k > 0$.
\end{corollary}
