% !TeX root = ../new.tex

% \begin{lemma}[Splitting Lemma (Hatcher p. 147)]\label{lem:splitting}
%   For a short exact sequence \[0\to A\xrightarrow{i} B\xrightarrow{j} C\to 0\] of abelian groups the following statements are equivalent
%   \begin{enumerate}
%     \item There is a homomorphism $p: B\to A$ such that $p\circ i = \mathbf{Id}_A$.
%     \item There is a homomorphism $s: C\to B$ such that $j\circ s = \mathbf{Id}_C$.
%     \item There is an isomorphism $B\cong A\oplus C$ making the commutative diagram below, where the maps in the lower row are the obvious ones $a\mapsto (a, 0)$ and $(a,c)\mapsto c$.
%
%     \[\begin{tikzcd}[column sep=small, row sep=small]
%               &                       & B\ar[dd, "\cong"]\ar[dr,"j"]  &         & \\
%       0\ar[r] & A\ar[ur, "i"]\ar[dr]  &                               & C\ar[r] & 0\\
%               &                       & A\oplus C\ar[ur]              &         &
%     \end{tikzcd}\]
%   \end{enumerate}
% \end{lemma}

\begin{lemma}[The Five-Lemma (Hatcher p. 129)]\label{lem:five}
  In a commutative diagram of abelian groups as below, if the two rows are exact and $\alpha,\beta,\delta$, and $\e$ are isomorphisms then $\gamma$ is an isomorphism.
  \[\begin{tikzcd}
      A\ar[r, "i"]\ar[d, "\alpha"]
    & B\ar[r, "j"]\ar[d, "\beta"]
    & C\ar[r, "k"]\ar[d, "\gamma"]
    & D\ar[r, "\ell"]\ar[d, "\delta"]
    & E\ar[d, "\e"]\\
    %
      A'\ar[r, "i'"]
    & B'\ar[r, "j'"]
    & C'\ar[r, "k'"]
    & D'\ar[r, "\ell'"]
    & E'\\
  \end{tikzcd}\]

  \begin{itemize}
    \item If $\beta$ and $\delta$ are surjective and $\e$ is injective then $\gamma$ is surjective.
    \item If $\beta$ and $\delta$ are injective and $\alpha$ is surjective then $\gamma$ is injective.
  \end{itemize}
\end{lemma}

% \begin{theorem}[Alexander Duality]\label{thm:alexander}
%   If $D$ is a compact, locally contractible, nonempty, proper subspace of $S^d$ then for all $k$ there is an isomorphism
%   \[ \Gamma_D^k : \tilde{\hom}_k(D)\to \tilde{\hom}^{d-k-1}(S^d\setminus D). \]
%
%   If $(D, B)$ is a pair of such subspaces of $S^d$ then for all $k$ there is an isomorphism
%   \[ \Gamma_{(D,B)}^k : \tilde{\hom}_k(D, B)\to \tilde{\hom}^{d-k}(S^d\setminus B, S^d\setminus D). \]
% \end{theorem}

\begin{lemma}[Lemma 3.2 from~\cite{chazal08towards}]\label{lem:sandwich}
    Given a sequence $A\to B\to C\to D\to E\to F$ of homomorphisms between finite-dimensional vector spaces, if $\rk(A\to F) = \rk(C\to D)$ then this quantity also equals the rank of $B\to E$.
    Similarly, if $A\to B\to C\to E\to F$ is a sequence of homomorphisms such that $\rk(A\to F) = \dim~C$ then $\rk(B\to E) = \dim~C$.
\end{lemma}

% \textbf{TODO
% \begin{itemize}
%   \item Excision
% \end{itemize}}

\subsection{Separation}

\begin{definition}[Separation (Munkres~\cite{munkres00topology})]
  Let $X$ be a topological space. A \textbf{separation} of $X$ is a pair $U, V$ of disjoint, nonempty, open subsets of $X$ whose union is $X$.
  The space $X$ is said to be \textbf{connected} if there does not exist a separation of $X$.
\end{definition}

Note that the sets $U, V$ that form a separation of $X$ are both open and closed in $X$.
For a subspace $Y$ of $X$ we will denote the interior and closure of a set $U$ in $Y$ with $\intr_Y(U)$ and $\cl_Y(X)$.
 % where $\intr(U)$ and $\cl(U)$ will refer to the interior and closure of $U$ in $X$, unless otherwise stated.

\begin{lemma}[23.1 (Munkres~\cite{munkres00topology})]
  If $Y$ is a subspace of $X$, a separation of $Y$ is a pair of disjoint, nonempty sets $A, B$ whose union is $Y$, neither of which contains a limit point of the other.
  The space $Y$ is connected if there exists no separation of $Y$.
\end{lemma}

If $A, B$ is a separation of a subspace $Y$ of $X$ then $A, B$ are both open and closed in $Y$, but not necessarily $X$.
The condition that neither $A$ nor $B$ contains a limit point of the other requires that $\cl_X(A)\cap B = \emptyset$ and $A\cap \cl_X(B) =\emptyset$ where $\cl_Y(A) = A$ and $\cl_Y(B) = B$.

% \begin{definition}[Components (Munkres~\cite{munkres00topology})]
%   Given $X$, define an equivalence relation on $X$ by setting $x\sim y$ if there is a connected subspace of $X$ containing both $x$ and $y$.
%   The equivalence class are called the \textbf{components} (or ``connected components'') of $X$.
% \end{definition}

For a disconnected topological space $X$ let $X_1, X_2, \ldots$ denote it's path-connected components.
For $A\subseteq X$ let $A_i = A\cap X_i$ denote the component of $A$ in $X_i$.

\begin{definition}[Separating Set]
  Let $X$ be a (possibly disconnected) topological space and $S\subset X$.
  $S$ \textbf{separates $X$ with a pair $(U, V)$} if $(U_i, V_i)$ is a separation of $X_i\setminus S_i$ for all $i$.
\end{definition}

If $S$ separates $X$ with a pair $(U, V)$ then $X = U\sqcup S\sqcup V$.
Note that while $U$ and $V$ are both open and closed in $X\setminus S$, each component $X_i = U_i\sqcup S_i\sqcup V_i$ is connected.
Therefore, if $S$ separates $X$ with a pair $(U, V)$, we require that $\cl_X(U)\cap V = \emptyset$ and $U\cap \cl_X(V) = \emptyset$.
If $S$ is an open set in $X$ then $U$ and $V$ are closed in $X$, therefore $\cl_X(U)\cap V = \emptyset$ and $U\cap \cl_X(V) = \emptyset$.
Otherwise, if $S$ is closed in $X$, then $U$ and $V$ are open in $X$.

% \begin{lemma}
%   If $S$ separates $X$ with a pair $(U, V)$ then
%   \[ \hom_k()
% \end{lemma}

Throughout we will use $U, S,$ and $V$ to denote subsets of $X$ analogous to the interior, boundary, and complement of $S\sqcup U$ in $X$, respectively.
The following definition, while equivalent to that of a separating set, makes this distinction explicit by defining the set $S$ relative to the set $S\sqcup U$.

\begin{definition}[Surrounding]
  Given $B\subset D \subset X$ the set $B$ \textbf{surrounds $D$ in $X$} if $B$ separates $X$ with the pair $(D\setminus B, X\setminus D)$.
  We will refer to such a pair as a \textbf{surrounding pair in $X$}.
\end{definition}

Now, the set $D\setminus B$ corresponds to the interior of $D$ and $X\setminus D$ corresponds to the complement of $D$ in $X$.
This allows us to clearly state the extension of a surrounding pair in a subspace of $X$ to a surrounding pair in $X$.

\begin{definition}[Extension]
  If $S$ surrounds $L$ in a subspace $D$ of $X$ let $\ext{S} := S\sqcup (D\setminus L)$ denote the (disjoint) union of the separating set $S$ with the complement of $L$ in $D$.
  The \textbf{extension of $(L, S)$ in $D$} is the pair
  \[ (D, \ext{S}) = (L\sqcup (D\setminus L), S\sqcup (D\setminus L)).\]
\end{definition}

\begin{lemma}\label{lem:excision}
  If $(L, S)$ is a surrounding pair in a subspace $D$ of $X$ and $L$ is open in $D$ then
  \[ \hom_k(L\cap A, S) \cong \hom_k(A, \ext{S}) \]
  for all $k$ and any $A\subseteq D$ such that $\ext{S}\subset A$.
\end{lemma}
\begin{proof}
  Because $S$ surrounds $L$ in $D$, $(L\setminus S, D\setminus L)$ is a separation of $D\setminus S$, a subspace of $D$.
  So $\cl_D(L\setminus S)\setminus L = \cl_D(L\setminus S) \cap (D\setminus L) = \emptyset$ which implies $\cl_D(L\setminus S)\subseteq L = \intr_D(L)$ as $L$ is open in $D$.
  Therefore,
  \begin{align*}
    \cl_D(D\setminus L) &= D\setminus \intr_D(L)\\
                        &\subseteq D\setminus \cl_D(L\setminus S)\\
                        &= \intr_D(D\setminus (L\setminus S))\\
                        &= \intr_D(\ext{S}).
  \end{align*}
  so,
  \begin{align*}
    \hom_k(L\cap A, S) &= \hom_k(A\setminus (D\setminus L), \ext{S}\setminus (D\setminus L))\\
      &\cong \hom_k(A, \ext{S})
  \end{align*}
  for all $k$ and any $A\subseteq D$ such that $\ext{S}\subset A$ by Excision.
\end{proof}
