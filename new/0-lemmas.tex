% !TeX root = new.tex

\begin{lemma}[Splitting Lemma (Hatcher p. 147)]\label{lem:splitting}
  For a short exact sequence \[0\to A\xrightarrow{i} B\xrightarrow{j} C\to 0\] of abelian groups the following statements are equivalent
  \begin{enumerate}
    \item There is a homomorphism $p: B\to A$ such that $p\circ i = \mathbf{Id}_A$.
    \item There is a homomorphism $s: C\to B$ such that $j\circ s = \mathbf{Id}_C$.
    \item There is an isomorphism $B\cong A\oplus C$ making the commutative diagram below, where the maps in the lower row are the obvious ones $a\mapsto (a, 0)$ and $(a,c)\mapsto c$.

    \[\begin{tikzcd}[column sep=small, row sep=small]
              &                       & B\ar[dd, "\cong"]\ar[dr,"j"]  &         & \\
      0\ar[r] & A\ar[ur, "i"]\ar[dr]  &                               & C\ar[r] & 0\\
              &                       & A\oplus C\ar[ur]              &         &
    \end{tikzcd}\]
  \end{enumerate}
\end{lemma}

\begin{lemma}[The Five-Lemma (Hatcher p. 129)]\label{lem:five}
  In a commutative diagram of abelian groups as below, if the two rows are exact and $\alpha,\beta,\delta$, and $\e$ are isomorphisms then $\gamma$ is an isomorphism.
  \[\begin{tikzcd}
      A\ar[r, "i"]\ar[d, "\alpha"]
    & B\ar[r, "j"]\ar[d, "\beta"]
    & C\ar[r, "k"]\ar[d, "\gamma"]
    & D\ar[r, "\ell"]\ar[d, "\delta"]
    & E\ar[d, "\e"]\\
    %
      A'\ar[r, "i'"]
    & B'\ar[r, "j'"]
    & C'\ar[r, "k'"]
    & D'\ar[r, "\ell'"]
    & E'\\
  \end{tikzcd}\]
\end{lemma}

\begin{theorem}[Alexander Duality]\label{thm:alexander}
  If $D$ is a compact, locally contractible, nonempty, proper subspace of $S^d$ then for all $k$ there is an isomorphism
  \[ \Gamma_D^k : \tilde{\hom}_k(D)\to \tilde{\hom}^{d-k-1}(S^d\setminus D). \]

  If $(D, B)$ is a pair of such subspaces of $S^d$ then for all $k$ there is an isomorphism
  \[ \Gamma_{(D,B)}^k : \tilde{\hom}_k(D, B)\to \tilde{\hom}^{d-k}(S^d\setminus B, S^d\setminus D). \]
\end{theorem}

\begin{lemma}[Lemma 3.2 from~\cite{chazal08towards}]\label{lem:sandwich}
    Given a sequence $A\to B\to C\to D\to E\to F$ of homomorphisms between finite-dimensional vector spaces, if $\rk(A\to F) = \rk(C\to D)$ then this quantity also equals the rank of $B\to E$.
    Similarly, if $A\to B\to C\to E\to F$ is a sequence of homomorphisms such that $\rk(A\to F) = \dim~C$ then $\rk(B\to E) = \dim~C$.
\end{lemma}

\textbf{TODO
\begin{itemize}
  \item Excision
\end{itemize}}
