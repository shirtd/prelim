% !TeX root = ../new.tex
\clearpage

Suppose $\b$ surrounds $D$ in $\X$ and $\Q^\of$ surrounds $\P$ in $D$ such that
\begin{itemize}
  \item $D\setminus \B\subseteq \P$,
  \item $\b\cap \P \subseteq \Q^\of\subseteq \B$, and
  \item $\B\cap \P\subseteq \QQ^\of\subseteq \BB$.
\end{itemize}

Because $\Q^\delta$ surrounds $\P$ in $D$ and $D\setminus \B\subseteq \P$ we have
\[ B_{\alpha-c\of} = (\P\cup (D\setminus \P))\cap B_{\alpha -c\of}\subseteq \P_\alpha \cup (B_{\alpha - c\of}\cap (D\setminus \P))\]
where $B_{\alpha - c\of}\cap (D\setminus \P) = (D\setminus \P)$ for $\alpha\geq \omega+c\of$.
So we define the extension $\ext{\P_\alpha} := \P_\alpha\cup (D\setminus \P)$ of $\P_\alpha$ for $\alpha\geq\omega+c\of$.
So our assumptions imply that we have the following sequence of inclusions by Lemma~\ref{lem:surround_and_cover}.
\[ \b\subseteq \ext{\Q^\of}\subseteq \B\subseteq\ext{\QQ^\of}\subseteq \BB.\]

Because the relative homology of a pair $(X, A)$ is not well defined for $A\not\subseteq X$ we introduce the following notation.

\begin{align*}
  \FQ_\alpha &:= \begin{cases}
    \ext{\P_\alpha}&\text{ if } \alpha < \omega-c\of\\
    \ext{\Q^\of}&\text{ otherwise.}
  \end{cases}&
  \FQ_\alpha' &:= \begin{cases}
    \FQ_\alpha&\text{ if } \alpha < \omega+c\of\\
    \ext{\QQ^\of}&\text{ otherwise. }
  \end{cases}
\end{align*}
\begin{align*}
  \FB_\alpha &:= \begin{cases}
    D_\alpha&\text{ if } \alpha < \omega-2c\of\\
    \b&\text{ otherwise.}
  \end{cases}&
  \FB_\alpha' &:= \begin{cases}
    D_\alpha&\text{ if } \alpha < \omega\\
    \B&\text{ otherwise.}
  \end{cases}&
% \end{align*}
% \begin{align*}
  \FB'' &:= \begin{cases}
    D_\alpha&\text{ if } \alpha < \omega+2c\of\\
    \BB&\text{ otherwise.}
  \end{cases}
\end{align*}
%
% % \begin{scriptstyle}
% \begin{align*}
%   \FP_\alpha &:= \begin{cases}
%     (\ext{\P_\alpha}, \ext{\P_\alpha)}&\text{ if } \alpha < \omega-c\of\\
%     (\ext{\P_\alpha},\ext{\Q^\of})&\text{ otherwise.}
%   \end{cases}&
%   \FP_\alpha' &:= \begin{cases}
%     \FP_\alpha&\text{ if } \alpha < \omega+c\of\\
%     (\ext{\P_\alpha},\ext{\QQ^\of})&\text{ otherwise. }
%   \end{cases}
% \end{align*}
% \begin{align*}
%   \D_\alpha &:= \begin{cases}
%     (B_\alpha, B_\alpha)&\text{ if } \alpha < \omega-2c\of\\
%     (B_\alpha, \b)&\text{ otherwise.}
%   \end{cases}&
%   \D_\alpha' &:= \begin{cases}
%     (B_\alpha, B_\alpha)&\text{ if } \alpha < \omega+2c\of\\
%     (B_\alpha, \BB)&\text{ otherwise.}
%   \end{cases}
% \end{align*}
% \begin{align*}
%   \D_\alpha'' &:= \begin{cases}
%     (B_\alpha, B_\alpha)&\text{ if } \alpha < \omega\\
%     (B_\alpha, \b)&\text{ otherwise.}
%   \end{cases}
% \end{align*}

% For fixed $k$ let
% \begin{align*}
%   \Lambda_\alpha &:= \hom_k(\FP_\alpha),& \Pi_\alpha

Now, for fixed $k$ and all $\alpha\in\R$ let
\[ \Lambda_\alpha := \hom_k(\P_\alpha, \FQ_\alpha),\ \ \Pi_\alpha := \hom_k(\P_\alpha,\FQ_\alpha'),\]
\[ \Gamma_\alpha := \hom_k(D_\alpha, \FB_\alpha),\ \ \Psi_\alpha := \hom_k(D_\alpha, \FB_\alpha'),\ \ \Sigma_\alpha:= \hom_k(D_\alpha, \FB_\alpha''),\]
and let
\[\lambda_\alpha^\beta : \Lambda_\alpha\to \Lambda_\beta,\ \ \pi_\alpha^\beta : \Pi_\alpha\to \Pi_\beta,\]
\[\psi_\alpha^\beta : \Psi_\alpha\to \Psi_\beta\]
be induced by inclusion for all $\beta\geq\alpha\in\R$.

Because
\[B_{\alpha-c\of} = (\ext{\P})_{\alpha-c\of}\subseteq\ext{\P_{\alpha}}\subseteq (\ext{\P})_{\alpha+c\of} = B_{\alpha+c\of}\]
for all $\alpha$ by Lemma~\ref{lem:ps_inter}, we have the following sequence of homomorphisms induced by inclusion for all \emph{all} $\alpha\in\R$.
\[\Gamma_{\alpha-2c\of}\xrightarrow{u_{\alpha-2c\of}}
  \Lambda_{\alpha-c\of}\xrightarrow{m_{\alpha-c\of}}
  \Psi_\alpha\xrightarrow{v_\alpha}
  \Pi_{\alpha+c\of}\xrightarrow{n_{\alpha+c\of}}
  \Sigma_{\alpha+2c\of}.\]
We also have the following sequence of homomorphisms induced by inclusion for all $\alpha\in\R$.
\[\Gamma_\alpha\xrightarrow{r_\alpha}
  \Psi_{\alpha}\xrightarrow{s_\alpha}
  \Sigma_\alpha. \]

The following diagrams commute for all $\beta\geq\alpha\in\R$ as all maps are induced by inclusions.

% \begin{centering}
% \begin{minipage}{0.5\textwidth}
% \begin{equation}\label{dgm:intr_tight4}
% \begin{tikzcd}
%   \Gamma_{\alpha-c\of}\arrow{r}{r_{\alpha-c\of}} \arrow{d}{u_{\alpha-c\of}} &
%   \Psi_{\alpha-c\of}\arrow{d}{v_{\alpha-c\of}} \\
%   %
%   \Lambda_\alpha\arrow{r}{t_{\alpha}}\arrow{d}{m_{\alpha}} &
%   \Pi_\alpha\arrow{d}{n_{\alpha}}\\
%   %
%   \Psi_{\alpha+c\of}\arrow{r}{s_{\alpha+c\of}} &
%   \Sigma_{\alpha+c\of}.
% \end{tikzcd}\end{equation}
% \end{minipage}
% \begin{minipage}{0.5\textwidth}
% \begin{equation}\label{dgm:intr_tight2a}
% \begin{tikzcd}
%   \Lambda_\alpha\arrow{r}{t_\alpha} \arrow{d}{\lambda_\alpha^\beta} &
%   \Pi_\alpha\arrow{d}{\pi_\alpha^\beta} \\
%   %
%   \Lambda_\beta\arrow{r}{t_\beta} &
%   \Pi_\beta
% \end{tikzcd}\end{equation}
% \end{minipage}
% \end{centering}

\begin{centering}
% start left
\begin{minipage}{0.5\textwidth}
\begin{equation}\label{dgm:intr_tight2a}
\begin{tikzcd}
  \Lambda_{\alpha-c\of}
    \arrow{rr}{t_{\alpha+c\of}\circ \lambda_{\alpha-c\of}^{\alpha+c\of}}
    \arrow{dr}{m_{\alpha-c\of}} &
  & \Pi_{\alpha+c\of}\\
  %
  & \Psi_\beta\arrow{ru}{v_\beta} &
\end{tikzcd}\end{equation}
% end left top
\begin{equation}\label{dgm:intr_tight2a}
\begin{tikzcd}[column sep=large]
  \Lambda_{\alpha-c\of}\arrow{r}{t_{\beta-c\of}\circ \lambda_{\alpha-c\of}^{\beta-c\of}} \arrow{d}{m_{\alpha-c\of}} &
  \Pi_{\beta-c\of}\arrow{d}{n_{\beta-c\of}} \\
  %
  \Psi_\alpha\arrow{r}{s_\beta\circ\psi_\alpha^\beta} &
  \Sigma_\beta
\end{tikzcd}\end{equation}
\end{minipage}
% end left
% start right
\begin{minipage}{0.5\textwidth}
\begin{equation}\label{dgm:intr_tight2a}
\begin{tikzcd}
  & \Lambda_{\alpha}
    \arrow{r}{t_{\beta}\circ \lambda_{\alpha}^{\beta}} &
  \Pi_{\beta}\arrow{dr}{n_\beta} &\\
  %
  \Gamma_{\alpha-c\of}
    \arrow{ur}{u_{\alpha-c\of}}
    \arrow[to=Gb, "s_{\beta+c\of}\circ\psi_{\alpha-c\of}^{\beta+c\of}\circ r_{\alpha-c\of}"] & & &
    |[alias=Gb]|
  \Sigma_{\beta+c\of}
\end{tikzcd}\end{equation}
% end right top
\begin{equation}\label{dgm:intr_tight2a}
\begin{tikzcd}[column sep=large]
  \Gamma_{\alpha-c\of}\arrow{r}{\psi_{\alpha-c\of}^{\beta-c\of}\circ r_{\alpha-c\of}} \arrow{d}{u_{\alpha-c\of}} &
  \Psi_{\beta-c\of}\arrow{d}{v_{\beta-c\of}} \\
  %
  \Lambda_\alpha\arrow{r}{t_\beta\circ \lambda_\alpha^\beta} &
  \Pi_\beta
\end{tikzcd}\end{equation}
% end right
\end{minipage}
\end{centering}

\begin{lemma}
  If $\hom_k(\b\to\B)$ is surjective and $\hom_k(\B)\cong \hom_k(\BB)$ for all $k$ then the $k$th persistent homology modules of $\{\Psi_\alpha\}_{\alpha\in\R}$ and $\{\Lambda_\alpha\to \Pi_\alpha\}_{\alpha\in\R}$ are $c\of$-interleaved for all $k$.
\end{lemma}
\begin{proof}
  Because $\pi_\alpha^\beta\circ t_\alpha = t_\beta\circ \lambda_\alpha$ let $\Phi_\alpha :=\im~t_\alpha$ and $\phi_\alpha^\beta := \pi_\alpha\rest_{\im~t_\alpha} = t_\beta\rest_{\im~\lambda_\alpha}$ for all $\beta\geq\alpha\in\R$.
  By applying Lemma~\ref{lem:five} to the long exact sequences of the pairs $\Gamma_\alpha$ and $\Psi_\alpha$ our assumption that $\hom_k(\b\to\B)$ is surjective for all $k$ implies $\hom_k(\Gamma_\alpha\to \Psi_\alpha)$ is surjective for all $\alpha\in\R$.
  Similarly, the assumption that $\hom_k(\B)\cong \hom_k(\BB)$ implies $\Psi_\alpha\cong \Sigma_\alpha$ by applying Lemma~\ref{lem:five} to the long exact sequences of the pairs $\Psi_\alpha$ and $\Sigma_\alpha$.


\end{proof}
