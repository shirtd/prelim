% !TeX root = ../new.tex

\section{The Topological Coverage Criterion}

In the following let $\dist(x, y) = \|x - y\|$ denote the Euclidean distance between points $x,y\in \R^d$.
For $A\subset\R^d$ and $x\in \R^d$ let
\[\dist_A(x) = \min_{a\in A}\dist(x, a)\]
denote the distance from $x$ to the set $A$.
% In the following, we will use open metric balls
% \[\ball_\e(x) = \{y\in\R^d\mid \dist(x, y) < \e\}\]
In the following, we will use closed metric balls
\[\ball_\e(x) = \{y\in\R^d\mid \dist(x, y)\leq \e\}\]
and offsets
% \[A^\e = \dist_A^{-1}[0, \e) = \{x\in \R^d\mid \dist_A(x) < \e\}.\]
\[A^\e = \dist_A^{-1}[0, \e) = \{x\in \R^d\mid \dist_A(x)\leq \e\}.\]

Let $D$ be a compact subset of $\R^d$ and $f : D\to \R$ be a $c$-Lipschitz function.
Let $B_a = f^{-1}(-\infty, a]$ denote the $a$-sublevel set of $f$.
For a subset $A$ of $D$ let $A_a := A\cap B_a$ denote the $a$-sublevelset of $f$ restricted to $A$.
Let $\overline{X} = \R^d\setminus X$ denote the complement of any subset $X\subset\R^d$ in $\R^d$.


Let $P$ be a finite subset of $D$ and $Q_a := P\cap B_a$ for $a\in\R$.
Let $\gamma\geq 3\delta > 0$ and $\omega\in\R$ be constants such that $P^\delta \subseteq D$.

% \textbf{If $B_{\omega - c\delta}^\delta \subseteq B_{\omega - c(\gamma - \delta)}^{\gamma - \delta}$} then we have the following commutative diagrams of inclusions between the pairs $(P,Q)$ and $(D, B)$ and their complements \emph{in $\R^d$} with increasing scale.

\textbf{If $B_{\omega - c\delta}^\delta \subseteq \bb$} then we have the following commutative diagrams of inclusions between the pairs $(P,Q)$ and $(D, B)$ and their complements \emph{in $\R^d$} with increasing scale.

% \[ \begin{tikzcd}
%   (P^\delta, Q^\delta) \arrow[hookrightarrow]{r}\arrow[hookrightarrow]{d} &
%   (P^\gamma, Q^\gamma) \arrow[hookrightarrow]{d} \\
%   %
%   (\D, \B) \arrow[hookrightarrow]{r} &
%   (D_1, B_1),
% \end{tikzcd}\begin{tikzcd}
%   (\overline{B_1}, \overline{D_1})\arrow[hookrightarrow]{r}{j}\arrow[hookrightarrow]{d} &
%   (\overline{\B}, \overline{\D}) \arrow[hookrightarrow]{d}\\
%   %
%   (\overline{Q^\gamma}, \overline{P^\gamma}) \arrow[hookrightarrow]{r}{i} &
%   (\overline{Q^\delta}, \overline{P^\delta}).
% \end{tikzcd}\]
%
% The following diagram is formed by applying the homology functor.
% \begin{equation}\label{dgm:1}\begin{tikzcd}
%     \hom_0(\overline{B_1}, \overline{D_1})\arrow{r}{j_*}\arrow{d} &
%     \hom_0(\overline{\B}, \overline{\D}) \arrow{d} \\
%     %
%     \hom_0(\overline{Q^\gamma}, \overline{P^\gamma}) \arrow{r}{i_*} &
%     \hom_0(\overline{Q^\delta}, \overline{P^\delta}).
% \end{tikzcd}\end{equation}
% Let $p_* : \im~j_*\to\im~i_*$.

% \[ \begin{tikzcd}
%   (P^\delta, Q_{\omega - c\delta}^\delta) \arrow[hookrightarrow]{r}\arrow[hookrightarrow]{d} &
%   (P^\delta, Q_{\omega + c\gamma}^\delta) \arrow[hookrightarrow]{d} \\
%   %
%   (D, B_{\omega - c(\gamma - \delta)}^{\gamma - \delta}) \arrow[hookrightarrow]{r} &
%   (D, B_{\omega + c(\gamma + \delta)}),
% \end{tikzcd}\begin{tikzcd}
%   (\overline{B_{\omega + c(\gamma + \delta)}}, \overline{D})\arrow[hookrightarrow]{r}{j}\arrow[hookrightarrow]{d} &
%   (\overline{B_{\omega - c(\gamma - \delta)}^{\gamma - \delta}}, \overline{D}) \arrow[hookrightarrow]{d}\\
%   %
%   (\overline{Q_{\omega + c\gamma}^\delta}, \overline{P^\delta}) \arrow[hookrightarrow]{r}{i} &
%   (\overline{Q_{\omega - c\delta}^\delta}, \overline{P^\delta}).
% \end{tikzcd}\]
%
% The following diagram is formed by applying the homology functor.
% \begin{equation}\label{dgm:1}\begin{tikzcd}
%     \hom_0(\overline{B_{\omega + c(\gamma + \delta)}}, \overline{D})\arrow{r}{j_*}\arrow{d} &
%     \hom_0(\overline{B_{\omega - c(\gamma - \delta)}^{\gamma - \delta}}, \overline{D}) \arrow{d} \\
%     %
%     \hom_0(\overline{Q_{\omega + c\gamma}^\delta}, \overline{P^\gamma}) \arrow{r}{i_*} &
%     \hom_0(\overline{Q_{\omega - c\delta}^\delta}, \overline{P^\delta}).
% \end{tikzcd}\end{equation}
% Let $p_* : \im~j_*\to\im~i_*$.

\[ \begin{tikzcd}
  (\P, \Q^\delta) \arrow[hookrightarrow]{r}\arrow[hookrightarrow]{d} &
  (\P, \QQ^\delta) \arrow[hookrightarrow]{d} \\
  %
  (D, \bb) \arrow[hookrightarrow]{r} &
  (D, \BB),
\end{tikzcd}\begin{tikzcd}
  (\cmp{\BB},\cmp{D})\arrow[hookrightarrow]{d}\arrow[hookrightarrow]{r} & %{j} &
  (\cmp{\bb}, \cmp{D}) \arrow[hookrightarrow]{d}\\
  %
  (\cmp{\QQ^\delta}, \cmp{\P}) \arrow[hookrightarrow]{r} & %{i} &
  (\cmp{\Q^\delta}, \cmp{\P}).
\end{tikzcd}\]

The following diagram is formed by applying the homology functor.
\begin{equation}\label{dgm:1}\begin{tikzcd}
  \hom_0(\cmp{\BB},\cmp{D})\arrow{d} \arrow{r}{j} & %{j_*} &
  \hom_0(\cmp{\bb}, \cmp{D}) \arrow{d} \\
  %
  \hom_0(\cmp{\QQ^\delta}, \cmp{\P}) \arrow{r}{i} & %{i_*} &
  \hom_0(\cmp{\Q^\delta}, \cmp{\P}).
\end{tikzcd}\end{equation}
Let $p : \im~j\to\im~i$.

\begin{lemma}\label{lem:psurj}
  Suppose $\b$ surrounds $D$ in $\R^d$ and $B_{\omega - c\delta}^\delta\subseteq \bb$.

  If $j$ is surjective then $p : \im~j\to\im~i$ is surjective.
\end{lemma}
\begin{proof}
  Choose a basis for $\im~i$ such that each basis element is represented by a point in $\P\setminus \QQ^\delta$.
  Let $x\in \P\setminus \QQ^\delta$ be such that $[x]$ is non-trivial in $\im~i_*$.
  Suppose $x\in\bb$ and let $y\in \b$ so that $\dist(x, y) < \off$.

  Because $x\in \P$ there exists some $p\in P$ such that $\dist(x, p) < \delta$.
  However, because $x\in\cmp{\QQ^\delta}$, $\dist(x, q) \geq \delta$ for all $q\in \QQ^\delta$.
  So $p$ is not in $\QQ$, so $f(p) > \ome + \offf$.
  Therefore, because $f$ is $c$-Lipschitz,
  \[ f(x)\geq f(p) - c\dist(x, p) > \ome + \offf - c\delta = \ome + \off.\]
  So $x\in \cmp{B_{\ome+\off}}$ where $\Q^\offf \subseteq B_{\fen}^\offf\subseteq B_{\ome+\off}$, so $x\in \cmp{\Q^\offf}$.

  Because $x\in\cmp{\Q^\offf}$ by hypothesis $\dist(x, q) > \offf$ for all $q\in \Q$.
  For any $z$ in the shortest path between $x$ and $y$ we have $\dist(x, z)\leq \dist(x, y)\leq \off$, so the following inequality holds for all $q\in \Q$
  \begin{align*}
    \dist(x, q) & \geq \dist(x, q) - \dist(x, z)\\
                & > \offf - \off\\
                & = \of.
  \end{align*}
  So $z\in \cmp{\Q^\delta}$ for all $z$ in the shortest path from $x$ to $y$.
  In particular, $x,y\in\cmp{\Q^\delta}$.

  Now, suppose $y\in \P$.
  So there exists some $p\in P$ such that $\dist(p, y) < \delta$.
  Because $f$ is $c$-Lipschitz and $y\in \b$
  \[ f(p)\leq f(y) + c\dist(p, y) < \o + c\off = \fen \]
  which implies $p\in \Q$, a contradiction, as we have shown $y\in\cmp{\Q^\delta}$, so we may assume that $y\in \cmp{\P}$.

  Because $x,y\in\cmp{\Q^\delta}$ we have corresponding chains $x,y\in C_0(\cmp{\Q^\delta})$ as well as $y\in\cmp{\P}$ generating a chain $y\in C_0(\P)$.
  As we have shown that $x\in \bb$ implies that the shortest path from $x$ to $y$ is contained in $\cmp{\Q^\delta}$ there exists a path $\zeta: [0,1]\to \cmp{\Q^\delta}$ with $\zeta(0) = x$ and $\zeta(1) = y$ that generates a chain $C_1(\cmp{\Q^\delta})$.
  So for $\zeta\in C_1(\cmp{\Q^\delta}, \cmp{\P})$ with $\partial \zeta = x + y$ we have that $x = \partial \zeta + y$.
  Thus $[x]$ is a relative boundary and is therefore trivial in $\hom_0(\cmp{\P}, \cmp{\Q^\delta})$, a contradiction, as we have assumed $[x]$ is non-trivial in $\im~i$.
  So we may conclude that $x\notin \bb$.

  So $x\in\cmp{\bb}$ and $x\in D\setminus\bb$.
  So $[x]$ is non-trivial in $\hom_0(\cmp{\bb},\cmp{D})$ and, because $j_*$ is surjective, $\im~j = \hom_0(\cmp{\bb},\cmp{D})$.
  So $p$ is surjective as $p[x] = [x]\in\im~p$ for all non-trivial $[x]\in\im~i$.
\end{proof}

\begin{theorem}[Geometric TCC]
  Let $D$ be a compact subset of $\R^d$ and $f : D\to\R$ be $c$-Lipschitz function.
  Let $\omega\in\R$, $\delta > 0$ be such that $\b$ surrounds $D$ in $\R^d$ and $j : \hom_0(D\setminus \BB)\to \hom_0(D\setminus\bb)$ is surjective.
  Let $P\subset D$ be a finite collection of points and $i : \hom_0(\cmp{\QQ^\delta}, \cmp{\P})\to \hom_0(\cmp{\Q^\delta}, \cmp{\P})$.

  If $\rk~i\geq \rk~j$ then $D\setminus \B\subseteq \P$ and $\Q^\delta$ surrounds $\P$ in $D$.
\end{theorem}
\begin{proof}

  By Lemma~\ref{lem:psurj} $p :\im~j\to \im~i$ is surjective.
  So, with the assumption that $\rk~i\geq \rk~j$, $\rk~i = \rk~j$.
  Because $P$ is a finite point set we know that $\im~i$ is finite-dimensional and, because $\rk~i = \rk~j$, $j$ is finite dimensional as well.
  So $p$ is bijective and therefore injective.

  As $j$ is surjective $\im~j = \hom_0(\cmp{\bb}, \cmp{D})$ and, because $\im~i$ is a subspace of $\hom_0(\cmp{\Q^\delta}, \cmp{\P})$, $p$ injective implies that the map $\hom_0(\cmp{\bb}, \cmp{D})\to \hom_0(\cmp{\Q^\delta}, \cmp{\P})$ must be injective, therefore $D\setminus\bb\subseteq \P$ by Lemma~\ref{lem:coverage}, and $\Q^\delta$ surrounds $\P$ in $D$ by Lemma~\ref{lem:cov_surrounds}.
  Finally, because $f$ is $c$-Lipschitz, $\bb\subseteq \B$, therefore $D\setminus\B\subseteq D\setminus \bb\subseteq \P$.

  % Because $j$ is surjective $\im~j = \hom_0(\cmp{\bb}, \cmp{D})$
  % so $p : \im~j\to \im~i$ is injective.
  % Therefore $\D\setminus\B\subseteq P^\delta$ by Lemma~\ref{lem:coverage} and $Q^\delta$ surrounds $P^\delta$ in $\D$ by Lemma~\ref{lem:separate}.
  % So we can extend $(P^\delta, Q^\delta)$ and $(P^\gamma, Q^\gamma)$ to pairs $(\D, \hat{Q^\delta})$ and $(\D, \hat{Q^\gamma})$ surrounding $\D$ in $\R^d$.
\end{proof}

\clearpage

\section{From Coverage Testing to the Analysis of Scalar Fields}

\begin{lemma}
  $\b\subseteq \widehat{\Q^\of}$.
\end{lemma}

\begin{lemma}
  $\B\subseteq \widehat{\QQ^\of}$.
\end{lemma}

\begin{corollary}
  If $\eta^k : \hom_k(\b)\to \hom_k(\BB)$ is surjective and $\im~\eta^k\cong \hom_k(\B)$ then for any $A\subset D$ such that $\BB\subset A$
  \[ \im~\hom_k((\P\cap A, \Q^\delta)\to (\P\cap A, \QQ^\delta))\cong \hom_k(A, \B). \]
\end{corollary}

\begin{lemma}
  $\{(\P\cap B_\alpha, \Q^\delta)\to (\P\cap B_\alpha, \QQ^\delta)\}_{\alpha\geq\oo}$ is $c\delta$-interleaved with $\{(P_\alpha^\delta, \Q^\delta)\to (P_\alpha^\delta, \QQ^\delta)\}_{\alpha\geq\oo}$.
\end{lemma}

\begin{corollary}
  If $\eta^k : \hom_k(\b)\to \hom_k(\BB)$ is surjective and $\im~\eta^k\cong \hom_k(\B)$ then $\{(B_\alpha, \B)\}_{\alpha\geq\oo}$ is $c\delta$-interleaved with $\{(P_\alpha^\delta, \Q^\delta)\to (P_\alpha^\delta, \QQ^\delta)\}_{\alpha\geq\oo}$.
\end{corollary}
