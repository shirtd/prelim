% !TeX root = new.tex

In the following let $\dist(x, y) = \|x - y\|$ denote the euclidean distance between points $x,y\in \R^d$.
For $A\subset\R^d$ and $x\in \R^d$ let
\[\dist_A(x) = \min_{a\in A}\dist(x, a)\]
denote the distance from $x$ to the set $A$.
In the following, we will use open metric balls
\[\ball_\e(x) = \{y\in\R^d\mid \dist(x, y) < \e\}\]
and offsets
\[A^\e = \dist_A^{-1}[0, \e) = \{x\in \R^d\mid \dist_A(x) < \e\}.\]% = \bigcup_{a\in A}\ball_\e(a) .\]

% Let $(D, B)$ be a surrounding pair in $\R^d$.
% For subsets $Y\subset X$ of $D$ suppose $D\setminus B \subseteq X$ and $Y$ separates $D$ with a pair $(U, V)$ such that $D\setminus B \subseteq U$.
% Let $(\hat{X},\hat{Y}) = (X\cup V, Y\cup V)$ denote the \emph{extension} of the pair $(X, Y)$ in $(D, B)$.
% Note that, in the following definition, the requirement that $Q^\delta$ separates $D$ with the pair $(P^\delta\setminus Q^\delta, V)$

\begin{definition}[Separating Cover]
  For $\delta > 0$, $\gamma > \delta$, and finite subsets $P\subset D$, $Q\subset P\cap B$ we say that $(P, Q)$ is an \textbf{(open) separating $(\delta,\gamma)$-cover} of a surrounding pair $(D, B)$ if
  \begin{enumerate}[label=(\alph*)]
    \item $D\setminus B \subseteq P^\delta$,
    % \item $Q^\delta$ separates $D$ with the pair $(P^\delta\setminus Q^\delta, V)$, and
    \item $Q^\delta$ surrounds $P^\delta$ in $D$, and
    % \item $(\hat{P^\delta}, \hat{Q^\delta})\subseteq (D, B)\subseteq (\hat{P^\gamma}, \hat{Q^\gamma})$.
    \item $\hat{Q^\delta}\subseteq B\subseteq \hat{Q^\gamma}$.
  \end{enumerate}
\end{definition}

% We note that the first two requirements of a separating $(\delta,\gamma)$-cover $(P, Q)$ imply that $D\setminus B\subseteq P^\delta\setminus Q^\delta$ so the third requirement is equivalent to an interleaving of extended pairs $(\hat{P^\delta}, \hat{Q^\delta})$ and $(\hat{P^\gamma}, \hat{Q^\gamma})$.

% {\color{red}
%
% \begin{lemma}
%   If $(P, Q)$ is an (open) separating $(\delta,\gamma)$-cover of a surrounding pair $(D, B)$ then
%   \[ \hom_k(P^\delta, Q^\delta)\cong \hom_k(\hat{P^\delta}, \hat{Q^\delta}). \]
% \end{lemma}
% \begin{proof}
%   Clearly $\hat{P^\delta}\setminus V = P^\delta$ and $\hat{Q^\delta}\setminus V = Q^\delta$.
%   Because $Q^\delta$ is an open set $V$ is closed\footnote{\textbf{TODO
%   $V = D\setminus (Q^\delta\cup U)$ for \emph{open} $D$. clopen? $D$ must be open for next excision.
%   options:\begin{itemize}
%     \item Define separating pair as separating $\R^d$ with $D\setminus B\subset U$ and $\overline{D}\subset V$
%     \item tricky bzns where $D$ is taken as a metric subspace (side effects?)
%   \end{itemize}}}, so $\cl(V) = V\subset \intr(\hat{Q^\delta})$.
%   The isomorphism follows by excision.
% \end{proof}
%
% For any separating $(\delta,\gamma)$-cover $(P, Q)$ of a surrounding pair $(D, B)$ clearly $Q^\gamma$ separates $D$ and $D\setminus B\subseteq P^\gamma$.
% Therefore, let $(\hat{P^\gamma}, \hat{Q^\gamma})$ denote the extension of $(P^\gamma, Q^\gamma)$ in $D$ and note that $\hom_k(P^\gamma, Q^\gamma)\cong \hom_k(\hat{P^\gamma}, \hat{Q^\gamma})$.
%
% \begin{lemma}\label{lem:excision1}
%   If $(D, B)$ is an open surrounding pair in $\R^d$ and $(P, Q)$ is an (open) separating $(\delta,\gamma)$-cover of $(D, B)$ then there is an isomorphism
%   \[ \hom_k(P^\delta, P^\delta\cap B)\to \hom_k(D, B) \]
%   induced by inclusion for all $k$.
% \end{lemma}
% \begin{proof}
%   Because $(D, B)$ is an open pair of subsets and $(P, Q)$ is a separating $(\delta,\gamma)$-cover of $(D, B)$ we know that $B\subset D$, $P^\delta \subseteq D$, and $D\setminus B \subseteq P^\delta$.
%   Moreover, because $B$ and $P^\delta$ are open sets $\intr(P^\delta) = P^\delta$ and $\intr(B) = B$.
%   So $P^\delta \cup B = \intr(P^\delta)\cup \intr(B)\subseteq D$ and
%   \[ D = (D\setminus B)\cup B \subseteq P^\delta \cup B \]
%   thus $\intr(P^\delta)\cup \intr(B) = D$ which implies the inclusion $(P^\delta, P^\delta\cap B)\hookrightarrow (D, B)$ induces an isomorphism in homology by excision.
% \end{proof}
%
% Because $(D, B)$ is a surrounding pair in $\R^d$ we know that $B$ separates $\R^d$ with the pair $(D\setminus B, \overline{D})$.
% So there is no path from $D\setminus B$ to $\overline{D}$ that does not cross $B$.
% As $D\setminus B\subseteq V$, $Q^\delta\subseteq B$, and $U, V$ and $Q^\delta$ partition $D$ it follows that $U\subset B$ and therefore that $(\hat{P^\delta}, \hat{Q^\delta}) \subseteq (D, B) \subseteq (\hat{P^\gamma}, \hat{Q^\gamma})$.\footnote{\textbf{TODO rigor.}}
% Similarly, $\hom_k(\hat{P^\delta}, \hat{P^\delta}\cap B)\cong \hom_k(D, B)$.
%
% }
