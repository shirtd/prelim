% !TeX root = ../new.tex

Let $D$ be a compact subset of $\R^d$.
Let $\dist(x, y) = \|x - y\|$ denote the Euclidean distance between points $x,y\in D$ as a subspace of $\R^d$.
For $A\subset D$ and $x\in D$ let
\[\dist_A(x) = \min_{a\in A}\dist(x, a)\]
denote the distance from $x$ to the set $A$.
% In the following, we will use open metric balls
% \[\ball_\e(x) = \{y\in\R^d\mid \dist(x, y) < \e\}\]
We will use open metric balls restricted to $D$ with the subspace topology
\[\ball_\e(x) = \{y\in D\mid \dist(x, y) < \e\}\]
and offsets
% \[A^\e = \dist_A^{-1}[0, \e) = \{x\in \R^d\mid \dist_A(x) < \e\}.\]
\[A^\e = \{x\in D\mid \dist_A(x) < \e\}.\]

Let $f : D\to \R$ be a $c$-Lipschitz function on $D$.
Let $B_a = f^{-1}(-\infty, a]$ denote the $a$-sublevel set of $f$.
For a subset $A$ of $D$ let $A_a := A\cap B_a$ denote the $a$-sublevelset of $f$ restricted to $A$.
Let $\overline{X} = \R^d\setminus X$ denote the complement of any subset $X\subset\R^d$ in $\R^d$.


Let $P$ be a finite subset of $D$ and $Q_a := P\cap B_a$ for $a\in\R$.
Let $\of > 0 $ and $\omega\in \R$ be constants such that $\P\subseteq D$.
Because $f$ is $c$ Lipschitz we have the following commutative diagrams induced by inclusion.

\[ \begin{tikzcd}
  (\P, \Q^\of) \arrow[hookrightarrow]{r}\arrow[hookrightarrow]{d} &
  (\P, \QQ^\of) \arrow[hookrightarrow]{d} \\
  %
  (D, \bb) \arrow[hookrightarrow]{r} &
  (D, \BB),
\end{tikzcd}\begin{tikzcd}
  (\cmp{\BB},\cmp{D})\arrow[hookrightarrow]{d}\arrow[hookrightarrow]{r} & %{j} &
  (\cmp{\bb}, \cmp{D}) \arrow[hookrightarrow]{d}\\
  %
  (\cmp{\QQ^\of}, \cmp{\P}) \arrow[hookrightarrow]{r} & %{i} &
  (\cmp{\Q^\of}, \cmp{\P}).
\end{tikzcd}\]

The following diagram is formed by applying the homology functor.
\begin{equation}\label{dgm:1}\begin{tikzcd}
  \hom_0(\cmp{\BB},\cmp{D})\arrow{d} \arrow{r}{j} & %{j_*} &
  \hom_0(\cmp{\bb}, \cmp{D}) \arrow{d} \\
  %
  \hom_0(\cmp{\QQ^\of}, \cmp{\P}) \arrow{r}{i} & %{i_*} &
  \hom_0(\cmp{\Q^\of}, \cmp{\P}).
\end{tikzcd}\end{equation}
Let $p : \im~j\to\im~i$.

\begin{lemma}\label{lem:psurj}
  Suppose $\B$ surrounds $D$ in $\R^d$ and let $\delta > 0$.

  If $j$ is surjective then $p : \im~j\to\im~i$ is surjective.
\end{lemma}
\begin{proof}
  Choose a basis for $\im~i$ such that each basis element is represented by a point in $\P\setminus \QQ^\of$.
  Let $x\in \P\setminus \QQ^\of$ be such that $[x]$ is non-trivial in $\im~i$.
  So there exits some $p\in P$ such that $\dist(p, x) < \delta$ and $p\notin \QQ$, otherwise $x\in\QQ^\of$.
  Therefore, because $f$ is $c$-Lipschitz,
  \[ f(x)\geq f(p) - c\dist(x, p) > \fenn - c\of =\omega.\]

  So $x\in\cmp{\B}$ and, because $x\in \P\subseteq D$, $x\in D\setminus \B$.
  That is, $[x]$ is non-trivial in $\hom_0(\cmp{\B},\cmp{D})$ which, by our assumption that $j$ is surjective, is equal to $\im~j$.
  As $i, j$ are induced by inclusion we may therefore conclude that $p$ is surjective as $p[x] = [x]$ for all non-trivial $[x]\in\im~i$.
  %
  % Because $x\in \P$ there exists some $p\in P$ such that $\dist(x, p) < \of$.
  % However, because $x\in\cmp{\QQ^\of}$, $\dist(x, q) \geq \of$ for all $q\in \QQ^\of$.
  % So $p$ is not in $\QQ$, so $f(p) > \ome + \offf$.
  % Therefore, because $f$ is $c$-Lipschitz,
  % \[ f(x)\geq f(p) - c\dist(x, p) > \ome + \offf - c\of = \ome + \off.\]
  % So $x\in \cmp{B_{\ome+\off}}$ where $\Q^\offf \subseteq B_{\fen}^\offf\subseteq B_{\ome+\off}$, so $x\in \cmp{\Q^\offf}$.
  %
  % Because $x\in\cmp{\Q^\offf}$ by hypothesis $\dist(x, q) > \offf$ for all $q\in \Q$.
  % For any $z$ in the shortest path between $x$ and $y$ we have $\dist(x, z)\leq \dist(x, y)\leq \off$, so the following inequality holds for all $q\in \Q$
  % \begin{align*}
  %   \dist(x, q) & \geq \dist(x, q) - \dist(x, z)\\
  %               & > \offf - (\off)\\
  %               & \geq \of.
  % \end{align*}
  % So $z\in \cmp{\Q^\of}$ for all $z$ in the shortest path from $x$ to $y$.
  % In particular, $x,y\in\cmp{\Q^\of}$.
  %
  % Now, suppose $y\in \P$.
  % So there exists some $p\in P$ such that $\dist(p, y) < \of$.
  % Because $f$ is $c$-Lipschitz and $y\in \b$
  % \[ f(p)\leq f(y) + c\dist(p, y) < \o + c\of\leq \fen \]
  % which implies $p\in \Q$, a contradiction, as we have shown $y\in\cmp{\Q^\of}$, so we may assume that $y\in \cmp{\P}$.
  %
  % Because $x,y\in\cmp{\Q^\of}$ we have corresponding chains $x,y\in C_0(\cmp{\Q^\of})$ as well as $y\in\cmp{\P}$ generating a chain $y\in C_0(\P)$.
  % As we have shown that $x\in \bb$ implies that the shortest path from $x$ to $y$ is contained in $\cmp{\Q^\of}$ there exists a path $\zeta: [0,1]\to \cmp{\Q^\of}$ with $\zeta(0) = x$ and $\zeta(1) = y$ that generates a chain $C_1(\cmp{\Q^\of})$.
  % So for $\zeta\in C_1(\cmp{\Q^\of}, \cmp{\P})$ with $\partial \zeta = x + y$ we have that $x = \partial \zeta + y$.
  % Thus $[x]$ is a relative boundary and is therefore trivial in $\hom_0(\cmp{\P}, \cmp{\Q^\of})$, a contradiction, as we have assumed $[x]$ is non-trivial in $\im~i$.
  % So we may conclude that $x\notin \bb$.
  %
  % So $x\in\cmp{\bb}$ and $x\in D\setminus\bb$.
  % So $[x]$ is non-trivial in $\hom_0(\cmp{\bb},\cmp{D})$ and, because $j_*$ is surjective, $\im~j = \hom_0(\cmp{\bb},\cmp{D})$.
  % So $p$ is surjective as $p[x] = [x]\in\im~p$ for all non-trivial $[x]\in\im~i$.
\end{proof}

\begin{theorem}[Geometric TCC]\label{thm:geo_tcc}
  Let $D$ be a compact subset of $\R^d$ and $f : D\to\R$ be $c$-Lipschitz function.
  Let $\omega\in\R$, $\of > 0$ be constants such that
  \begin{enumerate}[label=(\alph*)]
    \item $\B$ surrounds $D$ in $\R^d$, and
    \item $j : \hom_0(\cmp{\BB},\cmp{D})\to \hom_0(\cmp{\B}, \cmp{D})$ is surjective.
  \end{enumerate}
  Let $P\subset D$ be a finite collection of points and $i : \hom_0(\cmp{\QQ^\of}, \cmp{\P})\to \hom_0(\cmp{\Q^\of}, \cmp{\P})$.

  If $\rk~i\geq \rk~j$ then $D\setminus \B\subseteq \P$ and $\Q^\of$ surrounds $\P$ in $D$.
\end{theorem}
\begin{proof}
  By Lemma~\ref{lem:psurj} $p :\im~j\to \im~i$ is surjective.
  So, with the assumption that $\rk~i\geq \rk~j$, $\rk~i = \rk~j$.
  Because $P$ is a finite point set we know that $\im~i$ is finite-dimensional and, because $\rk~i = \rk~j$, $j$ is finite dimensional as well.
  So $p$ is bijective and therefore injective.

  As $j$ is surjective $\im~j = \hom_0(\cmp{\B}, \cmp{D})$ and, because $\im~i$ is a subspace of $\hom_0(\cmp{\Q^\of}, \cmp{\P})$, $p$ injective implies that the map $\hom_0(\cmp{\B}, \cmp{D})\to \hom_0(\cmp{\Q^\of}, \cmp{\P})$ must be injective, therefore $D\setminus\bb\subseteq \P$ by Lemma~\ref{lem:coverage}, and $\Q^\of$ surrounds $\P$ in $D$ by Lemma~\ref{lem:cov_surrounds}.
  % Finally, because $f$ is $c$-Lipschitz, $\bb\subseteq \B$, therefore $D\setminus\B\subseteq D\setminus \bb\subseteq \P$.
\end{proof}
