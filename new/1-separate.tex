% !TeX root = ../new.tex

\subsection{Separation}

\begin{definition}[Separation (Munkres~\cite{munkres00topology})]
  Let $X$ be a topological space. A \textbf{separation} of $X$ is a pair $U, V$ of disjoint, nonempty, open subsets of $X$ whose union is $X$.
  The space $X$ is said to be \textbf{connected} if there does not exist a separation of $X$.
\end{definition}

Note that the sets $U, V$ that form a separation of $X$ are both open and closed in $X$.
For a subspace $Y$ of $X$ we will denote the interior and closure of a set $U$ in $Y$ with $\intr_Y(U)$ and $\cl_Y(X)$.
 % where $\intr(U)$ and $\cl(U)$ will refer to the interior and closure of $U$ in $X$, unless otherwise stated.

\begin{lemma}[23.1 (Munkres~\cite{munkres00topology})]
  If $Y$ is a subspace of $X$, a separation of $Y$ is a pair of disjoint, nonempty sets $A, B$ whose union is $Y$, neither of which contains a limit point of the other.
  The space $Y$ is connected if there exists no separation of $Y$.
\end{lemma}

If $A, B$ is a separation of a subspace $Y$ of $X$ then $A, B$ are both open and closed in $Y$, but not necessarily $X$.
The condition that neither $A$ nor $B$ contains a limit point of the other requires that $\cl_X(A)\cap B = \emptyset$ and $A\cap \cl_X(B) =\emptyset$ where $\cl_Y(A) = A$ and $\cl_Y(B) = B$.

% \begin{definition}[Components (Munkres~\cite{munkres00topology})]
%   Given $X$, define an equivalence relation on $X$ by setting $x\sim y$ if there is a connected subspace of $X$ containing both $x$ and $y$.
%   The equivalence class are called the \textbf{components} (or ``connected components'') of $X$.
% \end{definition}

For a disconnected topological space $X$ let $X_1, X_2, \ldots$ denote it's path-connected components.
For $A\subseteq X$ let $A_i = A\cap X_i$ denote the component of $A$ in $X_i$.

\begin{definition}[Separating Set]
  Let $X$ be a (possibly disconnected) topological space and $S\subset X$.
  $S$ \textbf{separates $X$ with a pair $(U, V)$} if $(U_i, V_i)$ is a separation of $X_i\setminus S_i$ for all $i$.
\end{definition}

If $S$ separates $X$ with a pair $(U, V)$ then $X = U\sqcup S\sqcup V$.
Note that while $U$ and $V$ are both open and closed in $X\setminus S$, each component $X_i = U_i\sqcup S_i\sqcup V_i$ is connected.
Therefore, if $S$ separates $X$ with a pair $(U, V)$, we require that $\cl_X(U)\cap V = \emptyset$ and $U\cap \cl_X(V) = \emptyset$.
If $S$ is an open set in $X$ then $U$ and $V$ are closed in $X$, therefore $\cl_X(U)\cap V = \emptyset$ and $U\cap \cl_X(V) = \emptyset$.
Otherwise, if $S$ is closed in $X$, then $U$ and $V$ are open in $X$.

% \begin{lemma}
%   If $S$ separates $X$ with a pair $(U, V)$ then
%   \[ \hom_k()
% \end{lemma}

Throughout we will use $U, S,$ and $V$ to denote subsets of $X$ analogous to the interior, boundary, and complement of $S\sqcup U$ in $X$, respectively.
The following definition, while equivalent to that of a separating set, makes this distinction explicit by defining the set $S$ relative to the set $S\sqcup U$.

\begin{definition}[Surrounding]
  Given $B\subset D \subset X$ the set $B$ \textbf{surrounds $D$ in $X$} if $B$ separates $X$ with the pair $(D\setminus B, X\setminus D)$.
  We will refer to such a pair as a \textbf{surrounding pair in $X$}.
\end{definition}

Now, the set $D\setminus B$ corresponds to the interior of $D$ and $X\setminus D$ corresponds to the complement of $D$ in $X$.
This allows us to clearly state the extension of a surrounding pair in a subspace of $X$ to a surrounding pair in $X$.

\begin{definition}[Extension]
  If $P$ surrounds $Q$ in a subspace $D$ of $X$ let $\hat{Q} := Q\sqcup (D\setminus P)$ denote the (disjoint) union of the separating set $Q$ with the complement of $P$ in $D$.
  The \textbf{extension of $(P, Q)$ in $D$} is the pair
  \[ (D, \hat{Q}) = (P\sqcup (D\setminus P), Q\sqcup (D\setminus P)).\]
\end{definition}

\begin{lemma}\label{lem:excision}
  If $(P, Q)$ is a surrounding pair in a subspace $D$ of $X$ and $P$ is open in $D$ then
  \[ \hom_k(P\cap A, Q) \cong \hom_k(A, \hat{Q}) \]
  for all $k$ and any $A\subseteq D$ such that $\hat{Q}\subset A$.
\end{lemma}
\begin{proof}
  Because $Q$ surrounds $P$ in $D$, $(P\setminus Q, D\setminus P)$ is a separation of $D\setminus Q$, a subspace of $D$.
  So $\cl_D(P\setminus Q)\setminus P = \cl_D(P\setminus Q) \cap (D\setminus P) = \emptyset$ which implies $\cl_D(P\setminus Q)\subseteq P = \intr_D(P)$ as $P$ is open in $D$.
  Therefore,
  \begin{align*}
    \cl_D(D\setminus P) &= D\setminus \intr_D(P)\\
                        &\subseteq D\setminus \cl_D(P\setminus Q)\\
                        &= \intr_D(D\setminus (P\setminus Q))\\
                        &= \intr_D(\hat{Q}).
  \end{align*}
  so,
  \begin{align*}
    \hom_k(P\cap A, Q) &= \hom_k(A\setminus (D\setminus P), \hat{Q}\setminus (D\setminus P))\\
      &\cong \hom_k(A, \hat{Q})
  \end{align*}
  for all $k$ and any $A\subseteq D$ such that $\hat{Q}\subset A$ by Excision.
\end{proof}

\begin{theorem}
  Suppose $B\subseteq B'\subseteq B''$ all surround $D$ in $X$ and $A\subseteq D$ such that $B''\subset A$.
  Suppose $Q$ and $Q'$ surround $P$ in $D$ such that $B\subseteq \hat{Q} \subseteq B'\subseteq \hat{Q'}\subseteq B''$.
  % \begin{enumerate}
  %   \item $D\setminus B'\subseteq P$,
  %   \item $B\cap P \subseteq Q\subseteq B'$, and
  %   \item $B'\cap P\subseteq Q'\subseteq B''$.
  % \end{enumerate}
  Let $\eta^k : \hom_k(B)\to \hom_k(B'')$ be induced by inclusion.

  If $\eta^k$ is surjective and $\im~\eta^k\cong \hom_k(B')$ then
  \[\im~\hom_k((P\cap A, Q)\to (P\cap A, Q'))\cong \hom_k(A, B')\]
  for $k > 0$.
\end{theorem}
\begin{proof}
  Consider the following commutative diagrams of long exact sequences of pairs $(A, B)$, $(A, B')$ and $(A, B'')$.
  \begin{equation}\begin{tikzcd}
    \hom_k(B)\arrow{r}{i}\arrow{d}{a} &
    \hom_k(A)\arrow{r}{j}\arrow{d}{b} &
    \hom_k(A, B)\arrow{r}{k}\arrow{d}{c} &
    \hom_{k-1}(B)\arrow{r}{\ell}\arrow{d}{d} &
    \hom_{k-1}(A)\arrow{d}{e} &\\
    %
    \hom_k(B')\arrow{r}{i'}\arrow{d}{a'} &
    \hom_k(A)\arrow{r}{j'}\arrow{d}{b'} &
    \hom_k(A, B')\arrow{r}{k'}\arrow{d}{c'} &
    \hom_{k-1}(B')\arrow{r}{\ell'}\arrow{d}{d'} &
    \hom_{k-1}(A)\arrow{d}{e'} \\
    %
    \hom_k(B'')\arrow{r}{i''}&
    \hom_k(A)\arrow{r}{j''}&
    \hom_k(A, B'')\arrow{r}{k''}&
    \hom_{k-1}(B'')\arrow{r}{\ell''} &
    \hom_{k-1}(A)
  \end{tikzcd}\end{equation}
  where vertical maps are induced by inclusion.

  If $\im~\eta^k\cong\hom_k(B')$ for all $k$ then, because $\eta^k = a'\circ a$ and $\eta^{k-1} = d'\circ d$, $a, d$ must be surjective and $a', d'$ must be injective.
  Moreover, if $\eta^k$ is surjective then $d'$ is surjective, and therefore an isomorphism.
  As $b,b',e$ and $e'$ are the identity map they are bijective, therefore $c$ must be surjective and $c'$ must be bijective by Lemma~\ref{lem:five}, thus $\im~c'\circ c\cong \hom_k(A, B')$.

  Because $Q$ and $Q'$ surround $P$ in $D$, $\hom_k(P\cap A, Q)\cong \hom_k(A, \hat{Q})$ and $\hom_k(P\cap A, Q')\cong \hom_k(A, \hat{Q'})$ for all $k$ by Lemma~\ref{lem:excision}.
  % Moreover, because $D\setminus B\subseteq P$,
  % \[ \hat{Q} = Q\sqcup (D\setminus P) \subseteq Q\cup (D\setminus (D\setminus B')) = Q\cup B'
  We have the following sequence of homomorphisms induced by inclusion of pairs
  \[ \hom_k(A, B)\xrightarrow{m}\hom_k(A, \hat{Q})\xrightarrow{n}\hom_k(A, B')\xrightarrow{p}\hom_k(A, \hat{Q'})\xrightarrow{q}\hom_k(A, B'').\]
  As $q\circ p\circ n\circ m = c' \circ c$ and $\im~c'\circ c\cong \hom_k(A, B')$, $\im~n\circ p\cong \hom_k(A, B')$ by Lemma~\ref{lem:sandwich}.
  The result follows from Lemma~\ref{lem:excision} as $\im~n\circ p\cong \im~\hom_k((P\cap A, Q)\to (P\cap A, Q'))$.

\end{proof}

\clearpage

In the following let $X$ be a topological space and $\overline{A} := X\setminus A$ denote the complement of a subset $A$ of $X$.

\begin{lemma}\label{lem:coverage}
  Let $(D, B)$ be a surrounding pair in $X$ and $P\subseteq D$, $Q\subseteq P\cap B$.

  If $p: \hom_0(\overline{B}, \overline{D})\to \hom_0(\overline{Q}, \overline{P})$ is injective then $D\setminus B\subseteq P$.
\end{lemma}
\begin{proof}
    Suppose, for the sake of contradiction, that $p$ is injective and there exists a point $x\in (D\setminus B)\setminus P$.
    Because $B$ surrounds $D$ in $X$ the pair $(D\setminus B, \overline{D})$ forms a separation of $\overline{B}$.
    Therefore, $\hom_0(\overline{B})\cong \hom_0(D\setminus B)\oplus \hom_0(\overline{D})$ so
    \[ \hom_0(\overline{B}, \overline{D})\cong \hom_0(D\setminus B). \]
    So $[x]$ is non-trivial in $\hom_0(\overline{B},\overline{D})\cong \hom_0(D\setminus B)$ as $x$ is in some connected component of $D\setminus B$.
    So we have the following sequence of maps induced by inclusions
    \[ \hom_0(\overline{B},\overline{D})\xrightarrow{f} \hom_0(\overline{B},\overline{D}\cup\{x\})\xrightarrow{g} \hom_0(\overline{Q},\overline{P}).\]
    As $f[x]$ is trivial in $\hom_0(\overline{B},\overline{D}\cup\{x\})$ we have that $p[x] = (g\circ f)[x]$ is trivial, contradicting our hypothesis that $p$ is injective.
\end{proof}

\begin{lemma}\label{lem:cov_surrounds}
  Let $(D, B)$ be a surrounding pair in $X$ and $P\subseteq D$, $Q\subseteq P\cap B$.

  If $p: \hom_0(\overline{B}, \overline{D})\to \hom_0(\overline{Q}, \overline{P})$ is injective then $Q$ surrounds $P$ in $D$.
\end{lemma}
\begin{proof}
  Suppose, for the sake of contradiction, that $Q$ does not surround $P$ in $D$.
  Then there exists a path $\gamma : [0,1]\to\overline{Q}$ with $\gamma(0)\in P\setminus Q$ and $\gamma(1)\in D\setminus P$.
  By Lemma~\ref{lem:coverage} we know that $D\setminus B\subseteq P$, so $D\setminus B\subseteq P\setminus Q$.

  Choose $x\in D\setminus B$ and $z\in \overline{D}$ such that there exist paths $\xi : [0,1]\to P\setminus Q$ with $\xi(0) = x$, $\xi(1) = \gamma(0)$ and $\zeta : [0,1]\to \overline{D}\cup (D\setminus P)$ with $\zeta(0) = z$, $\zeta(1) = \gamma(1)$.
  $\xi, \gamma$ and $\zeta$ all generate chains in $C_1(\overline{Q}, \overline{P})$ and $\xi + \gamma + \zeta = \gamma^*\in C_1(\overline{Q}, \overline{P})$ with $\partial\gamma^* = x + z$.
  Moreover, $z$ generates a chain in $C_0(\overline{P})$ as $\overline{D}\subseteq\overline{P}$.
  So $x = \partial\gamma^* + z$ is a relative boundary in $C_0(\overline{Q}, \overline{P})$ thus $[x] = [z] = 0$ in $\hom_0(\overline{Q}, \overline{P})$.
  However, because $B$ surrounds $D$, $[x]\neq [y]$ in $\hom_0(\overline{B}, \overline{D})\cong \im~j_*$, contradicting our assumption that $p_*$ is injective.
\end{proof}
