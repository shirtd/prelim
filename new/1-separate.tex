% !TeX root = new.tex

\begin{definition}[Separated]
  Two subsets $U, V$ of a topological space $X$ are \textbf{separated in $X$} if each is disjoint from the other's closure: $\cl(U)\cap V =\emptyset$ and $U\cap\cl(V) = \emptyset$.
\end{definition}

\begin{definition}[Separation]
  We say that a subset $B$ \textbf{separates} a topological space $X$ with the pair $(U, V)$ if $B, U, V$ partitions $X$, $U, V$ are separated in $X$, and any path from $U$ to $V$ in $X$ intersects $B$.
\end{definition}

\begin{lemma}\label{lem:separate_splits}
  If $B$ separates $X$ with the pair $(U, V)$ then for all $k$ the short exact sequence
  \[ 0\to \hom_k(V)\xrightarrow{i_*} \hom_k(X\setminus B)\xrightarrow{j_*} \hom_k(U)\to 0\]
  splits.
\end{lemma}
\begin{proof}
  Because $X\setminus B$ is the disjoint union of $U$ and $V$ we know that $i_* : \hom_k(V)\to \hom_k(X\setminus B)$ is the map induced by inclusion and $p_* : \hom_k(X\setminus B)\to \hom_k(V)$ is induced by the restriction of the identity on $X\setminus B$ to $V$.
  Thus $p_*\circ i_* = \mathbf{Id}_{\hom_k(V)}$ and therefore, by Lemma~\ref{lem:splitting} the sequence splits.%we have a natural isomorphism $\hom_k(X\setminus )
\end{proof}

\begin{corollary}\label{cor:oplus_separates}
    If $B$ separates $X$ with the pair $(U, V)$ then for all $k$
    \[ \hom_k(X\setminus B) \cong H_k(U)\oplus \hom_k(V). \]
\end{corollary}

\begin{lemma}\label{lem:iso_separates}
  If $B$ separates $X$ with the pair $(U, V)$ then for all $k$
  \[ \hom_k(U)\cong \hom_k(X\setminus B, V).\]
\end{lemma}
\begin{proof}
  First note that the short exact sequence
  \[ 0\to \hom_k(V)\to \hom_k(U)\oplus\hom_k(V)\to \hom_k(U)\to 0\]
  extends to a long exact sequence with the zero map $\partial_*^k : \hom_k(U)\to \hom_k(V)$ as $\im~j^k_* = \hom_k(U) = \ker~\partial_*^k$ and $\im~\partial_*^k = \ker~ i^{k-1}_* = \mathbf{0}_{\hom_{k-1}(V)}$.
  Consider the following commutative diagram where the bottom row is the long exact sequence of the pair $(X\setminus B, V)$
  \begin{small}
  \[\begin{tikzcd}[column sep=small]
    \ldots\ar[r]  & \hom_k(V)\ar[r, "i_*^k"]\ar[d, "f_*^k"]
                  & \hom_k(U)\oplus\hom_k(V)\ar[r, "j_*^k"]\ar[d, "g_*^k"]
                  & \hom_k(U)\ar[r, "\partial_*^k"]\ar[d, "h_*^k"]
                  & \hom_{k-1}(V)\ar[r, "i_*^{k-1}"]\ar[d, "f_*^{k-1}"]
                  & \hom_{k-1}(U)\oplus\hom_{k-1}(V)\ar[r]\ar[d, "g_*^{k-1}"]
                  & \ldots\\
    \ldots\ar[r]  & \hom_k(V)\ar[r, "\widehat{i_*^k}"]
                  & \hom_k(X\setminus B)\ar[r, "\widehat{j_*^k}"]
                  & \hom_k(U)\ar[r, "\widehat{\partial_*^k}"]
                  & \hom_{k-1}(V)\ar[r, "\widehat{i_*^{k-1}}"]
                  & \hom_{k-1}(X\setminus B)\ar[r]
                  & \ldots\\
  \end{tikzcd}\]
  \end{small}
  As $f_*^k$ is the identity map and, by Corollary~\ref{cor:oplus_separates}, $g^k_*$ is an isomorphism for all $k$ it follows that $h_*^k$ is an isomorphism for all $k$ by Lemma~\ref{lem:five}.
\end{proof}

\begin{definition}[Surrounding]
  We say that $B\subset D$ \textbf{surrounds} $D\subset X$ in $X$ if $B$ separates $X$ with the pair $(D\setminus B, X\setminus D)$.
  We will refer to such a pair $(D, B)$ as a \textbf{surrounding pair} in $X$.
\end{definition}

We define the \emph{extension} of a surrounding pair $(D, B)$ in $X$ to be the pair
\[ (X, \hat{B}) = (D\cup (X\setminus D), B\cup (X\setminus D)).\]
Note that, for $X\subset\mathcal{X}$ the extension of a surrounding pair in $X$ is a surrounding pair in $\mathcal{X}$.

% {\color{red}
\begin{lemma}\label{lem:extension_iso}
  If $(D, B)$ is a surrounding pair of open sets in a topological space $X$ then for all $k$
  \[ \hom_k(D, B)\cong \hom_k(X, \hat{B}).\]
\end{lemma}
\begin{proof}
  Note,
  \begin{align*}
    X\setminus (\intr(D)\cup\intr(\hat{B})) &= X\cap \overline{(\intr(D)\cup\intr(\hat{B}))}\\
      &= X\cap \cl(\overline{D})\cap\cl(\overline{\hat{B}})) \\
      &= X\cap \overline{D}\cap\cl(\overline{\hat{B}})),
  \end{align*}
  because $D$ is open, thus $\overline{D}$ is closed, and therefore equal to its closure.

  Now,
  \begin{align*}
    \cl(\overline{\hat{B}}) &= \cl(\overline{B\cup (X\setminus D)})\\
      &= \cl(\overline{B\cup (X\cap \overline{D})})\\
      &= \cl(\overline{(B\cup X)\cap (B\cup\overline{D})})\\
      &= \cl(\overline{X}\cup (D\cap\overline{B}))\\
      &= \cl(D\setminus B).
  \end{align*}
  Therefore,
  \[ X\setminus (\intr(D)\cup\intr(\hat{B})) = (X\setminus D)\cap \cl(D\setminus B) =\emptyset \]
  for a surrounding pair $(D, B)$ in $X$ which requires that $(X\setminus D)$ and $D\setminus B$ are separated in $X$.

  So $\intr(D)\cup\intr(\hat{B}) = X$ thus $\hom_k(D, B)\cong \hom_k(X, \hat{B})$ by excision.
\end{proof}
% }

The following is a corollary of Theorem~\ref{thm:alexander} (Alexander Duality).

\begin{corollary}\label{cor:alexander_surrounds}
  If $(D,B)$ is a surrounding pair of locally contractible, nonempty, proper subspaces in $S^d$ then for all $k$
  \[ \tilde{\hom}_k(D, B) \cong \tilde{\hom}^{d-k}(D\setminus B). \]
\end{corollary}

In the following we will assume that $(D,B)$ is a surrounding pair of locally contractible, nonempty, proper subspaces in $S^d$.
Let $(\overline{B}, \overline{D}) = (S^d\setminus B, S^d\setminus D)$ denote the complement of the pair $(D, B)$ in $S^d$.

\begin{lemma}\label{lem:alexander_comm}
  If $(D,B)$ is a surrounding pair of locally contractible, nonempty, proper subspaces of $S^d$ then
  \[i_*^{k} : \tilde{\hom}_{k+1}(D, B)\to \tilde{\hom}_k(B)\]
  is injective and
  \[j_*^{k} : \tilde{\hom}_{k}(B)\to \tilde{\hom}_k(D)\]
  is surjective for all $k$.
\end{lemma}
\begin{proof}
  We have the following commutative diagram of long exact sequences of the pairs $(D, B)$ and $(\overline{B}, \overline{D})$.

  \begin{equation}\begin{tikzcd}\label{dgm:alexander}
    \tilde{\hom}_{k+1}(D, B)\arrow{r}{\partial_*^{k+1}}\arrow{d}{\Gamma^{k+1}_{(D, B)}} &
    \tilde{\hom}_{k}(B)\arrow{r}{i_*^{k}}\arrow{d}{\Gamma^{k}_B} &
    \tilde{\hom}_{k}(D)\arrow{d}{\Gamma^{k}_D} \\
    %
    \tilde{\hom}^{d-k-1}(\overline{B}, \overline{D})\arrow{r}{\overline{j_*^{d-k-1}}} &
    \tilde{\hom}^{d-k-1}(\overline{B})\arrow{r}{\overline{i_*^{d-k-1}}} &
    \tilde{\hom}^{d-k-1}(\overline{D}) \\
  \end{tikzcd}\end{equation}

  Because $B$ surrounds $D$ we have that
  \[\tilde{\hom}^{d-k-1}(\overline{B}) \cong \tilde{\hom}^{d-k-1}(D\setminus B)\oplus \hom_k(\overline{D})\]
  by Corollary~\ref{cor:oplus_separates}\footnote{\textbf{TODO show/state this holds for cohomology as well.}}, where $\tilde{\hom}^{d-k-1}(D\setminus B) \cong \tilde{\hom}^{d-k-1}(\overline{B}, \overline{D})$ by Lemma~\ref{lem:iso_separates}.
  It follows that $\overline{j_*^{d-k-1}}$ is injective and $\overline{i_*^{d-k-1}}$ is surjective.

  By commutativity of Diagram~\ref{dgm:alexander} and because $\Gamma_{(D,B)}^{k+1}, \Gamma_B^k$ and $\Gamma_D^k$ are isomorphisms we have that
  \[\partial_*^{k+1} = (\Gamma_B^{k})^{-1} \circ\overline{j_*^{d-k-1}}\circ \Gamma_{(D,B)}^{k+1}\]
  is injective and
  \[i_*^{k} = (\Gamma_D^{k})^{-1} \circ\overline{i_*^{d-k-1}}\circ \Gamma_B^{k+1}\]
  is surjective.
\end{proof}

We note that this implies the following for non-reduced homology\footnote{\textbf{TODO
  reasoning:\begin{itemize}
    \item consider $\hom_1(D, B)\to \hom_0(B)$.
    \item $\tilde{\hom}_0(B)\to \tilde{\hom}_0(D)$ surjective implies $\hom_0(B)\to \hom_0(D)$ surjective (right?).
  \end{itemize}}}
for subsets of $\R^d$.\footnote{\textbf{TODO
  reasoning:\begin{itemize}
    \item $S^d\cong \R^d\cup\{\infty\}$.
    \item Only requires spaces \emph{and} complements remain compact?
  \end{itemize}}}

\begin{corollary}\label{cor:alexander_comm}
  If $(D,B)$ is a surrounding pair of locally contractible, nonempty, proper subspaces of $\R^d$ then
  \[i_*^{k} : \hom_{k+1}(D, B)\to \hom_k(B)\]
  is injective and
  \[j_*^{k} : \hom_{k}(B)\to \hom_k(D)\]
  is surjective for all $k$.
\end{corollary}
