% !TeX root = ../new.tex

\section{Persistence Modules}

A \textbf{persistence module} $\V$ over $\R$ is an indexed family of vector spaces $\{V_\alpha\}$ and linear maps $\{v_\alpha^\beta : V_\alpha\to V_\beta\}$ such that $v^\e_\beta\circ v_\alpha^\beta = v_\alpha^\e$ whenever $\alpha\leq\beta\leq\e$ and $v_\alpha^\alpha$ is the identity on $V_\alpha$.
A \textbf{homomorphism} $\Gamma$ between two $\R$-persistence modules $\U, \V$ is a collection of linear mapw $\{\gamma_\alpha : U_\alpha\to V_\alpha$ such that the following diagram commutes for all $\alpha\leq\beta$.
\begin{equation}\label{dgm:homomorphism}
  \begin{tikzcd}
    U_\alpha\arrow{r}{u_\alpha^\beta}\arrow{d}{\gamma_\alpha} &
    U_\beta\arrow{d}{\gamma_\beta}\\
    %
    V_\alpha\arrow{r}{v_\alpha^\beta} &
    V_\beta
\end{tikzcd}\end{equation}
The space of homomorphisms from $\U$ to $\V$ will be denoted $\Hom(\U, \V)$.

A \textbf{homomorphism of degree $\delta$} is a collection $F$ of linear maps $f_\alpha : U_\alpha\to V_{\alpha+\delta}$ such that the following diagram commutes for all $\alpha\leq\beta$.
\begin{equation}\label{dgm:shifted_homomorphism}
  \begin{tikzcd}
    U_\alpha\arrow{r}{u_\alpha^\beta}\arrow{d}{f_\alpha} &
    U_\beta\arrow{d}{f_\beta}\\
    %
    V_{\alpha+\delta}\arrow{r}{v_{\alpha+\delta}^{\beta+\delta}} &
    V_{\beta +\delta}
\end{tikzcd}\end{equation}
The space of homomorphisms of degree $\delta$ from $\U$ to $\V$ will be denoted $\Hom^\delta(\U, \V)$.

Two persistence modules $\U$ and $\V$ are \textbf{$\delta$-interleaved} if there exist homomorphisms $F\in\Hom^\delta(\U, \V)$ and $G \in\Hom^\delta(\V,\U)$ such that the following diagrams commute for all $\alpha$.
\begin{minipage}{0.45\textwidth}
\begin{equation}\label{dgm:interleaving1}
  \begin{tikzcd}
    U_{\alpha-\delta}\arrow{rr}{u_{\alpha-\delta}^{\alpha+\delta}}\arrow{dr}{f_{\alpha-\delta}} & &
    U_{\alpha+\delta}\\
    %
    & V_{\alpha}\arrow{ur}{g_\alpha} &
\end{tikzcd}\end{equation}
\end{minipage}
\begin{minipage}{0.45\textwidth}
\begin{equation}\label{dgm:interleaving2}
  \begin{tikzcd}
    & U_{\alpha}\arrow{dr}{f_\alpha} &\\
    %
    V_{\alpha-\delta}\arrow{rr}{v_{\alpha-\delta}^{\alpha+\delta}}\arrow{ur}{g_{\alpha-\delta}} & &
    V_{\alpha+\delta}
\end{tikzcd}\end{equation}
\end{minipage}

\subsection{Image Persistence Modules}

The \textbf{image persistence module} of $\Gamma\in\Hom(\U,\V)$ is the family of subspaces $\{\Gamma_\alpha :=\im~\gamma_\alpha\}$ in $\V$ along with linear maps $\{\gamma_\alpha^\beta := v_\alpha^\beta\rest_{\im~\gamma_\alpha} : \Gamma_\alpha\to\Gamma_\beta\}$ and will be denoted by $\im~\Gamma$.

Given $\Gamma\in\Hom(\U,\V)$ and $\Lambda\in\Hom(\S,\T)$ along with $F\in\Hom^\delta(\U,\S)$ and $G\in\Hom^\delta(\V,\T)$ let $\Phi(F, G) : \im~\Gamma\to\im~\Lambda$ denote the family of linear maps $\{\phi_\alpha := g_\alpha\rest_{\Gamma_\alpha} : \Gamma_\alpha\to\Lambda_{\alpha+\delta}\}$.
$\Phi(F, G)$ is an \textbf{image module homomorphism of degree $\delta$} if the following diagram commutes.
for all $\alpha\leq\beta$.
\begin{equation}\label{dgm:shifted_homomorphism}
  \begin{tikzcd}[column sep=large]
    U_\alpha\arrow{r}{v_\alpha^\beta\circ\gamma_\alpha}\arrow{d}{f_\alpha} &
    V_\beta\arrow{d}{g_\beta}\\
    %
    S_{\alpha+\delta}\arrow{r}{t_{\alpha+\delta}^{\beta+\delta}\circ\lambda_{\alpha+\delta}} &
    T_{\beta +\delta}
\end{tikzcd}\end{equation}

Note that $\im~v_\alpha^\beta\circ\gamma_\alpha = \im~\gamma_\alpha^\beta$ and $\im~t_{\alpha}^{\beta}\circ\lambda_{\alpha} = \im~\lambda_\alpha^\beta$ for all $\alpha\leq\beta$.
Therefore, if $\Phi(F, G)$ is an image module homomorphism of degree $\delta$ then the following diagram commutes.
\begin{equation}\label{dgm:shifted_homomorphism}
  \begin{tikzcd}[column sep=large]
    \Gamma_\alpha\arrow{r}{\gamma_\alpha^\beta}\arrow{d}{\phi_\alpha} &
    \Gamma_\beta\arrow{d}{\phi_\beta}\\
    %
    \Lambda_{\alpha+\delta}\arrow{r}{\lambda_{\alpha+\delta}^{\beta+\delta}} &
    \Lambda_{\beta +\delta}
\end{tikzcd}\end{equation}

Let $\Gamma\in\Hom(\U,\V)$, $\Lambda\in\Hom(\S,\T)$ and $F\in\Hom^\delta(\U, \S)$, $G\in\Hom^\delta(\V, \T)$ be such that $\Phi(F,G) : \im~\Gamma\to\im~\Lambda$ is a image module homomorphism of degree $\delta$.
$\Phi(F, G)$ is said to be a \textbf{partial $\delta$ interleaving of image modules} $\im~\Gamma$ and $\im~\Lambda$ if there exists some $M\in\Hom^\delta(\S,\V)$ such that the following diagrams commute for all $\alpha$.

\begin{minipage}{0.45\textwidth}
\begin{equation}\label{dgm:partial_interleaving1}
  \begin{tikzcd}
    U_{\alpha-\delta}\arrow{rr}{v_{\alpha-\delta}^{\alpha+\delta}\circ\gamma_{\alpha-\delta}}\arrow{dr}{f_{\alpha-\delta}} & &
    V_{\alpha+\delta}\\
    %
    & S_{\alpha}\arrow{ur}{m_\alpha} &
\end{tikzcd}\end{equation}
\end{minipage}
\begin{minipage}{0.45\textwidth}
\begin{equation}\label{dgm:partial_interleaving2}
  \begin{tikzcd}
    & V_{\alpha}\arrow{dr}{g_\alpha} &\\
    %
    S_{\alpha-\delta}\arrow{rr}{t_{\alpha-\delta}^{\alpha+\delta}\circ\lambda_{\alpha-\delta}}\arrow{ur}{m_{\alpha-\delta}} & &
    T_{\alpha+\delta}
\end{tikzcd}\end{equation}
\end{minipage}

An image module homomorphism of degree $\delta$ that provides a partial $\delta$-interleaving will be denoted $\Phi_M(F, G)$.

\begin{theorem}\label{thm:interleaving_main}
  For $\R$-persistence modules $\S, \T, \U, \V, \W$ let
  \[ \Gamma\in\Hom(\U,\V),\ \Lambda\in\Hom(\S, \T),\ \Pi\in\Hom(\V,\W)\]
  be homomorphisms of persistence modules and
  \begin{align*}
    F\in\Hom^\delta(\U,\S)& & M\in\Hom^\delta(\S,\V)\\
    G\in\Hom^\delta(\V,\T)& & N\in\Hom^\delta(\T,\W)
  \end{align*}
  be homomorphisms of degree $\delta$.

  If $\Phi(F, G; M) :\im~\Gamma\to\im~\Lambda$ and $\Psi_G(M, N): \im~\Lambda\to\im~\Pi$ are partial $\delta$-interleavings of image modules such that $\Gamma$ is a epimorphism and $\Pi$ is a monomorphism then $\im~\Lambda$ is $\delta$-interleaved with $\V$.
\end{theorem}
\begin{proof}
  For ease of notation let $\Phi$ denote $\Phi_M(F, G)$ and $\Psi$ denote $\Psi_G(M, N)$.

  If $\Gamma$ is an epimorphism $\gamma_\alpha$ is surjective so $\Gamma_\alpha = V_\alpha$ and $\phi_{\alpha} = g_{\alpha}\rest_{\Gamma_\alpha} = g_\alpha$ for all $\alpha$.
  So $\im~\Gamma = \V$ and $\Phi\in\Hom^\delta(\V,\im~\Lambda)$.

  If $\Pi$ is a monomorphism then $\pi_\alpha$ is injective so we can define a natural isomorphism $\pi_\alpha^{-1} : \Pi_\alpha\to V_\alpha$ for all $\alpha$.
  Let $\Psi^*$ be defined as the family of linear maps $\{\psi_\alpha^* := \pi^{-1}_\alpha \circ \psi_\alpha : \Lambda_\alpha\to V_{\alpha+\delta}\}$.
  Because $\Psi$ is a partial $\delta$-interleaving of image modules, $n_\alpha\circ\lambda_\alpha = \pi_{\alpha+\delta}\circ m_\alpha$.
  So, because $\psi_\alpha = n_\alpha\rest_{\Lambda_\alpha}$ for all $\alpha$,
  \begin{align*}
    \im~\psi_\alpha^* &= \im~\pi^{-1}_{\alpha+\delta}\circ\psi_\alpha\\
                      &= \im~\pi^{-1}\circ (n_\alpha\circ\lambda_\alpha)\\
                      &= \im~\pi^{-1}\circ (\pi_{\alpha+\delta}\circ m_\alpha)\\
                      &= \im~ m_\alpha.
  \end{align*}
  It follows that $\im~v_{\alpha+\delta}^{\beta+\delta}\circ\psi_\alpha^* = \im~v_{\alpha+\delta}^{\beta+\delta}\circ m_\alpha$

  % Now,
  % \begin{align*}
  %   v_{\alpha+\delta}^{\beta+\delta}\circ\psi_\alpha^* &= v_{\alpha+\delta}^{\beta+\delta}\circ (\pi^{-1}_\alpha \circ \psi_\alpha)\\
  %     &= v_{\alpha+\delta}^{\beta+\delta}\circ \pi^{-1}_\alpha \circ n_\alpha\rest_{\Lambda_\alpha}
  % \end{align*}
  % where, because $\Psi$ is a partial $\delta$-interleaving of image modules, $n_\alpha\circ\lambda_\alpha = \pi_{\alpha+\delta}\circ m_\alpha$.
  % So,
  % \begin{align*}
  %   \im~v_{\alpha+\delta}^{\beta+\delta}\circ\psi_\alpha^* &= \im~ v_{\alpha+\delta}^{\beta+\delta}\circ \pi^{-1}_\alpha \circ n_\alpha\rest_{\Lambda_\alpha}\\
  %     &= \im~v_{\alpha+\delta}^{\beta+\delta}\circ \pi^{-1}_{\alpha+\delta}\circ(\pi_{\alpha+\delta}\circ m_\alpha)\\
  %     &= \im~v_{\alpha+\delta}^{\beta+\delta}\circ m_\alpha.
  % \end{align*}

  Similarly, because $\Psi$ is a $\delta$-interleaving of image modules $n_\beta\circ t_\alpha^\beta\circ \lambda_\alpha = w_{\alpha+\delta}^{\beta+\delta}\circ\pi_{\alpha+\delta}\circ m_\alpha$.
  Moreover, because $\Pi$ is a homomorphism of persistence modules, $w_{\alpha+\delta}^{\beta+\delta}\circ\pi_{\alpha+\delta} = \pi_{\beta+\delta}\circ v_{\alpha+\delta}^{\beta+\delta}$, so
  \[ n_\beta\circ t_\alpha^\beta\circ \lambda_\alpha = \pi_{\beta+\delta}\circ v_{\alpha+\delta}^{\beta+\delta}\circ m_\alpha.\]
  As $\psi_\beta\circ\lambda_\alpha^\beta = n_\beta\circ\lambda_\alpha^\beta = n_\beta\circ t_\alpha^\beta\rest_{\Lambda_\alpha}$ it follows
  \begin{align*}
    % \im~\psi_\beta^*\circ\lambda_\alpha^\beta &= \im~\pi^{-1}_{\beta+\delta}\circ\psi_\beta\circ\lambda_\alpha^\beta\\
    \im~\psi_\beta^*\circ\lambda_\alpha^\beta &= \im~\pi^{-1}_{\beta+\delta}\circ (n_\beta\circ t_\alpha^\beta\circ\lambda_\alpha)\\
      % &= \im~\pi^{-1}_{\beta+\delta}\circ (w_{\alpha+\delta}^{\beta+\delta}\circ\pi_{\alpha+\delta})\circ m_\alpha\\
      &= \im~\pi^{-1}_{\beta+\delta}\circ (\pi_{\beta+\delta}\circ v_{\alpha+\delta}^{\beta+\delta})\circ m_\alpha\\
      &= \im~v_{\alpha+\delta}^{\beta+\delta}\circ m_\alpha\\
      &= \im~v_{\alpha+\delta}^{\beta+\delta}\circ\psi_\alpha^*.
  \end{align*}
  So we may conclude that $\Psi^*\in\Hom^\delta(\im~\Lambda,\V)$.

  % \begin{align*}
  %   \psi_{\beta}^*\circ\lambda_\alpha^\beta &=\pi^{-1}_{\beta+\delta}\circ\psi_\beta\circ\lambda_\alpha^\beta\\
  %     &= \pi^{-1}_{\beta+\delta}\circ (n_\beta\rest_{\Lambda_\beta})\circ\lambda_\alpha^\beta\\
  %     &= \pi^{-1}_{\beta+\delta}\circ n_\beta\circ (t_\alpha^\beta\rest_{\Lambda_\alpha}).
  % \end{align*}
  % Because $\Psi$ is a $\delta$-interleaving of image modules $n_\beta\circ t_\alpha^\beta\circ \lambda_\alpha = w_{\alpha+\delta}^{\beta+\delta}\circ\pi_{\alpha+\delta}\circ m_\alpha$ and, because $\Pi$ is a homomorphism of persistence modules, $w_{\alpha+\delta}^{\beta+\delta}\circ\pi_{\alpha+\delta} = \pi_{\beta+\delta}\circ v_{\alpha+\delta}^{\beta+\delta}$.
  % So,
  % \begin{align*}
  %   \im~\psi_{\beta}^*\circ\lambda_\alpha^\beta &= \im~\pi^{-1}_{\beta+\delta}\circ (n_\beta\circ t_\alpha^\beta\rest_{\Lambda_\alpha})\\
  %     &= \im~\pi^{-1}_{\beta+\delta}\circ (\pi_{\beta+\delta}\circ v_{\alpha+\delta}^{\beta+\delta}\circ m_\alpha)\\
  %     &= \im~v_{\alpha+\delta}^{\beta+\delta}\circ m_\alpha\\
  %     &= \im~v_{\alpha+\delta}^{\beta+\delta}\circ\psi_\alpha^*.
  % \end{align*}
  % As the images of $v_{\alpha+\delta}^{\beta+\delta}\circ\psi_\alpha^*, \psi_{\beta}^*\circ\lambda_\alpha^\beta : \Lambda_\alpha\to V_{\beta+\delta}$ are equal we may conclude that $v_{\alpha+\delta}^{\beta+\delta}\circ\psi_\alpha^* = \psi_{\beta}^*\circ\lambda_\alpha^\beta$ for all $\alpha\leq\beta$, so $\Psi^*\in\Hom^\delta(\im~\Lambda,\V)$.

  So $\Phi\in\Hom^\delta(\V,\im~\Lambda)$ and $\Psi_G^*\in\Hom^\delta(\im~\Lambda,\V)$.
  As we have shown, $\im~\psi_{\alpha-\delta}^* = \im~m_{\alpha-\delta}$ so $\im~\phi_\alpha\circ\psi_{\alpha-\delta}^* = \im~\phi_\alpha\circ m_{\alpha-\delta}$.
  Moreover, because $\gamma_\alpha$ is surjective $\phi_\alpha = g_\alpha$ and, because $\Phi$ is a partial $\delta$-interleaving of image modules, $g_\alpha\circ m_{\alpha-\delta} = t_{\alpha-\delta}^{\alpha+\delta}\circ \lambda_{\alpha-\delta}$.
  As $\lambda_{\alpha-\delta}^{\alpha+\delta} = t_{\alpha-\delta}^{\alpha+\delta}\rest_{\im~\lambda_{\alpha-\delta}}$ it follows that $\im~\phi_\alpha\circ\psi_{\alpha-\delta}^* = \im~\lambda_{\alpha-\delta}^{\alpha+\delta}$.

  Finally, $\psi_\alpha^*\circ\phi_\alpha = \pi_{\alpha+\delta}^{-1}\circ n_\alpha\circ g_{\alpha-\delta}$ where, because $\Psi$ is a partial $\delta$-interleaving of image modules, $n_\alpha\circ g_{\alpha-\delta} = w_{\alpha-\delta}^{\alpha+\delta}\circ\pi_{\alpha-\delta}$.
  Because $\Pi$ is a homomorphism of persistence modules $w_{\alpha-\delta}^{\alpha+\delta}\circ \pi_{\alpha-\delta} = \pi_{\alpha+\delta}\circ v_{\alpha-\delta}^{\alpha+\delta}$.
  Therefore,
  \begin{align*}
    \psi_\alpha^*\circ\phi_\alpha &= \pi_{\alpha+\delta}^{-1}\circ n_\alpha\circ g_{\alpha-\delta}\\
      &= \pi_{\alpha+\delta}^{-1}\circ (\pi_{\alpha+\delta}\circ v_{\alpha-\delta}^{\alpha+\delta})\\
      &= v_{\alpha-\delta}^{\alpha+\delta}
  \end{align*}
  which, along with $\phi_\alpha\circ\im~\psi_{\alpha-\delta}^* = \lambda_{\alpha-\delta}^{\alpha+\delta}$ implies Diagrams~\ref{dgm:interleaving1} and~\ref{dgm:interleaving2} commute with $\Phi\in\Hom^\delta(\V,\im~\Lambda)$ and $\Psi^*\in\Hom^\delta(\im~\Lambda, \V)$.
  We may therefore conclude that $\im~\Lambda$ and $\V$ are $\delta$-interleaved.

\end{proof}

% \begin{theorem}\label{thm:interleaving}
%   Suppose $(\{S_\alpha\},\{T_\alpha\})$, $(\{X_\alpha\}, \{Y_\alpha\})$, and $(\{Y_\alpha\}, \{Z_\alpha\})$ are all pairs of compatible filtrations such that the following diagram commutes for all $\alpha\in\R$.
%   \begin{equation}\label{dgm:interleaving}
%   \begin{tikzcd}
%     \hom_k(X_{\alpha-\e})\arrow{r}{u_{\alpha-\e}}\arrow{d}{f_{\alpha-\e}} &
%     \hom_k(S_\alpha)\arrow{r}{m_\alpha}\arrow{d}{g_\alpha} &
%     \hom_k(Y_{\alpha+\e})\arrow{d}{h_{\alpha+\e}}\\
%     %
%     \hom_k(Y_{\alpha-\e})\arrow{r}{v_{\alpha-\e}} &
%     \hom_k(T_\alpha)\arrow{r}{n_\alpha} &
%     \hom_k(Z_{\alpha+\e})
%   \end{tikzcd}\end{equation}
%
%   If $f_\alpha : \hom_k(X_\alpha)\to\hom_k(Y_\alpha)$ is surjective and $h_\alpha : \hom_k(Y_\alpha)\to \hom_k(Z_\alpha)$ is injective for all $\alpha\in\R$ then the $k$th persistent homology modules of $\{S_\alpha\to T_\alpha\}$ and $\{Y_\alpha\}$ are $\e$-interleaved.
% \end{theorem}
% \begin{proof}
%   Let $\Phi := (\{\Phi_\alpha\}, \{\phi_\alpha^\beta : \Phi_\alpha\to\Phi_\beta\})$ denote the $k$th persistent homology module of $\{S_\alpha\to T_\alpha\}$ where $\Phi_\alpha := \im~g_\alpha$  and $\phi_\alpha^\beta := t_\alpha^\beta\rest_{\im~g_\alpha}$ for $t_\alpha^\beta : \hom_k(T_\alpha)\to\hom_k(T_\beta)$.
%   Note that this map is well defined for all $\alpha\leq\beta$ as $\{S_\alpha\}$ and $\{T_\alpha\}$ are compatible filtrations.
%   Let $\Psi := (\{\Psi_\alpha\}, \{\psi_\alpha^\beta : \Psi_\alpha\to\Psi_\beta\})$ be the $k$th persistent homology module of $\{Y_\alpha\}$ where $\Psi_\alpha := \hom_k(Y_\alpha)$ and $\psi_\alpha^\beta$ induced by the map $Y_\alpha\to Y_\beta$.
%   Once again, because $(\{X_\alpha\},\{Y_\alpha\})$ and $(\{Y_\alpha\},\{Z_\alpha\})$ are compatible pairs of filtrations this map commutes with both $X_\alpha\to X_\beta$ and $Z_\alpha\to Z_\beta$ for all $\alpha\leq\beta$.
%
%   Because $h_\alpha$ is injective for all $\alpha$ we have an isomorphism $h_\alpha^{-1} : \im~h_\alpha\to \Psi_\alpha$.
%   Because the right square of Diagram~\ref{dgm:interleaving} commutes $n_\alpha\circ g_\alpha = h_\alpha\circ m_\alpha$, so $\im~n_\alpha\rest_{\Phi_\alpha}$ is a subspace of $\im~h_\alpha$.
%   We therefore define $\mu_\alpha : \Phi_\alpha\to\Psi_{\alpha+\e}$ as $\mu_\alpha := h^{-1}_{\alpha+\e}\circ n_\alpha\rest_{\Phi_\alpha}$.
%   Similarly, because the left square of Diagram~\ref{dgm:interleaving} commutes and $f_{\alpha-\e}$ is surjective $\im~v_{\alpha-\e}\circ f_{\alpha-\e} = \im~v_{\alpha-\e}$ is a subspace of $\Phi_\alpha$.
%   We may therefore define $\nu_{\alpha-\e} : \Psi_{\alpha-\e}\to \Phi_\alpha$ as $\nu_{\alpha-\e} = v_{\alpha-\e}$ for all $\alpha$.
%
%   \textbf{It follows that the following diagrams commute for all $\alpha\leq\beta$}
%
%   \begin{minipage}{0.5\textwidth}
%   \begin{equation}\label{dgm:intr1}
%   \begin{tikzcd}
%     \Phi_{\alpha-\e}\arrow{rr}{\phi_{\alpha-\e}^{\alpha+\e}}\arrow{dr}{\mu_{\alpha-\e}} & &
%     \Phi_{\alpha+\e}\\
%     %
%     & \Psi_{\alpha}\arrow{ur}{\nu_\alpha} &
%   \end{tikzcd}\end{equation}
%   \begin{equation}\label{dgm:intr3}
%   \begin{tikzcd}
%     \Phi_{\alpha-\e}\arrow{r}{\phi_{\alpha-\e}^{\beta-\e}}\arrow{dr}{\mu_{\alpha-\e}} &
%     \Phi_{\beta-\e}\arrow{dr}{\mu_{\beta-\e}} &\\
%     %
%     & \Psi_{\alpha}\arrow{r}{\psi_\alpha^\beta} &
%     \Psi_{\beta}
%   \end{tikzcd}\end{equation}
%   \end{minipage}
%   \begin{minipage}{0.5\textwidth}
%   \begin{equation}\label{dgm:intr2}
%   \begin{tikzcd}
%     & \Phi_{\alpha}\arrow{dr}{\mu_{\alpha}} &\\
%     %
%     \Psi_{\alpha-\e}\arrow{ur}{\nu_{\alpha-\e}}\arrow{rr}{\psi_{\alpha-\e}^{\alpha+\e}} & &
%     \Psi_{\alpha+\e}
%   \end{tikzcd}\end{equation}
%   \begin{equation}\label{dgm:intr4}
%   \begin{tikzcd}
%     & \Phi_{\alpha}\arrow{r}{\phi_{\alpha}^{\beta}} &
%     \Phi_{\beta}\\
%     %
%     \Psi_{\alpha-\e}\arrow{ur}{\nu_{\alpha-\e}}\arrow{r}{\psi_{\alpha-\e}^{\beta-\e}} &
%     \Psi_{\beta-\e}\arrow{ur}{\nu_{\beta-\e}}
%   \end{tikzcd}\end{equation}
%   \end{minipage}
%
%   % Because
%   % $\nu_\alpha\circ\mu_{\alpha-\e} = v_\alpha\circ (h^{-1}_{\alpha-\e}\circ n_\alpha\rest_{\Phi_\alpha})
% \end{proof}
%
\clearpage

In the following let $D\subi{a,b} := B_\e\cup f\rest_{\comp{B_a}}^{-1}(-\infty,b]$ and $P\subi{a,b} := P\cap D\subi{a,b}$.

Suppose $\b$ surrounds $D$ in $\X$ and $\Q^\of$ surrounds $\P$ in $D$ such that
\begin{itemize}
  \item $D\setminus \B\subseteq \P$,
  \item $\b\cap \P \subseteq \Q^\of\subseteq \B$, and
  \item $\B\cap \P\subseteq \QQ^\of\subseteq \BB$.
\end{itemize}

Because $\Q^\delta$ surrounds $\P$ in $D$ and $D\setminus \B\subseteq \P$ we have
\[ D\subi{\omega,\alpha} = (\P\cup (D\setminus \P))\cap D\subi{\omega,\alpha}\subseteq \P\subi{\omega,\alpha} \cup (D\subi{\omega,\alpha}\cap (D\setminus \P))\]
where $D\subi{\omega,\alpha}\cap (D\setminus \P) = (D\setminus \P)$ for all $\alpha\in\R$.
So we define the extensions
\[\ext{\P\subi{\omega-c\delta, \alpha}} := \P\subi{\omega-c\delta, \alpha}\cup (D\setminus \P)\]
and
\[\ext{\P\subi{\omega+c\delta, \alpha}} := \P\subi{\omega+c\delta, \alpha}\cup (D\setminus \P)\]
of $\P_\alpha$ for all $\alpha\in\R$
Moreover, our assumptions imply that we have the following sequence of inclusions by Lemma~\ref{lem:surround_and_cover}.
\[ \b\subseteq \ext{\Q^\of}\subseteq \B\subseteq\ext{\QQ^\of}\subseteq \BB.\]

% Because $\Q^\delta$ surrounds $\P$ in $D$ and $D\setminus \B\subseteq \P$ we have
% \[ B_{\alpha-c\of} = (\P\cup (D\setminus \P))\cap B_{\alpha -c\of}\subseteq \P\subi{\alpha} \cup (B_{\alpha - c\of}\cap (D\setminus \P))\]
% where $B_{\alpha - c\of}\cap (D\setminus \P) = (D\setminus \P)$ for $\alpha\geq \omega+c\of$.
% So we define the extension $\ext{\P_\alpha} := \P_\alpha\cup (D\setminus \P)$ of $\P_\alpha$ for $\alpha\geq\omega+c\of$.
% So our assumptions imply that we have the following sequence of inclusions by Lemma~\ref{lem:surround_and_cover}.
% \[ \b\subseteq \ext{\Q^\of}\subseteq \B\subseteq\ext{\QQ^\of}\subseteq \BB.\]
%
% Because the relative homology of a pair $(X, A)$ is not well defined for $A\not\subseteq X$ we introduce the following notation.
%
% \begin{align*}
%   \FQ_\alpha &:= \begin{cases}
%     \ext{\P_\alpha}&\text{ if } \alpha < \omega-c\of\\
%     \ext{\Q^\of}&\text{ otherwise.}
%   \end{cases}&
%   \FQ_\alpha' &:= \begin{cases}
%     \FQ_\alpha&\text{ if } \alpha < \omega+c\of\\
%     \ext{\QQ^\of}&\text{ otherwise. }
%   \end{cases}
% \end{align*}
% \begin{align*}
%   \FB_\alpha &:= \begin{cases}
%     D_\alpha&\text{ if } \alpha < \omega-2c\of\\
%     \b&\text{ otherwise.}
%   \end{cases}&
%   \FB_\alpha' &:= \begin{cases}
%     D_\alpha&\text{ if } \alpha < \omega\\
%     \B&\text{ otherwise.}
%   \end{cases}&
% % \end{align*}
% % \begin{align*}
%   \FB'' &:= \begin{cases}
%     D_\alpha&\text{ if } \alpha < \omega+2c\of\\
%     \BB&\text{ otherwise.}
%   \end{cases}
% \end{align*}
%
% Because all maps are induced by inclusion the following diagram commutes for all $\alpha$.
% % \[(\{(\ext{P_\alpha},\FQ_\alpha)\},\{(\ext{P_\alpha},\FQ_\alpha')\}),\]
% % \[(\{(D_\alpha, \FB_\alpha)\}, \{(D_\alpha,\FB_\alpha')\}),\]
% % \[(\{(D_\alpha, \FB_\alpha')\},\{(D_\alpha,\FB_\alpha'')\}),\]
% % such that the following diagram commutes for all $\alpha\in\R$.
% \begin{equation}\label{dgm:interleaving2}
% \begin{tikzcd}
%   \hom_k(D_{\alpha-c\of}, \FB_{\alpha-c\of})\arrow{r}{f_{\alpha-c\delta}}\arrow{d}{\gamma_{\alpha-c\delta}} &
%   \hom_k(\ext{P^\of_\alpha},\FQ_\alpha)\arrow{r}{m_\alpha}\arrow{d}{\lambda_\alpha} &
%   \hom_k(D_{\alpha+c\of}, \FB_{\alpha+c\of}')\arrow{d}{\pi_{\alpha+c\of}}\\
%   %
%   \hom_k(D_{\alpha-c\of}, \FB_{\alpha-c\of}')\arrow{r}{g_{\alpha-c\of}} &
%   \hom_k(\ext{P^\of_\alpha}, \FQ_\alpha')\arrow{r}{n_\alpha} &
%   \hom_k(D_{\alpha+c\of}, \FB_{\alpha+c\of}'')
% \end{tikzcd}\end{equation}


Because all maps are induced by inclusion the following diagram commutes for all $\alpha$.
% \[(\{(\ext{P_\alpha},\FQ_\alpha)\},\{(\ext{P_\alpha},\FQ_\alpha')\}),\]
% \[(\{(D_\alpha, \FB_\alpha)\}, \{(D_\alpha,\FB_\alpha')\}),\]
% \[(\{(D_\alpha, \FB_\alpha')\},\{(D_\alpha,\FB_\alpha'')\}),\]
% such that the following diagram commutes for all $\alpha\in\R$.
\begin{equation}\label{dgm:interleaving2}
\begin{tikzcd}
  \hom_k(D\subi{\omega-2c\of, \alpha-c\of}, \b)\arrow{r}{f_{\alpha-c\delta}}\arrow{d}{\gamma_{\alpha-c\delta}} &
  \hom_k(\ext{P^\of\subi{\omega-c\of,\alpha}},\ext{\Q^\of})\arrow{r}{m_\alpha}\arrow{d}{\lambda_\alpha} &
  \hom_k(D\subi{\omega,\alpha+c\of}, \B)\arrow{d}{\pi_{\alpha+c\of}}\\
  %
  \hom_k(D\subi{\omega,\alpha-c\of}, \B)\arrow{r}{g_{\alpha-c\of}} &
  \hom_k(\ext{P^\of\subi{\omega+c\of,\alpha}}, \ext{\QQ^\of})\arrow{r}{n_\alpha} &
  \hom_k(D\subi{\omega+2c\of,\alpha+c\of}, \BB)
\end{tikzcd}\end{equation}

\begin{lemma}
  If $\hom_k(\b\to\B)$ is surjective and $\hom_k(\B)\cong \hom_k(\BB)$ for all $k$ then the $k$th persistent homology modules of \[\{(\P\subi{\omega-c\of,\alpha},\Q^\of)\to(\P\subi{\omega+c\of,\alpha},\QQ^\of)\}\] and $\{(D\subi{\omega,\alpha}, \B)\}$ are $c\of$-interleaved for all $k$.
\end{lemma}
\begin{proof}
  Let $\U, \V, \W$ be the $k$th persistent homology modules of $\{(D\subi{\omega-2c\of, \alpha}, \b)\}$, $\{(D\subi{\omega,\alpha}, \B)\}$, $\{(D\subi{\omega+2c\of,\alpha}, \BB)\}$, and $\S, \T$ the $k$th persistent homology modules of $\{(\ext{\P\subi{\omega-c\of,\alpha}},\ext{\Q^\of})\}$ and $\{(\ext{\P\subi{\omega+c\of,\alpha}},\ext{\QQ^\of})\}$ respectively.
  Let $\Gamma\in\Hom(\U,\V)$, $\Lambda\in\Hom(\S, \T)$, and $\Pi\in\Hom(\V,\W)$ be homomorphisms induced by inclusion.
  Because Diagram~\ref{dgm:interleaving2} commutes for all $\alpha$ we have the following homomorphisms of degree $c\delta$ induced by inclusion.
  \begin{align*}
    F\in\Hom^{c\delta}(\U,\S)& & M\in\Hom^{c\delta}(\S,\V)\\
    G\in\Hom^{c\delta}(\V,\T)& & N\in\Hom^{c\delta}(\T,\W)
  \end{align*}
  Because all maps are induced by inclusion we have partial $c\delta$-interleavings $\Phi(F, G; M) : \im~\Gamma\to \im~\Lambda$ and $\Psi(M, N; G) : \im~\Lambda\to\im~\Pi$.

  By applying Lemma~\ref{lem:five} to the long exact sequences of the pairs $(D\subi{\omega-2c\of,\alpha},\b)$ and $(D\subi{\omega,\alpha},\B)$ our assumption that $\hom_k(\b\to\B)$ is surjective for all $k$ implies $\gamma_\alpha : \hom_k(D\subi{\omega-2c\of,\alpha},\b)\to \hom_k(D\subi{\omega,\alpha},\B)$ is surjective for all $\alpha\in\R$.
  So $\Gamma : \U\to \V$ is an epimorphism.
  Similarly, the assumption that $\hom_k(\B)\cong \hom_k(\BB)$ implies $\pi_\alpha : \hom_k(D\subi{\omega,\alpha},\B)\to \hom_k(D\subi{\omega+2c\of,\alpha},\BB)$ is an isomorphism by applying Lemma~\ref{lem:five} to the long exact sequences of the pairs $(D\subi{\omega,\alpha},\B)$ and $(D\subi{\omega+2c\of,\alpha},\BB)$.
  So $\Pi : \V\to\W$ is an isomorphism of persistence modules (and therefore a monomorphism).
  The result follows from Theorem~\ref{thm:interleaving_main}.

  % As we have shown
  % As $(\{(\ext{P_\alpha},\FQ_\alpha)\},\{(\ext{P_\alpha},\FQ_\alpha')\})$, $(\{(D_\alpha, \FB_\alpha)\}, \{(D_\alpha,\FB_\alpha')\})$, and $(\{(D_\alpha, \FB_\alpha')\},\{(D_\alpha,\FB_\alpha'')\})$ are compatible pairs of filtrations such that Digram~\ref{dgm:interleaving} commutes with $f_\alpha$ surjective and $h_\alpha$ injective for all $\alpha\in\R$ the result follows from Theorem~\ref{thm:interleaving}.


\end{proof}
