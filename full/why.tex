\clearpage
\subsection{Why $\D\setminus\B^{2\delta}$ (TODO)?}

% In order to verify coverage we therefore need an element in $\hom_2(P^\delta, Q^\delta)$ that \emph{persists} until $\hom_2(P^\gamma, Q^\gamma)$.
%
% An element in $\hom_1(P^\delta\hookrightarrow P^\gamma)$ that does not correspond to an element of $\hom_1(\B)$ surely indicates a gap in coverage.
% However,
%
% Any element of $\hom_1(P^\delta)$ that does not correspond to an element of $\hom_1(\D)$ indicates a gap in coverage of the \emph{whole} domain $\D$.
% These are the elements which ``kill'' the void we expect in $\hom_2(P^\delta, Q^\delta)$.
%
% Any element of $\hom_1(Q^\delta)$ that is trivial in $\hom_1(P^\delta)$ must be the boundary of an element of $\hom_2(P^\delta, Q^\delta)$ by exactness.
%
% Because $\B$ is the topological boundary of $\D$ any element of $\hom_1(\B)$ that is trivial in $\hom_1(\D)$ must be the boundary of an element


% Suppose there exists a non-trivial element $[x]$ in $\im~\hom_2((P^\delta, Q^\delta)\hookrightarrow (P^\gamma, Q^\gamma))$.
% Clearly $[x]\notin\hom_2(P^\delta)$ and $[x]\notin\hom_2(P^\gamma)$ so there must be chains in $C_2(P^\delta)$ and $C_2(P^\gamma)$ with boundaries in $C_1(Q^\delta)$ and $C_1(Q^\gamma)$.
% If these chains correspond to those in $C_2(\D)$ and $C_1(\B)$ we have a good representation of our domain.
% However, boundaries

Suppose there is a non-trivial element $[x]$ in $\hom_2(P^\delta, Q^\delta)$.
Because we are working in $\R^2$ we know that $\hom_2(P^\delta, Q^\delta) = Z_2(P^\delta, Q^\delta)$.
So $[x]$ maps to a boundary in $C_1(P^\delta, Q^\delta) = C_1(P^\delta) / C_1(Q^\delta)$.
Clearly $[x]$ does not map to a cycle in $Z_2(Q^\delta)$ as it is non-zero.

\textbf{TODO} It feels like we cannot have any additional cycles in $\hom_1(P^\delta)$ that are not in $\hom_1(Q^\delta)$.
The gaps in coverage corresponding to cycles in $\hom_1(Q^\delta)$ must be in $\B^\delta$.
Does this mean that $\D\setminus\B^\delta$ is covered?

% so there must exist a chain in $C_2(P^\delta)$ with its boundary in $C_1(Q^\delta)$.
% If this chain corresponds to the maximal chain in $C_2(\D)$ with boundary equal to the topological boundary, an element of $C_1(\B)$, we have no issue.
% However, consider a cycle in $C_1(P^\delta)$ that corresponds to a gap in coverage

% The elements of $\im~\hom_2((P^\delta, Q^\delta)\hookrightarrow (P^\gamma, Q^\gamma))$ are those in $\hom_2(P^\delta, Q^\delta)$ which \emph{persist} until $\hom_2(P^\gamma, Q^\gamma)$.
% Any additional elements in $\hom_2(P^\gamma, Q^\gamma)$ that are not in $\hom_2(P^\delta, Q^\delta)$ will not be found in the inclusion.


% As we are interested in the coordinate-free setting we cannot compute the homology of the pair of \emph{offsets} $(P^\delta, Q^\delta)$ directly.
%
%
% Two fundamental issues arise in this process.
% First, we cannot compute the homology of the pair of \emph{offsets} $(P^\delta, Q^\delta)$ directly.
% Given precise coordinates of the sensors we could use the pair of \v Cech complexes $\cech_\delta(P, Q) = (\cech_\delta(P), \cech_\delta(Q))$ which is homotopy equivalent by the nerve theorem.
% However, we are interested in the coordinate-free case in which we have only pairwise proximity data between sensors.
% Second,
%
% Therefore, we must use the interleaving of the Rips and \v Cech to
%
%
% However, we are
% Even by using the nerve lemma to replace the pair of offsets with the pair of \v Cech
% Recall that the nerve lemma states that the
% Although we can compute the homology of the pair of \v Cech complexes, denoted , with the precise coordinates of our sensors we
% % % Given such a pair $(P, Q)$ let $\rips_\delta(P, Q)$ denote the pair of complexes $(\rips_\delta(P), \rips_\delta(Q))$ throughout.
% % Our goal now is to check if the relative homology $\hom_2(P^\delta, Q^\delta)$ of the pair $(P^\delta, Q^\delta)$ reflects that of our domain $(\D, \B)$.
% % A gap in coverage can be thought of as
%
% % Now suppose our network $P$ covers the domain at some scale $\delta > 0$ such that there are no gaps (1-cycles) in $\rips_\delta(P)$.
% % The subset $Q = \{p\in P\mid \ball_\delta(p)\cap\B\neq\emptyset\}$ of points within distance $\delta$ of $\B$ induces a subcomplex $\rips_\delta(Q)$ of $\rips_\delta(P)$.
% % Given such a pair $(P, Q)$ let $\rips_\delta(P, Q)$ denote the pair of complexes $(\rips_\delta(P), \rips_\delta(Q))$ throughout.
% % Under our assumptions, the relative homology $H_2(\rips_\delta(P, Q))$ should reflect that of the domain.
% % A gap in coverage can be thought of as ``popping the balloon'' in the sense that, if we wrap the simplices of $\rips_\delta(P)$ around those in $\rips_\delta(Q)$ we would have no void---the gap provides a hole through which the ``air'' can escape, as illustrated in Fig.~\ref{fig:balloons2}.
