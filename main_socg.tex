\documentclass[a4paper,UKenglish,cleveref, autoref, thm-restate]{lipics/socg-lipics-v2019}
%This is a template for producing LIPIcs articles.
%See lipics-manual.pdf for further information.
%for A4 paper format use option "a4paper", for US-letter use option "letterpaper"
%for british hyphenation rules use option "UKenglish", for american hyphenation rules use option "USenglish"
%for section-numbered lemmas etc., use "numberwithinsect"
%for enabling cleveref support, use "cleveref"
%for enabling autoref support, use "autoref"
%for anonymousing the authors (e.g. for double-blind review), add "anonymous"
%for enabling thm-restate support, use "thm-restate"

\usepackage{amsmath,amssymb,amsthm,xspace,enumitem,xcolor,tikz-cd,xspace}

% % !TeX root = main.tex

% \newtheorem{theorem}{Theorem}
% \newtheorem{lemma}{Lemma}
% \newtheorem{corollary}{Corollary}
% \newtheorem{definition}{Definition}


\newcommand{\R}{\mathbb{R}}
\renewcommand{\S}{\mathbb{S}}
\newcommand{\T}{\mathbb{T}}
\newcommand{\W}{\mathbb{W}}
\newcommand{\X}{\mathbb{X}}
\newcommand{\Y}{\mathbb{Y}}
\newcommand{\Z}{\mathbb{Z}}

\renewcommand{\AA}{\mathbb{A}}
\newcommand{\BB}{\mathbb{B}}
\newcommand{\FF}{\mathbb{F}}
\newcommand{\LL}{\mathbb{L}}
\newcommand{\UU}{\mathbb{U}}
\newcommand{\VV}{\mathbb{V}}

\newcommand{\A}{\mathcal{A}}
\newcommand{\E}{\mathcal{E}}
\newcommand{\F}{\mathcal{F}}
\newcommand{\I}{{\mathcal{I}}}
\newcommand{\J}{\mathcal{J}}
\newcommand{\N}{\mathcal{N}}
\newcommand{\U}{\mathcal{U}}
\newcommand{\V}{\mathcal{V}}

\newcommand{\e}{\varepsilon}
\newcommand{\im}{\mathbf{im}\xspace}
\renewcommand{\ker}{\mathbf{ker}\xspace}
\newcommand{\rk}{\mathbf{rk\xspace}}
\renewcommand{\dim}{\mathbf{dim}\xspace}
\newcommand{\rest}{\mathord{\mid}}
\newcommand{\id}{\mathbf{1}}

\newcommand{\cech}{\check{\mathcal{C}}}
\newcommand{\rips}{\mathcal{R}}
\newcommand{\ball}{\mathbf{ball}}
\newcommand{\dist}{\mathbf{d}}

\newcommand{\cl}{\mathbf{cl\xspace}}
\newcommand{\intr}{\mathbf{int\xspace}}

\newcommand{\Hom}{\mathrm{Hom}}
\renewcommand{\hom}{\mathrm{H}}

\newcommand{\subi}[2]{_{\scriptscriptstyle #2\mid #1}}

\newcommand{\D}[2]{\mathcal{D}\subi{#1}{#2}}
\newcommand{\DD}[1]{\mathbb{D}_{#1}}

\renewcommand{\P}[3]{\mathcal{P}\subi{#1}{#3}^{#2}}
\newcommand{\CP}[3]{\cech\P{#1}{#2}{#3}}
\newcommand{\RP}[3]{\rips\P{#1}{#2}{#3}}

\newcommand{\PP}[2]{\mathbb{P}_{#1}^{#2}}
\newcommand{\CPP}[2]{\cech\PP{#1}{#2}}
\newcommand{\RPP}[2]{\rips\PP{#1}{#2}}

\newcommand{\ext}[1]{\E\xspace #1}


\newcommand{\R}{\mathbb{R}}
\renewcommand{\S}{\mathbb{S}}
\newcommand{\T}{\mathbb{T}}
\newcommand{\W}{\mathbb{W}}
\newcommand{\X}{\mathbb{X}}
\newcommand{\Y}{\mathbb{Y}}
\newcommand{\Z}{\mathbb{Z}}

\renewcommand{\AA}{\mathbb{A}}
\newcommand{\BB}{\mathbb{B}}
\newcommand{\FF}{\mathbb{F}}
\newcommand{\LL}{\mathbb{L}}
\newcommand{\UU}{\mathbb{U}}
\newcommand{\VV}{\mathbb{V}}

\newcommand{\A}{\mathcal{A}}
\newcommand{\E}{\mathcal{E}}
\newcommand{\F}{\mathcal{F}}
\newcommand{\I}{{\mathcal{I}}}
\newcommand{\J}{\mathcal{J}}
\newcommand{\N}{\mathcal{N}}
\newcommand{\U}{\mathcal{U}}
\newcommand{\V}{\mathcal{V}}

\newcommand{\e}{\varepsilon}
\newcommand{\im}{\mathbf{im}\xspace}
\renewcommand{\ker}{\mathbf{ker}\xspace}
\newcommand{\rk}{\mathbf{rk\xspace}}
\renewcommand{\dim}{\mathbf{dim}\xspace}
\newcommand{\rest}{\mathord{\mid}}

\newcommand{\cech}{\check{\mathcal{C}}}
\newcommand{\rips}{\mathcal{R}}
\newcommand{\ball}{\mathbf{ball}}
\newcommand{\dist}{\mathbf{d}}

\newcommand{\cl}{\mathbf{cl\xspace}}
\newcommand{\intr}{\mathbf{int\xspace}}

\newcommand{\Hom}{\mathrm{Hom}}
\renewcommand{\hom}{\mathrm{H}}

\newcommand{\subi}[2]{_{\scriptscriptstyle #2\mid #1}}

\newcommand{\D}[2]{\mathcal{D}\subi{#1}{#2}}
\newcommand{\DD}[1]{\mathbb{D}_{#1}}

\renewcommand{\P}[3]{\mathcal{P}\subi{#1}{#3}^{#2}}
\newcommand{\CP}[3]{\cech\P{#1}{#2}{#3}}
\newcommand{\RP}[3]{\rips\P{#1}{#2}{#3}}

\newcommand{\PP}[2]{\mathbb{P}_{#1}^{#2}}
\newcommand{\CPP}[2]{\cech\PP{#1}{#2}}
\newcommand{\RPP}[2]{\rips\PP{#1}{#2}}



% % \newcommand{\ext}[1]{\widehat{#1}}
\newcommand{\ext}[1]{\E\xspace #1}
% \renewcommand{\I}{\mathcal{I}}
% \newcommand{\J}{\mathcal{J}}
%
%
% \newcommand{\cU}{\mathcal{U}}
% \newcommand{\cV}{\mathcal{V}}
% \newcommand{\cF}{\mathcal{V}}
% \newcommand{\A}{\mathcal{A}}






\newcommand{\fullversion}{Appendix} %{full version of this paper}

\bibliographystyle{plainurl}

\title{From Coverage Testing to Topological Scalar Field Analysis}

\author{Kirk P. Gardner}{North Carolina State University, United States}{kpgardn2@ncsu.edu}{https://orcid.org/0000-0001-5306-2174}{}
\author{Donald R. Sheehy}{North Carolina State University, United States}{don.r.sheehy@gmail.com}{https://orcid.org/0000-0002-9177-2713}{}

\authorrunning{K.\,P. Gardner and D.\,R. Sheehy}

\Copyright{Kirk P. Gardner and Donald R. Sheehy} %TODO mandatory, please use full first names. LIPIcs license is "CC-BY";  http://creativecommons.org/licenses/by/3.0/


\begin{CCSXML}
<ccs2012>
<concept>
<concept_id>10002950.10003741.10003742.10003744</concept_id>
<concept_desc>Mathematics of computing~Algebraic topology</concept_desc>
<concept_significance>500</concept_significance>
</concept>
<concept>
<concept_id>10003752.10010061.10010063</concept_id>
<concept_desc>Theory of computation~Computational geometry</concept_desc>
<concept_significance>300</concept_significance>
</concept>
</ccs2012>
\end{CCSXML}

\ccsdesc[500]{Mathematics of computing~Algebraic topology}
\ccsdesc[300]{Theory of computation~Computational geometry}
% \ccsdesc[500]{Theory of computation~Computational geometry}
% % \ccsdesc[500]{Mathematics of computing~Algebraic topology}


\keywords{topology, homology, coverage, scalar fields, persistent homology, relative homology}

% \category{} %optional, e.g. invited paper

% \relatedversion{} %optional, e.g. full version hosted on arXiv, HAL, or other respository/website
%\relatedversion{A full version of the paper is available at \url{...}.}

% \supplement{}%optional, e.g. related research data, source code, ... hosted on a repository like zenodo, figshare, GitHub, ...

\funding{This research was supported by the NSF under grants CCF-1525978 and CCF-1652218.}%optional, to capture a funding statement, which applies to all authors. Please enter author specific funding statements as fifth argument of the \author macro.

% \acknowledgements{I want to thank \dots}%optional

%\nolinenumbers %uncomment to disable line numbering

%\hideLIPIcs  %uncomment to remove references to LIPIcs series (logo, DOI, ...), e.g. when preparing a pre-final version to be uploaded to arXiv or another public repository

%Editor-only macros:: begin (do not touch as author)%%%%%%%%%%%%%%%%%%%%%%%%%%%%%%%%%%
\EventEditors{John Q. Open and Joan R. Access}
\EventNoEds{2}
\EventLongTitle{42nd Conference on Very Important Topics (CVIT 2016)}
\EventShortTitle{CVIT 2016}
\EventAcronym{CVIT}
\EventYear{2016}
\EventDate{December 24--27, 2016}
\EventLocation{Little Whinging, United Kingdom}
\EventLogo{}
\SeriesVolume{42}
\ArticleNo{23}
%%%%%%%%%%%%%%%%%%%%%%%%%%%%%%%%%%%%%%%%%%%%%%%%%%%%%%

\begin{document}

\maketitle

% !TeX root = ../main.tex

\begin{abstract}
  The topological coverage criterion (TCC) can be used to test whether an underlying space is sufficiently well covered by a given data set.
  Given a sufficiently dense sample, topological scalar field analysis (SFA) can give a summary of the shape of a real-valued function on a space.
  The goal of this paper is to put these theories together so that one can test coverage with the TCC and then compute a summary with SFA.
  The challenge is that the TCC requires a well-defined boundary that is not generally available in the SFA settings.
  To overcome this, we show how the scalar field itself can be used to define a boundary that can then be used in the TCC.
  This requires a generalization of the TCC proof and resolves one of the major barriers to wider use of the TCC.
  % It also extends SFA methods to a wider class of spaces.
  It also extends SFA methods to the setting in which coverage is only confirmed in a subset of a space surrounded by a sub-levelset.
  We show how the intersection of these two theories can be used to approximate the persistent homology of the scalar field with respect to a static sub-levelset.
  We then discuss how this persistent relative homology relates to that of the scalar field as a whole.
\end{abstract}


\section{Introduction}\label{sec:introduction}
% !TeX root = ../main.tex

In the topological analysis of scalar fields (SFA), one computes a topological summary capturing qualitative and quantitative shape information from a set of points endowed with a metric and a real-valued function.
That is, we have points with distances and a real number assigned to each point.
More generally, it usually suffices to have a neighborhood graph on the points identifying the pairs of points that close.
The topological computation often takes the form of persistent homology and integrates the local information from the function into global information about the behavior of the function on the entire space.
In prior work, Chazal et al.~\cite{chazal09analysis} showed that for sufficiently dense samples on sufficiently smooth spaces, the persistence diagram can be computed with some guarantees.
In followup work, Buchet et al.~\cite{buchet15topological} extended this result to show how to work with noisy inputs.
A fundamental assumption required to have strong guarantees on the output of these methods is that the underlying space be sufficiently well-sampled.
In this paper, we show how to combine scalar fields analysis with the theory of topological coverage testing to simultaneously compute the persistence diagram and also to test that the underlying space is sufficiently well-sampled.

Initiated by De Silva and Ghrist~\cite{desilva06coordinate,desilva07coverage,desilva07homological}, the theory of homological sensor networks addresses the problem of testing coverage of a bounded domain by a collection of sensors without coordinates.
The main result is the topological coverage criterion, which, in its most general form, states that under reasonable geometric assumptions, the $d$-dimensional homology of a pair of simplicial complexes built on the neighborhood graph will be nontrivial if and only if there is sufficient coverage (see Section~\ref{sec:tcc} for the precise statements).
This relative persistent homology test is called the Topological Coverage Criterion (TCC).

Superficially, the methods of SFA and TCC are very similar.
Both construct similar complexes and compute the persistent homology of the homological image of a complex on one scale into that of a larger scale.
They even overlap on some common techniques in their analysis including the use of the Nerve theorem and the Rips-\v{C}ech interleaving.
However, they differ in some fundamental way that makes it difficult to combine them into a single technique.
The main difference is that the TCC requires a clearly defined boundary.
Not only must the underlying space be a bounded subset of $\R^d$, the data must also be labeled to indicate which input points are close to the boundary.
This requirement is perhaps the main reason why the TCC can so rarely be applied in practice.
Cavanna et al.~\cite{cavanna2017when} generalized the TCC to allow for more general spaces and robust coverage guarantees.
That work gave a different approach to proving the correctness of the TCC which allows much more freedom in how the boundary is defined.


% Moreover, as a necessary but not sufficient condition for coverage there is room to question what can go wrong in the case of false positives.
% In fact, the efficacy of the condition relies on having enough sensors close enough to approximate the boundary in homology.
% This leads us to believe the condition checks for something more specific than coverage alone.
% Specifically, that we have a sample as well as a subsample that reflect the topology of the space and its boundary as a pair.

% !TeX root = ../main.tex

% % Homology provides a collection of invariants that represent global properties of a space.
% % Under specific assumptions the presence of a property can be assumed when the space itself is largely unknown.
% % Then, given a sample of an unknown space satisfying these assumptions, once can use the homology of the sample in order to confirm it is representative of the space with respect to that property.
% %
% % The Topological Coverage Criterion (TCC) is one example of this technique in which homology can be used in order to verify that a collection of subsets provided by a sample covers the space.
% % Specifically, by assuming the top dimensional relative homology is known one can check that of the sample is what we expect in order to certify coverage by relating the top dimensional relative to the boundary of a space to the zero dimensional homology of the complement.
%
% % The assumptions made about the boundary are central to the TCC.
% % While it certainly demonstrates an interesting application of homology to coordinate-free sensor networks these assumptions are unrealistic in practice.
% % Specifically, it requires that sensors are capable of detecting the physical presence of a boundary.
% % In a coordinate-free setting, where sensors are unable to measure distance nor communicate their coordinates, this requirement seems unnatural.
%
% The requirement that ``sensors'' can detect the presence of the boundary is unnatural in more general applications to data analysis.
% It is more natural assume that our ``sensors'' measure some unknown quantity---points sampling a scalar field.
% By requiring that our function is related to the metric of the space in a specific way (Lipschitz) we can replace this requirement with assumptions about the function itself.
% Indeed, these assumptions could relate the behavior of the function to the topological boundary of the space.
% However, the condition holds for any subset which ``surrounds'' the space in a specific sense~\cite{cavanna2017when}.
%
% % That is, it allows for false negatives, but not false positives.
% % We identify
% %
% % we can confirm coverage of the
% % % We take this property of relative homology, which can be understood in terms of excision, to its logical extreme.
%
%
% % We replace this requirement with the ability to measure some unknown quantity that .
% % This is done primarily by replacing the condition that sensors can detect the boundary
% % The original statement of the TCC allowed for false negatives as a condition for coverage.
% % % Consider the application to sensor networks as an example, where we take our sample points as sensors dropped in an unknown environment.
% % % We now endow our sensors with the ability to measure some unknown quantity that is related to metric of the space in a specific way (Lipschitz).
% % In this way we can replace the requirement that sensors can detect the presence of the true topological boundary
%
% % The TCC uses these assumptions to make the \emph{relative} homology known.
% % Specifically, we can confirm coverage of the ``interior'' of the space by effectively quarantining the surrounding space we take homology with respect to.
% % We take this property of relative homology, which can be understood in terms of excision, to its logical extreme.
%
% % The TCC uses relative homology to confirm coverage of the ``interior'' of the space by effectively quarantining the surrounding space we take homology with respect to.
% % This property can be understood in terms of excision, and is useful
% % In particular, by requiring that the subset of points close to the boundary resemble the boundary, the TCC does more than just confirm coverage.
% % We take this property of relative homology, which can be understood in terms of excision, to its logical extreme.
% % We consider the case in which we have incomplete data for a particular sublevel set an unknown function.
%
% % Returning to our sensor network analogy, suppose our sensors measure heat, or some other volatile quantity, and cannot approach a region without failing.
%
% We consider the case in which we have incomplete data for a particular sublevel set an unknown function.
% From this perspective, the TCC verifies that we not only have coverage, but that the sample we have is topologically representative of the region near, and above this sublevel set.
% We can then re-use the same machinery that was used to verify coverage to analyze a \emph{part} of the function in a very specific way.
%
% % also that our sample includes a subset of points that resembles the boundary.
% % Specifically, we can confirm coverage of the ``interior'' of the space by effectively quarantining the surrounding space we take homology with respect to.
% %
% % By properly isolating the region we can confirm that the sample we have is topologically representative of the region near, and above this sublevel set.
% % % Specifically, we can extend the statement of the TCC as a condition for coverage to a condition that verifies our sample
% %
% % % In a superficial sense, the assumptions and machinery required to approximate a function's persistent homology are precisely those confirmed and constructed in the TCC. %\footnote{\textbf{TODON} worst sentence ever, but exactly what I want to say. Also I stole ``superficial'' from your intro because it's perfect. Maybe some of your intro can fill this in.}.
%
% While the approximation of a function's persistent homology is well studied in general~\cite{chazal08towards}, the presence of incomplete data can severely impact the quality of the approximation.
% This is primarily due to the nature of homology as a measure of \emph{global} structure.
% While the obvious solution is to simply remove the un-verified data, the question of what precisely this would approximate is important to consider.
% % By simply restricting the function to the verified \emph{super}-levelset we accept that there may be global structure to the function itself that we are missing.
% % If we are then provided with the missing data it may prove more difficult to reconstruct the full diagram in this case.

% We consider the case in which we have incomplete data from a particular sublevel set of our function.
% We can replace the boundary with this sublevel set by requiring that it ``surrounds'' the space, with data points with function values close to serving as samples of our boundary\textbf{FUCK}
%
%
% While the approximation of a function's persistent homology is well studied in general~\cite{chazal08towards}, the presence of incomplete data can severely impact the quality of the approximation.
% This is primarily due to the nature of homology as a measure of \emph{global} structure.
% While the obvious solution is to simply remove the un-verified data, the question of what precisely this would approximate is important to consider.


\paragraph*{Contribution}

We will re-cast the TCC as a way to verify that the persistent homology of a scalar field can be \emph{partially} approximated by a given sample. % of a scalar field can adequately approximate part of the persistent homology collection of sample points can adequately approximate \emph{part} of the persistent homology of a scalar field.}\footnote{\textbf{TODO} fixit.}
Specifically, we will relate the persistent homology of a function relative to a \emph{static} sublevel set to a \emph{truncation} of the full diagram.
That is, beyond a certain point the full diagram remains unchanged, allowing for possible reconstruction.
% While the approximation of a function's persistent homology is well studied in general~\cite{chazal08towards}, the presence of incomplete data can severely impact the quality of the approximation.
% That is, due to the nature of homology as a measure of global structure, the \emph{restricted} diagram resulting from removing un-verified data fills in missing global structure with potentially spurious features.
This is in comparison with the \emph{restricted} diagram obtained by simply ignoring part of the domain. % removing the un-verified data.
% Due to the nature of homology as a measure of global structure that the restricted diagram may attempt to fill in, resulting in spurious features.
We therefore present relative persistent homology as an alternative to restriction in a way that extends the TCC to the analysis of scalar fields.
 % the presence of incomplete data can severely impact the quality of an approximation.
% This is primarily due to the nature of homology as a measure of global structure.
% That is, beyond a certain point the full diagram remains unchanged, allowing for possible reconstruction.
% While the obvious solution is to simply remove the un-verified data, the resulting \emph{restricted} diagram fills in missing global structure with potentially spurious features.
% Indeed, it can be shown that the truncated diagram is captured by the restriction in a specific way~\cite{cohen09extending, desilva11duality}.
% We therefore provide experimental evidence that compares the approximation of the restricted function directly provides a worse interleaving with the corresponding subset of the \emph{full} diagram.

% We present relative persistent homology as an alternative to this restricted diagram in a way that extends the TCC to the analysis of scalar fields.
% We will first provide some background on important topological, geometric, and algebraic structures required for our re-formulation of the TCC, and the approximation of the relative diagram.
Section~\ref{sec:summary} establishes notation and provides an overview of our main results in Sections~\ref{sec:tcc} and~\ref{sec:middle}.
In Section~\ref{sec:truncations} we introduce an interpretation of the relative diagram as a truncation of the full diagram that is motivated by a number of experiments in Section~\ref{sec:experiments}.% in terms of the sublevel set filtration as a \emph{truncation}.
% Finally, Section~\ref{sec:experiments} motivates our approach through a number of experiments.

% % We then introduce the notion of a \emph{surrounding set}, or pair, which will be central in or re-formulation of the TCC.
% % Section~\ref{sec:middle} establishes the algebraic tools that will be used to approximate the truncated diagram, and discuss how they fit in the context of the TCC.
% % After providing the proof of the interleaving in Section~\ref{sec:interleaving}
% We will then show how this approximation is related to the full diagram and state our main Theorem in Section~\ref{sec:truncations}. %of our function introduce an interpretation of this relative diagram of the full diagram that will be used in the statement of our main Theorem in
% Finally, Section~\ref{sec:experiments} details a number of examples that motivate our approach.%motivated by a number of experiments which demonstrate the relationship between the relative, restricted, and truncated diagrams in Section~\ref{sec:experiments}.
% % Finally, in Section~\ref{sec:experiments} we compare

\begin{figure}[htbp]
  \centering
  % \includegraphics[trim=50 190 0 200, clip, scale=0.2]{scripts/figures/scalar.png}
  % \includegraphics[trim=100 25 75 0, clip, angle=280, scale=0.25]{scripts/figures/scalar_contour.png}
  \includegraphics[trim=200 200 200 200, clip, width=0.5\textwidth]{scripts/figures/surf/side.png}
  \includegraphics[trim=200 0 200 200, clip, width=0.3\textwidth]{scripts/figures/surf/top.png}
  \includegraphics[scale=0.75]{scripts/figures/scalar_barcode_true.png}
  % \includegraphics[trim=0 310 270 0, clip, scale=0.8]{scripts/figures/scalar_restricted.png}
  % \includegraphics[scale=0.55]{scripts/figures/scalar_barcode_super_0.png}
  % \includegraphics[scale=0.55]{scripts/figures/scalar_barcode_sub_1.png}
\end{figure}


\section{Summary}\label{sec:summary}
% !TeX root = ../main_socg.tex

Let $\X$ denote an orientable $d$-manifold and $D\subset\X$ a compact subspace.
For a $c$-Lipschitz functon $f : D\to \R$ and $\alpha\in\R$ let $B_\alpha := f^{-1}((-\infty,\alpha])$ denote the $\alpha$-sublevel set of $f$.
Our sample will be denoted $P$, and the subset of points sampling $B_\alpha$ will be denoted $Q_\alpha := P\cap B_\alpha$.
For $\e > 0$ let $P^\e$ denote the union of open metric balls centered at points in $P$.
For ease of exposition let
\[ D\subi{z}{\alpha} := B_\alpha\cup B_z \]
denote the \emph{$z$-truncated} sublevel sets of $f$ and % of the restricted function $f\rest_{D\setminus B_w}$ for all $\alpha,w\in\R$.
\[ P\subi{z}{\alpha} := Q_\alpha\cup Q_z\]
for all $z,\alpha\in\R$.\footnote{\textbf{I'm starting to think:} \emph{For ease of exposition let pairs $(D_\alpha, B_z)$ denote $(B_{\scriptscriptstyle\max\{\alpha,z\}},B_z)$ so that $B_z\subseteq D_\alpha$ for all $\alpha\in\R$. Outside of a pair, we will refer to $D_\alpha$ as \_. Similarly, let $(P_\alpha^\e, Q_z^\e)$ denote $(Q_{\scriptscriptstyle\max\{\alpha,z\}}, Q_z)$.}}
\footnote{\textbf{Options:}
  $(P_{\scriptscriptstyle \lfloor\alpha\rfloor_{z+c\e}}^\e, Q_{z+c\e}^\e)$,
  $(P_{\scriptscriptstyle\alpha\mid z+c\e}^\e, Q_{z+c\e}^\e)$,
  $(P_{\scriptscriptstyle\max\{\alpha, z+c\e\}}^\e, Q_{z+c\e}^\e)$,
  $(P_{\alpha {\scriptscriptstyle  > z+c\e}}^\e, Q_{z+c\e}^\e)$,
  $(P_{\scriptscriptstyle \alpha; z+c\e}^\e, Q_{z+c\e}^\e)$,
  $(P_{\scriptscriptstyle z+c\e, \alpha}^\e, Q_{z+c\e}^\e)$,
  $((Q_\alpha\cup Q_{z+c\e})^\e, Q_{z+c\e}^\e)$;
  Throughout, let $(P_\alpha^\e, Q_{z+c\e}^\e)$ denote the pair $(Q_{\max\{\alpha, z+c\e\}}^\e, Q_{z+c\e}^\e)$ so that $Q_{z+c\e}^\e \subseteq P_\alpha^\e$ for all $\alpha\in\R$;
  Define filtrations for $\alpha\geq z+c\e$ and handle all of the edge cases by hand (there are a lot and it's gross).
}

We will select a sublevel set $B_\omega$ of $f$ that \emph{surrounds} $D$ to serve as our boundary.
Given a sample of $f$ at a finite number of points $P$ in $D$ we would like to confirm $P^\delta$ not only covers the interior $D\setminus B_\omega$, but also that $Q^\delta$ surrounds $P^\delta$ for some $Q\subset P$.
That is, we would like to verify that a pair $(P^\delta, Q^\delta)$ is %homologically
representative of the pair $(D,B_\omega)$ in homology.
Our goal is to use this fact to approximate the persistence of $f$ relative to $B_\omega$.

\paragraph*{Results}

Our approximation will be by a nested pair of (Vietoris-)Rips complexes, denoted $\rips^\e(P, Q) = (\rips^\e(P), \rips^\e(Q))$ for $\e > 0$.
Under mild regularity assumptions it can be shown that
\[ \rk~\hom_d(\rips^\delta(P, Q_{\omega - 2c\delta})\hookrightarrow \rips^{2\delta}(P, Q_{\omega+c\delta}))\geq \dim~\hom_0(\rips^\delta(P\setminus Q_{\omega-2c\delta}))\]
implies $D\setminus B_\omega\subseteq P^\delta$ and $Q_{\omega-2c\delta}^\delta$ surrounds $P^\delta$ in $D$.
Proof of this fact generalizing the proof of the TCC to boundaries defined in terms of a function $f$, eliminating unnatural assumptions made in previous work.
Not only are our subsamples $Q_{\omega-2c\delta}$ and $Q_{\omega+c\delta}$ defined in terms of their function values, but our regularity assumptions can be stated directly in terms of the persistent homology of $f$.

% Given a sample $P$ such that $D\setminus B_\omega\subseteq P^\delta$ and $Q_{\omega-2c\delta}^\delta$ surrounds $P^\delta$ in $D$ we can approximate the persistent homology of $f$ in a specific way.
Given a sample $P$ that satisfies the TCC we can approximate the persistent homology of $f$ in a specific way.
The nested pair of Rips complexes used to confirm coverage can be extended to a filtration
\[ \{\rips^{2\delta}(P\subi{\omega-2c\delta}{\alpha}, Q_{\omega-2c\delta})\hookrightarrow \rips^{4\delta}(P\subi{\omega+c\delta}{\alpha}, Q_{\omega+c\delta})\}_{\alpha\in\R}\]
that can be used to approximate the persistent homology of $\{(D\subi{\omega}{\alpha}, B_\omega)\}_{\alpha\in\R}$.
Indeed, we could use existing methods to approximate the persistent homology of $f$ \emph{restricted} to the subspace $D\setminus B_\omega$ that we cover.
However, the question of what this would approximate is important to consider.
{\color{red} Restricting the domain of the function can not only introduce noise close to the boundary, \textbf{but also perturb global structure in our signature.}\footnote{\color{red} close.}}
As an alternative, we approximate the persistence of $f$ \emph{relative} to the sublevel set $B_\omega$.
This is not only to eliminate noise introduced by the restriction, but also to \emph{truncate} the persistence of $f$ in a way that isolates global structure.

% By restricting the domain important global structure of the function near $\omega$ can appear as noise
%
% We would like to confirm that a sample $P$ not only covers the interior $D\setminus B_\omega$ at some scale $\delta$, but that there is a subset $Q$ of $P$ that serves as a sampled boundary.
% That is, we would like to confirm that a pair $(P^\delta, Q^\delta)$ is a good approximation of $(D, B_\omega)$ topologically.
% We can then use this fact to approximate the persistent homology of the relative filtration $\{(D\subi{\omega}{\alpha}, B_\omega)\}_{\alpha\in\R}$.
%
% This will be done using an approximation by nested pairs of \emph{(Vietoris-)Rips complexes}, denoted $\rips^\e(P, Q) := (\rips^\e(P), \rips^\e(Q))$ for $\e > 0$.
% Specifically, we will show that the condition
% \[ \rk~\hom_d(\rips^\delta(P, Q_{\omega - 2c\delta})\hookrightarrow \rips^{2\delta}(P, Q_{\omega+c\delta}))\geq \dim~\hom_0(\rips^\delta(P\setminus Q_{\omega-2c\delta}))\]
% verifies that $D\setminus B_\omega\subseteq P^\delta$ and $Q_{\omega-2c\delta}^\delta$ surrounds $P^\delta$ in $D$.
% This requires generalizing the TCC to boundaries defined as sublevel sets, a setting that applies more naturally to applications in data analysis.
%
% Given a verified sample we can then re-use our Rips complex as a filtration
% \[ \{\rips^{2\delta}(P\subi{\omega-2c\delta}{\alpha}, Q_{\omega-2c\delta})\hookrightarrow \rips^{4\delta}(P\subi{\omega+c\delta}{\alpha}, Q_{\omega+c\delta})\}_{\alpha\in\R}\]
% to approximate the persistent homology of the relative filtration $\{(D\subi{\omega}{\alpha}, B_\omega)\}_{\alpha\in\R}$.
% Indeed, we could use existing methods to approximate the persistent homology of $f$ restricted to the subspace $D\setminus B_\omega$ that we cover.
% In a way that mirrors the TCC we instead approximate the persistent relative homology in order to cancel out noise introduced by the restriction.
% This approach utilizes the property that the subsample $Q^\delta$ \emph{surrounds} $P^\delta$, allowing us to isolate the un-verified region without restriction.

\paragraph*{Outline}

We will begin with our statement of the TCC in Section~\ref{sec:tcc}.
Part of the proof of the TCC will be generalized to properties of \emph{surrounding pairs}, simplifying our reformulation of the TCC in Theorem~\ref{thm:algo_tcc}.
% This requires the introduction of surrounding pairs before proving our reformulation of the TCC (Theorem~\ref{thm:algo_tcc}).
Section~\ref{sec:middle} introduces extensions of surrounding pairs, as well as partial interleavings of image modules.
This is to show that a positive result from the TCC verifies that a surrounding pair of samples can be used to approximate the persistence of a function relative to a sublevel set in Theorem~\ref{thm:interleaving_main_2}.
In Section~\ref{sec:truncations} we provide an interpretation of this relative persistence as a truncation of the full diagram that is motivated by examples in Section~\ref{sec:experiments}.


\section{The Topological Coverage Criterion (TCC)}\label{sec:tcc}
% !TeX root = ../../main_socg.tex

A positive result from the TCC requires that we have a subset of our cover to serve as the boundary.
That is, the the condition not only checks that we have coverage, but also that we have a pair of spaces that reflects the pair $(D, B)$ topologically.
We call such a pair a \emph{surrounding pair} defined in terms of separating sets.
It has been shown that the TCC can be stated in terms of these surrounding pairs~\cite{cavanna2017when}.
Moreover, this work made assumptions directly in terms of the \emph{zero dimensional} persistent homology of the domain close to the boundary.
This allows allows us enough flexibility to define our surrounding set as a sublevel of a $c$-Lipschitz function $f$ and state our assumptions in terms of its persistent homology.

\begin{definition}[Surrounding Pair]
  Let $X$ be a topological space and $(D,B)$ a pair $X$.
  The set $B$ \textbf{surrounds $D$ in $X$} if $B$ separates $X$ with the pair $(D\setminus B, X\setminus D)$.
  We will refer to such a pair as a \textbf{surrounding pair in $X$}.
\end{definition}

For a surrounding pair $(D,B)$ in $\X$  the complement $\overline{B} = \X\setminus B$ is the union of disconnected sets $\X\setminus D$ and $D\setminus B$.
Therefore, $\hom_k(\overline{B}) \cong \hom_k(\overline{D})\oplus \hom_k(D\setminus B)$ thus $\hom_k(\overline{B},\overline{D})\cong \hom_k(D\setminus B)$ for all $k$.
The following lemma generalizes the proof of the TCC as a property of surrounding sets.%TODO, its proof can be found in the \fullversion.
We will then combine these results on the homology of surrounding pairs with information about both $\X$ as a metric space and our function.

\begin{lemma}\label{lem:coverage}
  Let $(D, B)$ be a surrounding pair in $X$ and $U\subseteq D$, $V\subseteq U\cap B$ be subsets.
  Let $\ell: \hom_0(X\setminus B, X\setminus D)\to \hom_0(X\setminus V, X\setminus U)$ be induced by inclusion.

  If $\ell$ is injective then $D\setminus B\subseteq U$ and $V$ surrounds $U$ in $D$.
\end{lemma}

Let $(\X,\dist)$ be a metric space and $D\subseteq \X$ be a compact subspace.
For a $c$-Lipschitz function $f : D\to \R$ we introduce a constant $\omega$ as a threshold that defines our ``boundary'' as a sublevel set $B_\omega$ of the function $f$.
Let $P$ be a finite subset of $D$ and $\zeta\geq\delta > 0 $ be constants such that $P^\delta\subseteq \intr_\X(D)$.
Here, $\delta$ will serve as our communication radius where $\zeta$ is reserved for use in Section~\ref{sec:middle}.
  \footnote{We will set $\zeta = 2\delta$ in the proof of our interleaving with Rips complexes but the TCC holds for all $\zeta\geq\delta$.}

\begin{lemma}\label{lem:psurj}
  Let $i : \hom_0(\overline{Q_{\omega+c\delta}^\delta}, \overline{P^\delta})\to \hom_0(\overline{Q_{\omega-c\zeta}^\delta}, \overline{P^\delta})$.

  If $B_\omega$ surrounds $D$ in $\X$ then $\dim~\hom_0(\overline{B_\omega}, \overline{D})\geq \rk~i$.
\end{lemma}
\begin{proof}
  Choose a basis for $\im~i$ such that each basis element is represented by a point in $P^\delta\setminus Q_{\omega+c\delta}^\delta$.
  Let $x\in P^\delta\setminus Q_{\omega+c\delta}^\delta$ be such that $i[x] \neq 0$.
  So there exits some $p\in P$ such that $\dist(p, x) < \delta$ and $p\notin Q_{\omega+c\delta}$, otherwise $x\in Q_{\omega+c\delta}^\delta$.
  Therefore, because $f$ is $c$-Lipschitz,
  \[ f(x)\geq f(p) - c\dist(x, p) > \omega.\]

  So $x\in\overline{B_\omega}$ and, because $x\in P^\delta\subseteq D$ it follows that $x\in D\setminus B_\omega$.
  Because $i$ and $\ell : \hom_0(\overline{B_\omega}, \overline{D})\to \hom_0(\overline{Q_{\omega-c\zeta}^\delta}, \overline{P^\delta})$ are induced by inclusion $\ell[x] = i[x]\neq 0$ in $\hom_0(\overline{Q_{\omega-c\zeta}^\delta}, \overline{P^\delta})$.
  That is, every element of $\im~i$ has a preimage in $\hom_0(\overline{B_\omega}, \overline{D})$, so we may conclude that $\dim~\hom_0(\overline{B_\omega}, \overline{D})\geq \rk~i$.
\end{proof}

While there is a surjective map from $\hom_0(\overline{B_\omega}, \overline{D})$ to $\im~i$ this map is not necessarily induced by inclusion.
We will therefore introduce a larger space $B_{\omega+c(\delta+\zeta)}$ that contains $Q_{\omega+c\delta}^\delta$ in order to provide a criteria for the injectivity of $\ell : \hom_0(\overline{B_\omega}, \overline{D})\to\hom_0(\overline{Q_{\omega-c\zeta}^\delta}, \overline{P^\delta})$ in terms of $\rk~i$.
We have the following commutative diagrams of inclusion maps and maps induced by inclusion between complements in $\X$.

\begin{equation}\label{dgm:1}
\begin{tikzcd}
  (P^\delta, Q_{\omega-c\zeta}^\delta) \arrow[hookrightarrow]{r}\arrow[hookrightarrow]{d} &
  (P^\delta, Q_{\omega+c\delta}^\delta) \arrow[hookrightarrow]{d} \\
  %
  (D, B_\omega) \arrow[hookrightarrow]{r} &
  (D, B_{\omega+c(\delta+\zeta)}),
\end{tikzcd}
\begin{tikzcd}
  \hom_0(\overline{B_{\omega+c(\delta+\zeta)}},\overline{D})\arrow{d}{m} \arrow{r}{j} &
  \hom_0(\overline{B_\omega}, \overline{D}) \arrow{d}{\ell} \\
  %
  \hom_0(\overline{Q_{\omega+c\delta}^\delta}, \overline{P^\delta}) \arrow{r}{i} &
  \hom_0(\overline{Q_{\omega-c\zeta}^\delta}, \overline{P^\delta}).
\end{tikzcd}\end{equation}

\paragraph*{Assumptions}

We will first require the map $\hom_0(D\setminus B_{\omega+c(\delta+\zeta)}\hookrightarrow D\setminus B_\omega)$ to be \emph{surjective}---as we approach $\omega$ from \emph{above} no components \emph{appear}.
This ensures that the rank of the map $j$ is equal to the dimension of $\dim~\hom_0(\overline{B_\omega}, \overline{D})$ so $\ell$ depends only on $\hom_0(\overline{B_\omega}, \overline{D})$ and $\im~i$.

We also assume that $\hom_0(D\setminus B_\omega\hookrightarrow D\setminus B_{\omega-c(\delta+\zeta)})$ is \emph{injective}---as we move away from $\omega$ moving \emph{down} no components \emph{disappear}.
Lemma~\ref{lem:assumption2} uses Assumption 2 to provide a computable upper bound on $\rk~j$.%TODO, its proof can be found in the \fullversion.

\begin{figure}[htbp]
  \centering
  \includegraphics[trim=200 300 200 200, clip, width=0.3\textwidth]{scripts/figures/surf/ass2_B_side.png}
  \includegraphics[trim=200 300 200 200, clip, width=0.3\textwidth]{scripts/figures/surf/ass1_C_side.png}
  \includegraphics[trim=200 300 200 200, clip, width=0.3\textwidth]{scripts/figures/surf/ass1_D_side.png}
  \includegraphics[trim=300 100 200 200, clip, width=0.3\textwidth]{scripts/figures/surf/ass2_B_top.png}
  \includegraphics[trim=300 150 300 200, clip, width=0.3\textwidth]{scripts/figures/surf/ass1_C_top.png}
  \includegraphics[trim=300 150 300 200, clip, width=0.3\textwidth]{scripts/figures/surf/ass1_D_top.png}
  \includegraphics[width=0.5\textwidth]{scripts/figures/scalar_barcode_H1-masked.png}
  \caption{The blue level set in the middle does not satisfy either assumption.
    The inclusion from the right is not \emph{surjective} as the smaller component appears in the middle (in the sublevel barcode, a $\hom_{d-1}$ feature dies in the purple region).
    The inclusion to the left is not \emph{injective} as the smaller component is merged with the large (in the sublevel barcode, a $\hom_{d-1}$ feature is born in the blue region).}\label{fig:assumption1}
\end{figure}

\begin{lemma}\label{lem:assumption2}
  If $\hom_0(D\setminus B_\omega\hookrightarrow D\setminus B_{\omega+c(\delta+\zeta)})$ is injective and each component of $D\setminus B_\omega$ contains a point in $P$ then $\dim~\hom_0(\rips^\delta(P\setminus Q_{\omega-c\zeta})) \geq \dim~\hom_0(D\setminus B_\omega)$.
\end{lemma}

\paragraph*{Nerves and Duality}

Recall that the Nerve Theorem states that for a good open cover $\cU$ of a space $X$ the inclusion map from the \emph{Nerve} of the cover to the space $\N(\cU)\hookrightarrow X$ is a homotopy equivalence.\footnote{In a good open cover every nonempty intersection of sets in the cover is contractible.}
The Persistent Nerve Lemma~\cite{chazal08towards} states that this homotopy equivalence commutes with inclusion on the level of homology.
The standard proof of the Nerve Theorem~\cite{kozlov07combinatorial}, and therefore the Persistent Nerve Lemma~\cite{chazal08towards}, extends directly to pairs of good open covers $(\cU, \cV)$ of pairs $(X, Y)$ such that $\cV$ is a subcover of $\cU$.\footnote{$\{V_i\}_{i\in I}$ is a subcover of $\{U_i\}_{i\in I}$ if $V_i\subseteq U_i$ for all $i\in I$.}

Recalling the definition of the strong convexity radius $\varrho_D$ (see Chazal et al.~\cite{chazal09analysis}) $\cU$ is a good open cover whenever $\varrho_D > \e$.
As the \v Cech complex is the Nerve of a cover by a union of balls we will let $\N_z^\e : \hom_k(\cech^\e(P,Q_z))\to \hom_k(P^\e, Q_z^\e)$ denote the isomorphism on homology provided by the Nerve Theorem for all $k$, $z\in\R$ and $\e < \varrho_D$.

Under certain conditions Alexander Duality provides an isomorphism between the $k$ relative cohomology of a compact pair in an orientable $d$-manifold $\X$ with the $d - k$ dimensional homology of their complements in $\X$ (see Spanier~\cite{spanier1989algebraic}).
For finitely generated (co)homology over a field the Universal Coefficient Theorem can be used with Alexander Duality to show $\hom_d(P^\e,Q_z^\e)\cong\hom_0(D\setminus Q_z^\e, D\setminus P^\e)$.
 % give a natural isomorphism $\xi_z^\e : \hom_d(P^\e,Q_z^\e)\to \hom_0(D\setminus Q_z^\e, D\setminus P^\e)$.\footnote{For the construction of this isomorphism see the \fullversion.}
This isomorphism holds in the specific case when $P^\e\subseteq \intr_\X(D)$ and $D\setminus P^\e$, $D\setminus Q_z^\e$ are locally contractible.
We therefore provide the following definition for ease of exposition.
\begin{definition}[$(\omega, \delta,\zeta)$-Sample]
  For $\zeta\geq \delta > 0$, $\omega\in\R$, and a $c$-Lipschitz function $f: D\to \R$ a finite point set $P\subset D$ is said to be an \textbf{$(\omega, \delta, \zeta)$-sublevel sample} of $f$ if \begin{itemize}
    \item $P^\delta\subset\intr_\X(D)$ and
    \item $D\setminus P^\delta$, $D\setminus Q_{\omega-c\zeta}^\delta$, and $D\setminus Q_{\omega+c\delta}^\delta$ are locally path connected in $\X$.
  \end{itemize}
\end{definition}

\begin{theorem}[Algorithmic TCC]\label{thm:algo_tcc}
  Let $\X$ be an orientable $d$-manifold and let $D$ be a compact subset of $\X$.
  Let $f : D\to\R$ be $c$-Lipschitz function and $\omega\in\R$, $\delta\leq\zeta < \varrho_D$ be constants such that $B_{\omega - c(\zeta +\delta)}$ surrounds $D$ in $\X$.
  Let $P$ be an $(\omega, \delta,\zeta)$-sample of $f$ such that every component of $D\setminus B_\omega$ contains a point in $P$.
  Suppose $\hom_0(D\setminus B_{\omega+c(\delta+\zeta)}\hookrightarrow D\setminus B_\omega)$ is surjective and $\hom_0(D\setminus B_\omega\hookrightarrow D\setminus B_{\omega-c(\delta+\zeta)})$ is injective.

   If $\rk~\hom_d(\rips^\delta(P, Q_{\omega -c\zeta})\hookrightarrow \rips^{2\delta}(P, Q_{\omega+c\delta})) \geq \dim~\hom_0(\rips^\delta(P\setminus Q_{\omega-c\zeta}))$ then $D\setminus B_\omega\subseteq P^\delta$ and $Q_{\omega-c\zeta}^\delta$ surrounds $P^\delta$ in $D$.
\end{theorem}
\begin{proof}
  Let $q : \hom_d(P^\delta, Q_{\omega-c\zeta}^\delta)\to \hom_d(P^\delta, Q_{\omega+c\delta}^\delta)$,
  $q_{\cech} : \hom_d(\cech^{\delta}(P, Q_{\omega-c\zeta}))\to\hom_d(\cech^{\delta}(P, Q_{\omega+c\delta}))$, and
  $q_{\rips} : \hom_d(\rips^{\delta}(P, Q_{\omega-c\zeta}))\to\hom_d(\rips^{2\delta}(P, Q_{\omega+c\delta}))$ be induced by inclusion.
  Then $\rk~q_{\cech} \geq\rk~q_{\rips}$ as $q_{\rips}$ factors through $q_{\cech}$ by the Rips-\v Cech interleaving.
  Moreover, $\rk~q = \rk~q_{\cech}$ by the persistent nerve lemma, so $\rk~q\geq \rk~q_{\rips}$.
  As we have assumed $\hom_0(D\setminus B_\omega\hookrightarrow D\setminus B_{\omega-c(\delta+\zeta)})$ Lemma~\ref{lem:assumption2} implies $\dim~\hom_0(\rips^\delta(P\setminus Q_{\omega-c\zeta}))\geq \dim~\hom_0(D\setminus B_\omega)$.
  Because $P$ is an $(\omega, \delta, \zeta)$-sample we have $\hom_d(P^\delta, Q_{\omega-c\zeta}^\delta)\cong \hom_0(D\setminus Q_{\omega-c\zeta}^\delta, D\setminus P^\delta)$ and $\hom_d(P^\delta, Q_{\omega+c\delta}^\delta)\cong \hom_0(D\setminus Q_{\omega-2c\delta}^\delta, D\setminus P^\delta)$ so $\rk~i\geq \rk~q$ by Alexander Duality and the Universal Coefficient Theorem.
  So, by our hypothesis that $\rk~q_{\rips}\geq\dim~\hom_0(\rips^\delta(P\setminus Q_{\omega-c\zeta}))$ we have $\rk~i\geq\dim~\hom_0(D\setminus B_\omega)$.

  Because $j : \hom_0(D\setminus B_{\omega+c(\delta+\zeta)}\hookrightarrow D\setminus B_\omega)$ is surjective by hypothesis $\rk~j = \dim~\hom_0(D\setminus B_\omega)$ so $\rk~j\geq \rk~i$ by Lemma~\ref{lem:psurj}.
  As we have shown $\rk~i\geq \dim~\hom_0(D\setminus B_\omega)$ it follows that $\rk~j = \rk~i$.
  Because $P$ is a finite point set we know that $\im~i$ is finite-dimensional and, because $\rk~i = \rk~j$, $\im~j=\hom_0(\overline{B_\omega}, \overline{D})$ is finite dimensional as well.
  So $\im~j$ is isomorphic to $\im~i$ as a subspace of $\hom_0(\overline{Q_{\omega-c\zeta}^\delta}, \overline{P^\delta})$ which, because $j$ is surjective, requires the map $\ell$ to be injective.
  Therefore $D\setminus B_\omega\subseteq P^\delta$ and $Q_{\omega-c\zeta}^\delta$ surrounds $P^\delta$ in $D$ by Lemma~\ref{lem:coverage}.
\end{proof}

% Because this isomorphism is natural with respect to maps induced by inclusion, and isomorphism provided by the Nerve Theorem commutes with maps induced by inclusion the composition $\xi\N_w^\e := \xi_w^\e\circ\N_w^\e$ gives an isomorphism that commutes with maps induced by inclusion for all $w\in\R$ and $\e < \varrho_D$.

% \begin{theorem}[Algorithmic TCC]\label{thm:algo_tcc}
%   Let $\X$ be an orientable $d$-manifold and let $D$ be a compact subset of $\X$.
%   Let $f : D\to\R$ be $c$-Lipschitz function and $\omega\in\R$, $\delta\leq\zeta < \varrho_D$ be constants such that $B_{\omega - c(\zeta +\delta)}$ surrounds $D$ in $\X$.
%   Let $P$ be an $(\omega, \delta,\zeta)$-sample of $f$ such that every component of $D\setminus B_\omega$ contains a point in $P$.
%   Suppose $\hom_0(D\setminus B_{\omega+c(\delta+\zeta)}\hookrightarrow D\setminus B_\omega)$ is surjective and $\hom_0(D\setminus B_\omega\hookrightarrow D\setminus B_{\omega-c(\delta+\zeta)})$ is injective.
%
%    If $\rk~\hom_d(\rips^\delta(P, Q_{\omega -c\zeta})\hookrightarrow \rips^{2\delta}(P, Q_{\omega+c\delta})) \geq \dim~\hom_0(\rips^\delta(P\setminus Q_{\omega-c\zeta}))$ then $D\setminus B_\omega\subseteq P^\delta$ and $Q_{\omega-c\zeta}^\delta$ surrounds $P^\delta$ in $D$.
% \end{theorem}
% \begin{proof}
%   % We have the following commutative diagram
%   % \[\begin{tikzcd}
%   %   \hom_d(\cech^\delta(P, Q_{\omega-c\zeta})) \arrow{r}{q_{\cech}}\arrow{d}{\N_{\omega-c\zeta}^{\delta}} &
%   %   \hom_d(\cech^\delta(P, Q_{\omega+c\delta})) \arrow{d}{\N_{\omega-c\zeta}^\delta}\\
%   %   %
%   %   \hom_d(P^\delta, Q_{\omega-c\zeta}^\delta))\arrow{r}{q} &
%   %   \hom_d(P^\delta, Q_{\omega+c\delta}^\delta).
%   % \end{tikzcd}\]
%   % where vertical maps are isomorphisms provided by the Nerve Theorem and horizontal maps are induced by inclusions.
%
%   Because $P$ is an $(\omega, \delta, \zeta)$-sublevel sample we have isomorphisms $\xi\N_{\omega-c\zeta}^\delta$ and $\xi\N_{\omega+c\delta}^\delta$ that commute with $q_{\cech} : \hom_d(\cech^{\delta}(P, Q_{\omega-c\zeta}))\to\hom_d(\cech^{2\delta}(P, Q_{\omega+c\delta}))$ and $i : \hom_0(D\setminus Q_{\omega+c\delta}^\delta, D\setminus P^\delta)\to \hom_0(D\setminus Q_{\omega-c\zeta}^\delta, D\setminus P^\delta)$.
%   Let $q_{\rips} : \hom_d(\rips^{\delta}(P, Q_{\omega-c\zeta}))\to\hom_d(\rips^{2\delta}(P, Q_{\omega+c\delta}))$ be induced by inclusion.
%   Then $\rk~q_{\cech} \geq\rk~q_{\rips}$ as $q_{\rips}$ factors through $q_{\cech}$.
%   As we have assumed $\hom_0(D\setminus B_\omega\hookrightarrow D\setminus B_{\omega-c(\delta+\zeta)})$ Lemma~\ref{lem:assumption2} implies $\dim~\hom_0(\rips^\delta(P\setminus Q_{\omega-c\zeta}))\geq \dim~\hom_0(D\setminus B_\omega)$.
%   It follows that, whenever $\rk~q_{\rips} \geq \dim~\hom_0(\rips^\delta(P\setminus Q_{\omega-c\zeta}))$, we have
%   \[ \rk~i = \rk~q_{\cech} \geq \rk~q_{\rips} \geq \dim~\hom_0(\rips^\delta(P\setminus Q_{\omega-c\zeta})) \geq \dim~\hom_0(D\setminus B_\omega).\]
%
%   Because $j$ is surjective by hypothesis $\rk~j = \dim~\hom_0(\overline{B_\omega},\overline{D}) = \dim~\hom_0(D\setminus B_\omega)$ so $\rk~j\geq \rk~i$ by Lemma~\ref{lem:psurj}.
%   As we have shown $\rk~i\geq \dim~\hom_0(D\setminus B_\omega)$ it follows that $\rk~j = \rk~i$.
%   Because $P$ is a finite point set we know that $\im~i$ is finite-dimensional and, because $\rk~i = \rk~j$, $\im~j=\hom_0(\overline{B_\omega}, \overline{D})$ is finite dimensional as well.
%   So $\im~j$ is isomorphic to $\im~i$ as a subspace of $\hom_0(\overline{Q_{\omega-c\zeta}^\delta}, \overline{P^\delta})$ which, because $j$ is surjective, requires the map $\ell$ to be injective.
%   Therefore $D\setminusB_\omega\subseteq P^\delta$ and $Q_{\omega-c\zeta}^\delta$ surrounds $P^\delta$ in $D$ by Lemma~\ref{lem:coverage}.
%   %, Lemma~\ref{lem:cov_surrounds}.
%   % As $j : \hom_0(D\setminus B_{\omega+c(\delta+\zeta)})\to \hom_0(D\setminus B_\omega)$ is surjective by assumption $\rk~j = \dim~\hom_0(D\setminus B_\omega)$, so $D\setminus B_\omega\subseteq P^\delta$ and $Q_{\omega-c\zeta}^\delta$ surrounds $P^\delta$ in $D$ by Theorem~\ref{thm:geo_tcc} as desired.
% \end{proof}


\section{From Coverage Testing to the Analysis of Scalar Fields}\label{sec:middle}
% !TeX root = ../../main_socg.tex

Because the TCC only confirms coverage of a \emph{superlevel} set $D\setminus B_\omega$, we cannot guarantee coverage of the entire domain.
Indeed, we could compute the persistent homology of the \emph{restriction} of $f$ to the superlevel set we cover in the standard way~\cite{chazal09analysis}.
Instead, we will approximate the persistent homology of the sublevel set filtration \emph{relative to} the sublevel set $B_\omega$.

\begin{figure}[htbp]
  \centering
  \begin{minipage}[b]{0.27\textwidth}
    \includegraphics[trim=200 200 200 100, clip, width=\textwidth]{scripts/figures/surf/ass2_C_side.png}\\
    \includegraphics[trim=200 100 200 200, clip, width=\textwidth]{scripts/figures/surf/ass2_C_top.png}
  \end{minipage}
  \begin{minipage}[b]{0.7\textwidth}
    \includegraphics[width=\textwidth]{scripts/figures/barcodes/res_rel.png}
  \end{minipage}
  \caption{Full, restricted, and relative barcodes of the function (left).}% on a $512\times 512$ grid.
    % The restricted barcode is of the function restricted to the region above the blue line.
    % The relative barcode is of the function relative to the blue sub-levelset below the blue line.}
    % We note that the additional features in restricted $\hom_0$ are artifacts of the restriction caused by the approximation.}
\end{figure}

We will first introduce the notion of an extension which will provide us with maps on relative homology induced by inclusion via excision.
However, even then, a map that factors through our pair $(D, B_\omega)$ is not enough to prove an interleaving of persistence modules by inclusion directly.
To address this we impose conditions on sublevel sets near $B_\omega$ which generalize the assumptions made in the TCC on maps induced by the inclusions
\[ D\setminus B_{\omega+c(\delta+\zeta)}\hookrightarrow D\setminus B_\omega\hookrightarrow D\setminus B_{\omega-c(\delta+\zeta)}\]
on $0$-dimensional homology, to assumptions on maps induced by the corresponding inclusions
\[ B_{\omega-c(\delta+\zeta)}\hookrightarrow B_\omega\hookrightarrow B_{\omega+c(\delta+\zeta)}\]
on homology in all dimensions $k$.

\subsection{Extensions and Image Persistence Modules}

Suppose $D$ is a subspace of $X$.
We define the extension of a surrounding pair in $D$ to a surrounding pair in $X$ with isomorphic relative homology.

\begin{definition}[Extension]
  If $V$ surrounds $U$ in a subspace $D$ of $X$ let $\ext{V} := V\sqcup (D\setminus U)$ denote the (disjoint) union of the separating set $V$ with the complement of $U$ in $D$.
  The \textbf{extension of $(U, V)$ in $D$} is the pair $(D, \ext{V}) = (U\sqcup (D\setminus U), V\sqcup (D\setminus U)).$
\end{definition}

Lemma~\ref{lem:surround_and_cover} states that we can use these extensions to interleave a pair $(U, V)$ with a sequence of subsets of $(D, B)$.
Lemma~\ref{excision} we can apply excision to the relative homology groups in order to get equivalent maps on homology that are induced by inclusions.
%TODO Proof of these facts, and their extensions to homomorphisms of persistence modules in the next section, can be found in the \fullversion.

\begin{lemma}\label{lem:surround_and_cover}
  Suppose $V$ surrounds $U$ in $D$ and $B'\subseteq B\subset D$.

  If $D\setminus B\subseteq U$ and $U\cap B'\subseteq V\subseteq B'$ then $B'\subseteq \ext{V}\subseteq B$.
\end{lemma}

\begin{lemma}\label{lem:excision}
  Let $(U, V)$ be an open surrounding pair in a subspace $D$ of $X$.

  Then $\hom_k((U\cap A, V)\hookrightarrow (A, \ext{V}))$ is an isomorphism for all $k$ and $A\subseteq D$ with $\ext{V}\subset A$.
\end{lemma}

In the TCC a nested pair of spaces is used in order to filter out noise introduced by the sample.
This same technique is used in the analysis of scalar fields~\cite{chazal09analysis} to interleave the persistent homology of a sequence of subspaces with that of a function.
These subspaces are simply the images of homomorphisms between homology groups induced by inclusion, and we refer to the resulting persistence module as an image persistence module.

% \paragraph{Image Persistence Modules}
\begin{definition}[Image Persistence Module]
  The \textbf{image persistence module} of a homomorphism $\Gamma\in\Hom(\U,\V)$ is the family of subspaces $\{\Gamma_\alpha :=\im~\gamma_\alpha\}$ in $\V$ along with linear maps $\{\gamma_\alpha^\beta := v_\alpha^\beta\rest_{\im~\gamma_\alpha} : \Gamma_\alpha\to\Gamma_\beta\}$ and will be denoted by $\im~\Gamma$.
\end{definition}

While we will primarily work with homomorphisms of persistence modules induced by inclusions, in general, defining homomorphisms between images simply as subspaces of the codomain is not sufficient.
Instead, we require that homomorphisms between image modules commute not only with shifts in scale, but also with the functions themselves.

\begin{definition}[Image Module Homomorphism]
  Given $\Gamma\in\Hom(\U,\V)$ and $\Lambda\in\Hom(\S,\T)$ along with $(F,G)\in\Hom^\delta(\U,\S)\times\Hom^\delta(\V,\T)$ let $\Phi(F, G) : \im~\Gamma\to\im~\Lambda$ denote the family of linear maps $\{\phi_\alpha := g_\alpha\rest_{\Gamma_\alpha} : \Gamma_\alpha\to\Lambda_{\alpha+\delta}\}$.
  $\Phi(F, G)$ is an \textbf{image module homomorphism of degree $\delta$} if the following diagram commutes for all $\alpha\leq\beta$.\footnote{We use the notation $\gamma_\alpha[\beta-\alpha] = v_\alpha^\beta\circ\gamma_\alpha$, $\lambda_\alpha[\beta-\alpha] = t_\alpha^\beta\circ\lambda_\alpha$ to denote the composition of homomorphisms between persistence modules and shifts in scale.}
  \begin{equation}\label{dgm:image_homomorphism}
    \begin{tikzcd}[column sep=large]
        U_\alpha\arrow{r}{\gamma_\alpha[\beta-\alpha]}\arrow{d}{f_\alpha} &
      V_\beta\arrow{d}{g_\beta}\\
      %
      S_{\alpha+\delta}\arrow{r}{\lambda_{\alpha+\delta}[\beta-\alpha]} &
      T_{\beta +\delta}
  \end{tikzcd}\end{equation}
  The space of image module homomorphisms of degree $\delta$ between $\im~\Gamma$ and $\im~\Lambda$ will be denoted $\Hom^\delta(\im~\Gamma,\im~\Lambda)$.
\end{definition}
%
% In the following the existence of an image module homomorphism $\Phi(F, G)\in\Hom^\delta(\im~\Gamma, \im~\Lambda)$ where $\Gamma\in\Hom(\U,\V)$ and $\Lambda\in\Hom(\S,\T)$  will imply that $(F,G)\in\Hom^\delta(\U,\S)\times \Hom^\delta(\V,\T)$.
%
%
The composition of image module homomorphisms are image module homomorphisms.
% That is, for $\Phi'(F', G')\in\Hom^\delta(\im~\Gamma, \im~\Lambda)$ and $\Phi(F, G)\in\Hom^{\delta'}(\im~\Lambda, \im~\Lambda')$ the composition of pairs $(F'\circ F, G'\circ G)$ is an image module homomorphism, denoted $\Phi'\circ \Phi\in\Hom^{\delta+\delta'}(\im~\Gamma,\im~\Lambda')$.
Proof of this fact can be found in the \fullversion.

\paragraph*{Partial Interleavings of Image Modules}

Image module homomorphisms introduce a direction to the traditional notion of interleaving.
% That is, given $\Gamma\in\Hom(\U,\V)$ and $\Lambda\in\Hom(\S,\T)$ and $\Phi(F, G)\in\Hom^\delta(\im~\Gamma, \im~\Lambda)$ we consider the case in which there is only a map $\S\to\V$ that commutes.
As we will see, our interleaving via Lemma~\ref{thm:interleaving_main} involves partially interleaving an image module to two other image modules whose composition is isomorphic to our target.

\begin{definition}[Partial Interleaving of Image Modules]
  % Let $\Gamma\in\Hom(\U,\V)$ and $\Lambda\in\Hom(\S,\T)$.
  % $\Phi(F, G)\in\Hom^\delta(\im~\Gamma,\im~\Lambda)$ is a \textbf{left $\delta$-interleaving of image modules} if there exists some $M\in\Hom^\delta(\S,\V)$ such that $\Gamma[2\delta] = M\circ F$.
  % If $\Lambda[2\delta] = G\circ M$ then $\Phi(F, G)$ is a \textbf{right $\delta$-interleaving of image modules}.
  An image module homomorphism $\Phi(F, G)$ is a \textbf{partial $\delta$-interleaving of image modules}, and denoted $\Phi_M(F, G)$, if there exists $M\in\Hom^\delta(\S,\V)$ such that $\Gamma[2\delta] = M\circ F$ and $\Lambda[2\delta] = G\circ M$.
\end{definition}

% For $I\in\Hom^{2\delta}(\U,\V)$ a pair $(F, M)\in \Hom^\delta(\U,\S)\times\Hom^\delta(\S,\V)$ is a said to factor $I$ through $\S$ with degree $\delta$ if $I = M\circ F$.
% Similarly, if $J\in\Hom^{\delta'}(\S,\T)$ a pair $(F,N)\in\Hom^\delta(\U,\S)\times\Hom^\delta(\T,\V)$ is said to factor $I$ through $J$ with degree $\delta$ if $I = N\circ J\circ F$.
% We will often omit the degree when it is clear from context.

% Proof of the following lemma can be found in the appendix.
% Proof of following lemma is straightforward and can be found in the \fullversion.

% uses partial interleavings surrounding a module $\V$ to prove an interleaving of an image module with $\V$.
%TODO Its proof is straightforward and can be found in the \fullversion.
Lemma~\ref{thm:interleaving_main} uses partial interleavings of a map $\Lambda$ with $\U\to\V$ and $\V\to\W$ along with the hypothesis that $\U\to \W$ is isomorphic to $\V$ to interleave $\im~\Lambda$ with $\V$.
When applied, this hypothesis will be satisfied by assumptions on our sublevel set similar to those made in the TCC.

\begin{lemma}\label{thm:interleaving_main}
  Suppose $\Gamma\in\Hom(\U,\V)$, $\Pi\in\Hom(\V,\W)$, and $\Lambda\in\Hom(\S, \T)$.

  If $\Phi_M(F, G)\in\Hom^\delta(\im~\Gamma, \im~\Lambda)$ and $\Psi_G(M, N)\in\Hom^\delta(\im~\Lambda, \im~\Pi)$ are partial $\delta$-interleavings of image modules such that $\Gamma$ is a epimorphism and $\Pi$ is a monomorphism then $\im~\Lambda$ is $\delta$-interleaved with $\V$.
\end{lemma}

% !TeX root = ../../main_socg.tex



\subsection{Proof of the Interleaving}

For $w,\alpha\in\R$ let $\DD{w}^k$ denote the $k$th persistent (relative) homology module of the filtration $\{(D\subi{w}{\alpha},B_w)\}_{\alpha\in\R}$ with respect to $B_w$, and let $\PP{w}{\e,k}$ denote the $k$th persistent (relative) homology module of $\{(P\subi{w}{\alpha}^\e,Q_w^\e)\}_{\alpha\in\R}$.
Similarly, let $\CPP{w}{\e,k}$ and $\RPP{w}{\e,k}$ denote the corresponding \Cech and Vietoris-Rips filtrations, respectively.
We will omit the dimension $k$ and write $\DD{w}$ (resp. $\PP{w}{\e}$) if a statement holds for all dimensions.
If $Q_w^\delta$ surrounds $P^\delta$ in $D$ let $\ext{\PP{w}{\e}}$ denote the $k$th persistent homology module of the filtration of extensions $\{(\ext{P\subi{w}{\alpha}^\e},\ext{Q_w^\e})\}$ for any $\e\geq\delta$, where $\ext{P\subi{w}{\alpha}^\e} = P\subi{w}{\alpha}^\e \cup (D\setminus P^\delta)$.


% \begin{lemma}\label{lem:extension_apply}
%   If $Q_w^\e$ surrounds $P^\e$ in $D$ then then there is an isomorphism $\E_w^\e \in \Hom(\PP{w}{\e},\ext{\PP{w}{\e}})$.
% \end{lemma}
% and will be used

Lemma~\ref{lem:inclusions} follows directly from the definition of truncated sublevel sets.
This is used to extend Lemma~\ref{lem:surround_and_cover} to persistence modules in Lemma~\ref{lem:inclusion_hom} in order to provide a sequence of shifted homomorphisms $\DD{\omega-3c\delta}\xrightarrow{F}\E\PP{\omega-2c\delta}{\e}\xrightarrow{M}\DD{\omega}\xrightarrow{G}\E\PP{\omega+c\delta}{2\e}\xrightarrow{N}\DD{\omega+5c\delta}$ of varying degree.
These homomorphisms are then combined with those given by the Nerve theorem and the Rips-\v Cech interleaving in Lemma~\ref{lem:partial_interleaving} to obtain partial interleavings required for our proof of Theorem~\ref{thm:interleaving_main_2}.

\begin{lemma}\label{lem:inclusions}
  If $\delta\leq\e$ and $t,\alpha\in\R$ then $P^\delta\cap D\subi{t-c\e}{\alpha-c\e}\subseteq P\subi{t}{\alpha}^\e\subseteq D\subi{t+c\e}{\alpha+c\e}$.
\end{lemma}

\begin{lemma}\label{lem:inclusion_hom}
  Let $s + 3c\delta\leq t + 2c\delta\leq u\leq v-c\delta\leq w-5c\delta$ and $\e\in [\delta,2\delta]$.
  If $Q_{t}^\delta$ surrounds $P^\delta$ in $D$ and $D\setminus B_u\subseteq P^\delta$ then the following homomorphisms are induced by inclusions:
  \[(F, G)\in \Hom^{c\delta}(\DD{s}, \E\PP{t}{\e})\times \Hom^{2c\delta}(\DD{u}, \E\PP{v}{2\e}),\ (M, N)\in \Hom^{c\e}(\E\PP{t}{\e},\DD{u})\times\Hom^{2c\e}(\E\PP{v}{2\e}, \DD{w}).\]
\end{lemma}

\begin{lemma}\label{lem:partial_interleaving}
  % Let all be induced by inclusions.
  For $\delta < \varrho_D$ let $\Gamma\in\Hom(\DD{s},\DD{u})$, $\Pi\in\Hom(\DD{u},\DD{w})$, and $\Lambda\in\Hom(\RPP{t}{2\delta}, \RPP{v}{4\delta})$ be induced by inclusions for $s + 3c\delta\leq t + 2c\delta\leq u\leq v-c\delta\leq w-5c\delta$.

  If $Q_{t}^\delta$ surrounds $P^\delta$ in $D$ and $D\setminus B_u\subseteq P^\delta$ then there is a partial $2c\delta$ interleaving $\Phi^*\in\Hom^{2c\delta}(\im~\Gamma, \im~\Lambda)$ and a partial $4c\delta$ interleaving $\Psi^*\in\Hom^{4c\delta}(\im~\Lambda, \im~\Pi)$.
\end{lemma}
\begin{proof}
  % Let $\E\PP{t}{\delta}\xrightarrow{A}\E\PP{t}{2\delta}\xrightarrow{C}\E\PP{v}{2\delta}\xrightarrow{E}\E\PP{v}{4\delta}$ be maps induced by inclusion.
  % Because $f$ is $c$-Lipschitz, $B_{s}\cap P^\delta\subseteq Q_{t}^\delta$ and $B_u\cap P^\delta\subseteq Q_{v}^{2\delta}$.
  % Similarly, $Q_{t}^{2\delta}\subseteq B_u$ and $Q_{v}^{4\delta}\subseteq B_{w}$.
  % Therefore, by Lemma~\ref{lem:surround_and_cover} $B_{s}\subseteq \E Q_{t}^\delta\subseteq\E Q_{t}^{2\delta}\subseteq B_u\subseteq \E Q_{v}^{2\delta}\subseteq \E Q_{v}^{4\delta}\subseteq B_{w}.$
  % It follows that the following diagrams commute for all $\alpha\leq\beta$ by Lemma~\ref{lem:p_interleave}.
  % % So the following diagrams commute for all $\alpha\leq\beta$ where all maps are induced by inclusions.
  Because the shifted homomorphisms provided by Lemma~\ref{lem:inclusion_hom} are all induced by inclusions the following diagram commutes for all $\alpha\leq\beta$.
  So we have image module homomorphisms $\Phi(F, G)\in\Hom^{2c\delta}(\im~\Gamma, \im~C\circ A)$ and $\Psi(M, N)\in\Hom^{4c\delta}(\im~E\circ C, \im~\Pi)$.
  \[\begin{tikzcd}
      \hom_k(D\subi{s}{\alpha-2c\delta}, B_s) \arrow{r}{f_{\alpha-2c\delta}}\arrow{d}{\gamma_{\alpha-2c\delta}[\beta-\alpha]} &
      \hom_k(\E P\subi{t}{\alpha}^\delta, \E Q_t^\delta)\arrow{d}{c_\alpha[\beta-\alpha]\circ a_\alpha}\\
      %
      \hom_k(D\subi{u}{\beta-2c\delta}, B_u)\arrow{r}{g_{\beta-2c\delta}} &
      \hom_k(\E P\subi{v}{\beta}^{2\delta}, \E Q_v^{2\delta})
    \end{tikzcd}
    \begin{tikzcd}
      \hom_k(\E P\subi{t}{\alpha}^{2\delta}, \E Q_t^{2\delta})\arrow{d}{e_\beta\circ c_\alpha[\beta-\alpha]}\arrow{r}{m_{\alpha}} &
      \hom_k(D\subi{u}{\alpha+4c\delta}, B_u)\arrow{d}{\gamma_{\alpha+4c\delta}[\beta-\alpha]}\\
      %
      \hom_k(\E P\subi{v}{\beta}^{4\delta}, \E Q_v^{4\delta})\arrow{r}{n_\beta} &
      \hom_k(D\subi{w}{\beta+4c\delta}, B_w)
    \end{tikzcd}\]

  Because the isomorphisms provided by Lemma~\ref{lem:excision} are given by excision they are induced by inclusion, and therefore give isomorphisms $\E_z^\e \in \Hom(\PP{z}{\e},\ext{\PP{z}{\e}})$ for any $z\in\R$ such that $Q_z^\e$ surrounds $P^\delta$ in $D$.
  For any $\e < \varrho_D$ we have isomorphisms $\N_z^\e\in\Hom(\CPP{z}{\e}, \PP{z}{\e})$ that commute with maps induced by inclusions by the Persistent Nerve Lemma.
  So the compositions $\E_z^\e\circ \N_z^\e$ isomorphisms that commute with maps induced by inclusion as well.
  These compositions, along with the Rips-\v Cech interleaving, provide maps $\E\PP{t}{\delta}\xrightarrow{F'}\RPP{t}{2\delta}\xrightarrow{M'} \E\PP{t}{2\delta}$ and $\E\PP{v}{2\delta}\xrightarrow{G'}\RPP{v}{4\delta}\xrightarrow{N'} \E\PP{v}{4\delta}$ that commute with maps induced by inclusions.
  So we have the following commutative diagram:
  % As all maps are induced by inclusions or commute with maps induced by inclusions we have the following commutative diagram.
  %
  \begin{equation}
    \begin{tikzcd}
      \E\PP{t}{\delta}\arrow{rr}{A}\arrow{dr}{F'} & &
      \E\PP{t}{2\delta}\arrow{r}{C} &
      \E\PP{v}{2\delta}\arrow{rr}{E}\arrow{dr}{G'} & &
      \E\PP{v}{4\delta}\\
      %
      & \RPP{t}{2\delta}\arrow{ur}{M'}\arrow[to=RV, "\Lambda"] & &
      & |[alias=RV]|\RPP{v}{4\delta}\arrow{ur}{N'} &
    \end{tikzcd}
  \end{equation}
  %
  That is, we have image module homomorphisms $\Phi'(F', G')$ and $\Psi'(M', N')$ such that $A = M'\circ F'$, $E = N'\circ G'$, and $\Lambda = G'\circ C\circ M'$.
  Because image module homomorphisms compose we have we have $\Phi^* = \Phi'\circ \Phi\in\Hom^{2c\delta}(\im~\Gamma, \im~\Lambda)$ and $\Psi^* = \Psi\circ\Psi'\in\Hom^{4c\delta}(\im~\Lambda, \im~\Pi)$.

  Because $G,M,C$ are induced by inclusions $C[3c\delta] = G\circ M$, so $\Lambda[3c\delta] = G'\circ C[3c\delta]\circ M' = G'\circ (G\circ M)\circ M'$ as $G', M'$ commute with maps induced by inclusions.
  In the same way, $\Gamma[3c\delta] = M\circ (A\circ F) = M\circ (M'\circ F')\circ F$ and $\Pi[5c\delta] = N\circ (E\circ G) = N\circ (N'\circ G')\circ G$.

  Let $F^*:= F'\circ F$, $G^*:= G'\circ G$, $M^*:=M'\circ M$, and $N^*:=N'\circ N$.
  So $\Phi^*_{M^*}$ is a partial $2c\delta$ interleaving as $\Gamma[3c\delta] = M^*\circ F^*$ and $\Lambda[3c\delta] = G^*\circ M^*$, and $\Psi^*_{G^*}$ is a partial $4c\delta$ interleaving as $\Lambda[3c\delta] = G^*\circ M^*$ and $\Pi[5c\delta] = N^*\circ G^*$.
\end{proof}

% If $\e < \varrho_D$ then we for any $\alpha\in\R$ the inclusion $\cech^\e(P\subi{w}{\alpha}, Q_w)\hookrightarrow (P\subi{w}{\alpha}^\e, Q_w^\e)$ is a homotopy equivalence by the Nerve Theorem.
% As the module homomorphisms of $\CPP{w}{\e}$ and $\PP{w}{\e}$ are induced by inclusion we have an isomorphism $\N_w^\e\in\Hom(\CPP{w}{\e}, \PP{w}{\e})$ of persistence modules that commutes with maps induced by inclusions by the Persistent Nerve Lemma
% As the isomorphisms of $\E_w^\e$ are given by excision they are induced by inclusions, so the composition $\E\N_w^\e := \E_w^\e\circ \N_w^\e$ is an isomorphism that commutes with maps induced by inclusion as well.
% The following lemma uses these isomorphisms along with inclusions $\I_w^\e\in\Hom(\CPP{w}{\e}, \RPP{w}{2\e})$ and $\J_w^\e\in\Hom(\RPP{w}{\e},\CPP{w}{\e})$ to establish image module homomorphisms by maps $\Sigma_w^\e\in\Hom(\PP{w}{\e},\RPP{w}{2\e})$ and $\Upsilon_w^\e\in \Hom(\RPP{w}{\e},\PP{w}{\e})$.
% %TODO Its proof, along with the existence of the maps $\E\N_w^\e$ can be found in the \fullversion.
%
% \begin{lemma}\label{lem:rips_homomorphism_left}
%   For $w\in\R$ and $\e \leq\varrho_D / 4$ let $\Lambda^\e\in\Hom(\ext{\PP{w}{\e}}, \ext{\PP{z}{2\e}})$ and $\rips\Lambda\in\Hom(\RPP{w}{2\e},\RPP{z}{4\e})$.
%   Then $\tilde{\Phi}(\Sigma_w^\e,\Sigma_z^{2\e})\in\Hom(\im~\Lambda^\e,\im~\rips\Lambda)$ and $\tilde{\Psi}(\Upsilon_w^{2\e},\Upsilon_z^{4\e})\in\Hom(\im~\rips\Lambda,\im~\Lambda^{2\e})$ are image module homomorphisms.
% \end{lemma}
%
% % Suppose $Q_w^\e$ surrounds $P^\e$ in $D$ for any $w\in\R$ and $\e > 0$.
% % Lemma~\ref{lem:surround_and_cover} can be extended to give homomorphisms $\DD{w-c\e}\xrightarrow{F}\E\PP{w}{\e}\xrightarrow{M}\DD{w+c\e}$ of degree $c\e$ induced by inclusions.
% Suppose $Q_{\omega-2c\delta}^\delta$ surrounds $P^\delta$ in $D$ and $D\setminus B_\omega\subseteq P^\delta$.% for $\zeta\geq 2\delta$.
% Then, because $f$ is $c$-Lipschitz, $B_{\omega-3c\delta}\cap P^\delta\subseteq Q_{\omega-2c\delta}^\delta$ and $B_\omega\cap P^\delta\subseteq Q_{\omega+c\delta}^{2\delta}$.
% Similarly, $Q_{\omega-2c\delta}^{2\delta}\subseteq B_\omega$ and $Q_{\omega+c\delta}^{4\delta}\subseteq B_{\omega+5c\delta}$.
% Therefore, by Lemma~\ref{lem:surround_and_cover}
% \[ B_{\omega-3c\delta}\subseteq \E Q_{\omega-2c\delta}^\delta\subseteq\E Q_{\omega-2c\delta}^{2\delta}\subseteq B_\omega
%   \subseteq \E Q_{\omega+c\delta}^{2\delta}\subseteq \E Q_{\omega+c\delta}^{4\delta}\subseteq B_{\omega+5c\delta}.\]
% We have the following commutative diagrams of persistence modules where all maps are induced by inclusions.
% Proof that inclusions given by Lemma~\ref{lem:surround_and_cover} extend to maps $(F, G)$ and $(M, N)$ of persistence modules can be found in the \fullversion.\\
%
% \begin{subequations}
%   \begin{minipage}{0.4\linewidth}
%     \begin{equation}\label{eq:partial_left}
%       \begin{tikzcd}
%         \DD{\omega-3c\delta} \arrow{r}{\Gamma}\arrow{d}{F} &
%         \DD{\omega} \arrow{d}{G}\\
%         %
%         \E\PP{\omega-2c\delta}{\delta}\arrow{r}{\Lambda} &
%         \E\PP{\omega+c\delta}{2\delta}
%       \end{tikzcd}
%     \end{equation}
%   \end{minipage}
%   \begin{minipage}{0.4\linewidth}
%     \begin{equation}\label{eq:partial_right}
%       \begin{tikzcd}
%         \E\PP{\omega-2c\delta}{2\delta} \arrow{r}{\Lambda'}\arrow{d}{M} &
%         \E\PP{\omega+c\delta}{4\delta}\arrow{d}{N}\\
%         %
%         \DD{\omega} \arrow{r}{\Pi} &
%         \DD{\omega+5c\delta}
%       \end{tikzcd}
%     \end{equation}
%   \end{minipage}
% \end{subequations}\vspace{2ex}
%
% In the following let $\rips\Lambda\in\Hom(\RPP{\omega-2c\delta}{2\delta},\RPP{\omega+c\delta}{4\delta})$ be induced by inclusion.
% Clearly, $\Phi(F, G)$ is an image module homomorphism of degree $2c\delta$ and $\Psi(M, N)$ is an image module homomorphism of degree $4c\delta$.
% By Lemma~\ref{lem:rips_homomorphism_left} we have image module homomorphisms $\tilde{\Phi}(\Sigma_{\omega-2c\delta}^\delta, \Sigma_{\omega+c\delta}^{2\delta})$ and $\tilde{\Psi}(\Upsilon_{\omega-2c\delta}^{2\delta}, \Upsilon_{\omega+c\delta}^{4\delta})$.
% Therefore, as the composition of image module homomorphisms are image module homomorphisms we have
% \[ \rips\Phi := \tilde{\Phi}\circ\Phi\in\Hom^{2c\delta}(\im~\Gamma,\im~\rips\Lambda)\ \text{ and }\ \rips\Psi :=\Psi\circ\tilde{\Psi}\in\Hom^{4c\delta}(\im~\rips\Lambda, \im~\Pi).\]
% % given by the compositions
% % \[ \rips\Phi(\rips F, \rips G) := (\Sigma_{\omega-2c\delta}^\delta\circ F, \Sigma_{\omega+c\delta}^{2\delta}\circ G)\ \text{ and }\ \rips\Psi(\rips M, \rips N) := (M\circ \Upsilon_{\omega-2c\delta}^{2\delta}, N\circ\Upsilon_{\omega+c\delta}^{4\delta}).\]

% Because all maps are induced by inclusions, or commute with maps induced by inclusions it can be shown that $\rips \Phi_{\rips M}$ is a partial $2c\delta$-interleaving of image modules and $\rips \Psi_{\rips G}$ is a partial $4c\delta$-interleaving of image modules by a straightforward diagram chasing argument.
% Proof of these facts can be found in the \fullversion.

The partial interleavings given by Lemma~\ref{lem:partial_interleaving}, along with assumptions that imply $\im(\DD{\omega-3c\delta}\to \DD{\omega+5c\delta})\cong \DD{\omega}$, provide the proof of Theorem~\ref{thm:interleaving_main_2} by Lemma~\ref{thm:interleaving_main}.

\begin{theorem}\label{thm:interleaving_main_2}
  Let $\X$ be a $d$-manifold, $D\subset\X$ and $f : D\to\R$ be a $c$-Lipschitz function.
  Let $\omega\in\R$, $\delta < \varrho_D/4$ be constants such that $B_{\omega-3c\delta}$ surrounds $D$ in $\X$.
  Let $P\subset D$ be a finite subset and suppose $\hom_k(B_{\omega-3c\delta}\hookrightarrow B_\omega)$ is surjective and $\hom_k(B_\omega\hookrightarrow B_{\omega+5c\delta})$ is an isomorphism for all $k$.

  If $D\setminus B_\omega\subseteq P^\delta$ and $Q_{\omega-2c\delta}^\delta$ surrounds $P^\delta$ in $D$ then the $k$th persistent homology module of $\{\rips^{2\delta}(P\subi{\omega-2c\delta}{\alpha}, Q_{\omega-2c\delta})\hookrightarrow \rips^{4\delta}(P\subi{\omega+c\delta}{\alpha}, Q_{\omega+c\delta})\}_{\alpha\in\R}$ is $4c\delta$-interleaved with that of $\{(D\subi{\omega}{\alpha}, B_\omega)\}_{\alpha\in\R}$.
\end{theorem}
\begin{proof}
  Let $\Lambda \in\Hom(\RPP{\omega-2c\delta}{2c\delta}, \RPP{\omega+c\delta}{4c\delta})$, $\Gamma\in\Hom(\DD{\omega-3c\delta}, \DD{\omega})$, and $\Pi\in\Hom(\DD{\omega},\DD{\omega+5c\delta})$ be induced by inclusions.
  Because $\delta < \varrho_D/4$, $D\setminus B_\omega\subseteq P^\delta$ and $Q_{\omega-2c\delta}^\delta$ surrounds $P^\delta$ in $D$ we have a partial $2c\delta$ interleaving $\Phi^*\in\Hom^{2c\delta}(\im~\Gamma, \im~\Lambda)$ and a partial $4c\delta$ interleaving $\Psi^*\in\Hom^{4c\delta}(\im~\Lambda, \im~\Pi)$ by Lemma~\ref{lem:partial_interleaving}.
  As we have assumed that $\hom_k(B_{\omega-3c\delta}\hookrightarrow B_\omega)$ is surjective and $\hom_k(B_\omega)\cong\hom_k(B_{\omega+5c\delta})$ the five-lemma implies $\gamma_\alpha$ is surjective and $\pi_\alpha$ is an isomorphism (and therefore injective) for all $\alpha$.
  So $\Gamma$ is an epimorphism and $\Pi$ is a monomorphism, thus $\im~\Lambda$ is $4c\delta$-interleaved with $\DD{\omega}$ by Lemma~\ref{thm:interleaving_main} as desired.
\end{proof}


\section{Approximation of the Truncated Diagram}\label{sec:truncations}
% !TeX root = ../../main_socg.tex

% In this section we consider the meaning of the $k$th persistent (relative) homology of a function $f : D\to \R$ modulo a fixed sublevel set $B_\omega := f^{-1}((-\infty,\omega])$.
% Unlike previous work~\cite{cohen09extending} we do not consider the persistent relative homology of $D$ modulo the sublevel set filtration $\{B_\alpha\}_{\alpha\in\R}$.
% Instead, we are interested in the role of a specific sublevel set $B_\omega$ in the context of the long exact sequences of $f$ modulo $B_\omega$ throughout the sublevel set filtration.

We will relate the relative persistence diagram that we have approximated in the previous section to a truncation of the full diagram.
% Recall that for fixed $\omega\in\R$ the \textbf{relative diagram} refers to the persistence diagram associated with the filtration $\{(D\subi{\omega}{\alpha}, B_\omega)\}_{\alpha\in\R}$.
% The \emph{full} diagram refers to that of the sublevel set filtration $\{B_\alpha\}_{\alpha\in\R}$ and the \textbf{truncated diagram (module)} refers to the subdiagram consisting of features born \emph{after} $\omega$.
% Naturally, the relative and truncated persistence modules refer to the persistence modules of associated with these diagrams.
Let $\LL^k$ denote the $k$th persistent homology module of the sublevel set filtration $\{B_\alpha\}_{\alpha\in\R}$.
As in the previous section, let $\DD{\omega}^k$ denote the $k$th persistent (relative) homology module of $\{(D\subi{\omega}{\alpha}, B_\omega)\}_{\alpha\in\R}$.
Throughout we will assume that we are taking homology in a field $\FF$ and that the homology groups $\hom_k(B_\alpha)$ and $\hom_k(D\subi{\omega}{\alpha}, B_\omega)$ are finite dimensional vector spaces for all $k$ and $\alpha\in\R$.
We will use the interval decomposition of $\LL^k$ to give a decomposition of the relative module $\DD{\omega}^k$ in terms of a \emph{truncation} of $\LL^k$.
Recall, the \emph{truncated diagram} is defined to be that of $\LL^k$ consisting only of those features born after $\omega$.
For fixed $\omega\in\R$ we will define the truncation $\T^k_\omega$ of $\LL^k$ in terms of the intervals decomposing $\LL^k$ that are in $[\omega, \infty)$.

% We find that the $k$th persistent (relative) homology of a function relative to a fixed sublevel set $B_\omega$ is equal to the submodule of features born after $\omega$ with additional infinite $k$-dimensional features which are paired with $(k-1)$-dimensonal features that are born before $\omega$ and die after $\omega$ in the full diagram.
% Unlike the persistent homology of the restriction $f\rest_{D\setminus B_\omega}$ this approach leaves features of the full diagram that are born after $\omega$ unchanged.
%
% For lack of a better analogy, this has the effect of ``quarantining'' the persistent homology of the function below $\omega$.
% Our hypothesis is that, given the persistent homology of the function up to $\omega$, one can recover the full diagram by pairing specific infinite $(k-1)$-dimensional features of $f\rest_{B_\omega}$ with specific infinite $k$-dimensional features of $f\rest_{D\setminus B_\omega}$ modulo $B_\omega$ via the long exact sequence(s) of pairs $(D\subi{\omega}{\alpha}, B_\omega)$.

\paragraph*{Truncated Interval Modules}

For an interval $I = [s,t)\subseteq \R$ let $I_+ := [t,\infty)$ and $I_- := (-\infty, s]$.
For $\omega\in\R$ let $\FF_{\omega}^I$ denote the interval module consisting of vector spaces $\{F\subi{\omega}{\alpha}^I\}_{\alpha\in\R}$ and linear maps $\{f\subi{\omega}{\alpha,\beta}^I : F\subi{\omega}{\alpha}^I\to F\subi{\omega}{\beta}^I\}_{\alpha\leq\beta}$ where
\[ F\subi{\omega}{\alpha}^I := \begin{cases} F_\alpha^I&\text{ if } \omega\in I_-\\ 0&\text{ otherwise,}\end{cases}\ \text{ and }\ \ f\subi{\omega}{\alpha,\beta}^I := \begin{cases} f_{\alpha,\beta}^I&\text{ if } \omega\in I_-\\ 0&\text{ otherwise.}\end{cases}\]
For a collection $\I$ of intervals let $\I_\omega := \{I\in\I\mid \omega\in I\}$.


% % \subsection{Decomposing the Persistent Relative Homology Module of Function Modulo a Sub-levelset}
% % \paragraph{
% \subsection{Interval Decomposition of the Relative Module}
%
% In the following we will assume that, for $\omega\in\R$ and taking homology in a field $\FF$, the homology groups $\hom_k(B_\alpha)$ and $\hom_k(D\subi{\omega}{\alpha}, B_\omega)$ are finite dimensional vector spaces for all $k$ and $\alpha\in\R$.
% Let $\LL^k$ denote the $k$th persistent homology module of the sublevel set filtration $\{B_\alpha\}_{\alpha\in\R}$ of $f$.
% Because all homology groups are finite dimensional we can decompose $\LL^k$ into a direct sum of interval modules
% \[ \LL^k = \bigoplus_{I\in\I^k} \FF^I.\] %\ \text{ and }\ \DD{\omega}^k = \bigoplus_{J\in\J^k} \FF^J\]
% for some collection $\I^k$ of intervals $I\subseteq \R$ for all $k$.
% As in the previous section, let $\DD{\omega}^k$ denote the $k$th persistent (relative) homology module of $\{(D\subi{\omega}{\alpha},B_\omega)\}_{\alpha\in\R}$, the sublevel set filtration of $f$ modulo $B_\omega$.
% % Moreover, for all $\alpha\in\R$ we have
% % \[ \hom_k(B_\alpha) = \bigoplus_{I\in \I^k}F_\alpha^I.\]%s\ \text{ and }\ \hom_k(D\subi{\omega}{\alpha}, B_\omega) = \bigoplus_{J\in\J^k} F_\alpha^J.\]
% % the sublevel set filtration $\{B_\alpha\}_{\alpha\in\R}$.
% % The following lemma decomposes the relative module $\DD{\omega}^k$ into the direct sum of the truncated module and a submodule consisting of infinite $k$-dimensional features that correspond to finite $(k-1)$-dimensional features of $\LL^k$ that are born before $\omega$ and die after $\omega$.

\begin{lemma}\label{lem:decomposition}
  % If $\I^k, \I^{k-1}$ decompose $\LL^k$ and $\LL^{k-1}$ then
  % \[\DD{\omega}^k = \bigoplus_{I\in\I^k \cup \I^{k-1}_+} \FF_{\omega}^I = \LL_{\omega}^k \oplus \bigoplus_{I\in \I^{k-1}} \FF_{\omega}^{I_+}.\]
  % Let $f : D\to \R$ be a function and let $B_\alpha := f^{-1}((-\infty,\alpha])$ for all $\alpha\in\R$.
  Suppose $\I^k$ and $\I^{k-1}$ are collections of intervals that decompose $\LL^k$ and $\LL^{k-1}$, respectively.
  Then for all $k$ the $k$th persistent homology module of $\{(D\subi{\omega}{\alpha}, B_\omega)\}_{\alpha\in\R}$ is equal to
  \[\bigoplus_{I\in\I^k} \FF_\omega^I \oplus \bigoplus_{I\in \I_\omega^{k-1}} \FF^{I_+}.\]
\end{lemma}
% \begin{proof}
%   Suppose $\alpha\leq\omega$.
%   So $\hom_k(D\subi{\omega}{\alpha}, B_\omega) = 0$ as $D\subi{\omega}{\alpha} = B_\omega\cup B_\alpha$ and $\T^k_\omega = 0$ as $F_\alpha^I = 0$ for any $I\in \I^k$ such that $\omega\in I_-$.
%   Moreover, $\omega\in I$ for all $I\in \I_\omega^{k-1}$, thus $F_\alpha^{I_+} = 0$ for all $\alpha\leq\omega$.
%   So it suffices to assume $\omega < \alpha$.
%
%   Consider the long exact sequence of the pair $\hom_k(D\subi{\omega}{\alpha}, B_\omega) = \hom_k(B_\alpha, B_\omega)$
%   \[ \ldots\to \hom_k(B_\omega)\xrightarrow{p_\alpha^k} \hom_k(B_\alpha)\xrightarrow{q_\alpha^k}\hom_k(D\subi{\omega}{\alpha}, B_\omega)\xrightarrow{r_\alpha^k} \hom_{k-1}(B_\omega)\xrightarrow{p_\alpha^{k-1}}\hom_{k-1}(B_\alpha)\to\ldots\]
%   where $\hom_k(B_\alpha) = \bigoplus_{I\in \I^k}F_\alpha^I$, $\hom_k(B_\omega) = \bigoplus_{I\in \I^k}F_\omega^I$, and $p_\alpha^k = \displaystyle\bigoplus_{I\in\I^k} f_{\omega,\alpha}^I$.
%
%   % By exactness $\ker~p_\alpha^k = \im~p_\alpha^k = \bigoplus_{I\in\I^k}\im~f_{\omega,\alpha}^I = \bigoplus_{I\in\I^k \mid \omega\in I} F_\alpha^I.$
%   % By exactness $\ker~r_\alpha^k = \im~q_\alpha^k \cong \hom_k(B_\alpha) / \ker~q_\alpha^k$ $ where the image of
%   % We first note that $\im~p_\alpha^k$ is equal to the direct sum of images $\im~f_{\omega,\alpha}^I$.
%   % By the definition of $F_\alpha^I$ we know $\im~f_{\omega,\alpha}^I$ is $F_\alpha^I$ if $\omega\in I$, 0 otherwise.
%   Noting that $\im~q_\alpha^k \cong \hom_k(B_\alpha) / \ker~q_\alpha^k$ where $\ker~q_\alpha^k = \im~p_\alpha^k$ by exactness we have $\ker~r_\alpha^k \cong \hom_k(B_\alpha) / \im~p_\alpha^k$.
%   By the definition of $F_\alpha^I$ and $f_{\omega,\alpha}^I$ we know $\im~f_{\omega,\alpha}^I$ is $F_\alpha^I$ if $\omega\in I$ and 0 otherwise.
%   As $\im~p_\alpha^k$ is equal to the direct sum of images $\im~f_{\omega,\alpha}^I$ over $I\in\I^k$ it follows that $\im~p_\alpha^k$ is the direct sum of those $F_\alpha^I$ over those $I\in\I^k$ such that $\omega\in I$.
%   Now, because $\hom_k(B_\alpha) = \bigoplus_{I\in \I^k}F_\alpha^I$ and each $F_\alpha^I$ is either 0 or $\FF$ the quotient $\hom_k(B_\alpha) / \im~p_\alpha^k$ is the direct sum of those $F_\alpha^I$ such that $\omega\notin I$.
%   Therefore, by the definition of $F\subi{\omega}{\alpha}^I$ we have
%   \[ \ker~r_\alpha^k = \bigoplus_{I\in\I_\omega^k} F\subi{\omega}{\alpha}^I.\]
%   % Thus, \[\ker~r_\alpha^k \cong \hom_k(B_\alpha) / \ker~q_\alpha^k = \bigoplus_{I\in \I^k\mid \omega\notin I} F_\alpha^I = \bigoplus_{I\in\I^k} F\subi{\omega}{\alpha}^I.\]
%
%   Similarly, $\im~r_\alpha^k = \ker~p_\alpha^{k-1}$ by exactness where $\ker~p_\alpha^{k-1}$ is the direct sum of kernels $\ker~f_{\omega,\alpha}^I$ over $I\in\I^{k-1}$.
%   By the definition of $F_\alpha^I$ and $f_{\omega,\alpha}^I$ we know that $\ker~f_{\omega,\alpha}^I$ is $F_\alpha^I$ if $\omega\notin I$ and $0$ otherwise.
%   % If $\ker~f_{\omega,\alpha}^I = 0$ then either $\alpha\in I$ and $\omega\notin I$, $\alpha\notin I$ and $\omega \in I$, or $\alpha\notin I$ and $\omega\notin I$.
%   % So it suffices to consider $I\in \I_\omega^{k-1}$ as $\ker~f_{\omega,\alpha}^I = 0$ for any $I\in \I^{k-1}$ such that $\omega\notin I$.
%   Noting that $\ker~f_{\omega,\alpha}^I = 0$ for any $I\in \I^{k-1}$ such that $\omega\notin I$ it suffices to consider only those $I\in \I_\omega^{k-1}$.
%   % Recalling that $I_+ = [t,\infty)$ for $I = [s,t)$
%   It follows that $\ker~f_{\omega,\alpha}^I = F_\alpha^{I_+}$ for any $I$ containing $\omega$ as $\omega < \alpha$.
%   Therefore,
%   \[\im~r_\alpha^k = \bigoplus_{I\in\I^{k-1}} F_\alpha^{I_+}.\]
%
%   We have the following split exact sequence associated with $r_\alpha^k$
%   % \[ 0\to \ker~r_\alpha^k\xrightarrow{\phi_\alpha^k}\bigoplus_{J\in\J^k} F_\alpha^J\xrightarrow{\psi_\alpha^k}\im~r_\alpha^k\to 0.\]
%   \[ 0\to \ker~r_\alpha^k\to \hom_k(D\subi{\omega}{\alpha}, B_\omega)\to\im~r_\alpha^k\to 0.\]
%   The desired result follows from the fact that for all $\alpha\in\R$
%   % \[ \bigoplus_{J\in\J^k} F_\alpha^J \cong \ker~r_\alpha^k\oplus \im~r_\alpha^k
%   %   \cong\left(\bigoplus_{I\in\I^k} F\subi{\omega}{\alpha}^I\right)\oplus\left(\bigoplus_{I\in\I^{k-1}} F\subi{\omega}{\alpha}^{I_+}\right).\]
%   \begin{align*}
%     \hom_k(D\subi{\omega}{\alpha}, B_\omega) &\cong \ker~r_\alpha^k\oplus \im~r_\alpha^k\\
%       &=\bigoplus_{I\in\I^k} F\subi{\omega}{\alpha}^I\oplus \bigoplus_{I\in\I_\omega^{k-1}} F_\alpha^{I_+}.
%       % &\cong\left(\bigoplus_{I\in\I^k} F\subi{\omega}{\alpha}^I\right)\oplus\left(\bigoplus_{I\in\I_\omega^{k-1}} F_\alpha^{I_+}\right).
%   \end{align*}
%     % thus $\DD{\omega}^k = \T^k_\omega \oplus \bigoplus_{I\in \I_\omega^{k-1}} \FF^{I_+}
% \end{proof}


% Letting $\AA_\omega^k := \displaystyle\bigoplus_{I\in\I^k} \FF_\omega^I$ and $\BB_\omega^k := \displaystyle\bigoplus_{I\in\I^k} \FF_\omega^{I_+}$ for all $k$ we have
% \[ \DD{\omega}^k \cong \AA_\omega^k\oplus \BB_\omega^{k-1}.\]

% \begin{theorem}
%   Let $\X$ be an orientable $d$-manifold and let $D$ be a compact subset of $\X$ with strong convexity radius $\varrho_D > \delta$.
%   Let $f : D\to\R$ be $c$-Lipschitz function and let $\omega\in\R$ and $2\delta\leq\zeta < \varrho_D/2$ be constants such that $B_{\omega - c(\zeta +\delta)}$ surrounds $D$ in $\X$.
%   Let $P\subset \intr_\X(D)$ and suppose $P^\delta$, $Q_{\omega-c\zeta}^\delta$, and $Q_{\omega+c\delta}^\delta$ satisfy the assumptions of Lemma~\ref{lem:duality_apply}.
%   Suppose $\hom_k(B_{\omega-c(\delta+\zeta)}\hookrightarrow B_\omega)$ and $\hom_k(B_\omega)\cong\hom_k(B_{\omega+c(\delta+2\zeta)})$ for all $k$.
%
%   If
%   \[\rk~\hom_d(\rips^\delta(P, Q_{\omega -c\zeta})\hookrightarrow \rips^{2\delta}(P, Q_{\omega+c\delta})) \geq \dim~\hom_0(\rips^\delta(P\setminus Q_{\omega-c\zeta}))\]
%   then the image module
%   \[ \im~(\RPP{\omega-c\zeta}{2\delta, k}\to \RPP{\omega+c\delta}{2\zeta, k})\]
%   is $2c\zeta$-interleaved with
%   \[ \LL_{\omega}^k \oplus \bigoplus_{I\in \I^{k-1}} \FF_{\omega}^{I_+}.\]
% \end{theorem}

\subsection*{Main Theorem}

Let $\I^k$ denote the decomposing intervals of $\LL^k$ for all $k$.
Let
\[\T_\omega^k := \bigoplus_{I\in\I^k} \FF_\omega^I\]
denote the \textbf{$\omega$-truncated $k$th persistent homology module} of $\LL^k$ and
\[ \LL_\omega^{k-1} := \bigoplus_{I\in \I_\omega^{k-1}} \FF^{I_+}.\]
denote the submodule of $\DD{\omega}^k$ consisting of intervals $[\beta,\infty)$ corresponding to features $[\alpha,\beta)$ in $\LL^{k-1}$ such that $\alpha\leq\omega <\beta$.
% % denote the submodule of $\DD{\omega}^k$ consisting of infinite $k$-dimensional features that correspond to finite $(k-1)$-dimensional features of $\LL^k$ that are born before $\omega$ and die after $\omega$.
Now, by Lemma~\ref{lem:decomposition} the $k$th persistent (relative) homology module of $\{(D\subi{\omega}{\alpha}, B_\omega)\}_{\alpha\in\R}$ is $\DD{\omega}^k = \T_\omega^k\oplus \LL_\omega^{k-1}.$
Our main theorem combines this decomposition with our coverage and interleaving results of Theorems~\ref{thm:algo_tcc} and~\ref{thm:interleaving_main_2}.% as a method for certified approximation of the truncated persistence diagram.\textbf{TODO: GROSS}

\begin{lemma}\label{lem:dual_ass}
  Let $\X$ be an orientable $d$-manifold and suppose $(D, B)$ and $(D, B')$ are compact, locally contractible, surrounding pairs in $\X$ such that $\hom_d(D, B)$ and $\hom_d(D, B')$ are finitely generated.

  If $\hom_{d-1}(B\hookrightarrow B')$ is surjective then $\hom_0(D\setminus B'\hookrightarrow D\setminus B)$ is injective.
  If $\hom_{d-1}(B\hookrightarrow B')$ is injective then $\hom_0(D\setminus B'\hookrightarrow D\setminus B)$ is surjective.
\end{lemma}
\begin{proof}
  If $\hom_{d-1}(B\hookrightarrow B')$ is surjective for all $k$ then $\hom_d((D, B)\hookrightarrow (D, B'))$ is surjective by the five lemma.
  Taking homology with coefficients in a field $\FF$ we can dualize to obtain an \emph{injective} map $\Hom(\hom_d(D,B'), \FF)\to \Hom(\hom_d(D, B), \FF)$.
  Therefore, because we are taking coefficients in a field, we have an injective map $\hom^d(D,B')\to \hom^d(D, B)$ by the Universal Coefficient Theorem.

  Because $(D, B)$ and $(D,B')$ are compact and locally connected we can apply Alexander Duality to obtain an injective map $\hom_0(\X\setminus B', \X\setminus D)\to\hom_0(\X\setminus B, \X\setminus D)$.
  Because $B,B'$ surround $D$ in $\X$ it follows that $\hom_0(D\setminus B'\hookrightarrow D\setminus B)$ is injective.
  It can be shown $\hom_{d-1}(B\hookrightarrow B')$ injective implies $\hom_0(D\setminus B'\hookrightarrow D\setminus B)$ surjective by a similar argument.
\end{proof}

\begin{theorem}\label{thm:main}
  Let $\X$ be an orientable $d$-manifold and let $D$ be a compact subset of $\X$.
  Let $f : D\to\R$ be a $c$-Lipschitz function and $\omega\in\R$, $\delta < \varrho_D/4$ be constants such that $P\subset D$ is a $(\delta, 2\delta,\omega)$-sublevel sample of $f$ and $B_{\omega-3c\delta}$ surrounds $D$ in $\X$.
  % Let $P$ be a finite subset of $D$ such that $(P, Q_{\omega-2c\delta})$ and $(P, Q_{\omega+c\delta})$ are $\delta$-good samples of $(D, B_\omega)$.
  % Let $P\subset \intr_\X(D)$ and suppose $D\setminus P^\delta$, $D\setminus Q_{\omega-2c\delta}^\delta$, and $D\setminus Q_{\omega+c\delta}^\delta$ are locally path connected.

  Suppose $\hom_k(B_{\omega-3c\delta}\hookrightarrow B_\omega)$ is surjective and $\hom_k(B_\omega\hookrightarrow B_{\omega+5c\delta})$ is an isomorphism for all $k$.
  If $\rk~\hom_d(\rips^\delta(P, Q_{\omega - 2c\delta})\hookrightarrow \rips^{2\delta}(P, Q_{\omega+c\delta})) \geq \dim~\hom_0(\rips^\delta(P\setminus Q_{\omega-2c\delta}))$ then the $k$th (relative) homology module of $\{\rips^{2\delta}(P\subi{\omega-2c\delta}{\alpha}, Q_{\omega-2c\delta})\hookrightarrow \rips^{4\delta}(P\subi{\omega+c\delta}{\alpha}, Q_{\omega+c\delta})\}_{\alpha\in\R}$ is $4c\delta$-interleaved with $\T_{\omega}^k \oplus \LL_\omega^{k-1}$: the $k$th persistent homology module of $\{(D\subi{\omega}{\alpha}, B_\omega)\}_{\alpha\in\R}$.
   % that of $\{(D\subi{\omega}{\alpha}, B_\omega)\}_{\alpha\in\R}$.
\end{theorem}
\begin{proof}
  If $\hom_k(B_{\omega-3c\delta}\hookrightarrow B_\omega)$ is surjective for all $k$ then, in particular, $\hom_{d-1}(B_{\omega-3c\delta}\hookrightarrow B_\omega)$ is surjective.
  If $\hom_k(B_{\omega-3c\delta}\hookrightarrow B_\omega)$ is surjective for all $k$ then, in particular, $\hom_{d-1}(B_{\omega-3c\delta}\hookrightarrow B_\omega)$ is surjective.
  Because $B_{\omega-3c\delta}, B_\omega$ are closed in $D$, and $D$ is compact, $(D, B_{\omega-3c\delta})$ and $(D,B_\omega)$ are compact pairs.
  If our pairs are locally contractible then $\hom_0(D\setminus B_\omega\hookrightarrow D\setminus B_{\omega-3c\delta})$ is injective and $\hom_0(D\setminus B_{\omega-5c\delta}\hookrightarrow D\setminus B_\omega)$ is surjective by Lemma~\ref{lem:dual_ass}.

  Because $\rk~\hom_d(\rips^\delta(P, Q_{\omega - 2c\delta})\hookrightarrow \rips^{2\delta}(P, Q_{\omega+c\delta})) \geq \dim~\hom_0(\rips^\delta(P\setminus Q_{\omega-2c\delta}))$ and $P\subset D$ is a $(\delta, 2\delta,\omega)$-sublevel sample of $f$ we have $D\setminus B_\omega\subseteq P^\delta$ and $Q_{\omega-2c\delta}^\delta$ surrounds $P^\delta$ in $D$ by Theorem~\ref{thm:algo_tcc}.
  So the persistent homology modules of $\{\rips^{2\delta}(P\subi{\omega-2c\delta}{\alpha}, Q_{\omega-2c\delta})\hookrightarrow \rips^{4\delta}(P\subi{\omega+c\delta}{\alpha}, Q_{\omega+c\delta})\}_{\alpha\in\R}$ are $4c\delta$ interleaved with those of $\{(D\subi{\omega}{\alpha}, B_\omega)\}_{\alpha\in\R}$ by Theorem~\ref{thm:interleaving_main_2}, and therefore $\T_{\omega}^k \oplus \LL_\omega^{k-1}$ by Lemma~\ref{lem:decomposition}.
\end{proof}


\section{Experiments}\label{sec:experiments}
\input{src/3-discussion/experiments-socg}

\section{Conclusion}
% !TeX root = ../main_socg.tex

We have extended the Topological Coverage Criterion to the setting of Topological Scalar Field Analysis.
By defining the boundary in terms of a sublevel set of a scalar field we provide an interpretation of the TCC that applies more naturally to data coverage.
We then showed how the assumptions and machinery of the TCC can be used to approximate the persistent homology of the scalar field relative to a static sublevel set.
This relative persistent homology is shown to be related to a truncation of that of the scalar field as whole, and therefore provides a way to approximate a part of its persistence diagram in the presence of un-verified data.

There are a number of unanswered questions and directions for future work.
From the theoretical perspective, our understanding of duality limited us in providing a more elegant extension of the TCC.
A better understanding of when and how duality can be applied would allow us to give a more rigorous statement of our assumptions.
Moreover, as duality plays a central role in the TCC it is natural to investigate its role in the analysis of scalar fields.
This would not only allow us to apply duality to persistent homology~\cite{desilva11duality}, but also allow us to provide a rigorous comparison between the relative approach and the persistent homology of the superlevel set filtration and explore connections with Extended Persistence~\cite{cohen09extending}.

From a computational perspective, we interested in exploring how to recover the full diagram as discussed in Section~\ref{sec:experiments}.
Our statements in terms of sublevel sets can be generalized to disjoint unions of sub and superlevel sets, where coverage is confirmed in an \emph{interlevel} set.
This, along with a better understanding of the relationship between sub and superlevel sets could lead to an iterative approach in which the persistent homology of a scalar field is constructed as data becomes available.
We are also interested in finding efficient ways to compute the image persistent (relative) homology that vary in both scalar and scale.

The problem of relaxing our assumptions on the boundary can be approached from both a theoretical and computational perspective.
Ways to avoid the isomorphism we require could be investigated in theory, and the interaction of relative persistent homology and the Persistent Nerve Lemma may be used tighten our assumptions.
We would also like to conduct a more rigorous investigation on the effect of these assumptions in practice.


\bibliography{bibliography}

\appendix

\section{Omitted Proofs}\label{apx:omit}
% !TeX root = ../../main_socg.tex

\begin{proof}[Proof of Lemma~\ref{lem:coverage}]
  This proof is in two parts.
  \begin{description}
    \item[$\ell$ injective $\implies$ $D\setminus B\subseteq U$] Suppose, for the sake of contradiction, that $p$ is injective and there exists a point $x\in (D\setminus B)\setminus U$.
      Because $B$ surrounds $D$ in $X$ the pair $(D\setminus B, \overline{D})$ forms a separation of $\overline{B}$.
      Therefore, $\hom_0(\overline{B})\cong \hom_0(D\setminus B)\oplus \hom_0(\overline{D})$ so
      \[ \hom_0(\overline{B}, \overline{D})\cong \hom_0(D\setminus B). \]
      So $[x]$ is non-trivial in $\hom_0(\overline{B},\overline{D})\cong \hom_0(D\setminus B)$ as $x$ is in some connected component of $D\setminus B$.
      So we have the following sequence of maps induced by inclusions
      \[ \hom_0(\overline{B},\overline{D})\xrightarrow{f} \hom_0(\overline{B},\overline{D}\cup\{x\})\xrightarrow{g} \hom_0(\overline{V},\overline{U}).\]
      As $f[x]$ is trivial in $\hom_0(\overline{B},\overline{D}\cup\{x\})$ we have that $\ell[x] = (g\circ f)[x]$ is trivial, contradicting our hypothesis that $\ell$ is injective.
    \item[$\ell$ injective $\implies$ $V$ surrounds $U$ in $D$.] Suppose, for the sake of contradiction, that $V$ does not surround $U$ in $D$.
      Then there exists a path $\gamma : [0,1]\to\overline{V}$ with $\gamma(0)\in U\setminus V$ and $\gamma(1)\in D\setminus U$.
      As we have shown, $D\setminus B\subseteq U$, so $D\setminus B\subseteq U\setminus V$.

      Choose $x\in D\setminus B$ and $z\in \overline{D}$ such that there exist paths $\xi : [0,1]\to U\setminus V$ with $\xi(0) = x$, $\xi(1) = \gamma(0)$ and $\zeta : [0,1]\to \overline{D}\cup (D\setminus U)$ with $\zeta(0) = z$, $\zeta(1) = \gamma(1)$.
      $\xi, \gamma$ and $\zeta$ all generate chains in $C_1(\overline{V}, \overline{U})$ and $\xi + \gamma + \zeta = \gamma^*\in C_1(\overline{V}, \overline{U})$ with $\partial\gamma^* = x + z$.
      Moreover, $z$ generates a chain in $C_0(\overline{U})$ as $\overline{D}\subseteq\overline{U}$.
      So $x = \partial\gamma^* + z$ is a relative boundary in $C_0(\overline{V}, \overline{U})$, thus $\ell[x] = \ell[z]$ in $\hom_0(\overline{V}, \overline{L})$.
      However, because $B$ surrounds $D$, $[x]\neq [y]$ in $\hom_0(\overline{B}, \overline{D})$ contradicting our assumption that $\ell$ is injective.
    \end{description}
\end{proof}

\begin{proof}[Proof of Lemma~\ref{lem:assumption2}]
  Assume there exist $p,q \in P\setminus Q_{\omega-c\zeta}$ such that $p$ and $q$ are connected in $\rips^\delta(P\setminus Q_{\omega-c\zeta})$ but not in $D\setminus B_\omega$.
  So the shortest path from $p, q$ is a subset of $(P\setminus Q_{\omega-c\zeta})^\delta$.
  For any $x\in (P\setminus Q_{\omega-c\zeta})^\delta$ there exists some $p\in P$ such that $f(p) > \omega - c\zeta$ and $\dist(p,x) < \delta$.
  Because $f$ is $c$-Lipschitz
  \[ f(x)\geq f(p) - c\dist(x,p) > \omega - c(\delta+\zeta)\]
  so there is a path from $p$ to $q$ in $D\setminus B_{\omega-c(\delta+\zeta)}$, thus $[p] = [q]$ in $\hom_0(D\setminus B_{\omega-c(\delta+\zeta)})$.

  But we have assumed that $[p]\neq[q]$ in $\hom_0(D\setminus B_\omega)$, contradicting our assumption that $\hom_0(D\setminus B_\omega\hookrightarrow D\setminus B_{\omega-c(\delta+\zeta)})$ is injective, so any $p,q$ connected in $\rips^\delta(P\setminus Q_{\omega-c\zeta})$ are connected in $D\setminus B_\omega$.
  That is, $\dim~\hom_0(\rips^\delta(P\setminus Q_{\omega-c\zeta}))\geq \dim~\hom_0(D\setminus B_\omega)$.
\end{proof}

\subsection{Extensions}

\begin{proof}[Proof of Lemma~\ref{lem:surround_and_cover}]
  Note that $B'\setminus (D\setminus U) = B'\cap U\subseteq V$ implies $B'\subseteq V\sqcup(D\setminus U) = \ext{V}$.
  Moreover, because $V\subseteq B$ and $D\setminus B\subseteq U$ implies $D\setminus U \subset D\setminus (D\setminus B) = B$ we have that
  $\ext{V} = V\sqcup (D\setminus U) \subseteq B\cup (D\setminus U) = B.$
  So $B' \subseteq \ext{V}\subseteq B$ as desired.
\end{proof}

\begin{proof}[Proof of Lemma~\ref{lem:excision}]
  Because $V$ surrounds $U$ in $D$, $(U\setminus V, D\setminus U)$ is a separation of $D\setminus V$, a subspace of $D$.
  Recall, $(U\setminus V, D\setminus U)$ is a \emph{separation} of $D\setminus V$ in $D$ if $\cl_D(U\setminus V)\cap (D\setminus U) = \emptyset$ and $(U\setminus V)\cap \cl_D(D\setminus U) = \emptyset$ (see Munkres~\cite{munkres00topology}, Lemma 23.1).
  So $\cl_D(U\setminus V)\setminus U = \cl_D(U\setminus V) \cap (D\setminus U) = \emptyset$ which implies $\cl_D(U\setminus V)\subseteq U = \intr_D(U)$ as $U$ is open in $D$.
  Therefore, $\cl_D(D\setminus U) = D\setminus \intr_D(U)\subseteq D\setminus \cl_D(U\setminus V) = \intr_D(\ext{V})$
  % \begin{align*}
  %   \cl_D(D\setminus U) = D\setminus \intr_D(U)\subseteq D\setminus \cl_D(U\setminus V) = \intr_D(\ext{V})
  % \end{align*}
  % \begin{align*}
  %   \cl_D(D\setminus U) &= D\setminus \intr_D(U)\\
  %                       &\subseteq D\setminus \cl_D(U\setminus V)\\
  %                       &= \intr_D(D\setminus (U\setminus V))\\
  %                       &= \intr_D(\ext{V}).
  % \end{align*}
  so for all $k$ and any $A\subseteq D$ such that $\ext{V}\subset A$ we have $\hom_k(U\cap A, V) = \hom_k(A\setminus (D\setminus U), \ext{V}\setminus (D\setminus U)) \cong \hom_k(A, \ext{V})$
  % \begin{align*}
  %   \hom_k(U\cap A, V) &= \hom_k(A\setminus (D\setminus U), \ext{V}\setminus (D\setminus U))\\
  %     &\cong \hom_k(A, \ext{V})
  % \end{align*}
   by excision.
\end{proof}

\subsection{Image Modules}

\begin{proof}[Proof of Lemma~\ref{lem:image_composition}]
  Because $\Phi(F, G)$ is an image module homomorphism of degree $\delta$ we have $g_{\beta-\delta}\circ\gamma_{\alpha-\delta}[\beta-\alpha] = \lambda_\alpha[\beta-\alpha]\circ f_{\alpha-\delta}$.
  Similarly, $g_{\beta}'\circ\lambda_{\alpha}[\beta-\alpha] = \lambda_{\alpha +\delta'}'[\beta-\alpha]\circ f_{\alpha}'$.
  So $\Phi''(F'\circ F, G'\circ G)\in\Hom^{\delta+\delta'}(\im~\Gamma,\im~\Lambda')$ as
  \[ g_\beta'\circ (g_{\beta-\delta}\circ \gamma_{\alpha-\delta}[\beta-\alpha]) = (g_\beta'\circ \lambda_\alpha[\beta-\alpha])\circ f_{\alpha-\delta} =\lambda_{\alpha+\delta'}[\beta-\alpha]\circ f_\alpha'\circ f_{\alpha-\delta}\]
  for all $\alpha\leq\beta$.
\end{proof}

\begin{proof}[Proof of Lemma~\ref{thm:interleaving_main}]
  % For ease of notation let $\Phi$ denote $\Phi_M(F, G)$ and $\Psi$ denote $\Psi_G(M, N)$.
  %
  If $\Gamma$ is an epimorphism $\gamma_\alpha$ is surjective so $\Gamma_\alpha = V_\alpha$ and $\phi_{\alpha} = g_{\alpha}\rest_{\Gamma_\alpha} = g_\alpha$ for all $\alpha$.
  So $\im~\Gamma = \VV$ and $\Phi\in\Hom^\delta(\VV,\im~\Lambda)$.

  If $\Pi$ is a monomorphism then $\pi_\alpha$ is injective so we can define an isomorphism $\pi_\alpha^{-1} : \Pi_\alpha\to V_\alpha$ for all $\alpha$.
  Let $\Psi^*$ be defined as the family of linear maps $\{\psi_\alpha^* := \pi^{-1}_\alpha \circ \psi_\alpha : \Lambda_\alpha\to V_{\alpha+\delta}\}$.
  Because $\Psi$ is a partial $\delta$-interleaving of image modules, $n_\alpha\circ\lambda_\alpha = \pi_{\alpha+\delta}\circ m_\alpha$.
  So, because $\psi_\alpha = n_\alpha\rest_{\Lambda_\alpha}$ for all $\alpha$,
  \begin{align*}
    \im~\psi_\alpha^* = \im~\pi^{-1}_{\alpha+\delta}\circ\psi_\alpha = \im~\pi^{-1}_{\alpha+\delta}\circ (n_\alpha\circ\lambda_\alpha) = \im~\pi^{-1}_{\alpha+\delta}\circ (\pi_{\alpha+\delta}\circ m_\alpha) = \im~ m_\alpha.
  \end{align*}
  It follows that $\im~v_{\alpha+\delta}^{\beta+\delta}\circ\psi_\alpha^* = \im~v_{\alpha+\delta}^{\beta+\delta}\circ m_\alpha$

  Similarly, because $\Psi$ is a $\delta$-interleaving of image modules $n_\beta\circ t_\alpha^\beta\circ \lambda_\alpha = w_{\alpha+\delta}^{\beta+\delta}\circ\pi_{\alpha+\delta}\circ m_\alpha$.
  Moreover, because $\Pi$ is a homomorphism of persistence modules, $w_{\alpha+\delta}^{\beta+\delta}\circ\pi_{\alpha+\delta} = \pi_{\beta+\delta}\circ v_{\alpha+\delta}^{\beta+\delta}$, so $n_\beta\circ t_\alpha^\beta\circ \lambda_\alpha = \pi_{\beta+\delta}\circ v_{\alpha+\delta}^{\beta+\delta}\circ m_\alpha.$
  As $\psi_\beta\circ\lambda_\alpha^\beta = n_\beta\circ\lambda_\alpha^\beta = n_\beta\circ t_\alpha^\beta\rest_{\Lambda_\alpha}$ it follows
  \begin{align*}
    \im~\psi_\beta^*\circ\lambda_\alpha^\beta &= \im~\pi^{-1}_{\beta+\delta}\circ (n_\beta\circ t_\alpha^\beta\circ\lambda_\alpha)\\
      &= \im~\pi^{-1}_{\beta+\delta}\circ (\pi_{\beta+\delta}\circ v_{\alpha+\delta}^{\beta+\delta})\circ m_\alpha\\
      &= \im~v_{\alpha+\delta}^{\beta+\delta}\circ m_\alpha\\
      &= \im~v_{\alpha+\delta}^{\beta+\delta}\circ\psi_\alpha^*.
  \end{align*}
  So we may conclude that $\Psi^*\in\Hom^\delta(\im~\Lambda,\VV)$.

  So $\Phi\in\Hom^\delta(\VV,\im~\Lambda)$ and $\Psi_G^*\in\Hom^\delta(\im~\Lambda,\VV)$.
  As we have shown, $\im~\psi_{\alpha-\delta}^* = \im~m_{\alpha-\delta}$ so $\im~\phi_\alpha\circ\psi_{\alpha-\delta}^* = \im~\phi_\alpha\circ m_{\alpha-\delta}$.
  Moreover, because $\gamma_\alpha$ is surjective $\phi_\alpha = g_\alpha$ and, because $\Phi$ is a partial $\delta$-interleaving of image modules, $g_\alpha\circ m_{\alpha-\delta} = t_{\alpha-\delta}^{\alpha+\delta}\circ \lambda_{\alpha-\delta}$.
  As $\lambda_{\alpha-\delta}^{\alpha+\delta} = t_{\alpha-\delta}^{\alpha+\delta}\rest_{\im~\lambda_{\alpha-\delta}}$ it follows that the following diagram commutes as $\im~\phi_\alpha\circ\psi_{\alpha-\delta}^* = \im~\lambda_{\alpha-\delta}^{\alpha+\delta}$:
  \begin{equation}\label{dgm:interleaving1}
    \begin{tikzcd}
      & V_{\alpha}\arrow{dr}{\phi_\alpha} &\\
      %
      \Lambda_{\alpha-\delta}\arrow{rr}{\lambda_{\alpha-\delta}^{\alpha+\delta}}\arrow{ur}{\psi_{\alpha-\delta}^*} & &
      \Lambda_{\alpha+\delta}.
  \end{tikzcd}\end{equation}

  Finally, $\psi_\alpha^*\circ\phi_\alpha = \pi_{\alpha+\delta}^{-1}\circ n_\alpha\circ g_{\alpha-\delta}$ where, because $\Psi$ is a partial $\delta$-interleaving of image modules, $n_\alpha\circ g_{\alpha-\delta} = w_{\alpha-\delta}^{\alpha+\delta}\circ\pi_{\alpha-\delta}$.
  Because $\Pi$ is a homomorphism of persistence modules $w_{\alpha-\delta}^{\alpha+\delta}\circ \pi_{\alpha-\delta} = \pi_{\alpha+\delta}\circ v_{\alpha-\delta}^{\alpha+\delta}$.
  Therefore,
  \begin{align*}
    \psi_\alpha^*\circ\phi_{\alpha-\delta} = \pi_{\alpha+\delta}^{-1}\circ n_\alpha\circ g_{\alpha-\delta} = \pi_{\alpha+\delta}^{-1}\circ (\pi_{\alpha+\delta}\circ v_{\alpha-\delta}^{\alpha+\delta}) = v_{\alpha-\delta}^{\alpha+\delta}
  \end{align*}
  so the following diagram commutes
  \begin{equation}\label{dgm:interleaving2}
    \begin{tikzcd}
      V_{\alpha-\delta}\arrow{rr}{v_{\alpha-\delta}^{\alpha+\delta}}\arrow{dr}{\phi_\alpha} & &
      V_{\alpha+\delta}.\\
      %
      & \Lambda_{\alpha}\arrow{ur}{\psi_\alpha^*} &
  \end{tikzcd}\end{equation}

  Because $\Phi\in\Hom^\delta(\VV,\im~\Lambda)$, $\Psi^*\in\Hom^\delta(\im~\Lambda, \VV)$, and Diagrams~\ref{dgm:interleaving1} and~\ref{dgm:interleaving2} commute we may conclude that $\im~\Lambda$ and $\VV$ are $\delta$-interleaved.

\end{proof}

\subsection{Partial Interleavings}

% \begin{proof}[Proof of Lemma~\ref{lem:extension_apply}]
%   Because $P\subi{w}{a} := P\cap D\subi{w}{a}$ and $B_w\subseteq D\subi{w}{a}$ we know $Q_w = P\cap B_w \subseteq P\subi{w}{a}$ for all $a\in\R$.
%   So
%   \[\ext{Q^\e_a} = Q^\e_a\cup (D\setminus P^\e) \subseteq P\subi{w}{a}^\e \cup (D\setminus P^\e) = \ext{P\subi{w}{a}^\e}.\]
%   As $(P^\e, Q_w^\e)$ is a surrounding pair in $D$, $P^\e$ is open in $D$ and $\ext{P\subi{w}{a}^\e}\subseteq D$ is such that $\ext{Q^\e_a}\subseteq \ext{P\subi{w}{a}^\e}$ it follows that
%   \[\hom_k(P\subi{w}{a}^\e, Q^\e_a) = \hom_k(P^\e\cap \ext{P\subi{w}{a}^\e}, Q^\e_a) \cong\hom_k(\ext{P\subi{w}{a}^\e}, \ext{Q^\e_a})\]
%   by Lemma~\ref{lem:excision}.
%
%   Because these isomorphisms commute with inclusions we have an isomorphism $\E\subi{w}{\cdot}^\e \in \Hom(\PP{w}{\e},\ext{\PP{w}{\e}})$ defined to be the family $\{\E\subi{w}{\alpha}^\e : \P{w}{\e}{a}\to \E\P{w}{\e}{a}\}$.
% \end{proof}

\begin{proof}[Proof of Lemma~\ref{lem:inclusions}]
  Suppose $x\in P^\delta\cap D\subi{t-c\e}{\alpha-c\e}$.
  Because $x$ in $P^\delta$ there exists some $p\in P$ such that $\dist(x,p) < \delta$.
  Because $f$ is $c$-Lipschitz $f(p)\leq f(x) + c\dist(x,p) < f(x) + c\delta$.
  If $\alpha\leq t$ then $x\in B_{t-c\e}$ implies $f(p) < t-c\e + c\delta \leq t$ so $x\in Q_t^\e$ as $\delta\leq\e$
  If $\alpha\geq t$ then $x\in B_{\alpha-c\e}$ which implies $f(p) \leq \alpha$ $x\in Q_\alpha^\e$.
  So $P^\delta\cap D\subi{t-c\e}{\alpha-c\e}\subseteq P\subi{t}{\alpha}^\e$ as $P\subi{t}{\alpha} = Q_t^\e\cup Q_\alpha^\e$.

  Now, suppose $x\in P\subi{t}{\alpha}^\e$.
  If $\alpha\leq t$ then $x\in Q_t^\e\subseteq B_{t+c\e}$ because $f$ is $c$-Lipschitz.
  Similarly, $\alpha > t$ implies $x\in Q_\alpha^\e\subseteq B_{\alpha+c\e}$, so $P\subi{t}{\alpha}^\e\subseteq D\subi{t+c\e}{\alpha+c\e}$ as $D\subi{t+c\e}{\alpha+c\e} = B_{t+c\e}\cup B_{\alpha+c\e}$.
\end{proof}

\begin{proof}[Proof of Lemma~\ref{lem:inclusion_hom}]
  Because $Q_t^\delta$ surrounds $P^\delta$ in $D$ and $\delta\leq\e$, $t < v$ we know $Q_t^\e$ and $Q_v^\e$ surround $P^\delta$ in $D$.
  As $P^\delta\cap B_s\subseteq Q_t^\e$ and $P^\delta\cap B_u\subseteq Q_v^{2\e}$ for all $\e\in[\delta,2\delta]$ Lemma~\ref{lem:surround_and_cover} implies that we have a sequence of inclusions $B_s\subseteq \E Q_t^\e\subseteq B_u\subseteq \E Q_v^{2\e}\subseteq B_w$.

  For any $\alpha\in\R$ we know that $D\setminus P^\delta \subseteq \ext{P\subi{t}{\alpha}^\e}$ by the definition of $\ext{P\subi{t}{\alpha}^\e}$.
  Moreover, $D\setminus P^\delta\subseteq D\subi{u}{\alpha}$ because $D\setminus B_u\subseteq P^\delta$.
  Lemma~\ref{lem:inclusions} therefore implies $D\subi{s}{\alpha-c\delta}\subseteq \E P\subi{t}{\alpha}^\e\subseteq D\subi{u}{\alpha+c\e}$ as $s + c\delta\leq t \leq u - c\e$.
  So the inclusions $(D\subi{s}{\alpha-c\delta}, B_s)\subseteq (\E P\subi{t}{\alpha}^\e, \E Q_t^\e)$ induce $F\in\Hom^{c\delta}(\DD{s},\E\PP{t}{\e})$ and $(\E P\subi{t}{\alpha}^\e, \E Q_t^\e)\subseteq (D\subi{u}{\alpha+c\e}, B_u)$ induce $M\in \Hom^{c\e}(\E\PP{t}{\e}, \DD{u})$.

  By an identical argument Lemma~\ref{lem:inclusions} implies $D\subi{u}{\alpha-2c\delta}\subseteq \E P\subi{v}{\alpha}^\e\subseteq D\subi{w}{\alpha+2c\e}$ as $u+c\delta\leq v\leq w-4c\delta$.
  So $(D\subi{u}{\alpha-2c\delta},B_u)\subseteq (\E P\subi{v}{\alpha}^\e, \E Q_v^{2\e})$ induce $G\in\Hom^{2c\delta}(\DD{u},\E\PP{v}{2\e})$ and $(\E P\subi{v}{\alpha}^\e, \E Q_v^{2\e})\subseteq (D\subi{w}{\alpha+2c\e},B_w)$ induce $N\in\Hom^{2c\e}(\E\PP{v}{2\e},\DD{u})$.
\end{proof}

% \subsection{Truncated Interval Modules}
%
% \begin{proof}[Proof of Lemma~\ref{lem:decomposition}]
%   Suppose $\alpha\leq\omega$.
%   So $\hom_k(D\subi{\omega}{\alpha}, B_\omega) = 0$ as $D\subi{\omega}{\alpha} = B_\omega\cup B_\alpha$ and $\T^k_\omega = 0$ as $F_\alpha^I = 0$ for any $I\in \I^k$ such that $\omega\in I_-$.
%   Moreover, $\omega\in I$ for all $I\in \I_\omega^{k-1}$, thus $F_\alpha^{I_+} = 0$ for all $\alpha\leq\omega$.
%   So it suffices to assume $\omega < \alpha$.
%
%   Consider the long exact sequence of the pair $\hom_k(D\subi{\omega}{\alpha}, B_\omega) = \hom_k(B_\alpha, B_\omega)$
%   \[ \ldots\to \hom_k(B_\omega)\xrightarrow{p_\alpha^k} \hom_k(B_\alpha)\xrightarrow{q_\alpha^k}\hom_k(D\subi{\omega}{\alpha}, B_\omega)\xrightarrow{r_\alpha^k} \hom_{k-1}(B_\omega)\xrightarrow{p_\alpha^{k-1}}\hom_{k-1}(B_\alpha)\to\ldots\]
%   where $\hom_k(B_\alpha) = \bigoplus_{I\in \I^k}F_\alpha^I$, $\hom_k(B_\omega) = \bigoplus_{I\in \I^k}F_\omega^I$, and $p_\alpha^k = \displaystyle\bigoplus_{I\in\I^k} f_{\omega,\alpha}^I$.
%
%   Noting that $\im~q_\alpha^k \cong \hom_k(B_\alpha) / \ker~q_\alpha^k$ where $\ker~q_\alpha^k = \im~p_\alpha^k$ by exactness we have $\ker~r_\alpha^k \cong \hom_k(B_\alpha) / \im~p_\alpha^k$.
%   By the definition of $F_\alpha^I$ and $f_{\omega,\alpha}^I$ we know $\im~f_{\omega,\alpha}^I$ is $F_\alpha^I$ if $\omega\in I$ and 0 otherwise.
%   As $\im~p_\alpha^k$ is equal to the direct sum of images $\im~f_{\omega,\alpha}^I$ over $I\in\I^k$ it follows that $\im~p_\alpha^k$ is the direct sum of those $F_\alpha^I$ over those $I\in\I^k$ such that $\omega\in I$.
%   Now, because $\hom_k(B_\alpha) = \bigoplus_{I\in \I^k}F_\alpha^I$ and each $F_\alpha^I$ is either 0 or $\FF$ the quotient $\hom_k(B_\alpha) / \im~p_\alpha^k$ is the direct sum of those $F_\alpha^I$ such that $\omega\notin I$.
%   Therefore, by the definition of $F\subi{\omega}{\alpha}^I$ we have
%   \[ \ker~r_\alpha^k = \bigoplus_{I\in\I_\omega^k} F\subi{\omega}{\alpha}^I.\]
%
%   Similarly, $\im~r_\alpha^k = \ker~p_\alpha^{k-1}$ by exactness where $\ker~p_\alpha^{k-1}$ is the direct sum of kernels $\ker~f_{\omega,\alpha}^I$ over $I\in\I^{k-1}$.
%   By the definition of $F_\alpha^I$ and $f_{\omega,\alpha}^I$ we know that $\ker~f_{\omega,\alpha}^I$ is $F_\alpha^I$ if $\omega\notin I$ and $0$ otherwise.
%   Noting that $\ker~f_{\omega,\alpha}^I = 0$ for any $I\in \I^{k-1}$ such that $\omega\notin I$ it suffices to consider only those $I\in \I_\omega^{k-1}$.
%   It follows that $\ker~f_{\omega,\alpha}^I = F_\alpha^{I_+}$ for any $I$ containing $\omega$ as $\omega < \alpha$.
%   Therefore,
%   \[\im~r_\alpha^k = \bigoplus_{I\in\I^{k-1}} F_\alpha^{I_+}.\]
%
%   We have the following split exact sequence associated with $r_\alpha^k$
%   \[ 0\to \ker~r_\alpha^k\to \hom_k(D\subi{\omega}{\alpha}, B_\omega)\to\im~r_\alpha^k\to 0.\]
%   The desired result follows from the fact that for all $\alpha\in\R$
%   \begin{align*}
%     \hom_k(D\subi{\omega}{\alpha}, B_\omega) &\cong \ker~r_\alpha^k\oplus \im~r_\alpha^k =\bigoplus_{I\in\I^k} F\subi{\omega}{\alpha}^I\oplus \bigoplus_{I\in\I_\omega^{k-1}} F_\alpha^{I_+}.
%   \end{align*}
% \end{proof}


\section{Duality}\label{apx:duality}
% !TeX root = ../main.tex

For a pair $(A, B)$ in a topological space $X$ and any $R$ module $G$ let $\hom^k(A, B; G)$ denote the \textbf{singular cohomology} of $(A,B)$ (with coefficients in $G$).
Let $\hom^k_c(A, B; G)$ denote the corresponding \textbf{singular cohomology with compact support}.
For any compact pair $(A,B)$ there is an isomorphism $\hom^k_c(A, B; G)\to\hom^k(A, B; G)$.

Corollary\ref{cor:univ_coef} follows from the Universal Coefficient Theorem for singular homology (and cohomology) as vector spaces over a field $\FF$, as the dual vector space $\Hom(\hom_k(A, B), \FF)$ is isomorphic to $\hom_k(A, B; \FF)$ for any finitely generated $\hom_k(A, B)$.

\begin{corollary}\label{cor:univ_coef}
  For a topological pair $(A, B)$ and a field $\FF$ such that $\hom_k(A, B)$ is finitely generated there is a natural isomorphism
  \[\nu : \hom^k(A, B; \FF)\to \hom_k(A, B; \FF).\]
\end{corollary}

Let $\overline{\hom}^k(A, B; G)$ be the \textbf{Alexander-Spanier cohomology} of the pair $(A,B)$, defined as the limit of the direct system of neighborhoods $(U,V)$ of the pair $(A, B)$.
Let $\overline{\hom}^k_c(A, B; G)$ denote the corresponding \textbf{Alexander-Spanier cohomology with compact support} where $\overline{\hom}^k_c(A, B; G)\cong\overline{\hom}^k(A, B; G)$ for any compact pair $(A, B)$.

\begin{theorem}[\textbf{Alexander-Poincar\'e-Lefschetz Duality} (Spanier~\cite{spanier1989algebraic}, Theorem 6.2.17)]\label{thm:alexander}
  Let $X$ be an orientable $d$-manifold and $(A, B)$ be a compact pair in $X$.
  Then for all $k$ and $R$ modules $G$ there is a (natural) isomorphism
  \[\lambda : \hom_k(X\setminus B, X\setminus A; G)\to \overline{\hom}^{d-k}(A, B; G).\]
\end{theorem}

A space $X$ is said to be \textbf{homologically locally connected in dimension $n$} if for every $x\in X$ and neighborhood $U$ of $x$ there exists a neighborhood $V$ of $x$ in $U$ such that $\tilde{\hom}_n(V)\to\tilde{\hom}_n(U)$ is trivial for $k\leq n$.

\begin{lemma}[Spanier p. 341, Corollary 6.9.6]\label{lem:alexander_iso}
  Let $A$ be a closed subset, homologically locally connected in dimension $n$, of a Hausdorff space $X$, homologically locally connected in dimension $n$.
  If $X$ has the property that every open subset is paracompact, $\mu : \overline{\hom}_c^k(X,A; G)\to \hom_c^k(X, A; G)$ is an isomorphism for $k\leq n$ and a monomorphism for $q = n+1$.
\end{lemma}

In the following we will assume homology (and cohomology) over a field $\FF$.

\begin{lemma}\label{cor:alexander_iso}
  Let $X$ be an orientable $d$-manifold and $(A,B)$ a compact pair of locally path connected subspaces in $X$.
  Then
  \[\xi : \hom_d(X\setminus B, X\setminus  A)\to \hom_0(A, B)\]
  is a natural isomorphism.
\end{lemma}
\begin{proof}
  Because $X$ is orientable and $(A,B)$ are compact $\lambda : \hom_d(X\setminus B, X\setminus A)\to \overline{\hom}^{0}(A, B)$ is an isomorphism by Theorem~\ref{thm:alexander}.
  Note that
  Moreover, because every subset of $X$ is (hereditarily) paracompact every open set in $A$, with the subspace topology, is paracompact.
  For any neighborhood $U$ of a point $x$ in a locally path connected space there must exist some neighborhood $V\subset U$ of $x$ that is path connected in the subspace topology.
  As $\tilde{\hom}_0(V) = 0$ for any nonempty, path connected topological space $V$ (see Spanier p. 175, Lemma 4.4.7) it follows that $A$ (resp. $B$) are homologically locally connected in dimension $0$.
  Because $(A,B)$ is a compact pair the singular and Alexander-spanier cohomology modules of $(A,B)$ with compact support are isomorphic to those without, thus $\mu:\overline{\hom}^{0}(A, B)\to \hom^0(A, B)$ is an isomorphism.
  By Corollary~\ref{cor:univ_coef} we have a natural isomorphism $\nu : \hom^0(A, B)\to\hom_0(A, B)$ thus the composition $\xi := \nu\circ\mu\circ\lambda : \hom_d(X\setminus B, X\setminus  A)\to \hom_0(A, B)$ is a natural isomorphism.
\end{proof}

\begin{lemma}\label{lem:duality_apply}
  Let $\X$ be an orientable $d$-manifold let $D$ be a compact subset of $\X$.
  Let $P$ be a finite subset of $D$ such that $P^\e\subset \intr_\X(D)$ and $Q\subseteq P$.

  If $D\setminus Q^\e$ and $D\setminus P^\e$ are locally path connected then there is a natural isomorphism
  \[ \xi : \hom_d(P^\e,Q^\e)\to \hom_0(D\setminus Q^\e, D\setminus P^\e).\]
\end{lemma}
\begin{proof}
  Because $Q^\e$ and $P^\e$ are open in $D$ and $D$ is compact in $\X$ the complement $D\setminus Q^\e$ is closed in $D$, and therefore compact in $\X$.
  Moreover, because $P^\e\subset \intr_\X(D)$, $\hom_d(\X\setminus(D\setminus P^\e), \X\setminus(D\setminus Q^\e)) = \hom_d(P^\e, Q^\e)$.
  As we have assumed these complements are locally path connected by assumption we have a natural isomorphism $\xi : \hom_d(P^\e, Q^\e)\to \hom_0(D\setminus Q^\e, D\setminus P^\e)$
  by Lemma~\ref{cor:alexander_iso}.
\end{proof}


\end{document}
