% !TeX root = main.tex

Let $X\subset \R^d$ and $A\subset X$.
A triple $(A,B,C)$ is a \textbf{partition} of $X$ if the sets $A, B$ and $C$ are disjoint and $A\cup B\cup C = X$.
\begin{definition}
    A subset $A\subset X$ \textbf{separates} $X$ if there exists a pair $(\I,\O)$ such that $(\I, A, \O)$ is a partition of $X$ and every path from $\I$ to $\O$ intersects $A$.
\end{definition}
The boundary $\B$ of any closed, bounded subset $\D$ of $\R^d$ separates $\R^d$ with the pair $(\D\setminus\B, \overline{\D})$, leading us to the following definition.
\begin{definition}
    A subset $\B\subset \D$ \textbf{surrounds} $\D\subset\R^d$ if $\B$ separates $\R^d$ with the pair $(\D\setminus\B, \overline{\D})$.
\end{definition}

\begin{lemma}
    If $A$ separates $X$ with the pair $(\I, \O)$ then for all $k\geq 0$
    \[ \hom_(\I)\cong\hom_k(\overline{A}, \O). \]
\end{lemma}
\begin{proof}
    If $A$ separates $X$ with the pair $(\I, \O)$ then $\I$ and $\O$ are not path connected and $\overline{A} = \I\cup\O$.

    For all $k > 0$ suppose $[x]\in \hom_k(\overline{A}, \O)$ is non-trivial.
    Because $\I$ and $\O$ are not path connected $\partial x \in C_{k-1}(\O)$ implies that $x\in C_k(\O)$ and therefore that $[x]$ is trivial in $\hom_k(\overline{A}, \O)$.
    So $[x]$ is not a relative cycle and therefore $[x]\in \hom_k(\overline{A})$, $[x]\notin\hom_k(\O)$.
    Because $\I$ and $\O$ are not path connected $[x]\notin\hom_k(\O)$ implies $[x]\in \hom_k(\overline{A}\setminus\O) = \hom_k(\I)$.

    On the other hand, if $[x]\in \hom_k(\I)$ then $[x]\in \hom_k(\overline{A}, \I)$ as $\I\subset\overline{A}$ and, because $\I$ and $\O$ are not path connected, $[x]$ is trivial in $\hom_k(\overline{A}, \O)$ if and only if $[x]$ is trivial in $\hom_k(\I)$.
\end{proof}

\begin{lemma}\label{lem:surrounds}
    If $A$ separates $X$ with the pair $(\I, \O)$ then for all $\e > 0$
    \[\hom_0(\I\setminus A^\e)\cong \hom_0(\overline{A^\e}, \overline{\I^\e}).\]
\end{lemma}
\begin{proof}
    Given some non-trivial $0$-chain $[x]\in\hom_0(\I\setminus A^\e)$ pick a representative point $x\in \I\setminus A^\e \subset \overline{A^\e}$.
    Because $A$ separates $X$ with the pair $(\I, \O)$ every path from $\I$ to $\O$ intersects $A$, therefore every path from $\I\setminus A^\e\subset\I$ to $\overline{\I^\e} \subset A\cup \O$ must intersect $A^\e$ for $\e > 0$.
    Thus $[x]$ is non-trivial in $\hom_0(\overline{A^\e}, \overline{\I^\e})$ so the map is injective.

    Any $[x]\in\hom_0(\overline{A^\e}, \overline{\I^\e})$ is represented by a point $x$ in a connected component of $\overline{A^\e}\setminus \overline{\I^\e} = \I\setminus A^\e$, and thus a homology class $[x]\in\hom_0(\D\setminus\B^\e)$, so the map is surjective.
\end{proof}

\begin{corollary}\label{lem:surrounds}
    If $\B$ surrounds $\D$ then for all $\e > 0$
    \[\hom_0(\D\setminus\B^\e)\cong\hom_0(\overline{\B^\e}, \overline{\D^\e}).\]
\end{corollary}

\subsection{Assumptions}\label{ssec:assumptions}

Let $\D\subset\R^d$ and $f:\D\to\R$ be a $c$-Lipschitz function satisfying the following assumptions for some $\omega\in\R$, $\delta > 0$ and $\gamma > 3\delta$.
Let $P\subset \D$ be a finite collection of sensors $p$ with the following capabilities.

\vspace{3ex}
\begin{center}
\setlength{\fboxsep}{2ex}
\fbox{\parbox{\textwidth}{
\begin{small}
\textbf{Sensor Capabilities}
    \begin{itemize}
        \item[a.]\textbf{(Communication Radii)} detect the presence, but not location or distance, of sensors within distances $\delta > 0$ and $\gamma \geq 3\delta$, and discriminate between sensors within each scale,
        \item[b.]\textbf{(Coverage Radius)} cover a radially symmetric subset of the domain with radius $\delta$,
        \item[c.]\textbf{(Measurement)} measure the scalar value $f(p)$.
    \end{itemize}
\end{small}
}}\end{center}\vspace{3ex}

The original TCC~\cite{desilva07coverage} and work on scalar fields~\cite{chazal09analysis} make strict assumptions about the geometry of the domain $\D$ and its boundary $\B$.
We re-interpreted the TCC by replacing the assumptions about the geometry of the domain with assumptions about its persistent homology~\cite{cavanna2017when}.
This amounted to a requirement that the zero-dimensional homology of the domain did not change as we shrunk the domain within a certain range.

In this work we would like to extend this idea from the zero-dimensional homology of a geometric domain to the homology of all dimensions of a domain defined as a sublevel-set of a function.
We can then re-interpret the geometric assumptions made in the work on scalar fields in a similar way which re-purposes the machinery used to confirm coverage.

\vspace{2ex}
\begin{center}
\setlength{\fboxsep}{2ex}
\fbox{\parbox{\textwidth}{
% \vspace{1ex}\hspace{1ex}
\textbf{Geometric Assumptions}
\begin{small}
    \begin{enumerate}
    \setcounter{enumi}{0}
        \item \textbf{(Domain)} $\D$ is a bounded, compact subset of $\R^d$ and $\D_{\omega-2c\delta}$ is closed and surrounds $\D$.
        \item \textbf{(Boundary)} $\im~\hom_k(\D_{\omega-2c\delta}\hookrightarrow \D_{\omega+{2c\delta}})$ is isomorphic to $\hom_k(\D_{\omega-2c\delta}^{2\delta})$ for all $k\in\N$.
        \item \textbf{(Retraction)} $\D_{\omega-c\delta}^\delta\to\D_{\omega-2c\delta}^{2\delta}$ is a deformation retraction.
        \item \textbf{(Inclusion)} $\D_{\omega - 2c\delta}\subseteq \D_{\omega - 2c\delta}^{2\delta} \subseteq \D_{\omega - 2c\delta}^{\delta + \gamma} \subseteq \D_{\omega + 2c\delta}.$
    \end{enumerate}
\end{small}
}}\end{center}\vspace{4ex}

In the following let $Q = P\cap \D_{\omega - c\delta}$ and $\B = \D_{\omega-2c\delta}$.

Assumption 1 replaces the requirement that $\B$ is the topological boundary of the domain $\D$.
As we did in previous work the notion of a topological boundary is replaced with that of a subset which surrounds the domain.
In addition to this we define this surrounding subset in terms of a scalar value $\omega$ which gives us a sublevel set that will serve as our boundary $\B$.
In doing so we can state the following assumptions in terms of the persistent homology of the function itself instead of the persistent homology of a geometric domain.

Assumption 2 is our main restriction on the persistent homology of the surrounding sub-level set and limits the size of features in terms of the offset parameters $\delta,\gamma$.
We note that this assumption is stronger than those made in our initial work which only required that the map $\hom_0(\D\setminus\D_{\omega+2c\delta}\hookrightarrow\D\setminus\B^{2\delta})$ be surjective which is not sufficient for the analysis of scalar fields.

Assumption 3 requires that the region containing the sampled boundary $Q^\delta$ resembles the extended boundary $\B^{2\delta}$.
While this assumption would not be necessary if $Q$ was defined as the set of points within distance $\delta$ of $\B$ we are particularily interested in the case in which the points cannot detect any features of the domain, such as the presence of a boundary, and can only measure some scalar value.

Finally, assumption 4 is our only additional restriction on the choice of the parameter $\gamma\geq 3\delta$ so that it does not extend beyond the sub-level set $\D_{\omega+2c\delta}$.


% The following is our first main theorem to be proved in the following section.
%
% \begin{theorem}[Geometric TCC]\label{thm:tcc}
%     Let $\D\subset\R^d$ and $f:\D\to\R$ be a $c$-Lipschitz function satisfying assumptions 1-3 for $\omega\in\R$, $\delta > 0$, and $\gamma > 3\delta$.
%     Let $P\subset\D$ be a collection of sensors and let $Q = P\cap \D_{\omega - c\delta}$, $\B = \D_{\omega - 2c\delta}$.
%     Let $p_* : \im~j_*\to\im~i_*$ for $j_*$, $i_*$ as defined in Diagram~\ref{dgm:1}.
%
%     If $p_*$ is injective then $\D\setminus\B^{2\delta}\subseteq P^\delta$ and $Q^\delta$ separates $\D$.
% \end{theorem}

% \begin{corollary}
%     If $\im~\hom_k(\D_{\omega-2c\delta}\hookrightarrow \D_{\omega+{2c\delta}})\cong \hom_k(\D_{\omega - 2c\delta}^{2\delta})$ and $p_*$ is injective then for all $k\geq 0$
%     \[ \im~i_* \cong \hom_k(\D^{2\delta}, \B^{2\delta}).\]
% \end{corollary}

% \begin{lemma}\label{lem:separate}
%     If $j_*$ is surjective and $p_*$ is injective then $Q^\delta$ separates $\D$.
% \end{lemma}
% \begin{proof}
%     Suppose, for the sake of contradiction, that $Q^\delta$ does not separate $\D$.
%     Then for all $(\I, \O)$ such that $\I \cup \O = \D\setminus Q^\delta$ there must exist some path from $\I$ to $\O$ that does not cross $Q^\delta$.
%     Formally, there exists a path $\pi : [0,1]\to\overline{Q^\delta}$ with $\pi(0)\in \I$ and $\pi(1)\in\O$.
%     Noting that $\overline{\B^{2\delta}}\subseteq \overline{Q^\delta}$ and, because $\overline{\B^{2\delta}} = \overline{\D^{2\delta}}\cup (\D\setminus\B^{2\delta})$ surrounds $\D^{2\delta}$ we can choose $(\I, \O)$ such that $\D\setminus \B^{2\delta}\subset \I$ and $\overline{\D^{2\delta}}\subset \O$.
%
%     Choose $x\in\D\setminus \B^{2\delta}$ and $y\in \overline{\D^{2\delta}}$ such that there exist paths $\pi_x : [0,1]\to \I$ with $\pi_x(0) = x$, $\pi_x(1) = \pi(0)$ and $\pi_y : [0,1]\to \O$ with $\pi_y(0) = y$, $\pi_y(1) = \pi(1)$.
%     $\pi_x, \pi_y$ and $\pi$ all generate chains in $C_1(\overline{Q^\delta}, \overline{P^\delta})$ and $\pi_x + \pi + \pi_y = \pi^*\in C_1(\overline{Q^\delta}, \overline{P^\delta})$ with $\partial\pi^* = x + y$.
%     Moreover, $y$ generates a chain in $C_0(\overline{P^\delta})$ as $\overline{\D^{2\delta}}\subseteq\overline{P^\delta}$.
%     So $x = \partial\pi^* + y$ is a relative boundary in $C_0(\overline{Q^\delta}, \overline{P^\delta})$ thus $[x] = 0 = [y]$ in $\hom_0(\overline{Q^\delta}, \overline{P^\delta})$ and therefore $[x] = [y]$ in $\im~i_*$.
%     However, because $\B^{2\delta}$ separates $\D$ with the pair $(\overline{\D^{2\delta}}, \D\setminus\B^{2\delta})$ we know that $[x]\neq [y]$ in $\hom_0(\overline{\B^{2\delta}}, \overline{\D^{2\delta}})\cong \im~j_*$, contradicting our assumption that $p_*$ is injective.
% \end{proof}
%
% In the following we will assume that $j_*$ is surjective and $p_*$ is injective.
% By Lemma~\ref{lem:separate} $Q^\delta$ separates $\D$ with a pair $(\I, \O)$.
% Let $\hat{Q}^\delta = Q^\delta \cup \O$, $\hat{P}^\delta = P^\delta \cup \O$ and $\hat{Q}^\gamma = Q^\gamma \cup \O$.
%
% \begin{lemma}
%     $\hom_*(\hat{P}^\delta, \hat{Q}^\delta)\cong \hom_*(P^\delta, Q^\delta)$ and $\hom_*(P^\delta\cup\hat{Q}^\gamma, \hat{Q}^\gamma)\cong \hom_*(P^\delta\cup Q^\gamma, Q^\gamma)$.
% \end{lemma}
% \begin{proof}
%     By the definition of $\hat{Q}^\delta$, $\hat{Q}^\gamma$ and $\O$ we have that $\hat{Q}^\delta\setminus\O = Q^\delta$ and $\hat{Q}^\gamma\setminus\O = Q^\gamma$ and $\cl~\O \subset \intr~\hat{Q}^\delta, \intr~\hat{Q}^\gamma$.
%     The desired results follow by Excision.
% \end{proof}
%
% \begin{lemma}
%     If $p_*$ is injective then $\B^{2\delta}\subseteq \hat{Q}^\gamma$.
% \end{lemma}
% \begin{proof}
%     Suppose $p_*$ is injective and there exists some $x\in \B^{2\delta}$ such that $x\notin \hat{Q}^\gamma$.
%     So there exists some $y\in U_{\omega - 2c\delta}$ such that $\dist(x, y) < 2\delta$.
%     Because $Q^\delta$ separates $\D = \O \cup Q^\delta \cup \I$ both $x$ and $y$ are either in $Q^\delta, \O$ or $\I$.
%     Moreover, $\overline{\I} = Q^\delta \cup \O = \hat{Q}^\delta$ so $\B^{2\delta}\cap\I = \B^{2\delta}\setminus \hat{Q}^\delta$ and $\B\cap\I = \B\setminus \overline{\I} = \B\setminus \hat{Q}^\delta = \emptyset$ as $\B\subset \hat{Q}^\delta$.
%     So $y\notin \I$ and, because $Q^\delta \subset Q^\gamma$ and $\O\subset \hat{Q}^\delta\subset\hat{Q}^\gamma$, we can assume w.l.o.g. that $x\in \B^{2\delta}\cap\I$.
%
%     If $y\in Q^\delta$ then there exists some $q\in Q$ such that $\dist(q, y) < \delta$ so
%     \[ \dist(q, x)\leq \dist(q, y) + \dist(x, y) < 3\delta\leq\gamma \]
%     which implies $x \in Q^\gamma$.
%
%     So we may assume that $y\in U_{\omega - 2c\delta}\cap\O$.
%     Because $Q^\delta$ separates $\D$ with $(\I, \O)$ there is no path from $x\in \I$ to $y\in\O$ that does not cross $Q^\delta$, so there must be some point $z\in Q^\delta$ in the shortest path from $x$ to $y$.
%     That is, there exists some $q\in Q$ such that $\dist(q, z) < \delta$ and $\dist(z, x) < \dist(x, y) < 2\delta$ so
%     \[ \dist(q, x)\leq \dist(q, z) + \dist(z, x) < \delta + 2\delta \leq \gamma. \]
%     So $y\in\O$ implies $x\in\hat{Q}^\gamma$.
%
% \end{proof}
