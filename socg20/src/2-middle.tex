% !TeX root = ../main.tex

Because the TCC only confirms coverage of a \emph{superlevel} set $D\setminus B_\omega$, we cannot guarantee coverage of the entire domain.
Indeed, with sufficient smoothness assumptions on the boundary we could compute the persistent homology of the \emph{restriction} of $f$ to the superlevel set we cover in the standard way~\cite{chazal09analysis}.
Instead, we will approximate the persistent homology of the sublevel set filtration \emph{relative to} the sublevel set $B_\omega$.
In the next section we will discuss an interpretation of the relative diagram that is motivated by examples in Section~\ref{sec:experiments}.

We will first introduce the notion of an extension which will provide us with maps on relative homology induced by inclusion via excision.
However, even then, a map that factors through our pair $(D, B_\omega)$ is not enough to prove an interleaving of persistence modules by inclusion directly.
To address this we impose conditions on sublevel sets near $B_\omega$ which generalize the assumptions made in the TCC.

\subsection{Extensions and Image Persistence Modules}

Suppose $D$ is a subspace of a topological space $X$.
We define the extension of a surrounding pair in $D$ to a surrounding pair in $X$ with isomorphic relative homology.

\begin{definition}[Extension]
  If $V$ surrounds $U$ in a subspace $D$ of $X$ let $\ext{V} := V\sqcup (D\setminus U)$ denote the (disjoint) union of the separating set $V$ with the complement of $U$ in $D$.
  The \textbf{extension of $(U, V)$ in $D$} is the pair $(D, \ext{V}) = (U\sqcup (D\setminus U), V\sqcup (D\setminus U)).$
\end{definition}

Lemma~\ref{lem:surround_and_cover} states that we can use these extensions to interleave a pair $(U, V)$ with a sequence of subsets of $(D, B)$.
Lemma~\ref{lem:excision} states that we can apply excision to the relative homology groups in order to get equivalent maps on homology that are induced by inclusions.

\begin{lemma}\label{lem:surround_and_cover}
  Suppose $V$ surrounds $U$ in $D$ and $B'\subseteq B\subset D$.

  If $D\setminus B\subseteq U$ and $U\cap B'\subseteq V\subseteq B'$ then $B'\subseteq \ext{V}\subseteq B$.
\end{lemma}

\begin{lemma}\label{lem:excision}
  Let $(U, V)$ be an open surrounding pair in a subspace $D$ of $X$.

  Then $\hom_k((U\cap A, V)\hookrightarrow (A, \ext{V}))$ is an isomorphism for all $k$ and $A\subseteq D$ with $\ext{V}\subset A$.
\end{lemma}

The TCC uses a nested pair of spaces in order to filter out noise introduced by the sample.
This same technique is used to approximate the persistent homology of a scalar field~\cite{chazal09analysis}.
As modules, these nested pairs are the images of homomorphisms between homology groups induced by inclusion, which we refer to as image persistence modules.
For a full background on persistence modules, shifted homomorphisms, and interleavings of persistence modules see Chazal et al.~\cite{chazal13structure}.

\begin{definition}[Image Persistence Module]
  The \textbf{image persistence module} of a homomorphism $\Gamma\in\Hom(\UU,\VV)$ is the family of subspaces $\{\Gamma_\alpha :=\im~\gamma_\alpha\}$ in $\VV$ along with linear maps $\{\gamma_\alpha^\beta := v_\alpha^\beta\rest_{\im~\gamma_\alpha} : \Gamma_\alpha\to\Gamma_\beta\}$ and will be denoted by $\im~\Gamma$.
\end{definition}

For a homomorphism $\Gamma\in\Hom(\UU, \VV)$ let $\Gamma[\delta]\in \Hom^{\delta}(\UU, \VV)$ denote the shifted homomorphism defined to be the family of linear maps $\{\gamma_\alpha[\delta] := v_\alpha^\delta\circ \gamma_\alpha : U_\alpha\to V_{\alpha+\delta}\}$.
While we will primarily work with homomorphisms of persistence modules induced by inclusions, in general, defining homomorphisms between images simply as subspaces of the codomain is not sufficient.
Instead, we require that homomorphisms between image modules commute not only with shifts in scale, but also with the functions themselves.

\begin{definition}[Image Module Homomorphism]
  Given $\Gamma\in\Hom(\UU,\VV)$ and $\Lambda\in\Hom(\S,\T)$ along with $(F,G)\in\Hom^\delta(\UU,\S)\times\Hom^\delta(\VV,\T)$ let $\Phi(F, G) : \im~\Gamma\to\im~\Lambda$ denote the family of linear maps $\{\phi_\alpha := g_\alpha\rest_{\Gamma_\alpha} : \Gamma_\alpha\to\Lambda_{\alpha+\delta}\}$.
  $\Phi(F, G)$ is an \textbf{image module homomorphism of degree $\delta$} if the following diagram commutes for all $\alpha\leq\beta$.
  \begin{equation}\label{dgm:image_homomorphism}
    \begin{tikzcd}[column sep=large]
        U_\alpha\arrow{r}{\gamma_\alpha[\beta-\alpha]}\arrow{d}{f_\alpha} &
      V_\beta\arrow{d}{g_\beta}\\
      %
      S_{\alpha+\delta}\arrow{r}{\lambda_{\alpha+\delta}[\beta-\alpha]} &
      T_{\beta +\delta}
  \end{tikzcd}\end{equation}
  The space of image module homomorphisms of degree $\delta$ between $\im~\Gamma$ and $\im~\Lambda$ will be denoted $\Hom^\delta(\im~\Gamma,\im~\Lambda)$.
\end{definition}

\begin{lemma}\label{lem:image_composition}
  Suppose $\Gamma\in\Hom(\UU,\VV)$, $\Lambda\in\Hom(\S,\T)$, and $\Lambda'\in\Hom(\S',\T')$.
  If $\Phi(F, G)\in\Hom^\delta(\im~\Gamma, \im~\Lambda)$ and $\Phi'(F', G')\in\Hom^{\delta'}(\im~\Lambda, \im~\Lambda')$ then $\Phi''(F'\circ F, G'\circ G) := \Phi'\circ\Phi\in\Hom^{\delta+\delta'}(\im~\Gamma,\im~\Lambda')$.
\end{lemma}

\paragraph*{Partial Interleavings of Image Modules}

Image module homomorphisms introduce a direction to the traditional notion of interleaving.
As we will see, our interleaving via Lemma~\ref{thm:interleaving_main} involves partially interleaving an image module to two other image modules whose composition is isomorphic to our target.

\begin{definition}[Partial Interleaving of Image Modules]
  An image module homomorphism $\Phi(F, G)$ is a \textbf{partial $\delta$-interleaving of image modules}, and denoted $\Phi_M(F, G)$, if there exists $M\in\Hom^\delta(\S,\VV)$ such that $\Gamma[2\delta] = M\circ F$ and $\Lambda[2\delta] = G\circ M$.
\end{definition}

Lemma~\ref{thm:interleaving_main} uses partial interleavings of a map $\Lambda$ with $\UU\to\VV$ and $\VV\to\W$ along with the hypothesis that $\im(\UU\to \W)$ is isomorphic to $\VV$ to interleave $\im~\Lambda$ with $\VV$.
When applied, this hypothesis will be satisfied by assumptions on our sublevel set similar to those made in the TCC.

\begin{lemma}\label{thm:interleaving_main}
  Suppose $\Gamma\in\Hom(\UU,\VV)$, $\Pi\in\Hom(\VV,\W)$, and $\Lambda\in\Hom(\S, \T)$.

  If $\Phi_M(F, G)\in\Hom^\delta(\im~\Gamma, \im~\Lambda)$ and $\Psi_G(M, N)\in\Hom^\delta(\im~\Lambda, \im~\Pi)$ are partial $\delta$-interleavings of image modules such that $\Gamma$ is a epimorphism and $\Pi$ is a monomorphism then $\im~\Lambda$ is $\delta$-interleaved with $\VV$.
\end{lemma}
