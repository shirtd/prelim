\documentclass[a4paper,UKenglish,cleveref, autoref, thm-restate]{lipics/socg-lipics-v2019}
%This is a template for producing LIPIcs articles.
%See lipics-manual.pdf for further information.
%for A4 paper format use option "a4paper", for US-letter use option "letterpaper"
%for british hyphenation rules use option "UKenglish", for american hyphenation rules use option "USenglish"
%for section-numbered lemmas etc., use "numberwithinsect"
%for enabling cleveref support, use "cleveref"
%for enabling autoref support, use "autoref"
%for anonymousing the authors (e.g. for double-blind review), add "anonymous"
%for enabling thm-restate support, use "thm-restate"

\usepackage{amsmath,amssymb,amsthm,xspace,enumitem,xcolor,tikz-cd,xspace}

\newcommand{\R}{\mathbb{R}}
\renewcommand{\S}{\mathbb{S}}
\newcommand{\T}{\mathbb{T}}
\newcommand{\W}{\mathbb{W}}
\newcommand{\X}{\mathbb{X}}
\newcommand{\Y}{\mathbb{Y}}
\newcommand{\Z}{\mathbb{Z}}

\renewcommand{\AA}{\mathbb{A}}
\newcommand{\BB}{\mathbb{B}}
\newcommand{\FF}{\mathbb{F}}
\newcommand{\LL}{\mathbb{L}}
\newcommand{\UU}{\mathbb{U}}
\newcommand{\VV}{\mathbb{V}}

\newcommand{\A}{\mathcal{A}}
\newcommand{\E}{\mathcal{E}}
\newcommand{\F}{\mathcal{F}}
\newcommand{\I}{{\mathcal{I}}}
\newcommand{\J}{\mathcal{J}}
\newcommand{\N}{\mathcal{N}}
\newcommand{\U}{\mathcal{U}}
\newcommand{\V}{\mathcal{V}}

\newcommand{\e}{\varepsilon}
\newcommand{\im}{\mathbf{im}\xspace}
\renewcommand{\ker}{\mathbf{ker}\xspace}
\newcommand{\rk}{\mathbf{rk\xspace}}
\renewcommand{\dim}{\mathbf{dim}\xspace}
\newcommand{\rest}{\mathord{\mid}}

\newcommand{\cech}{\check{\mathcal{C}}}
\newcommand{\rips}{\mathcal{R}}
\newcommand{\ball}{\mathbf{ball}}
\newcommand{\dist}{\mathbf{d}}

\newcommand{\cl}{\mathbf{cl\xspace}}
\newcommand{\intr}{\mathbf{int\xspace}}

\newcommand{\Hom}{\mathrm{Hom}}
\renewcommand{\hom}{\mathrm{H}}

\newcommand{\subi}[2]{_{\scriptscriptstyle #2\mid #1}}

\newcommand{\D}[2]{\mathcal{D}\subi{#1}{#2}}
\newcommand{\DD}[1]{\mathbb{D}_{#1}}

\renewcommand{\P}[3]{\mathcal{P}\subi{#1}{#3}^{#2}}
\newcommand{\CP}[3]{\cech\P{#1}{#2}{#3}}
\newcommand{\RP}[3]{\rips\P{#1}{#2}{#3}}

\newcommand{\PP}[2]{\mathbb{P}_{#1}^{#2}}
\newcommand{\CPP}[2]{\cech\PP{#1}{#2}}
\newcommand{\RPP}[2]{\rips\PP{#1}{#2}}

\newcommand{\ext}[1]{\E\xspace #1}

\newcommand{\fullversion}{Appendix} %{full version of this paper}

\bibliographystyle{plainurl}

\title{From Coverage Testing to Topological Scalar Field Analysis}

% Please enter author specific funding statements as fifth argument of the \author macro.
\author{Kirk P. Gardner}{North Carolina State University, United States}{kpgardn2@ncsu.edu}{https://orcid.org/0000-0001-5306-2174}{}
\author{Donald R. Sheehy}{North Carolina State University, United States}{don.r.sheehy@gmail.com}{https://orcid.org/0000-0002-9177-2713}{}

\authorrunning{K.\,P. Gardner and D.\,R. Sheehy}

\Copyright{Kirk P. Gardner and Donald R. Sheehy}

\begin{CCSXML}
  <ccs2012>
    <concept>
      <concept_id>10002950.10003741.10003742.10003744</concept_id>
      <concept_desc>Mathematics of computing~Algebraic topology</concept_desc>
      <concept_significance>500</concept_significance>
    </concept>
    <concept>
      <concept_id>10003752.10010061.10010063</concept_id>
      <concept_desc>Theory of computation~Computational geometry</concept_desc>
      <concept_significance>300</concept_significance>
    </concept>
  </ccs2012>
\end{CCSXML}

\ccsdesc[500]{Mathematics of computing~Algebraic topology}
\ccsdesc[300]{Theory of computation~Computational geometry}


\keywords{topology, homology, coverage, scalar fields, persistent homology, relative homology}

% \category{} %optional, e.g. invited paper

% \relatedversion{} %optional, e.g. full version hosted on arXiv, HAL, or other respository/website
%\relatedversion{A full version of the paper is available at \url{...}.}

% \supplement{}%optional, e.g. related research data, source code, ... hosted on a repository like zenodo, figshare, GitHub, ...

 %optional, to capture a funding statement, which applies to all authors.
\funding{This research was supported by the NSF under grants CCF-1525978 and CCF-1652218.}

% \acknowledgements{I want to thank \dots}%optional

%\nolinenumbers %uncomment to disable line numbering

%\hideLIPIcs  %uncomment to remove references to LIPIcs series (logo, DOI, ...), e.g. when preparing a pre-final version to be uploaded to arXiv or another public repository

%Editor-only macros:: begin (do not touch as author)%%%%%%%%%%%%%%%%%%%%%%%%%%%%%%%%%%
\EventEditors{John Q. Open and Joan R. Access}
\EventNoEds{2}
\EventLongTitle{42nd Conference on Very Important Topics (CVIT 2016)}
\EventShortTitle{CVIT 2016}
\EventAcronym{CVIT}
\EventYear{2016}
\EventDate{December 24--27, 2016}
\EventLocation{Little Whinging, United Kingdom}
\EventLogo{}
\SeriesVolume{42}
\ArticleNo{23}
%%%%%%%%%%%%%%%%%%%%%%%%%%%%%%%%%%%%%%%%%%%%%%%%%%%%%%

\begin{document}

\maketitle

% !TeX root = ../main.tex

\begin{abstract}
  The topological coverage criterion (TCC) can be used to test whether an underlying space is sufficiently well covered by a given data set.
  Given a sufficiently dense sample, topological scalar field analysis (SFA) can give a summary of the shape of a real-valued function on a space.
  The goal of this paper is to put these theories together so that one can test coverage with the TCC and then compute a summary with SFA.
  The challenge is that the TCC requires a well-defined boundary that is not generally available in the SFA settings.
  To overcome this, we show how the scalar field itself can be used to define a boundary that can then be used in the TCC.
  This requires a generalization of the TCC proof and resolves one of the major barriers to wider use of the TCC.
  % It also extends SFA methods to a wider class of spaces.
  It also extends SFA methods to the setting in which coverage is only confirmed in a subset of a space surrounded by a sub-levelset.
  We show how the intersection of these two theories can be used to approximate the persistent homology of the scalar field with respect to a static sub-levelset.
  We then discuss how this persistent relative homology relates to that of the scalar field as a whole.
\end{abstract}


\section{Introduction}\label{sec:introduction}
% !TeX root = ../main.tex

In the topological analysis of scalar fields (SFA), one computes a topological summary capturing qualitative and quantitative shape information from a set of points endowed with a metric and a real-valued function.
That is, we have points with distances and a real number assigned to each point.
More generally, it usually suffices to have a neighborhood graph on the points identifying the pairs of points that close.
The topological computation often takes the form of persistent homology and integrates the local information from the function into global information about the behavior of the function on the entire space.
In prior work, Chazal et al.~\cite{chazal09analysis} showed that for sufficiently dense samples on sufficiently smooth spaces, the persistence diagram can be computed with some guarantees.
In followup work, Buchet et al.~\cite{buchet15topological} extended this result to show how to work with noisy inputs.
A fundamental assumption required to have strong guarantees on the output of these methods is that the underlying space be sufficiently well-sampled.
In this paper, we show how to combine scalar fields analysis with the theory of topological coverage testing to simultaneously compute the persistence diagram and also to test that the underlying space is sufficiently well-sampled.

Initiated by De Silva and Ghrist~\cite{desilva06coordinate,desilva07coverage,desilva07homological}, the theory of homological sensor networks addresses the problem of testing coverage of a bounded domain by a collection of sensors without coordinates.
The main result is the topological coverage criterion, which, in its most general form, states that under reasonable geometric assumptions, the $d$-dimensional homology of a pair of simplicial complexes built on the neighborhood graph will be nontrivial if and only if there is sufficient coverage (see Section~\ref{sec:tcc} for the precise statements).
This relative persistent homology test is called the Topological Coverage Criterion (TCC).

Superficially, the methods of SFA and TCC are very similar.
Both construct similar complexes and compute the persistent homology of the homological image of a complex on one scale into that of a larger scale.
They even overlap on some common techniques in their analysis including the use of the Nerve theorem and the Rips-\v{C}ech interleaving.
However, they differ in some fundamental way that makes it difficult to combine them into a single technique.
The main difference is that the TCC requires a clearly defined boundary.
Not only must the underlying space be a bounded subset of $\R^d$, the data must also be labeled to indicate which input points are close to the boundary.
This requirement is perhaps the main reason why the TCC can so rarely be applied in practice.
Cavanna et al.~\cite{cavanna2017when} generalized the TCC to allow for more general spaces and robust coverage guarantees.
That work gave a different approach to proving the correctness of the TCC which allows much more freedom in how the boundary is defined.


% Moreover, as a necessary but not sufficient condition for coverage there is room to question what can go wrong in the case of false positives.
% In fact, the efficacy of the condition relies on having enough sensors close enough to approximate the boundary in homology.
% This leads us to believe the condition checks for something more specific than coverage alone.
% Specifically, that we have a sample as well as a subsample that reflect the topology of the space and its boundary as a pair.


\section{Summary}\label{sec:summary}
% !TeX root = ../main.tex

Let $\X$ denote an orientable $d$-manifold and $D\subset\X$ a compact subspace.
For a $c$-Lipschitz functon $f : D\to \R$ and $\alpha\in\R$ let $B_\alpha := f^{-1}((-\infty,\alpha])$ denote the $\alpha$-sublevel set of $f$.
Our sample will be denoted $P$, and the subset of points sampling $B_\alpha$ will be denoted $Q_\alpha := P\cap B_\alpha$.
For $\e > 0$ let $P^\e$ denote the union of open metric balls centered at points in $P$.
For ease of exposition let
\[ D\subi{z}{\alpha} := B_\alpha\cup B_z \]
denote the \emph{$z$-truncated} sublevel sets of $f$ and % of the restricted function $f\rest_{D\setminus B_w}$ for all $\alpha,w\in\R$.
\[ P\subi{z}{\alpha} := Q_\alpha\cup Q_z\]
for all $z,\alpha\in\R$.

We will select a sublevel set $B_\omega$ of $f$ that \emph{surrounds} $D$ to serve as our boundary.
Given a sample of $f$ at a finite number of points $P$ in $D$ we would like to confirm $P^\delta$ not only covers the interior $D\setminus B_\omega$, but also that $Q^\delta$ surrounds $P^\delta$ for some $Q\subset P$.
That is, we would like to verify that a pair $(P^\delta, Q^\delta)$ is representative of the pair $(D,B_\omega)$ in homology.
Our goal is to use this fact to approximate the persistence of $f$ relative to $B_\omega$.

Our approximation will be by a nested pair of (Vietoris-)Rips complexes, denoted $\rips^\e(P, Q) = (\rips^\e(P), \rips^\e(Q))$ for $\e > 0$.
Under mild regularity assumptions it can be shown that
\[ \rk~\hom_d(\rips^\delta(P, Q_{\omega - 2c\delta})\hookrightarrow \rips^{2\delta}(P, Q_{\omega+c\delta}))\geq \dim~\hom_0(\rips^\delta(P\setminus Q_{\omega-2c\delta}))\]
implies $D\setminus B_\omega\subseteq P^\delta$ and $Q_{\omega-2c\delta}^\delta$ surrounds $P^\delta$ in $D$.
Proof of this fact generalizes that of the TCC to boundaries defined in terms of a function $f$, eliminating unnatural assumptions made in previous work.
Not only are our subsamples $Q_{\omega-2c\delta}$ and $Q_{\omega+c\delta}$ defined by their function values, but our regularity assumptions can now be stated directly in terms of the persistent homology of $f$.

Given a sample $P$ that satisfies the TCC we can approximate the persistent homology of $f$ as follows.
The nested pair of Rips complexes used to confirm coverage can be extended to a filtration
\[ \{\rips^{2\delta}(P\subi{\omega-2c\delta}{\alpha}, Q_{\omega-2c\delta})\hookrightarrow \rips^{4\delta}(P\subi{\omega+c\delta}{\alpha}, Q_{\omega+c\delta})\}_{\alpha\in\R}\]
that can be used to approximate the persistent homology of $\{(D\subi{\omega}{\alpha}, B_\omega)\}_{\alpha\in\R}$.
The use of images of relative persistence modules is not only to eliminate noise at the boundary, but also to \emph{truncate} the persistence of $f$ in a way that isolates global structure.

\paragraph*{Outline}

We will begin with our statement of the TCC in Section~\ref{sec:tcc}.
A Part of the proof of the TCC will be generalized to properties of \emph{surrounding pairs}, simplifying our reformulation of the TCC in Theorem~\ref{thm:algo_tcc}.
Section~\ref{sec:middle} introduces extensions of surrounding pairs, as well as partial interleavings of image modules.
These are the main technical results used to show that a positive result from the TCC verifies that a surrounding pair of samples can be used to approximate the persistence of a function relative to a sublevel set in Theorem~\ref{thm:interleaving_main_2}.
In Section~\ref{sec:truncations} we provide an interpretation of this relative persistence as a truncation of the full diagram (i.e., the persistence of $f$ on all of $D$) that is motivated by examples from a proof-of-concept implementation in Section~\ref{sec:experiments}.


\section{The Topological Coverage Criterion (TCC)}\label{sec:tcc}
\input{src/1-tcc}

\section{From Coverage Testing to the Analysis of Scalar Fields}\label{sec:middle}
\input{src/2-middle}
% !TeX root = ../main.tex

\subsection{Proof of the Interleaving}

For $z,\alpha\in\R$ let $\DD{z}^k$ denote the $k$th persistent (relative) homology module of the filtration $\{(D\subi{z}{\alpha},B_z)\}_{\alpha\in\R}$ with respect to $B_z$, and let $\PP{z}{\e,k}$ denote the $k$th persistent (relative) homology module of $\{(P\subi{z}{\alpha}^\e,Q_z^\e)\}_{\alpha\in\R}$.
Similarly, let $\CPP{z}{\e,k}$ and $\RPP{z}{\e,k}$ denote the corresponding \v Cech and Rips filtrations, respectively.
We will omit the dimension $k$ and write $\DD{z}$ (resp. $\PP{z}{\e}$) if a statement holds for all dimensions.
If $Q_z^\delta$ surrounds $P^\delta$ in $D$ let $\ext{\PP{z}{\e}}$ denote the $k$th persistent homology module of the filtration of extensions $\{(\ext{P\subi{z}{\alpha}^\e},\ext{Q_z^\e})\}$ for any $\e\geq\delta$, where $\ext{P\subi{z}{\alpha}^\e} = P\subi{z}{\alpha}^\e \cup (D\setminus P^\delta)$.

Lemma~\ref{lem:inclusions} follows directly from the definition of truncated sublevel sets.
This is used to extend Lemma~\ref{lem:surround_and_cover} to persistence modules in Lemma~\ref{lem:inclusion_hom} in order to provide a sequence of shifted homomorphisms $\DD{\omega-3c\delta}\xrightarrow{F}\E\PP{\omega-2c\delta}{\e}\xrightarrow{M}\DD{\omega}\xrightarrow{G}\E\PP{\omega+c\delta}{2\e}\xrightarrow{N}\DD{\omega+5c\delta}$ of varying degree.
These homomorphisms are then combined with those given by the Nerve Theorem and the Rips-\v Cech interleaving in Lemma~\ref{lem:partial_interleaving} to obtain partial interleavings required for our proof of Theorem~\ref{thm:interleaving_main_2}.

\begin{lemma}\label{lem:inclusions}
  If $\delta\leq\e$ and $t,\alpha\in\R$ then $P^\delta\cap D\subi{t-c\e}{\alpha-c\e}\subseteq P\subi{t}{\alpha}^\e\subseteq D\subi{t+c\e}{\alpha+c\e}$.
\end{lemma}

\begin{lemma}\label{lem:inclusion_hom}
  Let $s + 3c\delta\leq t + 2c\delta\leq u\leq v-c\delta\leq w-5c\delta$ and $\e\in [\delta,2\delta]$.
  If $Q_{t}^\delta$ surrounds $P^\delta$ in $D$ and $D\setminus B_u\subseteq P^\delta$ then the following homomorphisms are induced by inclusions:
  \[(F, G)\in \Hom^{c\delta}(\DD{s}, \E\PP{t}{\e})\times \Hom^{2c\delta}(\DD{u}, \E\PP{v}{2\e}),\ (M, N)\in \Hom^{c\e}(\E\PP{t}{\e},\DD{u})\times\Hom^{2c\e}(\E\PP{v}{2\e}, \DD{w}).\]
\end{lemma}

\begin{lemma}\label{lem:partial_interleaving}
  For $\delta < \varrho_D$ and $s + 3c\delta\leq t + 2c\delta\leq u\leq v-c\delta\leq w-5c\delta$ let
  $\Gamma\in\Hom(\DD{s},\DD{u})$,
  $\Pi\in\Hom(\DD{u},\DD{w})$, and
  $\Lambda\in\Hom(\RPP{t}{2\delta}, \RPP{v}{4\delta})$ be induced by inclusions.

  If $Q_{t}^\delta$ surrounds $P^\delta$ in $D$ and $D\setminus B_u\subseteq P^\delta$ then there is a partial $2c\delta$ interleaving $\Phi^*\in\Hom^{2c\delta}(\im~\Gamma, \im~\Lambda)$ and a partial $4c\delta$ interleaving $\Psi^*\in\Hom^{4c\delta}(\im~\Lambda, \im~\Pi)$.
\end{lemma}
\begin{proof}
  Because the shifted homomorphisms provided by Lemma~\ref{lem:inclusion_hom} are all induced by inclusions the following diagram commutes for all $\alpha\leq\beta$.
  So we have image module homomorphisms $\Phi(F, G)\in\Hom^{2c\delta}(\im~\Gamma, \im~C\circ A)$ and $\Psi(M, N)\in\Hom^{4c\delta}(\im~E\circ C, \im~\Pi)$.
  \[\begin{tikzcd}
      \hom_k(D\subi{s}{\alpha-2c\delta}, B_s) \arrow{r}{f_{\alpha-2c\delta}}\arrow{d}{\gamma_{\alpha-2c\delta}[\beta-\alpha]} &
      \hom_k(\E P\subi{t}{\alpha}^\delta, \E Q_t^\delta)\arrow{d}{c_\alpha[\beta-\alpha]\circ a_\alpha}\\
      %
      \hom_k(D\subi{u}{\beta-2c\delta}, B_u)\arrow{r}{g_{\beta-2c\delta}} &
      \hom_k(\E P\subi{v}{\beta}^{2\delta}, \E Q_v^{2\delta})
    \end{tikzcd}
    \begin{tikzcd}
      \hom_k(\E P\subi{t}{\alpha}^{2\delta}, \E Q_t^{2\delta})\arrow{d}{e_\beta\circ c_\alpha[\beta-\alpha]}\arrow{r}{m_{\alpha}} &
      \hom_k(D\subi{u}{\alpha+4c\delta}, B_u)\arrow{d}{\gamma_{\alpha+4c\delta}[\beta-\alpha]}\\
      %
      \hom_k(\E P\subi{v}{\beta}^{4\delta}, \E Q_v^{4\delta})\arrow{r}{n_\beta} &
      \hom_k(D\subi{w}{\beta+4c\delta}, B_w)
    \end{tikzcd}\]

  Because the isomorphisms provided by Lemma~\ref{lem:excision} are given by excision they are induced by inclusion, and therefore give isomorphisms $\E_z^\e \in \Hom(\PP{z}{\e},\ext{\PP{z}{\e}})$ for any $z\in\R$ such that $Q_z^\e$ surrounds $P^\delta$ in $D$.
  For any $\e < \varrho_D$ we have isomorphisms $\N_z^\e\in\Hom(\CPP{z}{\e}, \PP{z}{\e})$ that commute with maps induced by inclusions by the Persistent Nerve Lemma.
  So the compositions $\E_z^\e\circ \N_z^\e$ are isomorphisms that commute with maps induced by inclusion as well.
  These compositions, along with the Rips-\v Cech interleaving, provide maps $\E\PP{t}{\delta}\xrightarrow{F'}\RPP{t}{2\delta}\xrightarrow{M'} \E\PP{t}{2\delta}$ and $\E\PP{v}{2\delta}\xrightarrow{G'}\RPP{v}{4\delta}\xrightarrow{N'} \E\PP{v}{4\delta}$ that commute with maps induced by inclusions.
  So we have the following commutative diagram:
  %
  \begin{equation}
    \begin{tikzcd}
      \E\PP{t}{\delta}\arrow{rr}{A}\arrow{dr}{F'} & &
      \E\PP{t}{2\delta}\arrow{r}{C} &
      \E\PP{v}{2\delta}\arrow{rr}{E}\arrow{dr}{G'} & &
      \E\PP{v}{4\delta}\\
      %
      & \RPP{t}{2\delta}\arrow{ur}{M'}\arrow[to=RV, "\Lambda"] & &
      & |[alias=RV]|\RPP{v}{4\delta}\arrow{ur}{N'} &
    \end{tikzcd}
  \end{equation}
  %
  That is, we have image module homomorphisms $\Phi'(F', G')$ and $\Psi'(M', N')$ such that $A = M'\circ F'$, $E = N'\circ G'$, and $\Lambda = G'\circ C\circ M'$.
  Because all maps are induced by inclusions or commute with maps induced by inclusions $\Phi^* = \Phi'\circ \Phi\in\Hom^{2c\delta}(\im~\Gamma, \im~\Lambda)$ and $\Psi^* = \Psi\circ\Psi'\in\Hom^{4c\delta}(\im~\Lambda, \im~\Pi)$ by Lemm~\ref{lem:image_composition}.

  Because $G,M,C$ are induced by inclusions $C[3c\delta] = G\circ M$, so $\Lambda[3c\delta] = G'\circ C[3c\delta]\circ M' = G'\circ (G\circ M)\circ M'$ as $G', M'$ commute with maps induced by inclusions.
  In the same way, $\Gamma[3c\delta] = M\circ (A\circ F) = M\circ (M'\circ F')\circ F$ and $\Pi[5c\delta] = N\circ (E\circ G) = N\circ (N'\circ G')\circ G$.

  Let $F^*:= F'\circ F$, $G^*:= G'\circ G$, $M^*:=M'\circ M$, and $N^*:=N'\circ N$.
  So $\Phi^*_{M^*}$ is a partial $2c\delta$ interleaving as $\Gamma[3c\delta] = M^*\circ F^*$ and $\Lambda[3c\delta] = G^*\circ M^*$, and $\Psi^*_{G^*}$ is a partial $4c\delta$ interleaving as $\Lambda[3c\delta] = G^*\circ M^*$ and $\Pi[5c\delta] = N^*\circ G^*$.
\end{proof}

The partial interleavings given by Lemma~\ref{lem:partial_interleaving}, along with assumptions that imply $\im(\DD{\omega-3c\delta}\to \DD{\omega+5c\delta})\cong \DD{\omega}$, provide the proof of Theorem~\ref{thm:interleaving_main_2} by Lemma~\ref{thm:interleaving_main}.

\begin{theorem}\label{thm:interleaving_main_2}
  Let $\X$ be a $d$-manifold, $D\subset\X$ and $f : D\to\R$ be a $c$-Lipschitz function.
  Let $\omega\in\R$, $\delta < \varrho_D/4$ be constants such that $B_{\omega-3c\delta}$ surrounds $D$ in $\X$.
  Let $P\subset D$ be a finite subset and suppose $\hom_k(B_{\omega-3c\delta}\hookrightarrow B_\omega)$ is surjective and $\hom_k(B_\omega\hookrightarrow B_{\omega+5c\delta})$ is an isomorphism for all $k$.

  If $D\setminus B_\omega\subseteq P^\delta$ and $Q_{\omega-2c\delta}^\delta$ surrounds $P^\delta$ in $D$ then the $k$th persistent homology module of $\{\rips^{2\delta}(P\subi{\omega-2c\delta}{\alpha}, Q_{\omega-2c\delta})\hookrightarrow \rips^{4\delta}(P\subi{\omega+c\delta}{\alpha}, Q_{\omega+c\delta})\}_{\alpha\in\R}$ is $4c\delta$-interleaved with that of $\{(D\subi{\omega}{\alpha}, B_\omega)\}_{\alpha\in\R}$.
\end{theorem}
\begin{proof}
  Let $\Lambda \in\Hom(\RPP{\omega-2c\delta}{2c\delta}, \RPP{\omega+c\delta}{4c\delta})$, $\Gamma\in\Hom(\DD{\omega-3c\delta}, \DD{\omega})$, and $\Pi\in\Hom(\DD{\omega},\DD{\omega+5c\delta})$ be induced by inclusions.
  Because $\delta < \varrho_D/4$, $D\setminus B_\omega\subseteq P^\delta$ and $Q_{\omega-2c\delta}^\delta$ surrounds $P^\delta$ in $D$ we have a partial $2c\delta$ interleaving $\Phi^*\in\Hom^{2c\delta}(\im~\Gamma, \im~\Lambda)$ and a partial $4c\delta$ interleaving $\Psi^*\in\Hom^{4c\delta}(\im~\Lambda, \im~\Pi)$ by Lemma~\ref{lem:partial_interleaving}.
  As we have assumed that $\hom_k(B_{\omega-3c\delta}\hookrightarrow B_\omega)$ is surjective and $\hom_k(B_\omega)\cong\hom_k(B_{\omega+5c\delta})$ the five-lemma implies $\gamma_\alpha$ is surjective and $\pi_\alpha$ is an isomorphism (and therefore injective) for all $\alpha$.
  So $\Gamma$ is an epimorphism and $\Pi$ is a monomorphism, thus $\im~\Lambda$ is $4c\delta$-interleaved with $\DD{\omega}$ by Lemma~\ref{thm:interleaving_main} as desired.
\end{proof}


\section{Approximation of the Truncated Diagram}\label{sec:truncations}
% !TeX root = ../main.tex

% \paragraph*{Relative, Truncated, and Restricted Persistence Diagrams}

For fixed $\omega\in\R$ we will refer to the persistence diagram associated with the filtration $\{(D\subi{\omega}{\alpha}, B_\omega)\}_{\alpha\in\R}$  as the \textbf{relative diagram} of $f$.
In this section we will relate the relative diagram to the \emph{full} diagram of the sublevel set filtration $\{B_\alpha\}_{\alpha\in\R}$.
Specifically, we define the \textbf{truncated diagram} to be the subdiagram of the full consisting of features born \emph{after} $\omega$.
In the following section we will compare the relative and truncated diagrams to the \textbf{restricted diagram}, defined to be that of the sublevel set filtration of $f\rest_{D\setminus B_\omega}$.%

Note that the truncated sublevel sets $D\subi{\omega}{\alpha}$ are equal to the union of $B_\omega$ and the restricted sublevel sets.
It is in this sense that $B_\omega$ is \emph{static} throughout---it is contained in every sublevel set of the relative filtration.
As we will not have verified coverage in $B_\omega$ we cannot analyze the function in this region directly.
We therefore have two alternatives: \emph{restrict} the domain of the function to $D\setminus B_\omega$, or use relative homology to analyze the function \emph{relative} to this region using excision.

\begin{figure}[htbp]
  \centering
  \begin{minipage}[b]{0.27\textwidth}
    \includegraphics[trim=200 200 200 100, clip, width=\textwidth]{figures/surf/ass2_C_side.png}\\
    \includegraphics[trim=200 100 200 200, clip, width=\textwidth]{figures/surf/ass2_C_top.png}
  \end{minipage}
  \begin{minipage}[b]{0.7\textwidth}
    \includegraphics[width=\textwidth]{figures/barcodes/res_rel.png}
  \end{minipage}
  \caption{Full, restricted, and relative barcodes of the function (left).}
\end{figure}

Let $\LL^k$ denote the $k$th persistent homology module of the sublevel set filtration $\{B_\alpha\}_{\alpha\in\R}$.
As in the previous section, let $\DD{\omega}^k$ denote the $k$th persistent (relative) homology module of $\{(D\subi{\omega}{\alpha}, B_\omega)\}_{\alpha\in\R}$.
Throughout we will assume that we are taking homology in a field $\FF$ and that the homology groups $\hom_k(B_\alpha)$ and $\hom_k(D\subi{\omega}{\alpha}, B_\omega)$ are finite dimensional vector spaces for all $k$ and $\alpha\in\R$.
We will use the interval decomposition of $\LL^k$ to give a decomposition of the relative module $\DD{\omega}^k$ in terms of a \emph{truncation} of $\LL^k$.
For fixed $\omega\in\R$ we will define the truncation $\T^k_\omega$ of $\LL^k$ in terms of the intervals decomposing $\LL^k$ that are in $[\omega, \infty)$.

\paragraph*{Truncated Interval Modules}

For an interval $I = [s,t)\subseteq \R$ let $I_+ := [t,\infty)$ and $I_- := (-\infty, s]$.
For $\omega\in\R$ let $\FF_{\omega}^I$ denote the interval module consisting of vector spaces $\{F\subi{\omega}{\alpha}^I\}_{\alpha\in\R}$ and linear maps $\{f\subi{\omega}{\alpha,\beta}^I : F\subi{\omega}{\alpha}^I\to F\subi{\omega}{\beta}^I\}_{\alpha\leq\beta}$ where
\[ F\subi{\omega}{\alpha}^I := \begin{cases} F_\alpha^I&\text{ if } \omega\in I_-\\ 0&\text{ otherwise,}\end{cases}\ \text{ and }\ \ f\subi{\omega}{\alpha,\beta}^I := \begin{cases} f_{\alpha,\beta}^I&\text{ if } \omega\in I_-\\ 0&\text{ otherwise.}\end{cases}\]
For a collection $\I$ of intervals let $\I_\omega := \{I\in\I\mid \omega\in I\}$.

\begin{lemma}\label{lem:decomposition}
  Suppose $\I^k$ and $\I^{k-1}$ are collections of intervals that decompose $\LL^k$ and $\LL^{k-1}$, respectively.
  Then for all $k$ the $k$th persistent homology module of $\{(D\subi{\omega}{\alpha}, B_\omega)\}_{\alpha\in\R}$ is isomorphic to
  \[\bigoplus_{I\in\I^k} \FF_\omega^I \oplus \bigoplus_{I\in \I_\omega^{k-1}} \FF^{I_+}.\]
\end{lemma}
\begin{proof}
  Suppose $\alpha\leq\omega$.
  So $\hom_k(D\subi{\omega}{\alpha}, B_\omega) = 0$ as $D\subi{\omega}{\alpha} = B_\omega\cup B_\alpha$ and $\T^k_\omega = 0$ as $F_\alpha^I = 0$ for any $I\in \I^k$ such that $\omega\in I_-$.
  Moreover, $\omega\in I$ for all $I\in \I_\omega^{k-1}$, thus $F_\alpha^{I_+} = 0$ for all $\alpha\leq\omega$.
  So it suffices to assume $\omega < \alpha$.

  Consider the long exact sequence of the pair $\hom_k(D\subi{\omega}{\alpha}, B_\omega) = \hom_k(B_\alpha, B_\omega)$
  \[ \ldots\to \hom_k(B_\omega)\xrightarrow{p_\alpha^k} \hom_k(B_\alpha)\xrightarrow{q_\alpha^k}\hom_k(D\subi{\omega}{\alpha}, B_\omega)\xrightarrow{r_\alpha^k} \hom_{k-1}(B_\omega)\xrightarrow{p_\alpha^{k-1}}\hom_{k-1}(B_\alpha)\to\ldots\]
  where $\hom_k(B_\alpha) = \bigoplus_{I\in \I^k}F_\alpha^I$, $\hom_k(B_\omega) = \bigoplus_{I\in \I^k}F_\omega^I$, and $p_\alpha^k = \displaystyle\bigoplus_{I\in\I^k} f_{\omega,\alpha}^I$.

  Noting that $\im~q_\alpha^k \cong \hom_k(B_\alpha) / \ker~q_\alpha^k$ where $\ker~q_\alpha^k = \im~p_\alpha^k$ by exactness we have $\ker~r_\alpha^k \cong \hom_k(B_\alpha) / \im~p_\alpha^k$.
  By the definition of $F_\alpha^I$ and $f_{\omega,\alpha}^I$ we know $\im~f_{\omega,\alpha}^I$ is $F_\alpha^I$ if $\omega\in I$ and 0 otherwise.
  As $\im~p_\alpha^k$ is equal to the direct sum of images $\im~f_{\omega,\alpha}^I$ over $I\in\I^k$ it follows that $\im~p_\alpha^k$ is the direct sum of those $F_\alpha^I$ over those $I\in\I^k$ such that $\omega\in I$.
  Now, because $\hom_k(B_\alpha) = \bigoplus_{I\in \I^k}F_\alpha^I$ and each $F_\alpha^I$ is either 0 or $\FF$ the quotient $\hom_k(B_\alpha) / \im~p_\alpha^k$ is the direct sum of those $F_\alpha^I$ such that $\omega\notin I$.
  Therefore, by the definition of $F\subi{\omega}{\alpha}^I$ we have
  \[ \ker~r_\alpha^k = \bigoplus_{I\in\I_\omega^k} F\subi{\omega}{\alpha}^I.\]

  Similarly, $\im~r_\alpha^k = \ker~p_\alpha^{k-1}$ by exactness where $\ker~p_\alpha^{k-1}$ is the direct sum of kernels $\ker~f_{\omega,\alpha}^I$ over $I\in\I^{k-1}$.
  By the definition of $F_\alpha^I$ and $f_{\omega,\alpha}^I$ we know that $\ker~f_{\omega,\alpha}^I$ is $F_\alpha^I$ if $\omega\notin I$ and $0$ otherwise.
  Noting that $\ker~f_{\omega,\alpha}^I = 0$ for any $I\in \I^{k-1}$ such that $\omega\notin I$ it suffices to consider only those $I\in \I_\omega^{k-1}$.
  It follows that $\ker~f_{\omega,\alpha}^I = F_\alpha^{I_+}$ for any $I$ containing $\omega$ as $\omega < \alpha$.
  Therefore,
  \[\im~r_\alpha^k = \bigoplus_{I\in\I^{k-1}} F_\alpha^{I_+}.\]

  We have the following split exact sequence associated with $r_\alpha^k$
  \[ 0\to \ker~r_\alpha^k\to \hom_k(D\subi{\omega}{\alpha}, B_\omega)\to\im~r_\alpha^k\to 0.\]
  The desired result follows from the fact that for all $\alpha\in\R$
  \begin{align*}
    \hom_k(D\subi{\omega}{\alpha}, B_\omega) &\cong \ker~r_\alpha^k\oplus \im~r_\alpha^k =\bigoplus_{I\in\I^k} F\subi{\omega}{\alpha}^I\oplus \bigoplus_{I\in\I_\omega^{k-1}} F_\alpha^{I_+}.
  \end{align*}
\end{proof}

Letting $\I^k$ denote the decomposing intervals of $\LL^k$ for all $k$ we can define the \textbf{$\omega$-truncated $k$th persistent homology module} of $\LL^k$ as
\[ \T_\omega^k := \bigoplus_{I\in\I^k} \FF_\omega^I\ \text{ and let }\ \LL_\omega^{k-1} := \bigoplus_{I\in \I_\omega^{k-1}} \FF^{I_+}\]
denote the submodule of $\DD{\omega}^k$ consisting of intervals $[\beta,\infty)$ corresponding to features $[\alpha,\beta)$ in $\LL^{k-1}$ such that $\alpha\leq\omega <\beta$.
Now, by Lemma~\ref{lem:decomposition} the $k$th persistent (relative) homology module of $\{(D\subi{\omega}{\alpha}, B_\omega)\}_{\alpha\in\R}$ is $\DD{\omega}^k = \T_\omega^k\oplus \LL_\omega^{k-1}.$
Theorems~\ref{thm:algo_tcc} and~\ref{thm:interleaving_main_2} can then be used to show that
\[ \{\rips^{2\delta}(P\subi{\omega-2c\delta}{\alpha}, Q_{\omega-2c\delta})\hookrightarrow \rips^{4\delta}(P\subi{\omega+c\delta}{\alpha}, Q_{\omega+c\delta})\}_{\alpha\in\R} \]
is $4c\delta$ interleaved with $\T_\omega^k\oplus \LL_\omega^{k-1}$ whenever
\[ \rk~\hom_d(\rips^\delta(P, Q_{\omega - 2c\delta})\hookrightarrow \rips^{2\delta}(P, Q_{\omega+c\delta})) \geq \dim~\hom_0(\rips^\delta(P\setminus Q_{\omega-2c\delta})).\]


\section{Experiments}\label{sec:experiments}
\input{src/5-experiments}

\section{Conclusion}
% !TeX root = ../main_socg.tex

We have extended the Topological Coverage Criterion to the setting of Topological Scalar Field Analysis.
By defining the boundary in terms of a sublevel set of a scalar field we provide an interpretation of the TCC that applies more naturally to data coverage.
We then showed how the assumptions and machinery of the TCC can be used to approximate the persistent homology of the scalar field relative to a static sublevel set.
This relative persistent homology is shown to be related to a truncation of that of the scalar field as whole, and therefore provides a way to approximate a part of its persistence diagram in the presence of un-verified data.

There are a number of unanswered questions and directions for future work.
From the theoretical perspective, our understanding of duality limited us in providing a more elegant extension of the TCC.
A better understanding of when and how duality can be applied would allow us to give a more rigorous statement of our assumptions.
Moreover, as duality plays a central role in the TCC it is natural to investigate its role in the analysis of scalar fields.
This would not only allow us to apply duality to persistent homology~\cite{desilva11duality}, but also allow us to provide a rigorous comparison between the relative approach and the persistent homology of the superlevel set filtration and explore connections with Extended Persistence~\cite{cohen09extending}.

From a computational perspective, we interested in exploring how to recover the full diagram as discussed in Section~\ref{sec:experiments}.
Our statements in terms of sublevel sets can be generalized to disjoint unions of sub and superlevel sets, where coverage is confirmed in an \emph{interlevel} set.
This, along with a better understanding of the relationship between sub and superlevel sets could lead to an iterative approach in which the persistent homology of a scalar field is constructed as data becomes available.
We are also interested in finding efficient ways to compute the image persistent (relative) homology that vary in both scalar and scale.

The problem of relaxing our assumptions on the boundary can be approached from both a theoretical and computational perspective.
Ways to avoid the isomorphism we require could be investigated in theory, and the interaction of relative persistent homology and the Persistent Nerve Lemma may be used tighten our assumptions.
We would also like to conduct a more rigorous investigation on the effect of these assumptions in practice.


\bibliography{bibliography}

\appendix

\section{Omitted Proofs}\label{apx:omit}
% !TeX root = ../main.tex

\clearpage

\section{Helpful Diagram}

% \begin{landscape}
% \begin{scriptsize}
% \begin{centering}
% \[\begin{tikzcd}[row sep=large, column sep=scriptsize]
%   |[alias=U]| \D{\omega-c(\delta+\zeta)}{\alpha-3c\delta}
%                                   \arrow[to=V, "\gamma_{\alpha-3c\delta}{[3c\delta]}"]
%                                   \arrow[to=Sa, "f_{\omega-c(\delta+\zeta}{[\alpha-3c\delta]}"]
%   &&&& |[alias=V]| \D{\omega}{\alpha}
%                                   \arrow[to=W, "\pi_\alpha{[3c\zeta]}"]
%                                   \arrow[to=Ta, "f_\omega{[\alpha]}"]
%   &&&& |[alias=W]|
%   \D{\omega+c(\delta+2\zeta)}{\alpha+3c\zeta}\\
%   %
%   & |[alias=Sa]|
%   \P{\omega-c\zeta}{\delta}{\alpha-2c\delta}
%                                   \arrow[to=Sb, "s_{\alpha-2c\delta}"]
%                                   \arrow[to=CSa, "(\E\N_{w-c\zeta}^\delta{[\alpha-2c\delta]})^{-1}"]
%   && |[alias=Sb]| \P{\omega-c\zeta}{2\delta}{\alpha-2c\delta}
%                                   \arrow[to=Ta, "\vartheta_{\alpha-2c\delta}{[2c\delta+\zeta]}"]
%                                   \arrow[to=V, "m_{\omega-c\zeta}^{2\delta}{[\alpha-2c\delta]}"]
%   && |[alias=Ta]| \P{\omega+c\delta}{\zeta}{\alpha+c\zeta}
%                                   \arrow[to=Tb, "t_{\alpha+c\zeta}"]
%                                   \arrow[to=CTa, "(\E\N_{w+c\delta}^\zeta{[\alpha+c\zeta]})^{-1}"]
%   && |[alias=Tb]| \P{\omega+c\delta}{2\zeta}{\alpha+c\zeta}
%                                   \arrow[to=W, "m_{\omega+c\delta}^{2\zeta}{[\alpha+c\zeta]}"]
%   &\\
%   %
%   & |[alias=CSa]|
%   \CP{\omega-c\zeta}{\delta}{\alpha-2c\delta}
%                                   \arrow[to=CSb, "\check{s}_{\alpha-2c\delta}"]
%                                   \arrow[to=RS, "\I_{\omega-c\zeta}^\delta{[\alpha-2c\delta]}"']
%   && |[alias=CSb]| \CP{\omega-c\zeta}{2\delta}{\alpha-2c\delta}
%                                   \arrow[to=Sb, "\E\N_{w-c\zeta}^{2\delta}{[\alpha-2c\delta]}"]
%                                   \arrow[to=CTa, "\check{\vartheta}_{\alpha-2c\delta}{[2c\delta+\zeta]}"]
%   && |[alias=CTa]| \CP{\omega+c\delta}{\zeta}{\alpha+c\zeta}
%                                   \arrow[to=CTb, "\check{t}_{\alpha+c\zeta}"]
%                                   \arrow[to=RT, "\I_{\omega+c\delta}^\zeta{[\alpha+c\zeta]}"']
%   && |[alias=CTb]| \CP{\omega+c\delta}{2\zeta}{\alpha+c\zeta}
%                                   \arrow[to=Tb, "\E\N_{w+c\delta}^{2\zeta}{[\alpha+2c\zeta]}"]
%   &\\
%   %
%   && |[alias=RS]| \CP{\omega-c\zeta}{2\delta}{\alpha-2c\delta}
%                                   \arrow[to=CSb, "\J_{\omega-c\zeta}^{2\delta}{[\alpha-2c\delta]}"]
%                                   \arrow[to=RT, "\rips\lambda_{\alpha-c\zeta}{[c(2\delta+\zeta)]}"]
%   &&&& |[alias=RT]| \RP{\omega+c\delta}{2\zeta}{\alpha+c\zeta}
%                                   \arrow[to=CTb, "\J_{\omega+c\delta}^{2\zeta}{[\alpha+c\zeta]}"]
%   &&\\
% \end{tikzcd}\]
% \end{centering}
% \end{scriptsize}
% \end{landscape}

\begin{scriptsize}
\begin{centering}
\[\begin{tikzcd}[row sep=large, column sep=scriptsize]
  |[alias=U]| \DD{\omega-c(\delta+\zeta)}
                                  \arrow[to=V, two heads, blue, "\Gamma{[3c\delta]}"]
                                  \arrow[to=Sa,blue, "F"]
  &&&& |[alias=V]| \DD{\omega}
                                  \arrow[to=W, hook,red, "\Pi{[3c\zeta]}"]
                                  \arrow[to=Ta,red, "G"]
  &&&& |[alias=W]|
  \DD{\omega+c(\delta+2\zeta)}\\
  %
  & |[alias=Sa]|
  \E\PP{\omega-c\zeta}{\delta}
                                  \arrow[to=Sb, blue, "\mathcal{S}"]
                                  \arrow[to=CSa, "(\E\N_{w-c\zeta}^\delta)^{-1}"]
  && |[alias=Sb]| \E\PP{\omega-c\zeta}{2\delta}
                                  \arrow[to=Ta, "\Theta{[2c\delta+\zeta]}"]
                                  \arrow[to=V, blue, "M"]
  && |[alias=Ta]| \E\PP{\omega+c\delta}{\zeta}
                                  \arrow[to=Tb, red, "\mathcal{T}"]
                                  \arrow[to=CTa, "(\E\N_{w+c\delta}^\zeta)^{-1}"]
  && |[alias=Tb]| \E\PP{\omega+c\delta}{2\zeta}
                                  \arrow[to=W, red, "N"]
  &\\
  %
  & |[alias=CSa]|
  \CPP{\omega-c\zeta}{\delta}
                                  \arrow[to=CSb, green, "\cech\mathcal{S}"]
                                  \arrow[to=RS, green, "\I_{\omega-c\zeta}^\delta"']
  && |[alias=CSb]| \CPP{\omega-c\zeta}{2\delta}
                                  \arrow[to=Sb, "\E\N_{w-c\zeta}^{2\delta}"]
                                  \arrow[to=CTa, "\cech\Theta{[2c\delta+c\zeta]}"]
  && |[alias=CTa]| \CPP{\omega+c\delta}{\zeta}
                                  \arrow[to=CTb, orange, "\cech\mathcal{T}"]
                                  \arrow[to=RT, orange, "\I_{\omega+c\delta}^\zeta"']
  && |[alias=CTb]| \CPP{\omega+c\delta}{2\zeta}
                                  \arrow[to=Tb, "\E\N_{w+c\delta}^{2\zeta}"]
  &\\
  %
  && |[alias=RS]| \RPP{\omega-c\zeta}{2\delta}
                                  \arrow[to=CSb, green, "\J_{\omega-c\zeta}^{2\delta}"]
                                  \arrow[to=RT, "\rips\Lambda"]
  &&&& |[alias=RT]| \RPP{\omega+c\delta}{2\zeta}
                                  \arrow[to=CTb, orange, "\J_{\omega+c\delta}^{2\zeta}"]
  &&\\
\end{tikzcd}\]
\end{centering}
\end{scriptsize}

\begin{enumerate}
  \item Blue commutes (all inclusions) + green commutes (all inclusions) + middle left square commutes with inclusion (pers. nerve lemma) $\implies$ weak interleaving of $\Gamma$ with $\RPP{\omega-c\zeta}{2\delta}$.
  \item Red commutes (all inclusions) + orange commutes (all inclusions) + middle right square commutes with inclusion (pers. nerve lemma) $\implies$ weak interleaving of $\Pi$ with $\RPP{\omega+c\delta}{2\zeta}$.
  \item Middle commutes using the same arguments $\implies$ weak interleaving of $\rips\Lambda$ with $\DD{\omega}$
  \item weak interleaving of $\Gamma$ with $\RPP{\omega-c\zeta}{2\delta}$ + weak interleaving of $\rips\Lambda$ with $\DD{\omega}$ + path \textbf{down} from $\Gamma$ to $\rips\Lambda$ is an image module homomorphism $\implies$ partial interleaving of image modules from $\Gamma$ to $\rips\Lambda$.
  \item weak interleaving of $\rips\Lambda$ with $\DD{\omega}$ + weak interleaving of $\Pi$ with $\RPP{\omega+c\delta}{2\zeta}$ + path \textbf{up} from $\rips\Lambda$ to $\Pi$ is an image module homomorphism $\implies$ partial interleaving of image modules from $\rips\Lambda$ to $\Pi$.
  \item these two partial interleavings + $\Gamma$ epi + $\Pi$ mono $\implies$ $\im~\rips\Lambda$ is interleaved with $\DD{\omega}$ by Lemma~\ref{thm:interleaving_main}.
\end{enumerate}


\section{Duality}\label{apx:duality}
% !TeX root = ../main.tex

For a pair $(A, B)$ in a topological space $X$ and any $R$ module $G$ let $\hom^k(A, B; G)$ denote the \textbf{singular cohomology} of $(A,B)$ (with coefficients in $G$).
Let $\hom^k_c(A, B; G)$ denote the corresponding \textbf{singular cohomology with compact support}.
For any compact pair $(A,B)$ there is an isomorphism $\hom^k_c(A, B; G)\to\hom^k(A, B; G)$.

Corollary\ref{cor:univ_coef} follows from the Universal Coefficient Theorem for singular homology (and cohomology) as vector spaces over a field $\FF$, as the dual vector space $\Hom(\hom_k(A, B), \FF)$ is isomorphic to $\hom_k(A, B; \FF)$ for any finitely generated $\hom_k(A, B)$.

\begin{corollary}\label{cor:univ_coef}
  For a topological pair $(A, B)$ and a field $\FF$ such that $\hom_k(A, B)$ is finitely generated there is a natural isomorphism
  \[\nu : \hom^k(A, B; \FF)\to \hom_k(A, B; \FF).\]
\end{corollary}

Let $\overline{\hom}^k(A, B; G)$ be the \textbf{Alexander-Spanier cohomology} of the pair $(A,B)$, defined as the limit of the direct system of neighborhoods $(U,V)$ of the pair $(A, B)$.
Let $\overline{\hom}^k_c(A, B; G)$ denote the corresponding \textbf{Alexander-Spanier cohomology with compact support} where $\overline{\hom}^k_c(A, B; G)\cong\overline{\hom}^k(A, B; G)$ for any compact pair $(A, B)$.

\begin{theorem}[\textbf{Alexander-Poincar\'e-Lefschetz Duality} (Spanier~\cite{spanier1989algebraic}, Theorem 6.2.17)]\label{thm:alexander}
  Let $X$ be an orientable $d$-manifold and $(A, B)$ be a compact pair in $X$.
  Then for all $k$ and $R$ modules $G$ there is a (natural) isomorphism
  \[\lambda : \hom_k(X\setminus B, X\setminus A; G)\to \overline{\hom}^{d-k}(A, B; G).\]
\end{theorem}

A space $X$ is said to be \textbf{homologically locally connected in dimension $n$} if for every $x\in X$ and neighborhood $U$ of $x$ there exists a neighborhood $V$ of $x$ in $U$ such that $\tilde{\hom}_n(V)\to\tilde{\hom}_n(U)$ is trivial for $k\leq n$.

\begin{lemma}[Spanier p. 341, Corollary 6.9.6]\label{lem:alexander_iso}
  Let $A$ be a closed subset, homologically locally connected in dimension $n$, of a Hausdorff space $X$, homologically locally connected in dimension $n$.
  If $X$ has the property that every open subset is paracompact, $\mu : \overline{\hom}_c^k(X,A; G)\to \hom_c^k(X, A; G)$ is an isomorphism for $k\leq n$ and a monomorphism for $q = n+1$.
\end{lemma}

In the following we will assume homology (and cohomology) over a field $\FF$.

\begin{lemma}\label{cor:alexander_iso}
  Let $X$ be an orientable $d$-manifold and $(A,B)$ a compact pair of locally path connected subspaces in $X$.
  Then
  \[\xi : \hom_d(X\setminus B, X\setminus  A)\to \hom_0(A, B)\]
  is a natural isomorphism.
\end{lemma}
\begin{proof}
  Because $X$ is orientable and $(A,B)$ are compact $\lambda : \hom_d(X\setminus B, X\setminus A)\to \overline{\hom}^{0}(A, B)$ is an isomorphism by Theorem~\ref{thm:alexander}.
  Note that
  Moreover, because every subset of $X$ is (hereditarily) paracompact every open set in $A$, with the subspace topology, is paracompact.
  For any neighborhood $U$ of a point $x$ in a locally path connected space there must exist some neighborhood $V\subset U$ of $x$ that is path connected in the subspace topology.
  As $\tilde{\hom}_0(V) = 0$ for any nonempty, path connected topological space $V$ (see Spanier p. 175, Lemma 4.4.7) it follows that $A$ (resp. $B$) are homologically locally connected in dimension $0$.
  Because $(A,B)$ is a compact pair the singular and Alexander-spanier cohomology modules of $(A,B)$ with compact support are isomorphic to those without, thus $\mu:\overline{\hom}^{0}(A, B)\to \hom^0(A, B)$ is an isomorphism.
  By Corollary~\ref{cor:univ_coef} we have a natural isomorphism $\nu : \hom^0(A, B)\to\hom_0(A, B)$ thus the composition $\xi := \nu\circ\mu\circ\lambda : \hom_d(X\setminus B, X\setminus  A)\to \hom_0(A, B)$ is a natural isomorphism.
\end{proof}

\begin{lemma}\label{lem:duality_apply}
  Let $\X$ be an orientable $d$-manifold let $D$ be a compact subset of $\X$.
  Let $P$ be a finite subset of $D$ such that $P^\e\subset \intr_\X(D)$ and $Q\subseteq P$.

  If $D\setminus Q^\e$ and $D\setminus P^\e$ are locally path connected then there is a natural isomorphism
  \[ \xi : \hom_d(P^\e,Q^\e)\to \hom_0(D\setminus Q^\e, D\setminus P^\e).\]
\end{lemma}
\begin{proof}
  Because $Q^\e$ and $P^\e$ are open in $D$ and $D$ is compact in $\X$ the complement $D\setminus Q^\e$ is closed in $D$, and therefore compact in $\X$.
  Moreover, because $P^\e\subset \intr_\X(D)$, $\hom_d(\X\setminus(D\setminus P^\e), \X\setminus(D\setminus Q^\e)) = \hom_d(P^\e, Q^\e)$.
  As we have assumed these complements are locally path connected by assumption we have a natural isomorphism $\xi : \hom_d(P^\e, Q^\e)\to \hom_0(D\setminus Q^\e, D\setminus P^\e)$
  by Lemma~\ref{cor:alexander_iso}.
\end{proof}


\end{document}
