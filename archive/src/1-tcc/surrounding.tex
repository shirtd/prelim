% !TeX root = ../../main.tex

\begin{definition}[Surrounding Pair]
  Let $X$ be a topological space and $(D,B)$ a pair in $X$.
  The set $B$ \textbf{surrounds $D$ in $X$} if $B$ separates $X$ with the pair $(D\setminus B, X\setminus D)$.
  We will refer to such a pair as a \textbf{surrounding pair in $X$}.
\end{definition}

Unlike the definition of a separating set, which simply breaks a space into disjoint subsets, we make the distinction between interior and exterior explicit by defining the subset $B$ relative $D$.
That is, the set $D\setminus B$ corresponds to the interior of $D$ and $X\setminus D$ corresponds to the complement of $D$ in $X$.
$B$ then serves as a boundary in the sense that there is no path from the ``interior'' to the ``complement,'' which is sufficient for a homological coverage criterion.

For a surrounding pair $(D,B)$ in $X$  the complement $\overline{B} = X\setminus B$ is the union of disconnected sets $X\setminus D$ and $D\setminus B$.
Therefore, $\hom_k(\overline{B}) \cong \hom_k(\overline{D})\oplus \hom_k(D\setminus B)$ thus $\hom_k(\overline{B},\overline{D})\cong \hom_k(D\setminus B)$ for all $k$.

The following lemma generalizes the proof of the TCC as a property of surrounding sets.
We will then combine these results on the homology of surrounding pairs with information about both $X$ as a metric space and our function.

\begin{lemma}\label{lem:coverage}
  Let $(D, B)$ be a surrounding pair in $X$ and $U\subseteq D$, $V\subseteq U\cap B$ be subsets.
  Let $\ell: \hom_0(X\setminus B, X\setminus D)\to \hom_0(X\setminus V, X\setminus U)$ be induced by inclusion.

  If $\ell$ is injective then $D\setminus B\subseteq U$ and $V$ surrounds $U$ in $D$.
\end{lemma}
\begin{proof}
  This proof is in two parts.
  \begin{description}
    \item[$\ell$ injective $\implies$ $D\setminus B\subseteq U$] Suppose, for the sake of contradiction, that $p$ is injective and there exists a point $x\in (D\setminus B)\setminus U$.
      So $[x]$ is non-trivial in $\hom_0(\overline{B},\overline{D})\cong \hom_0(D\setminus B)$ as $x$ is in some connected component of $D\setminus B$.
      So we have the following sequence of maps induced by inclusions
      \[ \hom_0(\overline{B},\overline{D})\xrightarrow{f} \hom_0(\overline{B},\overline{D}\cup\{x\})\xrightarrow{g} \hom_0(\overline{V},\overline{U}).\]
      As $f[x]$ is trivial in $\hom_0(\overline{B},\overline{D}\cup\{x\})$ we have that $\ell[x] = (g\circ f)[x]$ is trivial, contradicting our hypothesis that $\ell$ is injective.
    \item[$\ell$ injective $\implies$ $V$ surrounds $U$ in $D$.] Suppose, for the sake of contradiction, that $V$ does not surround $U$ in $D$.
      Then there exists a path $\gamma : [0,1]\to\overline{V}$ with $\gamma(0)\in U\setminus V$ and $\gamma(1)\in D\setminus U$.
      As we have shown, $D\setminus B\subseteq U$, so $D\setminus B\subseteq U\setminus V$.

      Choose $x\in D\setminus B$ and $z\in \overline{D}$ such that there exist paths $\xi : [0,1]\to U\setminus V$ with $\xi(0) = x$, $\xi(1) = \gamma(0)$ and $\zeta : [0,1]\to \overline{D}\cup (D\setminus U)$ with $\zeta(0) = z$, $\zeta(1) = \gamma(1)$.
      $\xi, \gamma$ and $\zeta$ all generate chains in $C_1(\overline{V}, \overline{U})$ and $\xi + \gamma + \zeta = \gamma^*\in C_1(\overline{V}, \overline{U})$ with $\partial\gamma^* = x + z$.
      Moreover, $z$ generates a chain in $C_0(\overline{U})$ as $\overline{D}\subseteq\overline{U}$.
      So $x = \partial\gamma^* + z$ is a relative boundary in $C_0(\overline{V}, \overline{U})$, thus $\ell[x] = \ell[z]$ in $\hom_0(\overline{V}, \overline{L})$.
      However, because $B$ surrounds $D$, $[x]\neq [y]$ in $\hom_0(\overline{B}, \overline{D})$ contradicting our assumption that $\ell$ is injective.
  \end{description}
\end{proof}

% \begin{lemma}\label{lem:cov_surrounds}
%   If $\ell$ injective then $V$ surrounds $U$ in $D$.
% \end{lemma}
% \begin{proof}
%   (See Appendix~\ref{apx:omit})
% \end{proof}
% \proofatend
%   Suppose, for the sake of contradiction, that $V$ does not surround $U$ in $D$.
%   Then there exists a path $\gamma : [0,1]\to\overline{V}$ with $\gamma(0)\in U\setminus V$ and $\gamma(1)\in D\setminus U$.
%   By Lemma~\ref{lem:coverage} we know that $D\setminus B\subseteq U$, so $D\setminus B\subseteq U\setminus V$.
%
%   Choose $x\in D\setminus B$ and $z\in \overline{D}$ such that there exist paths $\xi : [0,1]\to U\setminus V$ with $\xi(0) = x$, $\xi(1) = \gamma(0)$ and $\zeta : [0,1]\to \overline{D}\cup (D\setminus U)$ with $\zeta(0) = z$, $\zeta(1) = \gamma(1)$.
%   $\xi, \gamma$ and $\zeta$ all generate chains in $C_1(\overline{V}, \overline{U})$ and $\xi + \gamma + \zeta = \gamma^*\in C_1(\overline{V}, \overline{U})$ with $\partial\gamma^* = x + z$.
%   Moreover, $z$ generates a chain in $C_0(\overline{U})$ as $\overline{D}\subseteq\overline{U}$.
%   So $x = \partial\gamma^* + z$ is a relative boundary in $C_0(\overline{V}, \overline{U})$, thus $\ell[x] = \ell[z]$ in $\hom_0(\overline{V}, \overline{L})$.
%   However, because $B$ surrounds $D$, $[x]\neq [y]$ in $\hom_0(\overline{B}, \overline{D})$ contradicting our assumption that $\ell$ is injective.
% \endproofatend

% In the following let $X$ be a topological space and $\overline{A} := X\setminus U$ denote the complement of a subset $U$ of $X$.
