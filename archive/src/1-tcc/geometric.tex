% !TeX root = ../../main.tex

We now combine these results on the homology of surrounding pairs with information about both $\X$ as a metric space and our function.
We will then apply these results as a computable variation using Vietoris-Rips complexes that requires only pair-wise connectivity information.

Let $(\X,\dist)$ be a metric space and $D\subseteq \X$ be a compact subspace.
For a $c$-Lipschitz function $f : D\to \R$ we introduce a constant $\omega$ as a threshold that defines our ``boundary'' as a sublevel set $B_\omega$ of the function $f$.
Let $P$ be a finite subset of $D$ and $\zeta\geq\delta > 0 $ be constants such that $P^\delta\subseteq \intr_\X(D)$.
Here, $\delta$ will serve as our communication radius where $\zeta$ is reserved for use in Section~\ref{sec:middle}.
  \footnote{We will set $\zeta = 2\delta$ in the proof of our interleaving with Rips complexes but the TCC holds for all $\zeta\geq\delta$.}

Unlike previous variations of the TCC~\cite{cavanna2017when} we do not require a change of scale in the geometric case.
Instead, we will enforce regularity close to the sublevel set $B_\omega$ in terms of sublevel sets $B_{\omega-c(\delta+\zeta)}$ and $B_{\omega+c(\delta+\zeta)}$.
Not only is this a more natural assumption, but it also allows us to replace the requirement that sensors detect the physical presence of a boundary with a threshold on the function values they observe.


\begin{lemma}\label{lem:psurj}
  Let $i : \hom_0(\overline{Q_{\omega+c\delta}^\delta}, \overline{P^\delta})\to \hom_0(\overline{Q_{\omega-c\zeta}^\delta}, \overline{P^\delta})$.

  If $B_\omega$ surrounds $D$ in $\X$ then $\dim~\hom_0(\overline{B_\omega}, \overline{D})\geq \rk~i$.
\end{lemma}
\begin{proof}
  Choose a basis for $\im~i$ such that each basis element is represented by a point in $P^\delta\setminus Q_{\omega+c\delta}^\delta$.
  Let $x\in P^\delta\setminus Q_{\omega+c\delta}^\delta$ be such that $i[x]$ is non-trivial.
  So there exits some $p\in P$ such that $\dist(p, x) < \delta$ and $p\notin Q_{\omega+c\delta}$, otherwise $x\in Q_{\omega+c\delta}^\delta$.
  Therefore, because $f$ is $c$-Lipschitz,
  \[ f(x)\geq f(p) - c\dist(x, p) > \omega.\]

  So $x\in\overline{B_\omega}$ and, because $x\in P^\delta\subseteq D$ it follows that $x\in D\setminus B_\omega$.
  Because $i$ and $\ell : \hom_0(\overline{B_\omega}, \overline{D})\to \hom_0(\overline{Q_{\omega-c\zeta}^\delta}, \overline{P^\delta})$ are induced by inclusion $\ell[x] = i[x]$ is non-trivial in $\hom_0(\overline{Q_{\omega-c\zeta}^\delta}, \overline{P^\delta})$.
  That is, every element of $\im~i$ has a preimage in $\hom_0(\overline{B_\omega}, \overline{D})$, so we may conclude that $\dim~\hom_0(\overline{B_\omega}, \overline{D})\geq \rk~i$.
\end{proof}

While there is a surjective map from $\hom_0(\overline{B_\omega}, \overline{D})$ to $\im~i$ this map is not necessarily induced by inclusion.
We will therefore introduce a larger space $B_{\omega+c(\delta+\zeta)}$ that contains $Q_{\omega+c\delta}^\delta$ in order to provide a criteria for the injectivity of $\ell : \hom_0(\overline{B_\omega}, \overline{D})\to\hom_0(\overline{Q_{\omega-c\zeta}^\delta}, \overline{P^\delta})$ in terms of $\rk~i$.
We have the following commutative diagrams of inclusion maps and maps induced by inclusion between complements in $\X$.

\begin{equation}\label{dgm:1}
\begin{tikzcd}
  (P^\delta, Q_{\omega-c\zeta}^\delta) \arrow[hookrightarrow]{r}\arrow[hookrightarrow]{d} &
  (P^\delta, Q_{\omega+c\delta}^\delta) \arrow[hookrightarrow]{d} \\
  %
  (D, B_\omega) \arrow[hookrightarrow]{r} &
  (D, B_{\omega+c(\delta+\zeta)}),
\end{tikzcd}
\begin{tikzcd}
  \hom_0(\overline{B_{\omega+c(\delta+\zeta)}},\overline{D})\arrow{d}{m} \arrow{r}{j} &
  \hom_0(\overline{B_\omega}, \overline{D}) \arrow{d}{\ell} \\
  %
  \hom_0(\overline{Q_{\omega+c\delta}^\delta}, \overline{P^\delta}) \arrow{r}{i} &
  \hom_0(\overline{Q_{\omega-c\zeta}^\delta}, \overline{P^\delta}).
\end{tikzcd}\end{equation}

\paragraph{Assumption 1}

We will require the map $\hom_0(D\setminus B_{\omega+c(\delta+\zeta)}\hookrightarrow D\setminus B_\omega)$ to be \emph{surjective}---as we approach $\omega$ from \emph{above} no components \emph{appear}.
That is, in terms of $\omega$ as a super-levelset monotonically decreasing, no components \emph{apear} right \emph{before} $\omega$.
We note that, for a function in two dimensions, this translates to $1$-dimensional features disappearing right after $\omega$ in the sublevel set filtration, as shown in Figure~\ref{fig:assumption1}.

\begin{figure}[htbp]\label{fig:assumption1}
  \centering
  \includegraphics[trim=200 300 200 200, clip, width=0.5\textwidth]{figures/surf-ass1_C_side.png}
  \includegraphics[trim=300 150 200 200, clip, width=0.3\textwidth]{figures/surf-ass1_C_top.png}
  \includegraphics[trim=200 300 200 200, clip, width=0.5\textwidth]{figures/surf-ass1_D_side.png}
  \includegraphics[trim=300 150 200 200, clip, width=0.3\textwidth]{figures/surf-ass1_D_top.png}
  \includegraphics[scale=0.7]{figures/scalar_barcode_H1-masked.png}
  \caption{\textbf{(Assumption 1)} The blue levelset does not satisfy Assumption 1 as the smaller component is ``pinched out'' in the orange region.
            This can be seen in the barcode shown as a feature that dies in the purple region.}
\end{figure}

Now, the rank of the map $j$ is equal to the dimension of $\dim~\hom_0(\overline{B_\omega}, \overline{D})$ and our map $\ell$ induced by inclusion depends only on $\hom_0(\overline{B_\omega}, \overline{D})$ and $\im~i$.
The second assumption, which requires that nothing appears right \emph{before} $\omega$, will be used in Theorem~\ref{thm:algo_tcc} to provide a computable upper bound on $\rk~j$.

\begin{theorem}[Geometric TCC]\label{thm:geo_tcc}
  Let $D$ be a compact subset of $\X$ and $f : D\to\R$ be $c$-Lipschitz function.
  Let $\omega\in\R$, $\zeta\geq\delta > 0$ be constants such that $B_{\omega}$ surrounds $D$ in $\X$ and let $P\subset D$ be a finite collection of points.
  Let $j : \hom_0(\overline{B_{\omega+c(\delta+\zeta)}},\overline{D})\to \hom_0(\overline{B_{\omega}},\overline{D})$ and $i : \hom_0(\overline{Q_{\omega+c\delta}^\delta}, \overline{P^\delta})\to \hom_0(\overline{Q_{\omega-c\zeta}^\delta}, \overline{P^\delta})$ be induced by inclusion.

  If $j$ is surjective and $\rk~i\geq \rk~j$ then $D\setminus B_{\omega}\subseteq P^\delta$ and $Q_{\omega-c\zeta}^\delta$ surrounds $P^\delta$ in $D$.
\end{theorem}
\begin{proof}
  Because $j$ is surjective by hypothesis $\rk~j = \dim~\hom_0(\overline{B_{\omega}},\overline{D})$ so $\rk~j\geq \rk~i$ by Lemma~\ref{lem:psurj}.
  So $\rk~j = \rk~i$ with our assumption that $\rk~i\geq \rk~j$.
  Because $P$ is a finite point set we know that $\im~i$ is finite-dimensional and, because $\rk~i = \rk~j$, $\im~j=\hom_0(\overline{B_{\omega}}, \overline{D})$ is finite dimensional as well.

  So $\im~j$ is isomorphic to $\im~i$ as a subspace of $\hom_0(\overline{Q_{\omega-c\zeta}^\delta}, \overline{P^\delta})$ which, because $j$ is surjective, requires the map $\ell$ induced by inclusion to be injective.
  Therefore, $D\setminus B_{\omega}\subseteq P^\delta$ and $Q_{\omega-c\zeta}^\delta$ surrounds $P^\delta$ in $D$ by Lemma~\ref{lem:coverage}. %, Lemma~\ref{lem:cov_surrounds}.
\end{proof}
