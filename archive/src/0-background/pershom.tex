% !TeX root = ../../main.tex

% We are interested in a particular persistence module that can be obtained by applying homology as a functor from the category of topological spaces to the category of abelian groups.
Given a topological space $X$ a \textbf{filtration} of $X$ is a sequence of subspaces $\emptyset = X_0\subseteq X_1\subseteq\ldots\subseteq X_n = X$.
The \textbf{$k$th persistent homology module} of a filtration $\{X_i\}_{i\in I}$ over some indexing set $I$ is the persistence module consisting of vector spaces $\hom_k(X_i)$ and linear maps $\hom_k(X_i)\to \hom_k(X_j)$ induced by inclusion for $i\leq j$.
Similarly, given a pair $(X,Y)$ and filtrations $\{X_i\}_{i\in I}$ and $\{Y_i\}_{i\in I}$ such that $Y_i\subseteq X_i$ for all $i\in I$ the $k$th persistent relative homology module is the persistence module of vector spaces $\hom_k(X_i, Y_i)$ and linear maps $\hom_k(X_i, Y_i)\to\hom_k(X_j, Y_j)$ induced by inclusions for all $i\leq j$.

Given a function $f: X\to \R$ we can define a filtration known as \textbf{sublevel set filtration} as a sequence of sublevel sets $\{f^{-1}((-\infty, \alpha])\}_{\alpha\in\R}$ where $f^{-1}((-\infty, \alpha]) := \{x\in X\mid f(x)\leq\alpha\}$.
The superlevel set filtration of $f$ is similarly defined $\{f^{-1}([\alpha,\infty))\}_{\alpha\in\R}$.
% The persistent homology of the sub(or super)-levelset filtration is typically referred to as the persistent homology of $f$.

The persistent homology of a filtration encodes the homology of a space as it changes monotonically in some way.
In the case of the sublevel set filtration, the homological structure of subspaces of $X$ that map to intervals $(-\infty,\alpha]$ as $\alpha$ increases.
These changes are usually referred to either the birth and death of a \emph{feature} in some dimension---homology classes that \emph{persist} over an interval $[b, d)\subset\R$.
This information can be summarized by a \textbf{persistence diagram} or \textbf{barcode} which visualizes these features either as pairs of points $(b, d)\in\R^2$, or as intervals $[b, d)\subset\R$.

There is a metric on the space of persistence diagrams known as the \textbf{bottleneck distance}.
In most cases we are concerned with, the bottleneck distance is equal to the interleaving distance---that is, if two persistent homology modules are $\delta$-interleaved then the bottleneck distance between their associated diagrams is $\delta$.
