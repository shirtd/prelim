% !TeX root = ../../main.tex

A \textbf{simplicial complex} $K$ is a collection of subsets, called \textbf{simplices}, of a vertex set $V$ such that for all $\sigma\in K$ and $\tau\subset\sigma$ it must follow that $\tau\in K$.
The \textbf{dimension} of a simplex $\sigma\in K$ is defined as $\dim(\sigma) := |\sigma|-1$ where $|\cdot|$ denotes set cardinality.
The dimension of a simplicial complex $K$ is the maximum dimension of any simplex in $K$.
That is, a graph is a 1-dimensional simplicial complex in which vertices and edges are 0 and 1-dimensional simplices, respectively.

It is natural to think of a $k$-dimensional simplicial complex as the generalization of an undirected graph consisting of vertices and edges, collections of at most 2 vertices, to collections of sets of at most $k+1$ vertices.

\paragraph{Nerves}

Let $X$ be a topological space.
For some index set $I$ and $U_i\subseteq X$ a collection $\U = \{U_i\}_{i\in I}$ is called a \textbf{cover} of $X$ if $\bigcup_{i\in I} U_i = X$.
The cover is said to be open if all of the sets $U_i$ are open in $X$.
A finite open cover $\U$ is a \textbf{good open cover} if every nonempty intersection of sets in $\U$ is contractible.

There is a simplicial complex associated with any cover $\U$ known as its \textbf{Nerve}, denoted $\N(\U)$, that is defined to be the abstract simplicial complex with vertex set $I$ and simplices $\sigma\subseteq I$ whenever $\bigcap_{i\in\sigma} U_i\neq \emptyset$.
The \textbf{Nerve Theorem} states that whenever $\U$ is a good open cover of a paracompact space $X$ then its nerve $\N(\U)$ is homotopy equivalent to $X = \bigcup_{i\in I} U_i$.
The following is an important result by Chazal et. al.~\cite{chazal08towards}.

\begin{lemma}[\textbf{Persistent Nerve Lemma} (Chazal et. al.~\cite{chazal08towards}, Lemma 3.4)]\label{lem:pers_nerve}
  Let $X\subseteq X'$ be two paracompact spaces, and let $\U = \{U_i\}_{i\in I}$ and $\mathcal{U}' = \{U_i'\}_{i\in I}$ be good open covers of $X$ and $X'$, respectively, based on some finite parameter set $I$, such that $U_i\subseteq U_i'$ for all $i\in I$.
  Then there exist homotopy equivalences of pairs $\N\U\to X$ and $\N\U'\to X'$ that commute with the canonical inclusions $X \hookrightarrow X'$ and $\N\U\hookrightarrow \N\U'$ at the homology and homotopy levels.
\end{lemma}

We say $\V = \{V_i\}_{i\in I}$ is a subcover of a cover $\U = \{U_i\}_{i\in I}$ if $V_i\subseteq U_i$ for all $i\in I$.
The standard proof of the Nerve Theorem, and therefore the Persistent Nerve Lemma, extends directly to pairs of good open covers $(\U, \V)$ of pairs $(X, Y)$ such that $\V$ is a subcover of $\U$.

\paragraph{The \v Cech Complex}

If $(X,\dist)$ is a metric space and $P\subset X$ is a finite collection we say that $P$ covers $X$ at scale $\e$ if the collection $\U = \{\ball^\e(p)\}_{p\in P}$ is a cover of $X$.
The Nerve of this cover is known as the \textbf{\v Cech complex} at scale $\e$, denoted $\cech^\e(P)$, and is defined formally as
\[ \cech^\e(P) := \left\{\sigma \subseteq P\mid \bigcap_{p\in \sigma}\ball^\e(p)\neq \emptyset \right\}. \]
For pairs $(P, Q)$ we will write $\cech^\e(P,Q) := (\cech^\e(P), \cech^\e(Q))$ to denote the corresponding pair of \v Cech complexes.

Recalling the definition of the strong convexity radius $\varrho_X$ of a metric space $X$ we note that $\U$ is a good open cover whenever $\varrho_X > \e$.
That is, by the Nerve Theorem, the inclusion $\cech^\e(P)\hookrightarrow P^\e$ is a homotopy equivalence whenever $\varrho_X > \e$.

\paragraph{The Vietoris-Rips Complex}

Given an undirected graph $G = (V, E)$ we can construct a simplicial complex $K$ with vertex set $V$ and simplices $\sigma\subseteq V$ whenever $\{u,v\}\in E$ for all distinct $u, v\in\sigma$.
This is known as the \emph{clique complex} of the graph $G$.

Just as the \v Cech complex is the Nerve of a cover by metric balls there is a closely related simplicial complex that is the clique complex of a neighborhood graph.
That is, a graph $G$ with vertices $V$ corresponding to points in a metric space $(X,\dist)$ and $\{u,v\}\in E$ whenever $\dist(u,v)\leq \e$ for some $\e > 0$.
This is known as the \textbf{(Vietoris-)Rips complex}, and is formally defined for a finite set $P$ at scale $\e > 0$ as
\[ \rips^\e(P) = \left\{\sigma \subseteq P\mid \forall p,q\in\sigma,\ \dist(p, q)\leq \e\right\}. \]
For a pair $(P, Q)$ we will write $\rips^\e(P,Q) := (\rips^\e(P), \rips^\e(Q))$ to denote the corresponding pair of Rips complexes.

By the Nerve Theorem we can use the \v Cech complex to compute the homotopy type of metric space $(X, \dist)$ covered by a finite set $P$ at any scale $\e < \varrho_X$.
Unfortunately, constructing the \v Cech complex is not feasible in practice.
We will therefore use pairs of Rips complex to capture to homotopy type of the \v Cech complex by making use of the following inclusion:
\[ \rips^\e(P) \subseteq \cech^\e(P)\subseteq \rips^{2\e}(P).\]
