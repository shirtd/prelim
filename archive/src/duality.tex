% !TeX root = ../main.tex

For a pair $(A, B)$ in a topological space $X$ and any $R$ module $G$ let $\hom^k(A, B; G)$ denote the \textbf{singular cohomology} of $(A,B)$ (with coefficients in $G$).
Let $\hom^k_c(A, B; G)$ denote the corresponding \textbf{singular cohomology with compact support}.
For any compact pair $(A,B)$ there is an isomorphism $\hom^k_c(A, B; G)\to\hom^k(A, B; G)$.

Corollary\ref{cor:univ_coef} follows from the Universal Coefficient Theorem for singular homology (and cohomology) as vector spaces over a field $\FF$, as the dual vector space $\Hom(\hom_k(A, B), \FF)$ is isomorphic to $\hom_k(A, B; \FF)$ for any finitely generated $\hom_k(A, B)$.

\begin{corollary}\label{cor:univ_coef}
  For a topological pair $(A, B)$ and a field $\FF$ such that $\hom_k(A, B)$ is finitely generated there is a natural isomorphism
  \[\nu : \hom^k(A, B; \FF)\to \hom_k(A, B; \FF).\]
\end{corollary}

Let $\overline{\hom}^k(A, B; G)$ be the \textbf{Alexander-Spanier cohomology} of the pair $(A,B)$, defined as the limit of the direct system of neighborhoods $(U,V)$ of the pair $(A, B)$.
Let $\overline{\hom}^k_c(A, B; G)$ denote the corresponding \textbf{Alexander-Spanier cohomology with compact support} where $\overline{\hom}^k_c(A, B; G)\cong\overline{\hom}^k(A, B; G)$ for any compact pair $(A, B)$.

\begin{theorem}[\textbf{Alexander-Poincar\'e-Lefschetz Duality} (Spanier~\cite{spanier1989algebraic}, Theorem 6.2.17)]\label{thm:alexander}
  Let $X$ be an orientable $d$-manifold and $(A, B)$ be a compact pair in $X$.
  Then for all $k$ and $R$ modules $G$ there is a (natural) isomorphism
  \[\lambda : \hom_k(X\setminus B, X\setminus A; G)\to \overline{\hom}^{d-k}(A, B; G).\]
\end{theorem}

A space $X$ is said to be \textbf{homologically locally connected in dimension $n$} if for every $x\in X$ and neighborhood $U$ of $x$ there exists a neighborhood $V$ of $x$ in $U$ such that $\tilde{\hom}_n(V)\to\tilde{\hom}_n(U)$ is trivial for $k\leq n$.

\begin{lemma}[Spanier p. 341, Corollary 6.9.6]\label{lem:alexander_iso}
  Let $A$ be a closed subset, homologically locally connected in dimension $n$, of a Hausdorff space $X$, homologically locally connected in dimension $n$.
  If $X$ has the property that every open subset is paracompact, $\mu : \overline{\hom}_c^k(X,A; G)\to \hom_c^k(X, A; G)$ is an isomorphism for $k\leq n$ and a monomorphism for $q = n+1$.
\end{lemma}

In the following we will assume homology (and cohomology) over a field $\FF$.

\begin{lemma}\label{cor:alexander_iso}
  Let $X$ be an orientable $d$-manifold and $(A,B)$ a compact pair of locally path connected subspaces in $X$.
  Then
  \[\xi : \hom_d(X\setminus B, X\setminus  A)\to \hom_0(A, B)\]
  is a natural isomorphism.
\end{lemma}
\begin{proof}
  Because $X$ is orientable and $(A,B)$ are compact $\lambda : \hom_d(X\setminus B, X\setminus A)\to \overline{\hom}^{0}(A, B)$ is an isomorphism by Theorem~\ref{thm:alexander}.
  Note that
  Moreover, because every subset of $X$ is (hereditarily) paracompact every open set in $A$, with the subspace topology, is paracompact.
  For any neighborhood $U$ of a point $x$ in a locally path connected space there must exist some neighborhood $V\subset U$ of $x$ that is path connected in the subspace topology.
  As $\tilde{\hom}_0(V) = 0$ for any nonempty, path connected topological space $V$ (see Spanier p. 175, Lemma 4.4.7) it follows that $A$ (resp. $B$) are homologically locally connected in dimension $0$.
  Because $(A,B)$ is a compact pair the singular and Alexander-spanier cohomology modules of $(A,B)$ with compact support are isomorphic to those without, thus $\mu:\overline{\hom}^{0}(A, B)\to \hom^0(A, B)$ is an isomorphism.
  By Corollary~\ref{cor:univ_coef} we have a natural isomorphism $\nu : \hom^0(A, B)\to\hom_0(A, B)$ thus the composition $\xi := \nu\circ\mu\circ\lambda : \hom_d(X\setminus B, X\setminus  A)\to \hom_0(A, B)$ is a natural isomorphism.
\end{proof}

\begin{lemma}\label{lem:duality_apply}
  Let $\X$ be an orientable $d$-manifold let $D$ be a compact subset of $\X$.
  Let $P$ be a finite subset of $D$ such that $P^\e\subset \intr_\X(D)$ and $Q\subseteq P$.

  If $D\setminus Q^\e$ and $D\setminus P^\e$ are locally path connected then there is a natural isomorphism
  \[ \xi : \hom_d(P^\e,Q^\e)\to \hom_0(D\setminus Q^\e, D\setminus P^\e).\]
\end{lemma}
\begin{proof}
  Because $Q^\e$ and $P^\e$ are open in $D$ and $D$ is compact in $\X$ the complement $D\setminus Q^\e$ is closed in $D$, and therefore compact in $\X$.
  Moreover, because $P^\e\subset \intr_\X(D)$, $\hom_d(\X\setminus(D\setminus P^\e), \X\setminus(D\setminus Q^\e)) = \hom_d(P^\e, Q^\e)$.
  As we have assumed these complements are locally path connected by assumption we have a natural isomorphism $\xi : \hom_d(P^\e, Q^\e)\to \hom_0(D\setminus Q^\e, D\setminus P^\e)$
  by Lemma~\ref{cor:alexander_iso}.
\end{proof}
